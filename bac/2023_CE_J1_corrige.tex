% Options for packages loaded elsewhere
\PassOptionsToPackage{unicode}{hyperref}
\PassOptionsToPackage{hyphens}{url}
\PassOptionsToPackage{dvipsnames,svgnames,x11names}{xcolor}
%
\documentclass[
  letterpaper,
  DIV=11,
  numbers=noendperiod]{scrartcl}

\usepackage{amsmath,amssymb}
\usepackage{iftex}
\ifPDFTeX
  \usepackage[T1]{fontenc}
  \usepackage[utf8]{inputenc}
  \usepackage{textcomp} % provide euro and other symbols
\else % if luatex or xetex
  \usepackage{unicode-math}
  \defaultfontfeatures{Scale=MatchLowercase}
  \defaultfontfeatures[\rmfamily]{Ligatures=TeX,Scale=1}
\fi
\usepackage{lmodern}
\ifPDFTeX\else  
    % xetex/luatex font selection
\fi
% Use upquote if available, for straight quotes in verbatim environments
\IfFileExists{upquote.sty}{\usepackage{upquote}}{}
\IfFileExists{microtype.sty}{% use microtype if available
  \usepackage[]{microtype}
  \UseMicrotypeSet[protrusion]{basicmath} % disable protrusion for tt fonts
}{}
\makeatletter
\@ifundefined{KOMAClassName}{% if non-KOMA class
  \IfFileExists{parskip.sty}{%
    \usepackage{parskip}
  }{% else
    \setlength{\parindent}{0pt}
    \setlength{\parskip}{6pt plus 2pt minus 1pt}}
}{% if KOMA class
  \KOMAoptions{parskip=half}}
\makeatother
\usepackage{xcolor}
\usepackage[top=20mm,bottom=20mm,left=20mm,right=20mm,heightrounded]{geometry}
\setlength{\emergencystretch}{3em} % prevent overfull lines
\setcounter{secnumdepth}{-\maxdimen} % remove section numbering
% Make \paragraph and \subparagraph free-standing
\ifx\paragraph\undefined\else
  \let\oldparagraph\paragraph
  \renewcommand{\paragraph}[1]{\oldparagraph{#1}\mbox{}}
\fi
\ifx\subparagraph\undefined\else
  \let\oldsubparagraph\subparagraph
  \renewcommand{\subparagraph}[1]{\oldsubparagraph{#1}\mbox{}}
\fi

\usepackage{color}
\usepackage{fancyvrb}
\newcommand{\VerbBar}{|}
\newcommand{\VERB}{\Verb[commandchars=\\\{\}]}
\DefineVerbatimEnvironment{Highlighting}{Verbatim}{commandchars=\\\{\}}
% Add ',fontsize=\small' for more characters per line
\usepackage{framed}
\definecolor{shadecolor}{RGB}{241,243,245}
\newenvironment{Shaded}{\begin{snugshade}}{\end{snugshade}}
\newcommand{\AlertTok}[1]{\textcolor[rgb]{0.68,0.00,0.00}{#1}}
\newcommand{\AnnotationTok}[1]{\textcolor[rgb]{0.37,0.37,0.37}{#1}}
\newcommand{\AttributeTok}[1]{\textcolor[rgb]{0.40,0.45,0.13}{#1}}
\newcommand{\BaseNTok}[1]{\textcolor[rgb]{0.68,0.00,0.00}{#1}}
\newcommand{\BuiltInTok}[1]{\textcolor[rgb]{0.00,0.23,0.31}{#1}}
\newcommand{\CharTok}[1]{\textcolor[rgb]{0.13,0.47,0.30}{#1}}
\newcommand{\CommentTok}[1]{\textcolor[rgb]{0.37,0.37,0.37}{#1}}
\newcommand{\CommentVarTok}[1]{\textcolor[rgb]{0.37,0.37,0.37}{\textit{#1}}}
\newcommand{\ConstantTok}[1]{\textcolor[rgb]{0.56,0.35,0.01}{#1}}
\newcommand{\ControlFlowTok}[1]{\textcolor[rgb]{0.00,0.23,0.31}{#1}}
\newcommand{\DataTypeTok}[1]{\textcolor[rgb]{0.68,0.00,0.00}{#1}}
\newcommand{\DecValTok}[1]{\textcolor[rgb]{0.68,0.00,0.00}{#1}}
\newcommand{\DocumentationTok}[1]{\textcolor[rgb]{0.37,0.37,0.37}{\textit{#1}}}
\newcommand{\ErrorTok}[1]{\textcolor[rgb]{0.68,0.00,0.00}{#1}}
\newcommand{\ExtensionTok}[1]{\textcolor[rgb]{0.00,0.23,0.31}{#1}}
\newcommand{\FloatTok}[1]{\textcolor[rgb]{0.68,0.00,0.00}{#1}}
\newcommand{\FunctionTok}[1]{\textcolor[rgb]{0.28,0.35,0.67}{#1}}
\newcommand{\ImportTok}[1]{\textcolor[rgb]{0.00,0.46,0.62}{#1}}
\newcommand{\InformationTok}[1]{\textcolor[rgb]{0.37,0.37,0.37}{#1}}
\newcommand{\KeywordTok}[1]{\textcolor[rgb]{0.00,0.23,0.31}{#1}}
\newcommand{\NormalTok}[1]{\textcolor[rgb]{0.00,0.23,0.31}{#1}}
\newcommand{\OperatorTok}[1]{\textcolor[rgb]{0.37,0.37,0.37}{#1}}
\newcommand{\OtherTok}[1]{\textcolor[rgb]{0.00,0.23,0.31}{#1}}
\newcommand{\PreprocessorTok}[1]{\textcolor[rgb]{0.68,0.00,0.00}{#1}}
\newcommand{\RegionMarkerTok}[1]{\textcolor[rgb]{0.00,0.23,0.31}{#1}}
\newcommand{\SpecialCharTok}[1]{\textcolor[rgb]{0.37,0.37,0.37}{#1}}
\newcommand{\SpecialStringTok}[1]{\textcolor[rgb]{0.13,0.47,0.30}{#1}}
\newcommand{\StringTok}[1]{\textcolor[rgb]{0.13,0.47,0.30}{#1}}
\newcommand{\VariableTok}[1]{\textcolor[rgb]{0.07,0.07,0.07}{#1}}
\newcommand{\VerbatimStringTok}[1]{\textcolor[rgb]{0.13,0.47,0.30}{#1}}
\newcommand{\WarningTok}[1]{\textcolor[rgb]{0.37,0.37,0.37}{\textit{#1}}}

\providecommand{\tightlist}{%
  \setlength{\itemsep}{0pt}\setlength{\parskip}{0pt}}\usepackage{longtable,booktabs,array}
\usepackage{calc} % for calculating minipage widths
% Correct order of tables after \paragraph or \subparagraph
\usepackage{etoolbox}
\makeatletter
\patchcmd\longtable{\par}{\if@noskipsec\mbox{}\fi\par}{}{}
\makeatother
% Allow footnotes in longtable head/foot
\IfFileExists{footnotehyper.sty}{\usepackage{footnotehyper}}{\usepackage{footnote}}
\makesavenoteenv{longtable}
\usepackage{graphicx}
\makeatletter
\def\maxwidth{\ifdim\Gin@nat@width>\linewidth\linewidth\else\Gin@nat@width\fi}
\def\maxheight{\ifdim\Gin@nat@height>\textheight\textheight\else\Gin@nat@height\fi}
\makeatother
% Scale images if necessary, so that they will not overflow the page
% margins by default, and it is still possible to overwrite the defaults
% using explicit options in \includegraphics[width, height, ...]{}
\setkeys{Gin}{width=\maxwidth,height=\maxheight,keepaspectratio}
% Set default figure placement to htbp
\makeatletter
\def\fps@figure{htbp}
\makeatother

\usepackage{fancyhdr} \pagestyle{fancy} \usepackage{lastpage}
\KOMAoption{captions}{tablesignature}
\makeatletter
\@ifpackageloaded{tcolorbox}{}{\usepackage[skins,breakable]{tcolorbox}}
\@ifpackageloaded{fontawesome5}{}{\usepackage{fontawesome5}}
\definecolor{quarto-callout-color}{HTML}{909090}
\definecolor{quarto-callout-note-color}{HTML}{0758E5}
\definecolor{quarto-callout-important-color}{HTML}{CC1914}
\definecolor{quarto-callout-warning-color}{HTML}{EB9113}
\definecolor{quarto-callout-tip-color}{HTML}{00A047}
\definecolor{quarto-callout-caution-color}{HTML}{FC5300}
\definecolor{quarto-callout-color-frame}{HTML}{acacac}
\definecolor{quarto-callout-note-color-frame}{HTML}{4582ec}
\definecolor{quarto-callout-important-color-frame}{HTML}{d9534f}
\definecolor{quarto-callout-warning-color-frame}{HTML}{f0ad4e}
\definecolor{quarto-callout-tip-color-frame}{HTML}{02b875}
\definecolor{quarto-callout-caution-color-frame}{HTML}{fd7e14}
\makeatother
\makeatletter
\makeatother
\makeatletter
\makeatother
\makeatletter
\@ifpackageloaded{caption}{}{\usepackage{caption}}
\AtBeginDocument{%
\ifdefined\contentsname
  \renewcommand*\contentsname{Table des matières}
\else
  \newcommand\contentsname{Table des matières}
\fi
\ifdefined\listfigurename
  \renewcommand*\listfigurename{Liste des Figures}
\else
  \newcommand\listfigurename{Liste des Figures}
\fi
\ifdefined\listtablename
  \renewcommand*\listtablename{Liste des Tables}
\else
  \newcommand\listtablename{Liste des Tables}
\fi
\ifdefined\figurename
  \renewcommand*\figurename{Figure}
\else
  \newcommand\figurename{Figure}
\fi
\ifdefined\tablename
  \renewcommand*\tablename{Tableau}
\else
  \newcommand\tablename{Tableau}
\fi
}
\@ifpackageloaded{float}{}{\usepackage{float}}
\floatstyle{ruled}
\@ifundefined{c@chapter}{\newfloat{codelisting}{h}{lop}}{\newfloat{codelisting}{h}{lop}[chapter]}
\floatname{codelisting}{Listing}
\newcommand*\listoflistings{\listof{codelisting}{Liste des Listings}}
\makeatother
\makeatletter
\@ifpackageloaded{caption}{}{\usepackage{caption}}
\@ifpackageloaded{subcaption}{}{\usepackage{subcaption}}
\makeatother
\makeatletter
\@ifpackageloaded{tcolorbox}{}{\usepackage[skins,breakable]{tcolorbox}}
\makeatother
\makeatletter
\@ifundefined{shadecolor}{\definecolor{shadecolor}{rgb}{.97, .97, .97}}
\makeatother
\makeatletter
\makeatother
\makeatletter
\makeatother
\ifLuaTeX
\usepackage[bidi=basic]{babel}
\else
\usepackage[bidi=default]{babel}
\fi
\babelprovide[main,import]{french}
% get rid of language-specific shorthands (see #6817):
\let\LanguageShortHands\languageshorthands
\def\languageshorthands#1{}
\ifLuaTeX
  \usepackage{selnolig}  % disable illegal ligatures
\fi
\IfFileExists{bookmark.sty}{\usepackage{bookmark}}{\usepackage{hyperref}}
\IfFileExists{xurl.sty}{\usepackage{xurl}}{} % add URL line breaks if available
\urlstyle{same} % disable monospaced font for URLs
\hypersetup{
  pdftitle={Centres étrangers 2023 Jour 1},
  pdflang={fr},
  colorlinks=true,
  linkcolor={blue},
  filecolor={Maroon},
  citecolor={Blue},
  urlcolor={Blue},
  pdfcreator={LaTeX via pandoc}}

\title{Centres étrangers 2023 Jour 1}
\author{}
\date{}

\begin{document}
\maketitle
\lhead{Spécialité NSI} \rhead{Terminale} \chead{} \cfoot{} \lfoot{Lycée \'Emile Duclaux} \rfoot{Page \thepage/\pageref{LastPage}} \renewcommand{\headrulewidth}{0pt} \renewcommand{\footrulewidth}{0pt} \thispagestyle{fancy} \vspace{-2cm}

\ifdefined\Shaded\renewenvironment{Shaded}{\begin{tcolorbox}[interior hidden, borderline west={3pt}{0pt}{shadecolor}, breakable, enhanced, frame hidden, boxrule=0pt, sharp corners]}{\end{tcolorbox}}\fi

\hypertarget{exercice-1}{%
\subsection{Exercice 1}\label{exercice-1}}

\begin{enumerate}
\def\labelenumi{\arabic{enumi}.}
\item
  Schéma relationnel de la relation \texttt{description} (la clé
  primaire est soulignée et la clé étrangère est précédée du symbole
  \#):

  description({id\_description} : INT, resume : TEXT, duree : INT,
  \#id\_emission : INT)
\item
  \begin{enumerate}
  \def\labelenumii{\alph{enumii}.}
  \tightlist
  \item
    La requête affiche le résultat suivant :
  \end{enumerate}

  \begin{longtable}[]{@{}
    >{\raggedright\arraybackslash}p{(\columnwidth - 2\tabcolsep) * \real{0.4375}}
    >{\centering\arraybackslash}p{(\columnwidth - 2\tabcolsep) * \real{0.5625}}@{}}
  \toprule\noalign{}
  \begin{minipage}[b]{\linewidth}\raggedright
  theme
  \end{minipage} & \begin{minipage}[b]{\linewidth}\centering
  annee
  \end{minipage} \\
  \midrule\noalign{}
  \endhead
  \bottomrule\noalign{}
  \endlastfoot
  Le système d'enseignement supérieur français est-il juste et efficace
  ? & 2022 \\
  Trois innovations pour la croissance future (1/3) : La révolution
  blockchain & 2021 \\
  \end{longtable}

  \begin{enumerate}
  \def\labelenumii{\alph{enumii}.}
  \setcounter{enumii}{1}
  \tightlist
  \item
    Requête permettant d'afficher les thèmes des podcasts de l'année
    2019 :
  \end{enumerate}

\begin{Shaded}
\begin{Highlighting}[]
\KeywordTok{SELECT}\NormalTok{ theme}
\KeywordTok{FROM}\NormalTok{ podcast}
\KeywordTok{WHERE}\NormalTok{ annee }\OperatorTok{=} \DecValTok{2019}
\end{Highlighting}
\end{Shaded}

  \begin{enumerate}
  \def\labelenumii{\alph{enumii}.}
  \setcounter{enumii}{2}
  \tightlist
  \item
    Requête permettant d'afficher la liste des thèmes et des années de
    diffusion des podcasts dans l'ordre chronologique des années :
  \end{enumerate}

\begin{Shaded}
\begin{Highlighting}[]
\KeywordTok{SELECT}\NormalTok{ theme, annee}
\KeywordTok{FROM}\NormalTok{ podcast}
\KeywordTok{ORDER} \KeywordTok{BY}\NormalTok{ annee}
\end{Highlighting}
\end{Shaded}
\item
  \begin{enumerate}
  \def\labelenumii{\alph{enumii}.}
  \item
    La requête proposée affiche la liste de tous les thèmes de la
    relation \texttt{podcast} \textbf{sans répétition}.
  \item
    Requête SQL supprimant la ligne contenant
    l'\texttt{id\_podcast\ =\ 40} de la relation \texttt{podcast} :
  \end{enumerate}

\begin{Shaded}
\begin{Highlighting}[]
\KeywordTok{DELETE} \KeywordTok{FROM}\NormalTok{ podcast}
\KeywordTok{WHERE}\NormalTok{ id\_podcast }\OperatorTok{=} \DecValTok{40}
\end{Highlighting}
\end{Shaded}
\item
  \begin{enumerate}
  \def\labelenumii{\alph{enumii}.}
  \item
    Requête SQL permettant de changer le nom de l'animateur de
    l'émission ``Le Temps de débat'' en ``Emmanuel L''.

\begin{Shaded}
\begin{Highlighting}[]
\KeywordTok{UPDATE}\NormalTok{ emission}
\KeywordTok{SET}\NormalTok{ animateur }\OperatorTok{=} \OtherTok{"Emmanuel L"}
\KeywordTok{WHERE}\NormalTok{ nom }\OperatorTok{=} \OtherTok{"Le Temps de débat"}
\end{Highlighting}
\end{Shaded}
  \item
    Requête SQL permettant d'ajouter l'émission ``Hashtag'' sur la radio
    ``France inter'' avec ``Mathieu V.'', avec un \texttt{id\_emission}
    égal à 12850.

\begin{Shaded}
\begin{Highlighting}[]
\KeywordTok{INSERT} \KeywordTok{INTO}\NormalTok{ emission (id\_emission, nom, radio, animateur)}
\KeywordTok{VALUES}\NormalTok{ (}\DecValTok{12850}\NormalTok{, }\OtherTok{"Hashtag"}\NormalTok{, }\OtherTok{"France inter"}\NormalTok{, }\OtherTok{"Mathieu V."}\NormalTok{)}
\end{Highlighting}
\end{Shaded}
  \end{enumerate}
\item
  Requête permettant de lister les thèmes, le nom des émissions et le
  résumé des podcasts pour lesquels la durée est strictement inférieure
  à 5 minutes.

\begin{Shaded}
\begin{Highlighting}[]
\KeywordTok{SELECT}\NormalTok{ podcast.theme, emission.nom, description.}\KeywordTok{resume}
\KeywordTok{FROM}\NormalTok{ podcast}
\KeywordTok{JOIN}\NormalTok{ emission }\KeywordTok{ON}\NormalTok{ podcast.id\_emission }\OperatorTok{=}\NormalTok{ emission.id\_emission}
\KeywordTok{JOIN}\NormalTok{ description }\KeywordTok{ON}\NormalTok{ emission.id\_emission }\OperatorTok{=}\NormalTok{ description.id\_emission}
\KeywordTok{WHERE}\NormalTok{ description.duree }\OperatorTok{\textless{}} \DecValTok{5}
\end{Highlighting}
\end{Shaded}
\end{enumerate}

\begin{tcolorbox}[enhanced jigsaw, colframe=quarto-callout-important-color-frame, colback=white, bottomtitle=1mm, colbacktitle=quarto-callout-important-color!10!white, toprule=.15mm, rightrule=.15mm, title=\textcolor{quarto-callout-important-color}{\faExclamation}\hspace{0.5em}{Important}, breakable, titlerule=0mm, left=2mm, bottomrule=.15mm, arc=.35mm, toptitle=1mm, leftrule=.75mm, opacityback=0, opacitybacktitle=0.6, coltitle=black]

La base de donnée telle que définie dans l'énoncé n'est bien construite.
Chaque description est en-effet reliée à une émission unique à travers
la clé étrangère \texttt{id\_emission}, mais pas au podcast
correspondant. Comme il existe plusieurs podcasts pour une émission, il
n'y a pas moyen de savoir à quel podcast correspond quelle description.

\end{tcolorbox}

\hypertarget{exercice-2}{%
\subsection{Exercice 2}\label{exercice-2}}

\begin{enumerate}
\def\labelenumi{\arabic{enumi}.}
\item
  \begin{enumerate}
  \def\labelenumii{\alph{enumii}.}
  \tightlist
  \item
    On convertit les entiers 2 et 13 en binaire sur 8 bits :
  \end{enumerate}

  \begin{longtable}[]{@{}ll@{}}
  \toprule\noalign{}
  2 & 13 \\
  \midrule\noalign{}
  \endhead
  \bottomrule\noalign{}
  \endlastfoot
  00000010 & 00001101 \\
  \end{longtable}

  L'adresse IP 164.178.2.13 est donc représentée par la chaîne de
  caractères :

  \texttt{10100100.10110010.00000010.00001101}

  \begin{enumerate}
  \def\labelenumii{\alph{enumii}.}
  \setcounter{enumii}{1}
  \tightlist
  \item
    L'adresse IP indique que les 24 premiers bits sont réservés à
    l'identifiant du réseau et les 8 derniers bits sont réservés à
    l'identifiant de l'hôte. La machine appartient donc au réseau dont
    l'adresse est 164.178.2.0
  \end{enumerate}
\item
  Pour un paquet émis par A à destination de G, les chemins optimaux en
  suivant le protocole RIP sont ceux qui minimisent le nombre se sauts :

  \begin{longtable}[]{@{}ll@{}}
  \toprule\noalign{}
  Chemin & Nombre de sauts \\
  \midrule\noalign{}
  \endhead
  \bottomrule\noalign{}
  \endlastfoot
  A -\textgreater{} B -\textgreater{} C -\textgreater{} H
  -\textgreater{} G & 4 \\
  A -\textgreater{} B -\textgreater{} E -\textgreater{} G & 3 \\
  A -\textgreater{} D -\textgreater{} E -\textgreater{} G & 3 \\
  A -\textgreater{} D -\textgreater{} F -\textgreater{} G & 3 \\
  \end{longtable}

  Les chemins optimaux sont donc A -\textgreater{} B -\textgreater{} E
  -\textgreater{} G, A -\textgreater{} D -\textgreater{} E
  -\textgreater{} G et A -\textgreater{} D -\textgreater{} F
  -\textgreater{} G.
\item
  \begin{enumerate}
  \def\labelenumii{\alph{enumii}.}
  \tightlist
  \item
    Le coût d'une liaison Ethernet est \(\frac{10^9}{10^8}=10\), celui
    d'une liaison Fast-Ethernet est \(\frac{10^9}{10^9}=1\) et celui
    d'une liaison fibre est \(\frac{10^9}{10^{10}}=0.1\)
  \end{enumerate}

  \includegraphics{2023_CE_J1_fig1.png}

  \begin{enumerate}
  \def\labelenumii{\alph{enumii}.}
  \setcounter{enumii}{1}
  \item
    Nous pouvons déterminer le chemin de parcours en utilisant
    l'algorithme de Dijkstra.

    \begin{longtable}[]{@{}lllllllll@{}}
    \toprule\noalign{}
    A & B & C & D & E & F & G & H & choix \\
    \midrule\noalign{}
    \endhead
    \bottomrule\noalign{}
    \endlastfoot
    0-A & - & - & - & - & - & - & - & A(0) \\
    X & 1-A & - & 10-A & - & - & - & - & B(1) \\
    X & X & 11-B & 10-A & 11-B & - & - & - & D(10) \\
    X & X & 11-B & X & 10.1-D & 11-D & - & - & E(10.1) \\
    X & X & 11-B & X & X & 11-D & 20.1-E & - & C(11) ou F(11) \\
    X & X & X & X & X & X & 12-F & 11.1-C & H(11.1) \\
    X & X & X & X & X & X & 12-F & X & G(12) \\
    \end{longtable}

    Le chemin optimal est donc A -\textgreater{} D -\textgreater{} F
    -\textgreater{} G de coût total 12.
  \item
    Le routeur F est en panne. Nous appliquons l'algorithme de Dijkstra
    en supprimant le routeur F

    \begin{longtable}[]{@{}llllllll@{}}
    \toprule\noalign{}
    A & B & C & D & E & G & H & choix \\
    \midrule\noalign{}
    \endhead
    \bottomrule\noalign{}
    \endlastfoot
    0-A & - & - & - & - & - & - & A(0) \\
    X & 1-A & - & 10-A & - & - & - & B(1) \\
    X & X & 11-B & 10-A & 11-B & - & - & D(10) \\
    X & X & 11-B & X & 10.1-D & - & - & E(10.1) \\
    X & X & 11-B & X & X & 20.1-E & - & C(11) \\
    X & X & X & X & X & 20.1-E & 11.1-C & H(11.1) \\
    X & X & X & X & X & 12.1-H & X & G(12.1) \\
    \end{longtable}

    Le chemin optimal est donc A -\textgreater{} B -\textgreater{} C
    -\textgreater{} H -\textgreater{} G de coût total 12.1.
  \end{enumerate}
\end{enumerate}

\hypertarget{exercice-3}{%
\subsection{Exercice 3}\label{exercice-3}}

\begin{enumerate}
\def\labelenumi{\arabic{enumi}.}
\item
  Fonction permettant l'ajout d'une couleur aléatoire dans la file
  \texttt{f} :

\begin{Shaded}
\begin{Highlighting}[]
\KeywordTok{def}\NormalTok{ ajout(f):}
\NormalTok{    couleurs }\OperatorTok{=}\NormalTok{ (}\StringTok{"bleu"}\NormalTok{, }\StringTok{"rouge"}\NormalTok{, }\StringTok{"jaune"}\NormalTok{, }\StringTok{"vert"}\NormalTok{)}
\NormalTok{    indice }\OperatorTok{=}\NormalTok{ randint(}\DecValTok{0}\NormalTok{, }\DecValTok{3}\NormalTok{)}
\NormalTok{    enfiler(f, couleurs[indice])}
    \ControlFlowTok{return}\NormalTok{ f}
\end{Highlighting}
\end{Shaded}
\item
  Fonction permettant de vider la séquence \texttt{f} :

\begin{Shaded}
\begin{Highlighting}[]
\KeywordTok{def}\NormalTok{ vider(f):}
    \ControlFlowTok{while} \KeywordTok{not}\NormalTok{ est\_vide(f):}
\NormalTok{        defiler(f)}
\end{Highlighting}
\end{Shaded}
\item
  Fonction \texttt{affich\_seq} complétée :

\begin{Shaded}
\begin{Highlighting}[]
\KeywordTok{def}\NormalTok{ affich\_seq(sequence):}
\NormalTok{    stock }\OperatorTok{=}\NormalTok{ creer\_file\_vide()}
\NormalTok{    ajout(sequence)}
    \ControlFlowTok{while} \KeywordTok{not}\NormalTok{ est\_vide(sequence):}
\NormalTok{        c }\OperatorTok{=}\NormalTok{ defiler(sequence)}
\NormalTok{        affichage(c)}
\NormalTok{        time.sleep(}\FloatTok{0.5}\NormalTok{)}
\NormalTok{        enfiler(stock, c)}
    \ControlFlowTok{while} \KeywordTok{not}\NormalTok{ est\_vide(stock):}
\NormalTok{        enfiler(sequence, defiler(stock))}
\end{Highlighting}
\end{Shaded}
\item
  \begin{enumerate}
  \def\labelenumii{\alph{enumii}.}
  \tightlist
  \item
    Fonction \texttt{tour\_de\_jeu} complétée :
  \end{enumerate}

\begin{Shaded}
\begin{Highlighting}[]
\KeywordTok{def}\NormalTok{ tour\_de\_jeu(sequence):}
\NormalTok{    affich\_seq(sequence) }\CommentTok{\# zone A}
\NormalTok{    stock }\OperatorTok{=}\NormalTok{ creer\_file\_vide()}
    \ControlFlowTok{while} \KeywordTok{not}\NormalTok{ est\_vide(sequence):}
\NormalTok{        c\_joueur }\OperatorTok{=}\NormalTok{ saisie\_joueur()}
\NormalTok{        c\_seq }\OperatorTok{=}\NormalTok{ defiler(sequence) }\CommentTok{\# zone B}
        \ControlFlowTok{if}\NormalTok{ c\_joueur }\OperatorTok{==}\NormalTok{ c\_seq:}
\NormalTok{            enfiler(stock, c\_seq) }\CommentTok{\# zone C}
        \ControlFlowTok{else}\NormalTok{:}
\NormalTok{            vider(sequence) }\CommentTok{\# zone D}
    \ControlFlowTok{while} \KeywordTok{not}\NormalTok{ est\_vide(stock): }\CommentTok{\# zone E}
\NormalTok{        enfiler(sequence, defiler(stock)) }\CommentTok{\# zone F}
\end{Highlighting}
\end{Shaded}

  \begin{enumerate}
  \def\labelenumii{\alph{enumii}.}
  \setcounter{enumii}{1}
  \tightlist
  \item
    Fonction modifiée :
  \end{enumerate}
\end{enumerate}

\begin{Shaded}
\begin{Highlighting}[]
\KeywordTok{def}\NormalTok{ tour\_de\_jeu(sequence):}
\NormalTok{    affich\_seq(sequence) }\CommentTok{\# zone A}
\NormalTok{    stock }\OperatorTok{=}\NormalTok{ creer\_file\_vide()}
\NormalTok{    gagne }\OperatorTok{=} \VariableTok{True}
    \ControlFlowTok{while} \KeywordTok{not}\NormalTok{ est\_vide(sequence):}
\NormalTok{        c\_joueur }\OperatorTok{=}\NormalTok{ saisie\_joueur()}
\NormalTok{        c\_seq }\OperatorTok{=}\NormalTok{ defiler(sequence) }\CommentTok{\# zone B}
        \ControlFlowTok{if}\NormalTok{ c\_joueur }\OperatorTok{==}\NormalTok{ c\_seq:}
\NormalTok{            enfiler(stock, c\_seq) }\CommentTok{\# zone C}
        \ControlFlowTok{else}\NormalTok{:}
\NormalTok{            vider(sequence) }\CommentTok{\# zone D}
\NormalTok{            gagne }\OperatorTok{=} \VariableTok{False}
    \ControlFlowTok{while} \KeywordTok{not}\NormalTok{ est\_vide(stock): }\CommentTok{\# zone E}
\NormalTok{        enfiler(sequence, defiler(stock)) }\CommentTok{\# zone F}
    \ControlFlowTok{if}\NormalTok{ gagne:}
\NormalTok{        tour\_de\_jeu(sequence)}
    \ControlFlowTok{else}\NormalTok{:}
\NormalTok{        vider(sequence)}
\NormalTok{        ajout(sequence)}
\NormalTok{        tour\_de\_jeu(sequence)}
\end{Highlighting}
\end{Shaded}




\end{document}
