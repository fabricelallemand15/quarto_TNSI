% Options for packages loaded elsewhere
\PassOptionsToPackage{unicode}{hyperref}
\PassOptionsToPackage{hyphens}{url}
\PassOptionsToPackage{dvipsnames,svgnames,x11names}{xcolor}
%
\documentclass[
  letterpaper,
  DIV=11,
  numbers=noendperiod]{scrartcl}

\usepackage{amsmath,amssymb}
\usepackage{iftex}
\ifPDFTeX
  \usepackage[T1]{fontenc}
  \usepackage[utf8]{inputenc}
  \usepackage{textcomp} % provide euro and other symbols
\else % if luatex or xetex
  \usepackage{unicode-math}
  \defaultfontfeatures{Scale=MatchLowercase}
  \defaultfontfeatures[\rmfamily]{Ligatures=TeX,Scale=1}
\fi
\usepackage{lmodern}
\ifPDFTeX\else  
    % xetex/luatex font selection
\fi
% Use upquote if available, for straight quotes in verbatim environments
\IfFileExists{upquote.sty}{\usepackage{upquote}}{}
\IfFileExists{microtype.sty}{% use microtype if available
  \usepackage[]{microtype}
  \UseMicrotypeSet[protrusion]{basicmath} % disable protrusion for tt fonts
}{}
\makeatletter
\@ifundefined{KOMAClassName}{% if non-KOMA class
  \IfFileExists{parskip.sty}{%
    \usepackage{parskip}
  }{% else
    \setlength{\parindent}{0pt}
    \setlength{\parskip}{6pt plus 2pt minus 1pt}}
}{% if KOMA class
  \KOMAoptions{parskip=half}}
\makeatother
\usepackage{xcolor}
\usepackage[top=20mm,bottom=20mm,left=20mm,right=20mm,heightrounded]{geometry}
\setlength{\emergencystretch}{3em} % prevent overfull lines
\setcounter{secnumdepth}{-\maxdimen} % remove section numbering
% Make \paragraph and \subparagraph free-standing
\ifx\paragraph\undefined\else
  \let\oldparagraph\paragraph
  \renewcommand{\paragraph}[1]{\oldparagraph{#1}\mbox{}}
\fi
\ifx\subparagraph\undefined\else
  \let\oldsubparagraph\subparagraph
  \renewcommand{\subparagraph}[1]{\oldsubparagraph{#1}\mbox{}}
\fi


\providecommand{\tightlist}{%
  \setlength{\itemsep}{0pt}\setlength{\parskip}{0pt}}\usepackage{longtable,booktabs,array}
\usepackage{calc} % for calculating minipage widths
% Correct order of tables after \paragraph or \subparagraph
\usepackage{etoolbox}
\makeatletter
\patchcmd\longtable{\par}{\if@noskipsec\mbox{}\fi\par}{}{}
\makeatother
% Allow footnotes in longtable head/foot
\IfFileExists{footnotehyper.sty}{\usepackage{footnotehyper}}{\usepackage{footnote}}
\makesavenoteenv{longtable}
\usepackage{graphicx}
\makeatletter
\def\maxwidth{\ifdim\Gin@nat@width>\linewidth\linewidth\else\Gin@nat@width\fi}
\def\maxheight{\ifdim\Gin@nat@height>\textheight\textheight\else\Gin@nat@height\fi}
\makeatother
% Scale images if necessary, so that they will not overflow the page
% margins by default, and it is still possible to overwrite the defaults
% using explicit options in \includegraphics[width, height, ...]{}
\setkeys{Gin}{width=\maxwidth,height=\maxheight,keepaspectratio}
% Set default figure placement to htbp
\makeatletter
\def\fps@figure{htbp}
\makeatother

\usepackage{fancyhdr} \pagestyle{fancy} \usepackage{lastpage}
\KOMAoption{captions}{tablesignature}
\makeatletter
\makeatother
\makeatletter
\makeatother
\makeatletter
\@ifpackageloaded{caption}{}{\usepackage{caption}}
\AtBeginDocument{%
\ifdefined\contentsname
  \renewcommand*\contentsname{Table des matières}
\else
  \newcommand\contentsname{Table des matières}
\fi
\ifdefined\listfigurename
  \renewcommand*\listfigurename{Liste des Figures}
\else
  \newcommand\listfigurename{Liste des Figures}
\fi
\ifdefined\listtablename
  \renewcommand*\listtablename{Liste des Tables}
\else
  \newcommand\listtablename{Liste des Tables}
\fi
\ifdefined\figurename
  \renewcommand*\figurename{Figure}
\else
  \newcommand\figurename{Figure}
\fi
\ifdefined\tablename
  \renewcommand*\tablename{Tableau}
\else
  \newcommand\tablename{Tableau}
\fi
}
\@ifpackageloaded{float}{}{\usepackage{float}}
\floatstyle{ruled}
\@ifundefined{c@chapter}{\newfloat{codelisting}{h}{lop}}{\newfloat{codelisting}{h}{lop}[chapter]}
\floatname{codelisting}{Listing}
\newcommand*\listoflistings{\listof{codelisting}{Liste des Listings}}
\makeatother
\makeatletter
\@ifpackageloaded{caption}{}{\usepackage{caption}}
\@ifpackageloaded{subcaption}{}{\usepackage{subcaption}}
\makeatother
\makeatletter
\makeatother
\ifLuaTeX
\usepackage[bidi=basic]{babel}
\else
\usepackage[bidi=default]{babel}
\fi
\babelprovide[main,import]{french}
% get rid of language-specific shorthands (see #6817):
\let\LanguageShortHands\languageshorthands
\def\languageshorthands#1{}
\ifLuaTeX
  \usepackage{selnolig}  % disable illegal ligatures
\fi
\IfFileExists{bookmark.sty}{\usepackage{bookmark}}{\usepackage{hyperref}}
\IfFileExists{xurl.sty}{\usepackage{xurl}}{} % add URL line breaks if available
\urlstyle{same} % disable monospaced font for URLs
\hypersetup{
  pdftitle={Épreuve de NSI au bac en Terminale},
  pdflang={fr},
  colorlinks=true,
  linkcolor={blue},
  filecolor={Maroon},
  citecolor={Blue},
  urlcolor={Blue},
  pdfcreator={LaTeX via pandoc}}

\title{Épreuve de NSI au bac en Terminale}
\author{}
\date{}

\begin{document}
\maketitle
\lhead{Spécialité NSI} \rhead{Terminale} \chead{} \cfoot{} \lfoot{Lycée \'Emile Duclaux} \rfoot{Page \thepage/\pageref{LastPage}} \renewcommand{\headrulewidth}{0pt} \renewcommand{\footrulewidth}{0pt} \thispagestyle{fancy} \vspace{-2cm}

\hypertarget{programme-officiel}{%
\subsection{Programme officiel}\label{programme-officiel}}

\href{../assets/pdf/Programme_Term_NSI.pdf}{Fichier à
télécharger\ldots{}}

\hypertarget{nature-de-luxe9preuve-de-nsi-au-bac-en-terminale}{%
\subsection{Nature de l'épreuve de NSI au bac en
Terminale}\label{nature-de-luxe9preuve-de-nsi-au-bac-en-terminale}}

D'après le
\href{https://www.education.gouv.fr/bo/22/Hebdo36/MENE2226770N.htm}{Bulletin
Officiel n°36 du 30 septembre 2022} :

\begin{itemize}
\item
  \textbf{Durée} : 3 heures 30 + 1 heure
\item
  \textbf{Coefficient} : 16
\item
  \textbf{Format} : L'épreuve terminale obligatoire de spécialité est
  composée de deux parties :

  \begin{itemize}
  \tightlist
  \item
    une partie écrite, comptant pour 12 points sur 20,
  \item
    une partie pratique comptant pour 8 points sur 20.
  \end{itemize}
\end{itemize}

\textbf{Partie écrite de l'épreuve de NSI au bac en terminale}

\begin{itemize}
\item
  \textbf{Durée} : 3 heures 30
\item
  \textbf{Modalités} : La partie écrite consiste en la résolution de
  trois exercices permettant d'évaluer les connaissances et les
  capacités attendues conformément aux programmes de première et de
  terminale de la spécialité.

  Chaque exercice est noté sur 4 points.

  Le sujet comporte trois exercices indépendants les uns des autres, qui
  permettent d'évaluer les connaissances et compétences des candidats.
\end{itemize}

\hypertarget{points-du-programme-uxe9valuables-lors-de-luxe9preuve-uxe9crite}{%
\subsubsection{Points du programme évaluables lors de l'épreuve
écrite}\label{points-du-programme-uxe9valuables-lors-de-luxe9preuve-uxe9crite}}

Référence :
\href{https://www.education.gouv.fr/bo/22/Hebdo36/MENE2227884N.htm}{Bulletin
officiel n°36 du 30 septembre 2022}

\begin{itemize}
\item
  Thème 2 -- Structures de données

  \begin{itemize}
  \tightlist
  \item
    Structures de données, interface et implémentation.
  \item
    Vocabulaire de la programmation objet : classes, attributs,
    méthodes, objets.
  \item
    Listes, piles, files : structures linéaires. Dictionnaires, index et
    clé.
  \item
    Arbres : structures hiérarchiques. Arbres binaires : nœuds, racines,
    feuilles, sous-arbres gauches, sous-arbres droits.
  \end{itemize}
\item
  Thème 3 -- Bases de données

  \begin{itemize}
  \tightlist
  \item
    Modèle relationnel : relation, attribut, domaine, clef primaire,
    clef étrangère, schéma relationnel.
  \item
    Base de données relationnelle.
  \item
    Langage SQL : requêtes d'interrogation et de mise à jour d'une base
    de données.
  \end{itemize}
\item
  Thème 4 -- Architectures matérielles, systèmes d'exploitation et
  réseaux

  \begin{itemize}
  \tightlist
  \item
    Gestion des processus et des ressources par un système
    d'exploitation.
  \item
    Protocoles de routage.
  \end{itemize}
\item
  Thème 5 -- Langages et programmation

  \begin{itemize}
  \tightlist
  \item
    Récursivité.
  \item
    Modularité.
  \item
    Mise au point des programmes. Gestion des bugs.
  \end{itemize}
\item
  Thème 6 -- Algorithmique

  \begin{itemize}
  \tightlist
  \item
    Algorithmes sur les arbres binaires et sur les arbres binaires de
    recherche.
  \item
    Méthode « diviser pour régner »
  \end{itemize}
\end{itemize}

\hypertarget{points-du-programme-non-uxe9valuuxe9s-uxe0-luxe9crit}{%
\subsubsection{Points du programme non évalués à
l'écrit}\label{points-du-programme-non-uxe9valuuxe9s-uxe0-luxe9crit}}

\begin{itemize}
\item
  Thème 1 -- Histoire de l'informatique
\item
  Thème 2 -- Structures de données

  \begin{itemize}
  \tightlist
  \item
    Graphes : structures relationnelles. Sommets, arcs, arêtes, graphes
    orientés ou non orientés.
  \end{itemize}
\item
  Thème 3 -- Bases de données

  \begin{itemize}
  \tightlist
  \item
    Système de gestion de bases de données relationnelles.
  \end{itemize}
\item
  Thème 4 -- Architectures matérielles, systèmes d'exploitation et
  réseaux

  \begin{itemize}
  \tightlist
  \item
    Composants intégrés d'un système sur puce.
  \item
    Sécurisation des communications.
  \end{itemize}
\item
  Thème 5 -- Langages et programmation

  \begin{itemize}
  \tightlist
  \item
    Notion de programme en tant que donnée. Calculabilité, décidabilité.
  \item
    Paradigmes de programmation
  \end{itemize}
\item
  Thème 6 -- Algorithmique

  \begin{itemize}
  \tightlist
  \item
    Algorithmes sur les graphes.
  \item
    Programmation dynamique.
  \item
    Recherche textuelle.
  \end{itemize}
\end{itemize}

\hypertarget{partie-pratique-de-luxe9preuve-de-nsi-au-bac-en-terminale}{%
\subsection{Partie pratique de l'épreuve de NSI au bac en
terminale}\label{partie-pratique-de-luxe9preuve-de-nsi-au-bac-en-terminale}}

\begin{itemize}
\item
  \textbf{Durée} : 1 heure
\item
  \textbf{Modalités} : La partie pratique consiste en la résolution de
  deux exercices sur ordinateur, chacun étant noté sur 4 points.

  Le candidat est évalué sur la base d'un dialogue avec un
  professeur-examinateur.

  Un examinateur évalue au maximum quatre élèves. L'examinateur ne peut
  pas évaluer un élève qu'il a eu en classe durant l'année en cours.

  L'évaluation de cette partie se déroule au cours du deuxième trimestre
  pendant la période de l'épreuve écrite de spécialité.

  \begin{itemize}
  \item
    \textbf{Premier exercice}

    Le premier exercice consiste à programmer un algorithme figurant
    explicitement au programme, ne présentant pas de difficulté
    particulière, dont on fournit une spécification.

    Il s'agit donc de restituer un algorithme rencontré et travaillé à
    plusieurs reprises en cours de formation.

    Le sujet peut proposer un jeu de test avec les réponses attendues
    pour permettre au candidat de vérifier son travail.
  \item
    \textbf{Deuxième exercice}

    Pour le second exercice, un programme est fourni au candidat.

    Cet exercice ne demande pas l'écriture complète d'un programme, mais
    permet de valider des compétences de programmation suivant des
    modalités variées : le candidat doit, par exemple, compléter un
    programme « à trous » afin de répondre à une spécification donnée,
    ou encore compléter un programme pour le documenter, ou encore
    compléter un programme en ajoutant des assertions, etc.
  \end{itemize}
\end{itemize}



\end{document}
