% Options for packages loaded elsewhere
\PassOptionsToPackage{unicode}{hyperref}
\PassOptionsToPackage{hyphens}{url}
\PassOptionsToPackage{dvipsnames,svgnames,x11names}{xcolor}
%
\documentclass[
  a4paper,
  DIV=11,
  numbers=noendperiod]{scrartcl}

\usepackage{amsmath,amssymb}
\usepackage{lmodern}
\usepackage{iftex}
\ifPDFTeX
  \usepackage[T1]{fontenc}
  \usepackage[utf8]{inputenc}
  \usepackage{textcomp} % provide euro and other symbols
\else % if luatex or xetex
  \usepackage{unicode-math}
  \defaultfontfeatures{Scale=MatchLowercase}
  \defaultfontfeatures[\rmfamily]{Ligatures=TeX,Scale=1}
\fi
% Use upquote if available, for straight quotes in verbatim environments
\IfFileExists{upquote.sty}{\usepackage{upquote}}{}
\IfFileExists{microtype.sty}{% use microtype if available
  \usepackage[]{microtype}
  \UseMicrotypeSet[protrusion]{basicmath} % disable protrusion for tt fonts
}{}
\makeatletter
\@ifundefined{KOMAClassName}{% if non-KOMA class
  \IfFileExists{parskip.sty}{%
    \usepackage{parskip}
  }{% else
    \setlength{\parindent}{0pt}
    \setlength{\parskip}{6pt plus 2pt minus 1pt}}
}{% if KOMA class
  \KOMAoptions{parskip=half}}
\makeatother
\usepackage{xcolor}
\usepackage[top=20mm,bottom=20mm,left=20mm,right=20mm,heightrounded]{geometry}
\setlength{\emergencystretch}{3em} % prevent overfull lines
\setcounter{secnumdepth}{-\maxdimen} % remove section numbering
% Make \paragraph and \subparagraph free-standing
\ifx\paragraph\undefined\else
  \let\oldparagraph\paragraph
  \renewcommand{\paragraph}[1]{\oldparagraph{#1}\mbox{}}
\fi
\ifx\subparagraph\undefined\else
  \let\oldsubparagraph\subparagraph
  \renewcommand{\subparagraph}[1]{\oldsubparagraph{#1}\mbox{}}
\fi

\usepackage{color}
\usepackage{fancyvrb}
\newcommand{\VerbBar}{|}
\newcommand{\VERB}{\Verb[commandchars=\\\{\}]}
\DefineVerbatimEnvironment{Highlighting}{Verbatim}{commandchars=\\\{\}}
% Add ',fontsize=\small' for more characters per line
\usepackage{framed}
\definecolor{shadecolor}{RGB}{241,243,245}
\newenvironment{Shaded}{\begin{snugshade}}{\end{snugshade}}
\newcommand{\AlertTok}[1]{\textcolor[rgb]{0.68,0.00,0.00}{#1}}
\newcommand{\AnnotationTok}[1]{\textcolor[rgb]{0.37,0.37,0.37}{#1}}
\newcommand{\AttributeTok}[1]{\textcolor[rgb]{0.40,0.45,0.13}{#1}}
\newcommand{\BaseNTok}[1]{\textcolor[rgb]{0.68,0.00,0.00}{#1}}
\newcommand{\BuiltInTok}[1]{\textcolor[rgb]{0.00,0.23,0.31}{#1}}
\newcommand{\CharTok}[1]{\textcolor[rgb]{0.13,0.47,0.30}{#1}}
\newcommand{\CommentTok}[1]{\textcolor[rgb]{0.37,0.37,0.37}{#1}}
\newcommand{\CommentVarTok}[1]{\textcolor[rgb]{0.37,0.37,0.37}{\textit{#1}}}
\newcommand{\ConstantTok}[1]{\textcolor[rgb]{0.56,0.35,0.01}{#1}}
\newcommand{\ControlFlowTok}[1]{\textcolor[rgb]{0.00,0.23,0.31}{#1}}
\newcommand{\DataTypeTok}[1]{\textcolor[rgb]{0.68,0.00,0.00}{#1}}
\newcommand{\DecValTok}[1]{\textcolor[rgb]{0.68,0.00,0.00}{#1}}
\newcommand{\DocumentationTok}[1]{\textcolor[rgb]{0.37,0.37,0.37}{\textit{#1}}}
\newcommand{\ErrorTok}[1]{\textcolor[rgb]{0.68,0.00,0.00}{#1}}
\newcommand{\ExtensionTok}[1]{\textcolor[rgb]{0.00,0.23,0.31}{#1}}
\newcommand{\FloatTok}[1]{\textcolor[rgb]{0.68,0.00,0.00}{#1}}
\newcommand{\FunctionTok}[1]{\textcolor[rgb]{0.28,0.35,0.67}{#1}}
\newcommand{\ImportTok}[1]{\textcolor[rgb]{0.00,0.46,0.62}{#1}}
\newcommand{\InformationTok}[1]{\textcolor[rgb]{0.37,0.37,0.37}{#1}}
\newcommand{\KeywordTok}[1]{\textcolor[rgb]{0.00,0.23,0.31}{#1}}
\newcommand{\NormalTok}[1]{\textcolor[rgb]{0.00,0.23,0.31}{#1}}
\newcommand{\OperatorTok}[1]{\textcolor[rgb]{0.37,0.37,0.37}{#1}}
\newcommand{\OtherTok}[1]{\textcolor[rgb]{0.00,0.23,0.31}{#1}}
\newcommand{\PreprocessorTok}[1]{\textcolor[rgb]{0.68,0.00,0.00}{#1}}
\newcommand{\RegionMarkerTok}[1]{\textcolor[rgb]{0.00,0.23,0.31}{#1}}
\newcommand{\SpecialCharTok}[1]{\textcolor[rgb]{0.37,0.37,0.37}{#1}}
\newcommand{\SpecialStringTok}[1]{\textcolor[rgb]{0.13,0.47,0.30}{#1}}
\newcommand{\StringTok}[1]{\textcolor[rgb]{0.13,0.47,0.30}{#1}}
\newcommand{\VariableTok}[1]{\textcolor[rgb]{0.07,0.07,0.07}{#1}}
\newcommand{\VerbatimStringTok}[1]{\textcolor[rgb]{0.13,0.47,0.30}{#1}}
\newcommand{\WarningTok}[1]{\textcolor[rgb]{0.37,0.37,0.37}{\textit{#1}}}

\providecommand{\tightlist}{%
  \setlength{\itemsep}{0pt}\setlength{\parskip}{0pt}}\usepackage{longtable,booktabs,array}
\usepackage{calc} % for calculating minipage widths
% Correct order of tables after \paragraph or \subparagraph
\usepackage{etoolbox}
\makeatletter
\patchcmd\longtable{\par}{\if@noskipsec\mbox{}\fi\par}{}{}
\makeatother
% Allow footnotes in longtable head/foot
\IfFileExists{footnotehyper.sty}{\usepackage{footnotehyper}}{\usepackage{footnote}}
\makesavenoteenv{longtable}
\usepackage{graphicx}
\makeatletter
\def\maxwidth{\ifdim\Gin@nat@width>\linewidth\linewidth\else\Gin@nat@width\fi}
\def\maxheight{\ifdim\Gin@nat@height>\textheight\textheight\else\Gin@nat@height\fi}
\makeatother
% Scale images if necessary, so that they will not overflow the page
% margins by default, and it is still possible to overwrite the defaults
% using explicit options in \includegraphics[width, height, ...]{}
\setkeys{Gin}{width=\maxwidth,height=\maxheight,keepaspectratio}
% Set default figure placement to htbp
\makeatletter
\def\fps@figure{htbp}
\makeatother

\usepackage{fancyhdr} \pagestyle{fancy} \usepackage{lastpage}
\KOMAoption{captions}{tablesignature}
\makeatletter
\@ifpackageloaded{tcolorbox}{}{\usepackage[many]{tcolorbox}}
\@ifpackageloaded{fontawesome5}{}{\usepackage{fontawesome5}}
\definecolor{quarto-callout-color}{HTML}{909090}
\definecolor{quarto-callout-note-color}{HTML}{0758E5}
\definecolor{quarto-callout-important-color}{HTML}{CC1914}
\definecolor{quarto-callout-warning-color}{HTML}{EB9113}
\definecolor{quarto-callout-tip-color}{HTML}{00A047}
\definecolor{quarto-callout-caution-color}{HTML}{FC5300}
\definecolor{quarto-callout-color-frame}{HTML}{acacac}
\definecolor{quarto-callout-note-color-frame}{HTML}{4582ec}
\definecolor{quarto-callout-important-color-frame}{HTML}{d9534f}
\definecolor{quarto-callout-warning-color-frame}{HTML}{f0ad4e}
\definecolor{quarto-callout-tip-color-frame}{HTML}{02b875}
\definecolor{quarto-callout-caution-color-frame}{HTML}{fd7e14}
\makeatother
\makeatletter
\makeatother
\makeatletter
\makeatother
\makeatletter
\@ifpackageloaded{caption}{}{\usepackage{caption}}
\AtBeginDocument{%
\ifdefined\contentsname
  \renewcommand*\contentsname{Table des matières}
\else
  \newcommand\contentsname{Table des matières}
\fi
\ifdefined\listfigurename
  \renewcommand*\listfigurename{Liste des Figures}
\else
  \newcommand\listfigurename{Liste des Figures}
\fi
\ifdefined\listtablename
  \renewcommand*\listtablename{Liste des Tables}
\else
  \newcommand\listtablename{Liste des Tables}
\fi
\ifdefined\figurename
  \renewcommand*\figurename{Figure}
\else
  \newcommand\figurename{Figure}
\fi
\ifdefined\tablename
  \renewcommand*\tablename{Tableau}
\else
  \newcommand\tablename{Tableau}
\fi
}
\@ifpackageloaded{float}{}{\usepackage{float}}
\floatstyle{ruled}
\@ifundefined{c@chapter}{\newfloat{codelisting}{h}{lop}}{\newfloat{codelisting}{h}{lop}[chapter]}
\floatname{codelisting}{Listing}
\newcommand*\listoflistings{\listof{codelisting}{Liste des Listings}}
\makeatother
\makeatletter
\@ifpackageloaded{caption}{}{\usepackage{caption}}
\@ifpackageloaded{subcaption}{}{\usepackage{subcaption}}
\makeatother
\makeatletter
\@ifpackageloaded{tcolorbox}{}{\usepackage[many]{tcolorbox}}
\makeatother
\makeatletter
\@ifundefined{shadecolor}{\definecolor{shadecolor}{rgb}{.97, .97, .97}}
\makeatother
\makeatletter
\makeatother
\makeatletter
\@ifpackageloaded{fontawesome5}{}{\usepackage{fontawesome5}}
\makeatother
\ifLuaTeX
\usepackage[bidi=basic]{babel}
\else
\usepackage[bidi=default]{babel}
\fi
\babelprovide[main,import]{french}
% get rid of language-specific shorthands (see #6817):
\let\LanguageShortHands\languageshorthands
\def\languageshorthands#1{}
\ifLuaTeX
  \usepackage{selnolig}  % disable illegal ligatures
\fi
\IfFileExists{bookmark.sty}{\usepackage{bookmark}}{\usepackage{hyperref}}
\IfFileExists{xurl.sty}{\usepackage{xurl}}{} % add URL line breaks if available
\urlstyle{same} % disable monospaced font for URLs
\hypersetup{
  pdftitle={Récursivité (Exercices)},
  pdflang={fr},
  colorlinks=true,
  linkcolor={blue},
  filecolor={Maroon},
  citecolor={Blue},
  urlcolor={Blue},
  pdfcreator={LaTeX via pandoc}}

\title{Récursivité (Exercices)}
\usepackage{etoolbox}
\makeatletter
\providecommand{\subtitle}[1]{% add subtitle to \maketitle
  \apptocmd{\@title}{\par {\large #1 \par}}{}{}
}
\makeatother
\subtitle{S1 - Langages et programmation}
\author{}
\date{}

\begin{document}
\maketitle
\lhead{Spécialité NSI} \rhead{Terminale} \chead{} \cfoot{} \lfoot{Lycée \'Emile Duclaux} \rfoot{Page \thepage/\pageref{LastPage}} \renewcommand{\headrulewidth}{0pt} \renewcommand{\footrulewidth}{0pt} \thispagestyle{fancy} \vspace{-2cm}

\ifdefined\Shaded\renewenvironment{Shaded}{\begin{tcolorbox}[frame hidden, interior hidden, boxrule=0pt, borderline west={3pt}{0pt}{shadecolor}, sharp corners, enhanced, breakable]}{\end{tcolorbox}}\fi

\emph{Les exercices précédés du symbole \faIcon{desktop} sont à faire
sur machine, en sauvegardant le fichier si nécessaire.}

\emph{Les exercices précédés du symbole \faIcon{pencil-alt} doivent être
résolus par écrit.}

\begin{tcolorbox}[enhanced jigsaw, breakable, toprule=.15mm, left=2mm, coltitle=black, toptitle=1mm, title=\textcolor{quarto-callout-warning-color}{\faExclamationTriangle}\hspace{0.5em}{Attention !}, arc=.35mm, leftrule=.75mm, colback=white, bottomrule=.15mm, colbacktitle=quarto-callout-warning-color!10!white, rightrule=.15mm, opacitybacktitle=0.6, opacityback=0, bottomtitle=1mm, titlerule=0mm]

Les exercices suivants comportent également quelques compléments de
cours.

Les exemples présentés dans ces exercices sont des exemples très
classiques qu'il faut connaître.

\end{tcolorbox}

\hypertarget{fa-solid-pencil-alt-exercice-1-factorielle}{%
\subsection{\texorpdfstring{\faIcon{pencil-alt} Exercice 1 :
factorielle}{ Exercice 1 : factorielle}}\label{fa-solid-pencil-alt-exercice-1-factorielle}}

On rappelle l'exemple du premier paragraphe du cours concernant le
calcul de la factorielle \(n!=1\times 2\times 3\times\ldots\times n\)
d'un entier naturel \(n\), dans sa version récursive.

\begin{Shaded}
\begin{Highlighting}[]
  \KeywordTok{def}\NormalTok{ fact(n):}
    \CommentTok{"""Renvoie la factorielle de n (méthode récursive)."""}
    \ControlFlowTok{if}\NormalTok{ n }\OperatorTok{==} \DecValTok{0}\NormalTok{:}
\NormalTok{      res }\OperatorTok{=} \DecValTok{1}
    \ControlFlowTok{else}\NormalTok{:}
\NormalTok{      res }\OperatorTok{=}\NormalTok{ n}\OperatorTok{*}\NormalTok{fact(n}\OperatorTok{{-}}\DecValTok{1}\NormalTok{)}
    \ControlFlowTok{return}\NormalTok{ res}
\end{Highlighting}
\end{Shaded}

\begin{enumerate}
\def\labelenumi{\arabic{enumi}.}
\item
  Dans cette fonction, quel est le cas de base ?
\item
  Démontrer que l'algorithme se termine (\textbf{preuve de terminaison})
  dès lors que l'argument \(n\) donné initialement est un entier
  naturel.
\item
  Que se passe-t-il si on appelle la fonction \texttt{fact} avec
  \(n=-2\) ? Proposer une modification de la fonction pour traiter ce
  type de cas.
\item
  Pour démontrer que cet algorithme renvoie bien \(n!\) lorsque \(n\)
  est un entier naturel, on peut procéder par un \textbf{raisonnement
  par récurrence}.

  \begin{itemize}
  \tightlist
  \item
    \emph{Cas de base} : pour \(n=0\), la fonction renvoie-t-elle \(0!\)
    ?
  \item
    \emph{Hypothèse} : on suppose que, pour une certaine valeur de
    l'entier naturel non nul \(n\), \texttt{fact(n-1)} renvoie
    \((n-1)!\). Montrer que, sous cette hypothèse, \texttt{fac(n)}
    renvoie bien \(n!\).
  \item
    \emph{Conclusion} : en déduire que \texttt{fac(n)} renvoie \(n!\)
    pour tout entier naturel \(n\).
  \end{itemize}
\item
  Pour évaluer la \textbf{complexité} de cet algorithme, nous allons
  compter le nombre de multiplications et de comparaisons effectuées.
  Démontrer, à l'aide d'un raisonnement pas récurrence, que la
  complexité de cet algorithme est en \(\mathcal{O}(n)\).
\end{enumerate}

\begin{tcolorbox}[enhanced jigsaw, breakable, toprule=.15mm, left=2mm, coltitle=black, toptitle=1mm, title=\textcolor{quarto-callout-note-color}{\faInfo}\hspace{0.5em}{À retenir \ldots{}}, arc=.35mm, leftrule=.75mm, colback=white, bottomrule=.15mm, colbacktitle=quarto-callout-note-color!10!white, rightrule=.15mm, opacitybacktitle=0.6, opacityback=0, bottomtitle=1mm, titlerule=0mm]

\begin{itemize}
\tightlist
\item
  Le principe de la preuve de terminaison.
\item
  Le principe du raisonnement pas récurrence
\end{itemize}

\end{tcolorbox}

\hypertarget{fa-solid-pencil-alt-fa-desktop-exercice-2-suite-de-fibonacci}{%
\subsection{\texorpdfstring{\faIcon{pencil-alt} \faIcon{desktop}
Exercice 2 : suite de
Fibonacci}{  Exercice 2 : suite de Fibonacci}}\label{fa-solid-pencil-alt-fa-desktop-exercice-2-suite-de-fibonacci}}

La suite de Fibonacci est une suite de nombres entiers notés \(F_n\),
définie par \(F_0=0\), \(F_1=1\) et dans laquelle chaque terme est égal
à la somme des deux termes qui le précèdent.

\begin{enumerate}
\def\labelenumi{\arabic{enumi}.}
\item
  Calculer \(F_n\) à la main pour les valeurs de \(n\) allant de 2
  jusqu'à 5.
\item
  Recopier et compléter le code de la fonction \texttt{fibo\_iter} qui
  retourne \(F_n\) en utilisant un algorithme itératif.

\begin{Shaded}
\begin{Highlighting}[]
\KeywordTok{def}\NormalTok{ fibo\_iter(n: }\BuiltInTok{int}\NormalTok{) }\OperatorTok{{-}\textgreater{}} \BuiltInTok{int}\NormalTok{:}
    \CommentTok{"""Suite de Fibonacci, version itérative"""}
    \ControlFlowTok{if}\NormalTok{ n }\OperatorTok{==} \DecValTok{0}\NormalTok{:}
        \ControlFlowTok{return} \DecValTok{0}
    \ControlFlowTok{else}\NormalTok{:}
\NormalTok{        f0, f1 }\OperatorTok{=} \DecValTok{0}\NormalTok{, }\DecValTok{1}
        \ControlFlowTok{for}\NormalTok{ k }\KeywordTok{in} \BuiltInTok{range}\NormalTok{(}\DecValTok{1}\NormalTok{, n):}
\NormalTok{            f0, f1 }\OperatorTok{=}\NormalTok{ ...  }\CommentTok{\# Ligne à compléter ...}
        \ControlFlowTok{return}\NormalTok{ f1}


\ControlFlowTok{for}\NormalTok{ k }\KeywordTok{in} \BuiltInTok{range}\NormalTok{(}\DecValTok{10}\NormalTok{):}
    \BuiltInTok{print}\NormalTok{(fibo\_iter(k))}
\end{Highlighting}
\end{Shaded}
\item
  Évaluer la complexité en termes de nombre d'additions.
\item
  D'après la définition de la suite, on a, pour tout entier naturel
  \(n\geqslant 2\) :

  \[F_{n}=F_{n-2}+F_{n-1}\]

  En déduire une version \textbf{récursive} de l'algorithme de calcul de
  \(F_n\). Cet algorithme a ceci de particulier que chaque fonction
  procède à \textbf{deux} appels récursifs. On pourra recopier et
  compléter le code ci-dessous.

\begin{Shaded}
\begin{Highlighting}[]
\KeywordTok{def}\NormalTok{ fibo\_rec(n: }\BuiltInTok{int}\NormalTok{) }\OperatorTok{{-}\textgreater{}} \BuiltInTok{int}\NormalTok{:}
    \CommentTok{"""Suite de Fibonacci version récursive"""}
    \CommentTok{\# Cas de base}
    \ControlFlowTok{if}\NormalTok{ ...:}
        \ControlFlowTok{return}\NormalTok{ n}
    \CommentTok{\# Récursion}
    \ControlFlowTok{else}\NormalTok{:}
        \ControlFlowTok{return}\NormalTok{ ...}


\ControlFlowTok{for}\NormalTok{ k }\KeywordTok{in} \BuiltInTok{range}\NormalTok{(}\DecValTok{10}\NormalTok{):}
    \BuiltInTok{print}\NormalTok{(fibo\_rec(k))}
\end{Highlighting}
\end{Shaded}
\item
  Utiliser chacune des deux versions pour calculer la valeur de
  \(F_{50}\). Que constate-t-on ? Expliquer.
\end{enumerate}

\begin{tcolorbox}[enhanced jigsaw, breakable, toprule=.15mm, left=2mm, coltitle=black, toptitle=1mm, title=\textcolor{quarto-callout-tip-color}{\faLightbulb}\hspace{0.5em}{Remarques et compléments}, arc=.35mm, leftrule=.75mm, colback=white, bottomrule=.15mm, colbacktitle=quarto-callout-tip-color!10!white, rightrule=.15mm, opacitybacktitle=0.6, opacityback=0, bottomtitle=1mm, titlerule=0mm]

La version récursive se révèle beaucoup moins efficace. Pour comprendre
pourquoi, nous pouvons représenter par un arbre les appels récursifs
nécessaires.

\includegraphics{fibo_mermaid.png}

On retrouve les cas de base dans les feuilles de l'arbre. Nous pouvons
constater que le nombre d'appels récursifs est très grand. Il est
possible de démontrer que ce nombre augmente de façon
\textbf{exponentielle}. Pour calculer \(F_{100}\), il y aurait environ
\(10^{20}\) opérations. À raison de \(10^9\) opérations par seconde, la
calcul prendra de l'ordre de \(10^{11}\) secondes, soit environ 3 000
ans !

Un autre constat qui montre l'inefficacité de ce programme : plusieurs
calculs identiques sont répétés plusieurs fois. On calcule par exemple
\(F_3\) deux fois et \(F_2\) trois fois. Une solution meilleure serait
de garder en mémoire les éléments déjà calculés et de ne calculer que
les nouveaux éléments encore jamais rencontrés. Une telle démarche
relève de la \textbf{programmation dynamique} qui sera abordée en fin
d'année.

Pour satisfaire votre curiosité insatiable, vous pouvez déjà observer et
tester le programme ci-dessous :

\begin{Shaded}
\begin{Highlighting}[]
\KeywordTok{def}\NormalTok{ fibo\_dyn(n: }\BuiltInTok{int}\NormalTok{, suite: }\BuiltInTok{dict} \OperatorTok{=}\NormalTok{ \{}\DecValTok{0}\NormalTok{: }\DecValTok{0}\NormalTok{, }\DecValTok{1}\NormalTok{: }\DecValTok{1}\NormalTok{\}) }\OperatorTok{{-}\textgreater{}} \BuiltInTok{int}\NormalTok{:}
    \CommentTok{"""Suite de Fibonacci version dynamique"""}
    \CommentTok{\# Cas de base}
    \ControlFlowTok{if}\NormalTok{ n }\OperatorTok{==} \DecValTok{0} \KeywordTok{or}\NormalTok{ n }\OperatorTok{==} \DecValTok{1}\NormalTok{:}
        \ControlFlowTok{return}\NormalTok{ n}
    \CommentTok{\# Récursion}
    \ControlFlowTok{else}\NormalTok{:}
        \CommentTok{\# Si Fn est déjà calculé, on le retourne}
        \ControlFlowTok{if}\NormalTok{ n }\KeywordTok{in}\NormalTok{ suite.keys():}
            \ControlFlowTok{return}\NormalTok{ suite[n]}
        \ControlFlowTok{else}\NormalTok{:}
            \CommentTok{\# Sinon, on le calcule et on le garde en mémoire}
\NormalTok{            f }\OperatorTok{=}\NormalTok{ fibo\_dyn(n}\OperatorTok{{-}}\DecValTok{2}\NormalTok{, suite) }\OperatorTok{+}\NormalTok{ fibo\_dyn(n}\OperatorTok{{-}}\DecValTok{1}\NormalTok{, suite)}
\NormalTok{            suite[n] }\OperatorTok{=}\NormalTok{ f}
            \ControlFlowTok{return}\NormalTok{ f}


\ControlFlowTok{for}\NormalTok{ k }\KeywordTok{in} \BuiltInTok{range}\NormalTok{(}\DecValTok{10}\NormalTok{):}
    \BuiltInTok{print}\NormalTok{(fibo\_dyn(k))}
\end{Highlighting}
\end{Shaded}

Une exécution dans PythonTutor est instructive :

Voir
\href{https://www.flallemand.fr/wp/2022/06/05/mesurer-le-temps-dexecution-dun-fragment-de-code/}{cet
article du blog} qui explique comment visualiser le temps d'exécution
d'une fonction.

\end{tcolorbox}

\hypertarget{fa-solid-pencil-alt-fa-desktop-exercice-3-calcul-de-xn}{%
\subsection{\texorpdfstring{\faIcon{pencil-alt} \faIcon{desktop}
Exercice 3 : calcul de
\(x^n\)}{  Exercice 3 : calcul de x\^{}n}}\label{fa-solid-pencil-alt-fa-desktop-exercice-3-calcul-de-xn}}

Pour tout nombre réel \(x\) et tout entier naturel \(n\), \(x^n\) est
défini par \(x^0=1\) et, pour \(n>0\),
\(x^n=x\times x\times x\times \ldots \times x\) : produit de \(n\)
facteurs tous égaux à \(x\).

Les règles de calcul sur les exposants permettent d'affirmer que, pour
\(n>0\), \(x^n=x\times x^{n-1}\).

\begin{enumerate}
\def\labelenumi{\arabic{enumi}.}
\tightlist
\item
  Écrire la fonction récursive \texttt{puissance(x,n)} qui calcule le
  nombre \(x^n\) pour tout entier naturel \(n\).
\item
  Dessiner l'arbre d'appels de cette fonction pour \(x=3\) et \(n=5\).
\item
  Pour les plus rapides
\end{enumerate}

\begin{tcolorbox}[enhanced jigsaw, breakable, toprule=.15mm, left=2mm, coltitle=black, toptitle=1mm, title=\textcolor{quarto-callout-caution-color}{\faFire}\hspace{0.5em}{Question bonus}, arc=.35mm, leftrule=.75mm, colback=white, bottomrule=.15mm, colbacktitle=quarto-callout-caution-color!10!white, rightrule=.15mm, opacitybacktitle=0.6, opacityback=0, bottomtitle=1mm, titlerule=0mm]

Un autre méthode de calcul de \(x^n\) consiste à distinguer le cas où
\(n\) est pair et celui où \(n\) est impair :

\begin{itemize}
\tightlist
\item
  si \(n=0\), alors \(x^n=1\) ;
\item
  si \(n\) est pair, alors \(x^n=\left(x^{n/2}\right)^2\) ;
\item
  si \(n\) est impair, alors \(x^n=x\times\left(x^{(n-1)/2}\right)^2\).
\end{itemize}

L'algorithme qui découle de cette définition porte également le nom
\textbf{d'exponentiation rapide}. Comme son nom l'indique, il s'agit
d'un algorithme particulièrement efficace pour calculer rapidement de
grandes puissances entières.

Écrire la fonction récursive \texttt{puissancev2(x,n)} qui calcule le
nombre \(x^n\) pour tout entier naturel n selon la méthode
d'exponentiation rapide.

\end{tcolorbox}

\hypertarget{fa-solid-pencil-alt-exercice-4-maximum-dune-liste}{%
\subsection{\texorpdfstring{\faIcon{pencil-alt} Exercice 4 : maximum
d'une
liste}{ Exercice 4 : maximum d'une liste}}\label{fa-solid-pencil-alt-exercice-4-maximum-dune-liste}}

On considère le programme ci-dessous :

\begin{Shaded}
\begin{Highlighting}[]
\KeywordTok{def}\NormalTok{ maximum(a, b):}
    \ControlFlowTok{if}\NormalTok{ a }\OperatorTok{\textgreater{}}\NormalTok{ b:}
        \ControlFlowTok{return}\NormalTok{ a}
    \ControlFlowTok{else}\NormalTok{:}
        \ControlFlowTok{return}\NormalTok{ b}

\KeywordTok{def}\NormalTok{ maximum\_tab(tab):}
    \ControlFlowTok{if} \BuiltInTok{len}\NormalTok{(tab) }\OperatorTok{==} \DecValTok{1}\NormalTok{:}
        \ControlFlowTok{return}\NormalTok{ tab[}\DecValTok{0}\NormalTok{]}
    \ControlFlowTok{else}\NormalTok{:}
        \ControlFlowTok{return}\NormalTok{ maximum(tab[}\DecValTok{0}\NormalTok{], maximum\_tab(tab[}\DecValTok{1}\NormalTok{:]))}

\ImportTok{from}\NormalTok{ random }\ImportTok{import}\NormalTok{ randint}

\NormalTok{mon\_tab }\OperatorTok{=}\NormalTok{ []}
\ControlFlowTok{for}\NormalTok{ i }\KeywordTok{in} \BuiltInTok{range}\NormalTok{(}\DecValTok{20}\NormalTok{):}
\NormalTok{    mon\_tab.append(randint(}\OperatorTok{{-}}\DecValTok{100}\NormalTok{, }\DecValTok{100}\NormalTok{))}
\BuiltInTok{print}\NormalTok{(mon\_tab)}
\BuiltInTok{print}\NormalTok{(maximum\_tab(mon\_tab))}
\end{Highlighting}
\end{Shaded}

\begin{enumerate}
\def\labelenumi{\arabic{enumi}.}
\tightlist
\item
  Décrire, en langage usuel, le principe de fonctionnement de la
  fonction \texttt{maximum\_tab}.
\item
  Expliquer en quoi la fonction \texttt{maximum\_tab} est récursive.
  Quel est le cas de base ?
\item
  Prouver la terminaison de cette fonction.
\item
  Effectuer par récurrence la preuve de cet algorithme (c'est-à-dire
  prouver que la fonction retourne bien le maximum du tableau donné en
  argument).
\item
  Dessiner l'arbre d'appels de cette fonction pour l'appel
  \texttt{maximum\_tab({[}-4,55,-1,-35,-52,31{]})}.
\end{enumerate}

\hypertarget{fa-desktop-exercice-5-palindromes}{%
\subsection{\texorpdfstring{\faIcon{desktop} Exercice 5 :
palindromes}{ Exercice 5 : palindromes}}\label{fa-desktop-exercice-5-palindromes}}

On appelle palindrome un mot qui se lit dans les deux sens comme « été »
ou « radar ».

Écrire une fonction récursive \texttt{palindrome} qui teste si un mot
est un palindrome.

\begin{itemize}
\tightlist
\item
  Entrée : Un mot (type \texttt{str}).
\item
  Sortie : Un booléen égal à \texttt{True} si le mot est un palindrome,
  \texttt{False} sinon.
\end{itemize}

On considérera les deux cas suivant comme cas de base :

\begin{itemize}
\tightlist
\item
  si le mot est la chaîne vide, c'est un palindrome ;
\item
  si le mot ne contient qu'une seule lettre, c'est un palindrome
\end{itemize}

\hypertarget{fa-solid-pencil-alt-fa-desktop-exercice-6-flocon-de-von-koch}{%
\subsection{\texorpdfstring{\faIcon{pencil-alt} \faIcon{desktop}
Exercice 6 : flocon de von
Koch}{  Exercice 6 : flocon de von Koch}}\label{fa-solid-pencil-alt-fa-desktop-exercice-6-flocon-de-von-koch}}

Une image qui a une apparence similaire quelle que soit l'échelle à
laquelle on l'observe est appelée une \textbf{fractale} (il y a d'autres
types de fractales).

Un exemple simple de fractale est le flocon de Von Koch, dont voici une
représentation (pour un degré 4).

\begin{figure}

{\centering \includegraphics{koch.png}

}

\caption{Flocon de von Koch}

\end{figure}

On peut la créer à partir d'un segment de droite, en modifiant
récursivement chaque segment de droite de la façon suivante :

\begin{itemize}
\tightlist
\item
  on divise le segment de droite en trois segments de longueurs égales ;
\item
  on construit un triangle équilatéral ayant pour base le segment médian
  de la première étape ;
\item
  on supprime le segment de droite qui était la base du triangle de la
  deuxième étape.
\end{itemize}

Voici le résultat obtenu en une étape :

\begin{figure}

{\centering \includegraphics{koch_etape.png}

}

\caption{Étape de construction}

\end{figure}

Pour continuer, il suffit de considérer chaque segment de cette dernière
figure comme segment de départ.

\begin{Shaded}
\begin{Highlighting}[]
\ImportTok{from}\NormalTok{ turtle }\ImportTok{import} \OperatorTok{*}


\KeywordTok{def}\NormalTok{ Koch(n, d):}
    \ControlFlowTok{if}\NormalTok{ n }\OperatorTok{==} \DecValTok{0}\NormalTok{:}
\NormalTok{        forward(d)}
    \ControlFlowTok{else}\NormalTok{:}
\NormalTok{        Koch(n}\OperatorTok{{-}}\DecValTok{1}\NormalTok{, d}\OperatorTok{/}\DecValTok{3}\NormalTok{)}
\NormalTok{        left(}\DecValTok{60}\NormalTok{)}
\NormalTok{        Koch(n}\OperatorTok{{-}}\DecValTok{1}\NormalTok{, d}\OperatorTok{/}\DecValTok{3}\NormalTok{)}
\NormalTok{        right(}\DecValTok{120}\NormalTok{)}
\NormalTok{        Koch(n}\OperatorTok{{-}}\DecValTok{1}\NormalTok{, d}\OperatorTok{/}\DecValTok{3}\NormalTok{)}
\NormalTok{        left(}\DecValTok{60}\NormalTok{)}
\NormalTok{        Koch(n}\OperatorTok{{-}}\DecValTok{1}\NormalTok{, d}\OperatorTok{/}\DecValTok{3}\NormalTok{)}
    \ControlFlowTok{return} \VariableTok{None}


\KeywordTok{def}\NormalTok{ flocon(n, d):}
    \ControlFlowTok{for}\NormalTok{ k }\KeywordTok{in} \BuiltInTok{range}\NormalTok{(}\DecValTok{3}\NormalTok{):}
\NormalTok{        Koch(n, d)}
\NormalTok{        right(}\DecValTok{120}\NormalTok{)}
    \ControlFlowTok{return} \VariableTok{None}


\NormalTok{flocon(}\DecValTok{4}\NormalTok{, }\DecValTok{300}\NormalTok{)}
\NormalTok{exitonclick()}
\end{Highlighting}
\end{Shaded}

\begin{enumerate}
\def\labelenumi{\arabic{enumi}.}
\tightlist
\item
  Identifier le cas de base de la fonction récursive
  \texttt{Koch(n,\ d)}. Que fait-il ?
\item
  Modifier les paramètres \texttt{n} et \texttt{d} lors de l'appel à la
  fonction \texttt{flocon} et observer l'impact de ces modifications sur
  le dessin.
\item
  Combien d'appels récursifs sont-ils réalisés lors de l'appel de la
  fonction \texttt{Koch(4,\ 300)} ?
\end{enumerate}

\hypertarget{fa-solid-pencil-alt-exercice-7-type-bac}{%
\subsection{\texorpdfstring{\faIcon{pencil-alt} Exercice 7 (type
bac)}{ Exercice 7 (type bac)}}\label{fa-solid-pencil-alt-exercice-7-type-bac}}



\end{document}
