% Options for packages loaded elsewhere
\PassOptionsToPackage{unicode}{hyperref}
\PassOptionsToPackage{hyphens}{url}
\PassOptionsToPackage{dvipsnames,svgnames,x11names}{xcolor}
%
\documentclass[
  letterpaper,
  DIV=11,
  numbers=noendperiod]{scrartcl}

\usepackage{amsmath,amssymb}
\usepackage{iftex}
\ifPDFTeX
  \usepackage[T1]{fontenc}
  \usepackage[utf8]{inputenc}
  \usepackage{textcomp} % provide euro and other symbols
\else % if luatex or xetex
  \usepackage{unicode-math}
  \defaultfontfeatures{Scale=MatchLowercase}
  \defaultfontfeatures[\rmfamily]{Ligatures=TeX,Scale=1}
\fi
\usepackage{lmodern}
\ifPDFTeX\else  
    % xetex/luatex font selection
\fi
% Use upquote if available, for straight quotes in verbatim environments
\IfFileExists{upquote.sty}{\usepackage{upquote}}{}
\IfFileExists{microtype.sty}{% use microtype if available
  \usepackage[]{microtype}
  \UseMicrotypeSet[protrusion]{basicmath} % disable protrusion for tt fonts
}{}
\makeatletter
\@ifundefined{KOMAClassName}{% if non-KOMA class
  \IfFileExists{parskip.sty}{%
    \usepackage{parskip}
  }{% else
    \setlength{\parindent}{0pt}
    \setlength{\parskip}{6pt plus 2pt minus 1pt}}
}{% if KOMA class
  \KOMAoptions{parskip=half}}
\makeatother
\usepackage{xcolor}
\usepackage[top=20mm,bottom=20mm,left=20mm,right=20mm,heightrounded]{geometry}
\setlength{\emergencystretch}{3em} % prevent overfull lines
\setcounter{secnumdepth}{-\maxdimen} % remove section numbering
% Make \paragraph and \subparagraph free-standing
\ifx\paragraph\undefined\else
  \let\oldparagraph\paragraph
  \renewcommand{\paragraph}[1]{\oldparagraph{#1}\mbox{}}
\fi
\ifx\subparagraph\undefined\else
  \let\oldsubparagraph\subparagraph
  \renewcommand{\subparagraph}[1]{\oldsubparagraph{#1}\mbox{}}
\fi

\usepackage{color}
\usepackage{fancyvrb}
\newcommand{\VerbBar}{|}
\newcommand{\VERB}{\Verb[commandchars=\\\{\}]}
\DefineVerbatimEnvironment{Highlighting}{Verbatim}{commandchars=\\\{\}}
% Add ',fontsize=\small' for more characters per line
\usepackage{framed}
\definecolor{shadecolor}{RGB}{241,243,245}
\newenvironment{Shaded}{\begin{snugshade}}{\end{snugshade}}
\newcommand{\AlertTok}[1]{\textcolor[rgb]{0.68,0.00,0.00}{#1}}
\newcommand{\AnnotationTok}[1]{\textcolor[rgb]{0.37,0.37,0.37}{#1}}
\newcommand{\AttributeTok}[1]{\textcolor[rgb]{0.40,0.45,0.13}{#1}}
\newcommand{\BaseNTok}[1]{\textcolor[rgb]{0.68,0.00,0.00}{#1}}
\newcommand{\BuiltInTok}[1]{\textcolor[rgb]{0.00,0.23,0.31}{#1}}
\newcommand{\CharTok}[1]{\textcolor[rgb]{0.13,0.47,0.30}{#1}}
\newcommand{\CommentTok}[1]{\textcolor[rgb]{0.37,0.37,0.37}{#1}}
\newcommand{\CommentVarTok}[1]{\textcolor[rgb]{0.37,0.37,0.37}{\textit{#1}}}
\newcommand{\ConstantTok}[1]{\textcolor[rgb]{0.56,0.35,0.01}{#1}}
\newcommand{\ControlFlowTok}[1]{\textcolor[rgb]{0.00,0.23,0.31}{#1}}
\newcommand{\DataTypeTok}[1]{\textcolor[rgb]{0.68,0.00,0.00}{#1}}
\newcommand{\DecValTok}[1]{\textcolor[rgb]{0.68,0.00,0.00}{#1}}
\newcommand{\DocumentationTok}[1]{\textcolor[rgb]{0.37,0.37,0.37}{\textit{#1}}}
\newcommand{\ErrorTok}[1]{\textcolor[rgb]{0.68,0.00,0.00}{#1}}
\newcommand{\ExtensionTok}[1]{\textcolor[rgb]{0.00,0.23,0.31}{#1}}
\newcommand{\FloatTok}[1]{\textcolor[rgb]{0.68,0.00,0.00}{#1}}
\newcommand{\FunctionTok}[1]{\textcolor[rgb]{0.28,0.35,0.67}{#1}}
\newcommand{\ImportTok}[1]{\textcolor[rgb]{0.00,0.46,0.62}{#1}}
\newcommand{\InformationTok}[1]{\textcolor[rgb]{0.37,0.37,0.37}{#1}}
\newcommand{\KeywordTok}[1]{\textcolor[rgb]{0.00,0.23,0.31}{#1}}
\newcommand{\NormalTok}[1]{\textcolor[rgb]{0.00,0.23,0.31}{#1}}
\newcommand{\OperatorTok}[1]{\textcolor[rgb]{0.37,0.37,0.37}{#1}}
\newcommand{\OtherTok}[1]{\textcolor[rgb]{0.00,0.23,0.31}{#1}}
\newcommand{\PreprocessorTok}[1]{\textcolor[rgb]{0.68,0.00,0.00}{#1}}
\newcommand{\RegionMarkerTok}[1]{\textcolor[rgb]{0.00,0.23,0.31}{#1}}
\newcommand{\SpecialCharTok}[1]{\textcolor[rgb]{0.37,0.37,0.37}{#1}}
\newcommand{\SpecialStringTok}[1]{\textcolor[rgb]{0.13,0.47,0.30}{#1}}
\newcommand{\StringTok}[1]{\textcolor[rgb]{0.13,0.47,0.30}{#1}}
\newcommand{\VariableTok}[1]{\textcolor[rgb]{0.07,0.07,0.07}{#1}}
\newcommand{\VerbatimStringTok}[1]{\textcolor[rgb]{0.13,0.47,0.30}{#1}}
\newcommand{\WarningTok}[1]{\textcolor[rgb]{0.37,0.37,0.37}{\textit{#1}}}

\providecommand{\tightlist}{%
  \setlength{\itemsep}{0pt}\setlength{\parskip}{0pt}}\usepackage{longtable,booktabs,array}
\usepackage{calc} % for calculating minipage widths
% Correct order of tables after \paragraph or \subparagraph
\usepackage{etoolbox}
\makeatletter
\patchcmd\longtable{\par}{\if@noskipsec\mbox{}\fi\par}{}{}
\makeatother
% Allow footnotes in longtable head/foot
\IfFileExists{footnotehyper.sty}{\usepackage{footnotehyper}}{\usepackage{footnote}}
\makesavenoteenv{longtable}
\usepackage{graphicx}
\makeatletter
\def\maxwidth{\ifdim\Gin@nat@width>\linewidth\linewidth\else\Gin@nat@width\fi}
\def\maxheight{\ifdim\Gin@nat@height>\textheight\textheight\else\Gin@nat@height\fi}
\makeatother
% Scale images if necessary, so that they will not overflow the page
% margins by default, and it is still possible to overwrite the defaults
% using explicit options in \includegraphics[width, height, ...]{}
\setkeys{Gin}{width=\maxwidth,height=\maxheight,keepaspectratio}
% Set default figure placement to htbp
\makeatletter
\def\fps@figure{htbp}
\makeatother

\usepackage{fancyhdr} \pagestyle{fancy} \usepackage{lastpage}
\KOMAoption{captions}{tablesignature}
\makeatletter
\@ifpackageloaded{tcolorbox}{}{\usepackage[skins,breakable]{tcolorbox}}
\@ifpackageloaded{fontawesome5}{}{\usepackage{fontawesome5}}
\definecolor{quarto-callout-color}{HTML}{909090}
\definecolor{quarto-callout-note-color}{HTML}{0758E5}
\definecolor{quarto-callout-important-color}{HTML}{CC1914}
\definecolor{quarto-callout-warning-color}{HTML}{EB9113}
\definecolor{quarto-callout-tip-color}{HTML}{00A047}
\definecolor{quarto-callout-caution-color}{HTML}{FC5300}
\definecolor{quarto-callout-color-frame}{HTML}{acacac}
\definecolor{quarto-callout-note-color-frame}{HTML}{4582ec}
\definecolor{quarto-callout-important-color-frame}{HTML}{d9534f}
\definecolor{quarto-callout-warning-color-frame}{HTML}{f0ad4e}
\definecolor{quarto-callout-tip-color-frame}{HTML}{02b875}
\definecolor{quarto-callout-caution-color-frame}{HTML}{fd7e14}
\makeatother
\makeatletter
\makeatother
\makeatletter
\makeatother
\makeatletter
\@ifpackageloaded{caption}{}{\usepackage{caption}}
\AtBeginDocument{%
\ifdefined\contentsname
  \renewcommand*\contentsname{Table des matières}
\else
  \newcommand\contentsname{Table des matières}
\fi
\ifdefined\listfigurename
  \renewcommand*\listfigurename{Liste des Figures}
\else
  \newcommand\listfigurename{Liste des Figures}
\fi
\ifdefined\listtablename
  \renewcommand*\listtablename{Liste des Tables}
\else
  \newcommand\listtablename{Liste des Tables}
\fi
\ifdefined\figurename
  \renewcommand*\figurename{Figure}
\else
  \newcommand\figurename{Figure}
\fi
\ifdefined\tablename
  \renewcommand*\tablename{Tableau}
\else
  \newcommand\tablename{Tableau}
\fi
}
\@ifpackageloaded{float}{}{\usepackage{float}}
\floatstyle{ruled}
\@ifundefined{c@chapter}{\newfloat{codelisting}{h}{lop}}{\newfloat{codelisting}{h}{lop}[chapter]}
\floatname{codelisting}{Listing}
\newcommand*\listoflistings{\listof{codelisting}{Liste des Listings}}
\makeatother
\makeatletter
\@ifpackageloaded{caption}{}{\usepackage{caption}}
\@ifpackageloaded{subcaption}{}{\usepackage{subcaption}}
\makeatother
\makeatletter
\@ifpackageloaded{tcolorbox}{}{\usepackage[skins,breakable]{tcolorbox}}
\makeatother
\makeatletter
\@ifundefined{shadecolor}{\definecolor{shadecolor}{rgb}{.97, .97, .97}}
\makeatother
\makeatletter
\makeatother
\makeatletter
\makeatother
\makeatletter
\@ifpackageloaded{fontawesome5}{}{\usepackage{fontawesome5}}
\makeatother
\ifLuaTeX
\usepackage[bidi=basic]{babel}
\else
\usepackage[bidi=default]{babel}
\fi
\babelprovide[main,import]{french}
% get rid of language-specific shorthands (see #6817):
\let\LanguageShortHands\languageshorthands
\def\languageshorthands#1{}
\ifLuaTeX
  \usepackage{selnolig}  % disable illegal ligatures
\fi
\IfFileExists{bookmark.sty}{\usepackage{bookmark}}{\usepackage{hyperref}}
\IfFileExists{xurl.sty}{\usepackage{xurl}}{} % add URL line breaks if available
\urlstyle{same} % disable monospaced font for URLs
\hypersetup{
  pdftitle={Paradigmes de programmation (Exercices)},
  pdflang={fr},
  colorlinks=true,
  linkcolor={blue},
  filecolor={Maroon},
  citecolor={Blue},
  urlcolor={Blue},
  pdfcreator={LaTeX via pandoc}}

\title{Paradigmes de programmation (Exercices)}
\usepackage{etoolbox}
\makeatletter
\providecommand{\subtitle}[1]{% add subtitle to \maketitle
  \apptocmd{\@title}{\par {\large #1 \par}}{}{}
}
\makeatother
\subtitle{S1 - Langages et programmation}
\author{}
\date{}

\begin{document}
\maketitle
\lhead{Spécialité NSI} \rhead{Terminale} \chead{} \cfoot{} \lfoot{Lycée \'Emile Duclaux} \rfoot{Page \thepage/\pageref{LastPage}} \renewcommand{\headrulewidth}{0pt} \renewcommand{\footrulewidth}{0pt} \thispagestyle{fancy} \vspace{-2cm}

\ifdefined\Shaded\renewenvironment{Shaded}{\begin{tcolorbox}[interior hidden, sharp corners, breakable, enhanced, borderline west={3pt}{0pt}{shadecolor}, boxrule=0pt, frame hidden]}{\end{tcolorbox}}\fi

\emph{Les exercices précédés du symbole \faIcon{desktop} sont à faire
sur machine, en sauvegardant le fichier si nécessaire.}

\emph{Les exercices précédés du symbole \faIcon{pencil-alt} doivent être
résolus par écrit.}

\hypertarget{fa-solid-pencil-alt-exercice-1}{%
\subsection{\texorpdfstring{\faIcon{pencil-alt} Exercice
1}{ Exercice 1}}\label{fa-solid-pencil-alt-exercice-1}}

Voici différents algorithmiques permettant l'affichage des 10 chiffres
entiers dans l'ordre décroissant.

Préciser pour chacun des algorithme le type de paradigme auquel il
correspond.

\textbf{Algorithme 1 :}

\begin{Shaded}
\begin{Highlighting}[]
\KeywordTok{def}\NormalTok{ decompter(n:}\BuiltInTok{int}\NormalTok{)}\OperatorTok{{-}\textgreater{}}\VariableTok{None}\NormalTok{:}
    \ControlFlowTok{if}\NormalTok{ n}\OperatorTok{\textgreater{}=}\DecValTok{0}\NormalTok{: }
        \BuiltInTok{print}\NormalTok{(n)}
\NormalTok{        decompter(n}\OperatorTok{{-}}\DecValTok{1}\NormalTok{)}
\NormalTok{decompter(}\DecValTok{9}\NormalTok{)}
\end{Highlighting}
\end{Shaded}

\textbf{Algorithme 2 :}

\begin{Shaded}
\begin{Highlighting}[]
\ControlFlowTok{for}\NormalTok{ i }\KeywordTok{in} \BuiltInTok{range}\NormalTok{(}\DecValTok{10}\NormalTok{):}
    \BuiltInTok{print}\NormalTok{(}\DecValTok{9}\OperatorTok{{-}}\NormalTok{i)}
\end{Highlighting}
\end{Shaded}

\textbf{Algorithme 3 :}

\begin{Shaded}
\begin{Highlighting}[]
\KeywordTok{class}\NormalTok{ Nombres():}

    \KeywordTok{def} \FunctionTok{\_\_init\_\_}\NormalTok{(}\VariableTok{self}\NormalTok{,valeur):}
        \VariableTok{self}\NormalTok{.valeur }\OperatorTok{=}\NormalTok{ valeur}

    \KeywordTok{def}\NormalTok{ diminuer(}\VariableTok{self}\NormalTok{):}
        \VariableTok{self}\NormalTok{.valeur }\OperatorTok{{-}=} \DecValTok{1}

    \KeywordTok{def} \FunctionTok{\_\_str\_\_}\NormalTok{(}\VariableTok{self}\NormalTok{):}
        \ControlFlowTok{return} \BuiltInTok{str}\NormalTok{(}\VariableTok{self}\NormalTok{.valeur)}

\NormalTok{n }\OperatorTok{=}\NormalTok{ Nombres(}\DecValTok{9}\NormalTok{)}
\ControlFlowTok{while}\NormalTok{ n.valeur }\OperatorTok{\textgreater{}=} \DecValTok{0}\NormalTok{ :}
    \BuiltInTok{print}\NormalTok{(n)}
\NormalTok{    n.diminuer()}
\end{Highlighting}
\end{Shaded}

\hypertarget{fa-solid-pencil-alt-exercice-2}{%
\subsection{\texorpdfstring{\faIcon{pencil-alt} Exercice
2}{ Exercice 2}}\label{fa-solid-pencil-alt-exercice-2}}

Voici deux versions d'une fonction \texttt{teste\_ordre\_liste} dont
l'objectif est de savoir si une liste est ordonnée par ordre croissant.
Indiquer quel paradigme est utilisé dans chacune des deux versions et
expliquer votre réponse.

\textbf{Version 1}

\begin{Shaded}
\begin{Highlighting}[]
\CommentTok{\# {-}*{-} coding: utf{-}8 {-}*{-}}

\KeywordTok{def}\NormalTok{ test\_ordre(a,b) :}
    \ControlFlowTok{if}\NormalTok{ a }\OperatorTok{\textless{}}\NormalTok{ b :}
        \ControlFlowTok{return} \VariableTok{True}
    \ControlFlowTok{else}\NormalTok{ :}
        \ControlFlowTok{return} \VariableTok{False}

\KeywordTok{def}\NormalTok{ test\_ordre\_liste(liste) :}
    \ControlFlowTok{if} \BuiltInTok{len}\NormalTok{(liste) }\OperatorTok{\textless{}} \DecValTok{2}\NormalTok{ :}
        \ControlFlowTok{return} \VariableTok{True}
    \ControlFlowTok{return}\NormalTok{ test\_ordre(liste[}\DecValTok{0}\NormalTok{],liste[}\DecValTok{1}\NormalTok{]) }\KeywordTok{and}\NormalTok{ test\_ordre\_liste(liste[}\DecValTok{1}\NormalTok{:])}

\NormalTok{test\_ordre\_liste([}\DecValTok{2}\NormalTok{,}\DecValTok{3}\NormalTok{,}\DecValTok{2}\NormalTok{])}
\end{Highlighting}
\end{Shaded}

\textbf{Version 2}

\begin{Shaded}
\begin{Highlighting}[]
\CommentTok{\# {-}*{-} coding: utf{-}8 {-}*{-}}

\KeywordTok{def}\NormalTok{ test\_ordre(a,b) :}
    \ControlFlowTok{if}\NormalTok{ a }\OperatorTok{\textless{}}\NormalTok{ b :}
        \ControlFlowTok{return} \VariableTok{True}
    \ControlFlowTok{else}\NormalTok{ :}
        \ControlFlowTok{return} \VariableTok{False}

\KeywordTok{def}\NormalTok{ test\_ordre\_liste(liste) :}
    \ControlFlowTok{if} \BuiltInTok{len}\NormalTok{(liste) }\OperatorTok{\textless{}} \DecValTok{2}\NormalTok{ :}
        \ControlFlowTok{return} \VariableTok{True}
    \ControlFlowTok{else}\NormalTok{ :}
        \ControlFlowTok{if}\NormalTok{ test\_ordre(liste[}\DecValTok{0}\NormalTok{],liste[}\DecValTok{1}\NormalTok{]) }\OperatorTok{==} \VariableTok{False}\NormalTok{ :}
            \ControlFlowTok{return} \VariableTok{False}
        \ControlFlowTok{else}\NormalTok{ :}
            \KeywordTok{del}\NormalTok{ liste[}\DecValTok{0}\NormalTok{]}
            \ControlFlowTok{return}\NormalTok{ test\_ordre\_liste(liste)}

\NormalTok{test\_ordre\_liste([}\DecValTok{2}\NormalTok{,}\DecValTok{3}\NormalTok{,}\DecValTok{2}\NormalTok{])}
\end{Highlighting}
\end{Shaded}

\hypertarget{fa-solid-pencil-alt-exercice-3}{%
\subsection{\texorpdfstring{\faIcon{pencil-alt} Exercice
3}{ Exercice 3}}\label{fa-solid-pencil-alt-exercice-3}}

Le programme ci-dessous ne respecte pas le paradigme fonctionnel.
Pourquoi ?

\begin{Shaded}
\begin{Highlighting}[]
\NormalTok{i }\OperatorTok{=} \DecValTok{5}

\KeywordTok{def}\NormalTok{ fct():}
  \ControlFlowTok{if}\NormalTok{ i }\OperatorTok{\textgreater{}} \DecValTok{5}\NormalTok{:}
    \ControlFlowTok{return} \VariableTok{True}
  \ControlFlowTok{else}\NormalTok{ :}
    \ControlFlowTok{return} \VariableTok{False}

\NormalTok{fct()}
\end{Highlighting}
\end{Shaded}

Modifier le programme pour qu'il respecte le paradigme fonctionnel.

\hypertarget{fa-solid-pencil-alt-exercice-4}{%
\subsection{\texorpdfstring{\faIcon{pencil-alt} Exercice
4}{ Exercice 4}}\label{fa-solid-pencil-alt-exercice-4}}

Même exercice avec le programme ci-dessous :

\begin{Shaded}
\begin{Highlighting}[]
\NormalTok{l }\OperatorTok{=}\NormalTok{ [}\DecValTok{4}\NormalTok{,}\DecValTok{7}\NormalTok{,}\DecValTok{3}\NormalTok{]}

\KeywordTok{def}\NormalTok{ ajout(i):}
\NormalTok{  l.append(i)}
\end{Highlighting}
\end{Shaded}

\begin{tcolorbox}[enhanced jigsaw, toprule=.15mm, breakable, titlerule=0mm, left=2mm, colbacktitle=quarto-callout-tip-color!10!white, opacityback=0, leftrule=.75mm, colback=white, opacitybacktitle=0.6, colframe=quarto-callout-tip-color-frame, bottomrule=.15mm, coltitle=black, bottomtitle=1mm, arc=.35mm, toptitle=1mm, title=\textcolor{quarto-callout-tip-color}{\faLightbulb}\hspace{0.5em}{Complément pour les curieux}, rightrule=.15mm]

Pour ceux qui voudraient découvrir un langage fonctionnel,
\href{http://sdz.tdct.org/sdz/ocaml-pour-les-zeros.html}{cette page}
fournit une introduction pas à pas aux bases de OCamL. Ce langage est
utilisé en CPGE scientifiques dans le cadre de l'option informatique.
Une autre introduction, pour la prépa, est disponible
\href{https://info-llg.fr/option-mpsi/pdf/01.les_bases.pdf}{ici}. On
peut programmer en OCamL en ligne \href{https://try.ocaml.pro/}{ici}.

\end{tcolorbox}



\end{document}
