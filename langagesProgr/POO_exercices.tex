% Options for packages loaded elsewhere
\PassOptionsToPackage{unicode}{hyperref}
\PassOptionsToPackage{hyphens}{url}
\PassOptionsToPackage{dvipsnames,svgnames,x11names}{xcolor}
%
\documentclass[
  letterpaper,
  DIV=11,
  numbers=noendperiod]{scrartcl}

\usepackage{amsmath,amssymb}
\usepackage{iftex}
\ifPDFTeX
  \usepackage[T1]{fontenc}
  \usepackage[utf8]{inputenc}
  \usepackage{textcomp} % provide euro and other symbols
\else % if luatex or xetex
  \usepackage{unicode-math}
  \defaultfontfeatures{Scale=MatchLowercase}
  \defaultfontfeatures[\rmfamily]{Ligatures=TeX,Scale=1}
\fi
\usepackage{lmodern}
\ifPDFTeX\else  
    % xetex/luatex font selection
\fi
% Use upquote if available, for straight quotes in verbatim environments
\IfFileExists{upquote.sty}{\usepackage{upquote}}{}
\IfFileExists{microtype.sty}{% use microtype if available
  \usepackage[]{microtype}
  \UseMicrotypeSet[protrusion]{basicmath} % disable protrusion for tt fonts
}{}
\makeatletter
\@ifundefined{KOMAClassName}{% if non-KOMA class
  \IfFileExists{parskip.sty}{%
    \usepackage{parskip}
  }{% else
    \setlength{\parindent}{0pt}
    \setlength{\parskip}{6pt plus 2pt minus 1pt}}
}{% if KOMA class
  \KOMAoptions{parskip=half}}
\makeatother
\usepackage{xcolor}
\usepackage[top=20mm,bottom=20mm,left=20mm,right=20mm,heightrounded]{geometry}
\setlength{\emergencystretch}{3em} % prevent overfull lines
\setcounter{secnumdepth}{-\maxdimen} % remove section numbering
% Make \paragraph and \subparagraph free-standing
\ifx\paragraph\undefined\else
  \let\oldparagraph\paragraph
  \renewcommand{\paragraph}[1]{\oldparagraph{#1}\mbox{}}
\fi
\ifx\subparagraph\undefined\else
  \let\oldsubparagraph\subparagraph
  \renewcommand{\subparagraph}[1]{\oldsubparagraph{#1}\mbox{}}
\fi

\usepackage{color}
\usepackage{fancyvrb}
\newcommand{\VerbBar}{|}
\newcommand{\VERB}{\Verb[commandchars=\\\{\}]}
\DefineVerbatimEnvironment{Highlighting}{Verbatim}{commandchars=\\\{\}}
% Add ',fontsize=\small' for more characters per line
\usepackage{framed}
\definecolor{shadecolor}{RGB}{241,243,245}
\newenvironment{Shaded}{\begin{snugshade}}{\end{snugshade}}
\newcommand{\AlertTok}[1]{\textcolor[rgb]{0.68,0.00,0.00}{#1}}
\newcommand{\AnnotationTok}[1]{\textcolor[rgb]{0.37,0.37,0.37}{#1}}
\newcommand{\AttributeTok}[1]{\textcolor[rgb]{0.40,0.45,0.13}{#1}}
\newcommand{\BaseNTok}[1]{\textcolor[rgb]{0.68,0.00,0.00}{#1}}
\newcommand{\BuiltInTok}[1]{\textcolor[rgb]{0.00,0.23,0.31}{#1}}
\newcommand{\CharTok}[1]{\textcolor[rgb]{0.13,0.47,0.30}{#1}}
\newcommand{\CommentTok}[1]{\textcolor[rgb]{0.37,0.37,0.37}{#1}}
\newcommand{\CommentVarTok}[1]{\textcolor[rgb]{0.37,0.37,0.37}{\textit{#1}}}
\newcommand{\ConstantTok}[1]{\textcolor[rgb]{0.56,0.35,0.01}{#1}}
\newcommand{\ControlFlowTok}[1]{\textcolor[rgb]{0.00,0.23,0.31}{#1}}
\newcommand{\DataTypeTok}[1]{\textcolor[rgb]{0.68,0.00,0.00}{#1}}
\newcommand{\DecValTok}[1]{\textcolor[rgb]{0.68,0.00,0.00}{#1}}
\newcommand{\DocumentationTok}[1]{\textcolor[rgb]{0.37,0.37,0.37}{\textit{#1}}}
\newcommand{\ErrorTok}[1]{\textcolor[rgb]{0.68,0.00,0.00}{#1}}
\newcommand{\ExtensionTok}[1]{\textcolor[rgb]{0.00,0.23,0.31}{#1}}
\newcommand{\FloatTok}[1]{\textcolor[rgb]{0.68,0.00,0.00}{#1}}
\newcommand{\FunctionTok}[1]{\textcolor[rgb]{0.28,0.35,0.67}{#1}}
\newcommand{\ImportTok}[1]{\textcolor[rgb]{0.00,0.46,0.62}{#1}}
\newcommand{\InformationTok}[1]{\textcolor[rgb]{0.37,0.37,0.37}{#1}}
\newcommand{\KeywordTok}[1]{\textcolor[rgb]{0.00,0.23,0.31}{#1}}
\newcommand{\NormalTok}[1]{\textcolor[rgb]{0.00,0.23,0.31}{#1}}
\newcommand{\OperatorTok}[1]{\textcolor[rgb]{0.37,0.37,0.37}{#1}}
\newcommand{\OtherTok}[1]{\textcolor[rgb]{0.00,0.23,0.31}{#1}}
\newcommand{\PreprocessorTok}[1]{\textcolor[rgb]{0.68,0.00,0.00}{#1}}
\newcommand{\RegionMarkerTok}[1]{\textcolor[rgb]{0.00,0.23,0.31}{#1}}
\newcommand{\SpecialCharTok}[1]{\textcolor[rgb]{0.37,0.37,0.37}{#1}}
\newcommand{\SpecialStringTok}[1]{\textcolor[rgb]{0.13,0.47,0.30}{#1}}
\newcommand{\StringTok}[1]{\textcolor[rgb]{0.13,0.47,0.30}{#1}}
\newcommand{\VariableTok}[1]{\textcolor[rgb]{0.07,0.07,0.07}{#1}}
\newcommand{\VerbatimStringTok}[1]{\textcolor[rgb]{0.13,0.47,0.30}{#1}}
\newcommand{\WarningTok}[1]{\textcolor[rgb]{0.37,0.37,0.37}{\textit{#1}}}

\providecommand{\tightlist}{%
  \setlength{\itemsep}{0pt}\setlength{\parskip}{0pt}}\usepackage{longtable,booktabs,array}
\usepackage{calc} % for calculating minipage widths
% Correct order of tables after \paragraph or \subparagraph
\usepackage{etoolbox}
\makeatletter
\patchcmd\longtable{\par}{\if@noskipsec\mbox{}\fi\par}{}{}
\makeatother
% Allow footnotes in longtable head/foot
\IfFileExists{footnotehyper.sty}{\usepackage{footnotehyper}}{\usepackage{footnote}}
\makesavenoteenv{longtable}
\usepackage{graphicx}
\makeatletter
\def\maxwidth{\ifdim\Gin@nat@width>\linewidth\linewidth\else\Gin@nat@width\fi}
\def\maxheight{\ifdim\Gin@nat@height>\textheight\textheight\else\Gin@nat@height\fi}
\makeatother
% Scale images if necessary, so that they will not overflow the page
% margins by default, and it is still possible to overwrite the defaults
% using explicit options in \includegraphics[width, height, ...]{}
\setkeys{Gin}{width=\maxwidth,height=\maxheight,keepaspectratio}
% Set default figure placement to htbp
\makeatletter
\def\fps@figure{htbp}
\makeatother

\usepackage{fancyhdr} \pagestyle{fancy} \usepackage{lastpage}
\KOMAoption{captions}{tablesignature}
\makeatletter
\makeatother
\makeatletter
\makeatother
\makeatletter
\@ifpackageloaded{caption}{}{\usepackage{caption}}
\AtBeginDocument{%
\ifdefined\contentsname
  \renewcommand*\contentsname{Table des matières}
\else
  \newcommand\contentsname{Table des matières}
\fi
\ifdefined\listfigurename
  \renewcommand*\listfigurename{Liste des Figures}
\else
  \newcommand\listfigurename{Liste des Figures}
\fi
\ifdefined\listtablename
  \renewcommand*\listtablename{Liste des Tables}
\else
  \newcommand\listtablename{Liste des Tables}
\fi
\ifdefined\figurename
  \renewcommand*\figurename{Figure}
\else
  \newcommand\figurename{Figure}
\fi
\ifdefined\tablename
  \renewcommand*\tablename{Tableau}
\else
  \newcommand\tablename{Tableau}
\fi
}
\@ifpackageloaded{float}{}{\usepackage{float}}
\floatstyle{ruled}
\@ifundefined{c@chapter}{\newfloat{codelisting}{h}{lop}}{\newfloat{codelisting}{h}{lop}[chapter]}
\floatname{codelisting}{Listing}
\newcommand*\listoflistings{\listof{codelisting}{Liste des Listings}}
\makeatother
\makeatletter
\@ifpackageloaded{caption}{}{\usepackage{caption}}
\@ifpackageloaded{subcaption}{}{\usepackage{subcaption}}
\makeatother
\makeatletter
\@ifpackageloaded{tcolorbox}{}{\usepackage[skins,breakable]{tcolorbox}}
\makeatother
\makeatletter
\@ifundefined{shadecolor}{\definecolor{shadecolor}{rgb}{.97, .97, .97}}
\makeatother
\makeatletter
\makeatother
\makeatletter
\makeatother
\makeatletter
\@ifpackageloaded{fontawesome5}{}{\usepackage{fontawesome5}}
\makeatother
\ifLuaTeX
\usepackage[bidi=basic]{babel}
\else
\usepackage[bidi=default]{babel}
\fi
\babelprovide[main,import]{french}
% get rid of language-specific shorthands (see #6817):
\let\LanguageShortHands\languageshorthands
\def\languageshorthands#1{}
\ifLuaTeX
  \usepackage{selnolig}  % disable illegal ligatures
\fi
\IfFileExists{bookmark.sty}{\usepackage{bookmark}}{\usepackage{hyperref}}
\IfFileExists{xurl.sty}{\usepackage{xurl}}{} % add URL line breaks if available
\urlstyle{same} % disable monospaced font for URLs
\hypersetup{
  pdftitle={Programmation orientée objets (Exercices)},
  pdflang={fr},
  colorlinks=true,
  linkcolor={blue},
  filecolor={Maroon},
  citecolor={Blue},
  urlcolor={Blue},
  pdfcreator={LaTeX via pandoc}}

\title{Programmation orientée objets (Exercices)}
\usepackage{etoolbox}
\makeatletter
\providecommand{\subtitle}[1]{% add subtitle to \maketitle
  \apptocmd{\@title}{\par {\large #1 \par}}{}{}
}
\makeatother
\subtitle{S1 - Langages et programmation}
\author{}
\date{}

\begin{document}
\maketitle
\lhead{Spécialité NSI} \rhead{Terminale} \chead{} \cfoot{} \lfoot{Lycée \'Emile Duclaux} \rfoot{Page \thepage/\pageref{LastPage}} \renewcommand{\headrulewidth}{0pt} \renewcommand{\footrulewidth}{0pt} \thispagestyle{fancy} \vspace{-2cm}

\ifdefined\Shaded\renewenvironment{Shaded}{\begin{tcolorbox}[boxrule=0pt, breakable, interior hidden, borderline west={3pt}{0pt}{shadecolor}, frame hidden, sharp corners, enhanced]}{\end{tcolorbox}}\fi

\emph{Les exercices précédés du symbole \faIcon{desktop} sont à faire
sur machine, en sauvegardant le fichier si nécessaire.}

\emph{Les exercices précédés du symbole \faIcon{pencil-alt} doivent être
résolus par écrit.}

\hypertarget{fa-desktop-exercice-1}{%
\subsection{\texorpdfstring{\faIcon{desktop} Exercice
1}{ Exercice 1}}\label{fa-desktop-exercice-1}}

On considère la classe suivante :

\begin{Shaded}
\begin{Highlighting}[]
\KeywordTok{class}\NormalTok{ Point:}
    \KeywordTok{def} \FunctionTok{\_\_init\_\_}\NormalTok{(}\VariableTok{self}\NormalTok{, x, y):}
        \VariableTok{self}\NormalTok{.x }\OperatorTok{=}\NormalTok{ x}
        \VariableTok{self}\NormalTok{.y }\OperatorTok{=}\NormalTok{ y}

    \KeywordTok{def}\NormalTok{ deplace(}\VariableTok{self}\NormalTok{, dx, dy):}
        \VariableTok{self}\NormalTok{.x }\OperatorTok{=} \VariableTok{self}\NormalTok{.x }\OperatorTok{+}\NormalTok{ dx}
        \VariableTok{self}\NormalTok{.y }\OperatorTok{=} \VariableTok{self}\NormalTok{.y }\OperatorTok{+}\NormalTok{ dy}

    \KeywordTok{def}\NormalTok{ symetrique(}\VariableTok{self}\NormalTok{):}
        \ControlFlowTok{return}\NormalTok{ Point(}\OperatorTok{{-}}\VariableTok{self}\NormalTok{.x, }\OperatorTok{{-}}\VariableTok{self}\NormalTok{.y)}

    \KeywordTok{def} \FunctionTok{\_\_repr\_\_}\NormalTok{(}\VariableTok{self}\NormalTok{):}
        \ControlFlowTok{return} \SpecialStringTok{f"Point(}\SpecialCharTok{\{}\VariableTok{self}\SpecialCharTok{.}\NormalTok{x}\SpecialCharTok{\}}\SpecialStringTok{, }\SpecialCharTok{\{}\VariableTok{self}\SpecialCharTok{.}\NormalTok{y}\SpecialCharTok{\}}\SpecialStringTok{)"}
\end{Highlighting}
\end{Shaded}

\begin{enumerate}
\def\labelenumi{\arabic{enumi}.}
\item
  Quelle instruction entrer dans la console pour créer le point a
  d'abscisse 2 et d'ordonnée 4 ?
\item
  Quels sont les attributs et les méthodes de cette classe ? Dresser le
  diagramme de classe de cette classe.
\item
  La méthode spéciale \texttt{\_\_repr\_\_} permet de définir comment
  l'objet sera affiché dans la console Python.

  Qu'affichent les instructions suivantes dont la sortie a été effacée ?

\begin{Shaded}
\begin{Highlighting}[]
\OperatorTok{\textgreater{}\textgreater{}\textgreater{}}\NormalTok{ b }\OperatorTok{=}\NormalTok{ Point(}\DecValTok{1}\NormalTok{, }\DecValTok{2}\NormalTok{)}
\OperatorTok{\textgreater{}\textgreater{}\textgreater{}}\NormalTok{ b}
\NormalTok{...}
\OperatorTok{\textgreater{}\textgreater{}\textgreater{}}\NormalTok{ b.deplace(}\DecValTok{3}\NormalTok{, }\DecValTok{5}\NormalTok{)}
\OperatorTok{\textgreater{}\textgreater{}\textgreater{}}\NormalTok{ b}
\NormalTok{...}
\end{Highlighting}
\end{Shaded}
\item
  Définir une méthode \texttt{abscisse} qui renvoie l'abscisse du point.
\item
  Recommencer avec la méthode \texttt{ordonnee}.
\end{enumerate}

\hypertarget{fa-desktop-exercice-2}{%
\subsection{\texorpdfstring{\faIcon{desktop} Exercice
2}{ Exercice 2}}\label{fa-desktop-exercice-2}}

Soit la classe \texttt{Date} définie par le diagramme de classe
(Figure~\ref{fig-classe}).

\begin{figure}[h]

{\centering \includegraphics[width=1.04167in,height=\textheight]{classe_mermaid_exo.png}

}

\caption{\label{fig-classe}Diagramme de classe de la classe Date}

\end{figure}

\begin{enumerate}
\def\labelenumi{\arabic{enumi}.}
\tightlist
\item
  Implémenter cette classe en Python.
\item
  Créer deux dates le 20 janvier 2012 et le 14 février 2022.
\item
  Dans la méthode d'initialisation d'instance de la classe, prévoir un
  dispositif pour éviter les dates impossibles (du genre 32/14/2020).
  Dans ce cas, la création doit provoquer une erreur, chose possible
  grâce à l'instruction \texttt{raise} (documentation à rechercher !).
\item
  Ajouter une méthode \texttt{\_\_repr\_\_} et une méthode
  \texttt{\_\_str\_\_}permettant d'afficher la date sous la forme ``25
  janvier 1989''. Les noms des mois seront définis en tant qu'attribut
  de classe à l'aide d'une liste.
\item
  Ajouter une méthode \texttt{\_\_lt\_\_} qui permet de comparer deux
  dates. L'expression \texttt{d1\ \textless{}\ d2} (\texttt{d1} et
  \texttt{d2} étant deux objets de type \texttt{Date}) doit grâce à
  cette méthode renvoyer \texttt{True} ou \texttt{False} .
\end{enumerate}

\hypertarget{fa-solid-pencil-alt-exercice-3-bac-2022-extrait}{%
\subsection{\texorpdfstring{\faIcon{pencil-alt} Exercice 3 (Bac 2022,
extrait)}{ Exercice 3 (Bac 2022, extrait)}}\label{fa-solid-pencil-alt-exercice-3-bac-2022-extrait}}

Simon souhaite créer en Python le jeu de cartes « la bataille » pour
deux joueurs. Les questions qui suivent demandent de reprogrammer
quelques fonctions du jeu. On rappelle ici les règles du jeu de la
bataille :

\emph{Préparation} :

\begin{itemize}
\tightlist
\item
  Distribuer toutes les cartes aux deux joueurs.
\item
  Les joueurs ne prennent pas connaissance de leurs cartes et les
  laissent en tas face cachée devant eux.
\end{itemize}

\emph{Déroulement} :

\begin{itemize}
\tightlist
\item
  À chaque tour, chaque joueur dévoile la carte du haut de son tas.
\item
  Le joueur qui présente la carte ayant la plus haute valeur emporte les
  deux cartes qu'il place sous son tas.
\item
  Les valeurs des cartes sont : dans l'ordre de la plus forte à la plus
  faible : As, Roi, Dame, Valet, 10, 9, 8, 7, 6, 5, 4, 3 et 2 (la plus
  faible).
\end{itemize}

\emph{Si deux cartes sont de même valeur, il y a ``bataille''.}

\begin{itemize}
\tightlist
\item
  Chaque joueur pose alors une carte face cachée, suivie d'une carte
  face visible sur la carte dévoilée précédemment.
\item
  On recommence l'opération s'il y a de nouveau une bataille sinon, le
  joueur ayant la valeur la plus forte emporte tout le tas.
\end{itemize}

Lorsque l'un des joueurs possède toutes les cartes du jeu, la partie
s'arrête et ce dernier gagne.

Pour cela Simon crée une classe Python \texttt{Carte}. Chaque instance
de la classe a deux attributs : un pour sa valeur et un pour sa couleur.
Il donne au valet la valeur 11, à la dame la valeur 12, au roi la valeur
13 et à l'as la valeur 14. La couleur est une chaîne de caractères:
``trefle'', ``carreau'', ``coeur'' ou ``pique''.

Simon a écrit la classe Python \texttt{Carte} suivante, ayant deux
attributs \texttt{valeur} et \texttt{couleur}, et dont le constructeur
prend deux arguments: \texttt{val} et \texttt{coul}.

\begin{enumerate}
\def\labelenumi{\arabic{enumi}.}
\item
  Recopier et compléter les pointillés des lignes ci-dessous.

\begin{Shaded}
\begin{Highlighting}[]
\KeywordTok{class}\NormalTok{ Carte:}
    \KeywordTok{def} \FunctionTok{\_\_init\_\_}\NormalTok{(}\VariableTok{self}\NormalTok{, val, coul):}
\NormalTok{        ... .valeur }\OperatorTok{=}\NormalTok{ ...}
\NormalTok{        ... . ... }\OperatorTok{=}\NormalTok{ coul}
\end{Highlighting}
\end{Shaded}
\item
  Parmi les propositions ci-dessous quelle instruction permet de créer
  l'objet « 7 de cœur » sous le nom c7 ?

  \begin{itemize}
  \tightlist
  \item
    \texttt{c7.\ init\ (self,\ 7,\ "coeur")}
  \item
    \texttt{c7\ =\ Carte(self,\ 7,\ "coeur")}
  \item
    \texttt{c7\ =\ Carte\ (\ 7,\ "coeur")}
  \item
    \texttt{from\ Carte\ import\ 7,\ "coeur"}
  \end{itemize}
\item
  On souhaite créer le jeu de cartes. Pour cela, on écrit une fonction
  \texttt{initialiser()} :

  \begin{itemize}
  \tightlist
  \item
    sans paramètre
  \item
    qui renvoie une liste de 52 objets de la classe \texttt{Carte}
    représentant les 52 cartes du jeu.
  \end{itemize}

  Voici une proposition de code. Recopier et compléter les lignes
  suivantes pour que la fonction réponde à la demande :

\begin{Shaded}
\begin{Highlighting}[]
\KeywordTok{def}\NormalTok{ initialiser() :}
\NormalTok{    jeu }\OperatorTok{=}\NormalTok{ [] }
    \ControlFlowTok{for}\NormalTok{ c }\KeywordTok{in}\NormalTok{ [}\StringTok{"coeur"}\NormalTok{, }\StringTok{"carreau"}\NormalTok{, }\StringTok{"trefle"}\NormalTok{, }\StringTok{"pique"}\NormalTok{]:}
        \ControlFlowTok{for}\NormalTok{ v }\KeywordTok{in} \BuiltInTok{range}\NormalTok{( ... ) :}
\NormalTok{            carte\_cree }\OperatorTok{=}\NormalTok{ ...}
\NormalTok{            jeu.append(carte\_cree)}
    \ControlFlowTok{return}\NormalTok{ jeu}
\end{Highlighting}
\end{Shaded}
\item
  Écrire une fonction \texttt{comparer(cartel,\ carte2)} qui prend en
  paramètres deux objets de la classe \texttt{Carte}. Cette fonction
  renvoie :

  \begin{itemize}
  \tightlist
  \item
    0 si la force des deux cartes est identique,
  \item
    1 si la carte cartel est strictement plus forte que carte2
  \item
    -1 si la carte carte2 est strictement plus forte que cartel
  \end{itemize}
\end{enumerate}

\href{https://flallemand.fr/notebook/?from=https://flallemand.fr/nsi/assets/notebooks/exo3_POO_CORR.ipynb}{Voir
le corrigé}

\hypertarget{fa-solid-pencil-alt-exercice-4-bac-2022}{%
\subsection{\texorpdfstring{\faIcon{pencil-alt} Exercice 4 (Bac
2022)}{ Exercice 4 (Bac 2022)}}\label{fa-solid-pencil-alt-exercice-4-bac-2022}}

Un fabricant de brioches décide d'informatiser sa gestion des stocks. Il
écrit pour cela un programme en langage Python. Une partie de son
travail consiste à développer une classe Stock dont la première version
est la suivante :

\begin{Shaded}
\begin{Highlighting}[]
\KeywordTok{class}\NormalTok{ Stock:}
    \KeywordTok{def} \FunctionTok{\_\_init\_\_}\NormalTok{(}\VariableTok{self}\NormalTok{):}
        \VariableTok{self}\NormalTok{.qt\_farine }\OperatorTok{=} \DecValTok{0} \CommentTok{\# quantité de farine initialisée à 0 g}
        \VariableTok{self}\NormalTok{.nb\_oeufs }\OperatorTok{=} \DecValTok{0} \CommentTok{\# nombre d’œufs (0 à l’initialisation)}
        \VariableTok{self}\NormalTok{.qt\_beurre }\OperatorTok{=} \DecValTok{0} \CommentTok{\# quantité de beurre initialisée à 0 g}
\end{Highlighting}
\end{Shaded}

\begin{enumerate}
\def\labelenumi{\arabic{enumi}.}
\item
  Écrire une méthode \texttt{ajouter\_beurre(self,\ qt)} qui ajoute la
  quantité \texttt{qt} de beurre à un objet de la classe \texttt{Stock}.

  On admet que l'on a écrit deux autres méthodes
  \texttt{ajouter\_farine} et \texttt{ajouter\_oeufs} qui ont des
  fonctionnements analogues.
\item
  Écrire une méthode \texttt{afficher(self)} qui affiche la quantité de
  farine, d'œufs et de beurre d'un objet de type \texttt{Stock}.
  L'exemple ci-dessous illustre l'exécution de cette méthode dans la
  console :

\begin{Shaded}
\begin{Highlighting}[]
\OperatorTok{\textgreater{}\textgreater{}\textgreater{}}\NormalTok{ mon\_stock }\OperatorTok{=}\NormalTok{ Stock() }
\OperatorTok{\textgreater{}\textgreater{}\textgreater{}}\NormalTok{ mon\_stock.afficher() }
\NormalTok{farine: }\DecValTok{0} 
\NormalTok{oeuf: }\DecValTok{0} 
\NormalTok{beurre: }\DecValTok{0} 
\OperatorTok{\textgreater{}\textgreater{}\textgreater{}}\NormalTok{ mon\_stock.ajouter\_beurre(}\DecValTok{560}\NormalTok{) }
\OperatorTok{\textgreater{}\textgreater{}\textgreater{}}\NormalTok{ mon\_stock.afficher() }
\NormalTok{farine: }\DecValTok{0} 
\NormalTok{oeuf: }\DecValTok{0} 
\NormalTok{beurre: }\DecValTok{560} 
\end{Highlighting}
\end{Shaded}
\item
  Pour faire une brioche, il faut 350 g de farine, 175 g de beurre et 4
  oeufs. Écrire une méthode \texttt{stock\_suffisant\_brioche(self)} qui
  renvoie un booléen : VRAI s'il y a assez d'ingrédients dans le stock
  pour faire une brioche et FAUX sinon.
\item
  On considère la méthode supplémentaire \texttt{produire(self)} de la
  classe \texttt{Stock} donnée par le code suivant :

\begin{Shaded}
\begin{Highlighting}[]
\KeywordTok{def}\NormalTok{ produire(}\VariableTok{self}\NormalTok{):}
\NormalTok{    res }\OperatorTok{=} \DecValTok{0} 
    \ControlFlowTok{while} \VariableTok{self}\NormalTok{.stock\_suffisant\_brioche():}
        \VariableTok{self}\NormalTok{.qt\_beurre }\OperatorTok{=} \VariableTok{self}\NormalTok{.qt\_beurre }\OperatorTok{{-}} \DecValTok{175} 
        \VariableTok{self}\NormalTok{.qt\_farine }\OperatorTok{=} \VariableTok{self}\NormalTok{.qt\_farine }\OperatorTok{{-}} \DecValTok{350} 
        \VariableTok{self}\NormalTok{.nb\_oeufs }\OperatorTok{=} \VariableTok{self}\NormalTok{.nb\_oeufs }\OperatorTok{{-}} \DecValTok{4} 
\NormalTok{        res }\OperatorTok{=}\NormalTok{ res }\OperatorTok{+} \DecValTok{1}
    \ControlFlowTok{return}\NormalTok{ res}
\end{Highlighting}
\end{Shaded}

  On considère un stock défini par les instructions suivantes :

\begin{Shaded}
\begin{Highlighting}[]
\OperatorTok{\textgreater{}\textgreater{}\textgreater{}}\NormalTok{ mon\_stock}\OperatorTok{=}\NormalTok{Stock()}
\OperatorTok{\textgreater{}\textgreater{}\textgreater{}}\NormalTok{ mon\_stock.ajouter\_beurre(}\DecValTok{1000}\NormalTok{) }
\OperatorTok{\textgreater{}\textgreater{}\textgreater{}}\NormalTok{ mon\_stock.ajouter\_farine(}\DecValTok{1000}\NormalTok{) }
\OperatorTok{\textgreater{}\textgreater{}\textgreater{}}\NormalTok{ mon\_stock.ajouter\_oeufs(}\DecValTok{10}\NormalTok{)}
\end{Highlighting}
\end{Shaded}

  \begin{enumerate}
  \def\labelenumii{\arabic{enumii}.}
  \tightlist
  \item
    On exécute ensuite l'instruction :
    \texttt{\textgreater{}\textgreater{}\textgreater{}\ mon\_stock.produire()}.
    Quelle valeur s'affiche dans la console ? Que représente cette
    valeur ?
  \item
    On exécute ensuite l'instruction :
    \texttt{\textgreater{}\textgreater{}\textgreater{}\ mon\_stock.afficher()}.
    Que s'affiche-t-il dans la console ?
  \end{enumerate}
\item
  L'industriel possède n lieux de production distincts et donc n stocks
  distincts.

  On suppose que ces stocks sont dans une liste dont chaque élément est
  un objet de type \texttt{Stock}. Écrire une fonction Python
  \texttt{nb\_brioches(liste\_stocks)} possédant pour unique paramètre
  la liste des stocks et qui renvoie le nombre total de brioches
  produites.
\end{enumerate}

\href{https://flallemand.fr/notebook/?from=https://flallemand.fr/nsi/assets/notebooks/exo4_POO_CORR.ipynb}{Voir
le corrigé}



\end{document}
