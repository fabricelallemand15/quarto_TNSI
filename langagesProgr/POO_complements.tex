% Options for packages loaded elsewhere
\PassOptionsToPackage{unicode}{hyperref}
\PassOptionsToPackage{hyphens}{url}
\PassOptionsToPackage{dvipsnames,svgnames,x11names}{xcolor}
%
\documentclass[
  a4paper,
  DIV=11,
  numbers=noendperiod]{scrartcl}

\usepackage{amsmath,amssymb}
\usepackage{iftex}
\ifPDFTeX
  \usepackage[T1]{fontenc}
  \usepackage[utf8]{inputenc}
  \usepackage{textcomp} % provide euro and other symbols
\else % if luatex or xetex
  \usepackage{unicode-math}
  \defaultfontfeatures{Scale=MatchLowercase}
  \defaultfontfeatures[\rmfamily]{Ligatures=TeX,Scale=1}
\fi
\usepackage{lmodern}
\ifPDFTeX\else  
    % xetex/luatex font selection
\fi
% Use upquote if available, for straight quotes in verbatim environments
\IfFileExists{upquote.sty}{\usepackage{upquote}}{}
\IfFileExists{microtype.sty}{% use microtype if available
  \usepackage[]{microtype}
  \UseMicrotypeSet[protrusion]{basicmath} % disable protrusion for tt fonts
}{}
\makeatletter
\@ifundefined{KOMAClassName}{% if non-KOMA class
  \IfFileExists{parskip.sty}{%
    \usepackage{parskip}
  }{% else
    \setlength{\parindent}{0pt}
    \setlength{\parskip}{6pt plus 2pt minus 1pt}}
}{% if KOMA class
  \KOMAoptions{parskip=half}}
\makeatother
\usepackage{xcolor}
\usepackage[top=20mm,bottom=20mm,left=20mm,right=20mm,heightrounded]{geometry}
\setlength{\emergencystretch}{3em} % prevent overfull lines
\setcounter{secnumdepth}{-\maxdimen} % remove section numbering
% Make \paragraph and \subparagraph free-standing
\ifx\paragraph\undefined\else
  \let\oldparagraph\paragraph
  \renewcommand{\paragraph}[1]{\oldparagraph{#1}\mbox{}}
\fi
\ifx\subparagraph\undefined\else
  \let\oldsubparagraph\subparagraph
  \renewcommand{\subparagraph}[1]{\oldsubparagraph{#1}\mbox{}}
\fi

\usepackage{color}
\usepackage{fancyvrb}
\newcommand{\VerbBar}{|}
\newcommand{\VERB}{\Verb[commandchars=\\\{\}]}
\DefineVerbatimEnvironment{Highlighting}{Verbatim}{commandchars=\\\{\}}
% Add ',fontsize=\small' for more characters per line
\usepackage{framed}
\definecolor{shadecolor}{RGB}{241,243,245}
\newenvironment{Shaded}{\begin{snugshade}}{\end{snugshade}}
\newcommand{\AlertTok}[1]{\textcolor[rgb]{0.68,0.00,0.00}{#1}}
\newcommand{\AnnotationTok}[1]{\textcolor[rgb]{0.37,0.37,0.37}{#1}}
\newcommand{\AttributeTok}[1]{\textcolor[rgb]{0.40,0.45,0.13}{#1}}
\newcommand{\BaseNTok}[1]{\textcolor[rgb]{0.68,0.00,0.00}{#1}}
\newcommand{\BuiltInTok}[1]{\textcolor[rgb]{0.00,0.23,0.31}{#1}}
\newcommand{\CharTok}[1]{\textcolor[rgb]{0.13,0.47,0.30}{#1}}
\newcommand{\CommentTok}[1]{\textcolor[rgb]{0.37,0.37,0.37}{#1}}
\newcommand{\CommentVarTok}[1]{\textcolor[rgb]{0.37,0.37,0.37}{\textit{#1}}}
\newcommand{\ConstantTok}[1]{\textcolor[rgb]{0.56,0.35,0.01}{#1}}
\newcommand{\ControlFlowTok}[1]{\textcolor[rgb]{0.00,0.23,0.31}{#1}}
\newcommand{\DataTypeTok}[1]{\textcolor[rgb]{0.68,0.00,0.00}{#1}}
\newcommand{\DecValTok}[1]{\textcolor[rgb]{0.68,0.00,0.00}{#1}}
\newcommand{\DocumentationTok}[1]{\textcolor[rgb]{0.37,0.37,0.37}{\textit{#1}}}
\newcommand{\ErrorTok}[1]{\textcolor[rgb]{0.68,0.00,0.00}{#1}}
\newcommand{\ExtensionTok}[1]{\textcolor[rgb]{0.00,0.23,0.31}{#1}}
\newcommand{\FloatTok}[1]{\textcolor[rgb]{0.68,0.00,0.00}{#1}}
\newcommand{\FunctionTok}[1]{\textcolor[rgb]{0.28,0.35,0.67}{#1}}
\newcommand{\ImportTok}[1]{\textcolor[rgb]{0.00,0.46,0.62}{#1}}
\newcommand{\InformationTok}[1]{\textcolor[rgb]{0.37,0.37,0.37}{#1}}
\newcommand{\KeywordTok}[1]{\textcolor[rgb]{0.00,0.23,0.31}{#1}}
\newcommand{\NormalTok}[1]{\textcolor[rgb]{0.00,0.23,0.31}{#1}}
\newcommand{\OperatorTok}[1]{\textcolor[rgb]{0.37,0.37,0.37}{#1}}
\newcommand{\OtherTok}[1]{\textcolor[rgb]{0.00,0.23,0.31}{#1}}
\newcommand{\PreprocessorTok}[1]{\textcolor[rgb]{0.68,0.00,0.00}{#1}}
\newcommand{\RegionMarkerTok}[1]{\textcolor[rgb]{0.00,0.23,0.31}{#1}}
\newcommand{\SpecialCharTok}[1]{\textcolor[rgb]{0.37,0.37,0.37}{#1}}
\newcommand{\SpecialStringTok}[1]{\textcolor[rgb]{0.13,0.47,0.30}{#1}}
\newcommand{\StringTok}[1]{\textcolor[rgb]{0.13,0.47,0.30}{#1}}
\newcommand{\VariableTok}[1]{\textcolor[rgb]{0.07,0.07,0.07}{#1}}
\newcommand{\VerbatimStringTok}[1]{\textcolor[rgb]{0.13,0.47,0.30}{#1}}
\newcommand{\WarningTok}[1]{\textcolor[rgb]{0.37,0.37,0.37}{\textit{#1}}}

\providecommand{\tightlist}{%
  \setlength{\itemsep}{0pt}\setlength{\parskip}{0pt}}\usepackage{longtable,booktabs,array}
\usepackage{calc} % for calculating minipage widths
% Correct order of tables after \paragraph or \subparagraph
\usepackage{etoolbox}
\makeatletter
\patchcmd\longtable{\par}{\if@noskipsec\mbox{}\fi\par}{}{}
\makeatother
% Allow footnotes in longtable head/foot
\IfFileExists{footnotehyper.sty}{\usepackage{footnotehyper}}{\usepackage{footnote}}
\makesavenoteenv{longtable}
\usepackage{graphicx}
\makeatletter
\def\maxwidth{\ifdim\Gin@nat@width>\linewidth\linewidth\else\Gin@nat@width\fi}
\def\maxheight{\ifdim\Gin@nat@height>\textheight\textheight\else\Gin@nat@height\fi}
\makeatother
% Scale images if necessary, so that they will not overflow the page
% margins by default, and it is still possible to overwrite the defaults
% using explicit options in \includegraphics[width, height, ...]{}
\setkeys{Gin}{width=\maxwidth,height=\maxheight,keepaspectratio}
% Set default figure placement to htbp
\makeatletter
\def\fps@figure{htbp}
\makeatother

\usepackage{fancyhdr} \pagestyle{fancy} \usepackage{lastpage}
\KOMAoption{captions}{tablesignature}
\makeatletter
\@ifpackageloaded{tcolorbox}{}{\usepackage[skins,breakable]{tcolorbox}}
\@ifpackageloaded{fontawesome5}{}{\usepackage{fontawesome5}}
\definecolor{quarto-callout-color}{HTML}{909090}
\definecolor{quarto-callout-note-color}{HTML}{0758E5}
\definecolor{quarto-callout-important-color}{HTML}{CC1914}
\definecolor{quarto-callout-warning-color}{HTML}{EB9113}
\definecolor{quarto-callout-tip-color}{HTML}{00A047}
\definecolor{quarto-callout-caution-color}{HTML}{FC5300}
\definecolor{quarto-callout-color-frame}{HTML}{acacac}
\definecolor{quarto-callout-note-color-frame}{HTML}{4582ec}
\definecolor{quarto-callout-important-color-frame}{HTML}{d9534f}
\definecolor{quarto-callout-warning-color-frame}{HTML}{f0ad4e}
\definecolor{quarto-callout-tip-color-frame}{HTML}{02b875}
\definecolor{quarto-callout-caution-color-frame}{HTML}{fd7e14}
\makeatother
\makeatletter
\makeatother
\makeatletter
\makeatother
\makeatletter
\@ifpackageloaded{caption}{}{\usepackage{caption}}
\AtBeginDocument{%
\ifdefined\contentsname
  \renewcommand*\contentsname{Table des matières}
\else
  \newcommand\contentsname{Table des matières}
\fi
\ifdefined\listfigurename
  \renewcommand*\listfigurename{Liste des Figures}
\else
  \newcommand\listfigurename{Liste des Figures}
\fi
\ifdefined\listtablename
  \renewcommand*\listtablename{Liste des Tables}
\else
  \newcommand\listtablename{Liste des Tables}
\fi
\ifdefined\figurename
  \renewcommand*\figurename{Figure}
\else
  \newcommand\figurename{Figure}
\fi
\ifdefined\tablename
  \renewcommand*\tablename{Tableau}
\else
  \newcommand\tablename{Tableau}
\fi
}
\@ifpackageloaded{float}{}{\usepackage{float}}
\floatstyle{ruled}
\@ifundefined{c@chapter}{\newfloat{codelisting}{h}{lop}}{\newfloat{codelisting}{h}{lop}[chapter]}
\floatname{codelisting}{Listing}
\newcommand*\listoflistings{\listof{codelisting}{Liste des Listings}}
\makeatother
\makeatletter
\@ifpackageloaded{caption}{}{\usepackage{caption}}
\@ifpackageloaded{subcaption}{}{\usepackage{subcaption}}
\makeatother
\makeatletter
\@ifpackageloaded{tcolorbox}{}{\usepackage[skins,breakable]{tcolorbox}}
\makeatother
\makeatletter
\@ifundefined{shadecolor}{\definecolor{shadecolor}{rgb}{.97, .97, .97}}
\makeatother
\makeatletter
\makeatother
\makeatletter
\makeatother
\ifLuaTeX
\usepackage[bidi=basic]{babel}
\else
\usepackage[bidi=default]{babel}
\fi
\babelprovide[main,import]{french}
% get rid of language-specific shorthands (see #6817):
\let\LanguageShortHands\languageshorthands
\def\languageshorthands#1{}
\ifLuaTeX
  \usepackage{selnolig}  % disable illegal ligatures
\fi
\IfFileExists{bookmark.sty}{\usepackage{bookmark}}{\usepackage{hyperref}}
\IfFileExists{xurl.sty}{\usepackage{xurl}}{} % add URL line breaks if available
\urlstyle{same} % disable monospaced font for URLs
\hypersetup{
  pdftitle={Programmation orientée objets (Compléments)},
  pdflang={fr},
  colorlinks=true,
  linkcolor={blue},
  filecolor={Maroon},
  citecolor={Blue},
  urlcolor={Blue},
  pdfcreator={LaTeX via pandoc}}

\title{Programmation orientée objets (Compléments)}
\usepackage{etoolbox}
\makeatletter
\providecommand{\subtitle}[1]{% add subtitle to \maketitle
  \apptocmd{\@title}{\par {\large #1 \par}}{}{}
}
\makeatother
\subtitle{S1 - Langages et programmation}
\author{}
\date{}

\begin{document}
\maketitle
\lhead{Spécialité NSI} \rhead{Terminale} \chead{} \cfoot{} \lfoot{Lycée \'Emile Duclaux} \rfoot{Page \thepage/\pageref{LastPage}} \renewcommand{\headrulewidth}{0pt} \renewcommand{\footrulewidth}{0pt} \thispagestyle{fancy} \vspace{-2cm}

\ifdefined\Shaded\renewenvironment{Shaded}{\begin{tcolorbox}[enhanced, sharp corners, frame hidden, boxrule=0pt, breakable, interior hidden, borderline west={3pt}{0pt}{shadecolor}]}{\end{tcolorbox}}\fi

\begin{tcolorbox}[enhanced jigsaw, left=2mm, colback=white, coltitle=black, rightrule=.15mm, colbacktitle=quarto-callout-warning-color!10!white, bottomtitle=1mm, title=\textcolor{quarto-callout-warning-color}{\faExclamationTriangle}\hspace{0.5em}{Avertissement}, arc=.35mm, colframe=quarto-callout-warning-color-frame, bottomrule=.15mm, breakable, toprule=.15mm, opacityback=0, titlerule=0mm, toptitle=1mm, leftrule=.75mm, opacitybacktitle=0.6]

Les compléments présentés ici sont hors programme. Ils peuvent néanmoins
apporter une connaissance et une compréhension plus fine de la POO et
être utiles dans le cadre du travail sur les projets.

\end{tcolorbox}

\hypertarget{principes-et-duxe9finitions}{%
\subsection{Principes et
définitions}\label{principes-et-duxe9finitions}}

\begin{description}
\tightlist
\item[\textbf{Objet}]
Un \textbf{objet} est une donnée manipulable par un programme : il
s'agit d'un conteneur pour une \emph{valeur} ou un \emph{état} auquel
est associé un \emph{ensemble d'opérations}. Cet objet est associé à un
\textbf{type}, défini comme l'ensemble des valeurs possibles, cette
liste d'opérations, ainsi que leur codage (binaire).
\end{description}

Un objet est identifié dans un programme par un \emph{nom} ou une
\emph{notation littérale}, mais peut parfois être \emph{anonyme} (comme
les variables temporaires ou les composantes d'un tableau).

Et pour une définition d'un langage orienté objet, l'idée première que
l'on retrouve dans la définition de wikipédia offre un cadre intéressant
: un langage objet doit permettre l'analyse et le développement logiciel
fondés sur des \emph{relations entre objets}.

Concrètement, un objet est une structure de données qui répond à un
ensemble de messages. Cette structure de données définit son état tandis
que l'ensemble des messages qu'il comprend décrit son comportement :

\begin{itemize}
\tightlist
\item
  les données, ou champs, qui décrivent sa structure interne sont
  appelées ses \textbf{attributs} ;
\item
  l'ensemble des messages forme ce que l'on appelle l'interface de
  l'objet ; c'est seulement au travers de celle-ci que les objets
  interagissent entre eux. La réponse à la réception d'un message par un
  objet est appelée une \textbf{méthode} (méthode de mise en œuvre du
  message) ; elle décrit quelle réponse doit être donnée au message.
\end{itemize}

Les attributs et les méthodes constituent les \textbf{membres} d'un
objet. Un objet possède un \textbf{type}.

En Python, un objet la création d'un objet se fait en utilisant une
\textbf{classe} : un objet est alors une instance de sa classe. La
\textbf{classe} est un \emph{type}, un ensemble d'objets partageant les
mêmes propriétés concrétisées par une liste de membres.

\begin{description}
\tightlist
\item[\textbf{Langage orienté objet}]
Un \textbf{langage orienté objet} est un langage de programmation qui
comporte de manière native les éléments suivants :
l'\emph{encapsulation}, l'\emph{héritage}, le \emph{polymorphisme} et la
\emph{programmation générique}.
\end{description}

\hypertarget{les-principes-cluxe9s-de-la-poo}{%
\subsection{Les principes clés de la
POO}\label{les-principes-cluxe9s-de-la-poo}}

\hypertarget{lencapsulation}{%
\subsubsection{L'encapsulation}\label{lencapsulation}}

Certains membres (ou plus exactement leur représentation informatique)
sont cachés : c'est le principe
d'\textbf{\href{https://fr.wikipedia.org/wiki/Encapsulation_(programmation)}{encapsulation}}.
Ainsi, le programme peut modifier la structure interne des objets ou
leurs méthodes associées sans avoir d'impact sur les utilisateurs de
l'objet. C'est un des principes fondamentaux notamment pour la
robustesse du code.

En particulier, les bonnes pratiques de POO recommandent de na pas
permettre un accès direct aux attributs d'un objet à l'extérieur de
celui-ci. On appelle \textbf{interface} d'un objet l'ensemble de ses
membres qui sont accessibles à l'extérieur de celui-ci. L'interface ne
devrait donc contenir que des méthodes. Pas forcément toutes, certaines
méthodes (comme \texttt{\_\_init\_\_}) restent privées.

Contrairement à d'autres langages, Python offre une totale liberté de
modification sur les membres d'un objet. C'est au programmeur de rester
vigilant. Il existe néanmoins des conventions permettant d'identifier
les membres de l'interface des autres membres d'un objet.

\begin{tcolorbox}[enhanced jigsaw, left=2mm, colback=white, coltitle=black, rightrule=.15mm, colbacktitle=quarto-callout-important-color!10!white, bottomtitle=1mm, title=\textcolor{quarto-callout-important-color}{\faExclamation}\hspace{0.5em}{Conventions de nommage en Python}, arc=.35mm, colframe=quarto-callout-important-color-frame, bottomrule=.15mm, breakable, toprule=.15mm, opacityback=0, titlerule=0mm, toptitle=1mm, leftrule=.75mm, opacitybacktitle=0.6]

\begin{itemize}
\tightlist
\item
  un nom d'attribut commençant par un double underscore \texttt{\_\_}
  désigne un attribut privé.
\item
  une méthode dont le nom est de la forme \texttt{\_\_nom\_\_} désigne
  une méthode privée.
\end{itemize}

\end{tcolorbox}

Mais alors si les attributs doivent rester privés, comment y accéder, et
comment les modifier ?

Il convient pour cela, en toute rigueur, de définir des méthodes ad-hoc
: une méthode qui permet d'accéder à un attribut est un \texttt{getter},
une période qui permet de changer la valeur d'un attribut est un
\texttt{setter}.

Voici par exemple une nouvelle définition de la classe ``Rectangle''
tenant compte des remarques précédentes.

\begin{Shaded}
\begin{Highlighting}[]
\KeywordTok{class}\NormalTok{ Rectangle:}
    \CommentTok{"""Représente un rectangle"""}

    \KeywordTok{def} \FunctionTok{\_\_init\_\_}\NormalTok{(}\VariableTok{self}\NormalTok{, largeur}\OperatorTok{=}\DecValTok{2}\NormalTok{, hauteur}\OperatorTok{=}\DecValTok{3}\NormalTok{):}
        \VariableTok{self}\NormalTok{.\_\_largeur }\OperatorTok{=}\NormalTok{ largeur}
        \VariableTok{self}\NormalTok{.\_\_hauteur }\OperatorTok{=}\NormalTok{ hauteur}

    \KeywordTok{def}\NormalTok{ get\_largeur(}\VariableTok{self}\NormalTok{):}
        \ControlFlowTok{return} \VariableTok{self}\NormalTok{.\_\_largeur}

    \KeywordTok{def}\NormalTok{ set\_largeur(}\VariableTok{self}\NormalTok{, largeur):}
        \VariableTok{self}\NormalTok{.\_\_largeur }\OperatorTok{=}\NormalTok{ largeur}

    \KeywordTok{def}\NormalTok{ get\_hauteur(}\VariableTok{self}\NormalTok{):}
        \ControlFlowTok{return} \VariableTok{self}\NormalTok{.\_\_hauteur}

    \KeywordTok{def}\NormalTok{ set\_hauteur(}\VariableTok{self}\NormalTok{, hauteur):}
        \VariableTok{self}\NormalTok{.\_\_hauteur }\OperatorTok{=}\NormalTok{ hauteur}

    \KeywordTok{def}\NormalTok{ perimetre(}\VariableTok{self}\NormalTok{):}
        \CommentTok{"""Retourne le périmètre"""}
        \ControlFlowTok{return} \DecValTok{2} \OperatorTok{*}\NormalTok{ (}\VariableTok{self}\NormalTok{.\_\_largeur }\OperatorTok{+} \VariableTok{self}\NormalTok{.\_\_hauteur)}

    \KeywordTok{def}\NormalTok{ aire(}\VariableTok{self}\NormalTok{):}
        \CommentTok{"""Retourne l\textquotesingle{}aire"""}
        \ControlFlowTok{return} \VariableTok{self}\NormalTok{.\_\_largeur }\OperatorTok{*} \VariableTok{self}\NormalTok{.\_\_hauteur}
\end{Highlighting}
\end{Shaded}

Utilisation :

\begin{Shaded}
\begin{Highlighting}[]
\OperatorTok{\textgreater{}\textgreater{}\textgreater{}}\NormalTok{ rec }\OperatorTok{=}\NormalTok{ Rectangle(}\DecValTok{10}\NormalTok{, }\DecValTok{5}\NormalTok{)}
\OperatorTok{\textgreater{}\textgreater{}\textgreater{}}\NormalTok{ rec.\_\_hauteur}
\PreprocessorTok{AttributeError}\NormalTok{: }\StringTok{\textquotesingle{}Rectangle\textquotesingle{}} \BuiltInTok{object}\NormalTok{ has no attribute }\StringTok{\textquotesingle{}\_\_hauteur\textquotesingle{}}
\OperatorTok{\textgreater{}\textgreater{}\textgreater{}}\NormalTok{ rec.get\_hauteur()}
\DecValTok{5}
\end{Highlighting}
\end{Shaded}

Nous voyons que l'accès direct à l'attribut n'est plus possible.

Cela n'est pas très pratique et change nos habitudes : nous aimerions en
effet pouvoir accéder à la valeur d'un attribut en utilisant la notation
pointée. Deux remarques à ces objections. D'une part, ces règles de
programmation ne sont pas là pour nous embêter ! Il s'agit de sécuriser
notre code : la définition d'un \texttt{setter} par exemple, peut
permettre de vérifier la validité des arguments entrés et afficher un
message d'erreur si besoin (par exemple si on appelle
\texttt{set\_hauteur(-10}). Deuxième remarque : Python propose une
fonctionnalité avancée, appelée \textbf{décorateurs} et qui permet de
retrouver, en apparence, l'accès direct aux attributs. Voici une
nouvelle version de la classe ``Rectangle'' avec l'utilisation du
décorateur \texttt{@property} et la redéfinition des \texttt{getter} et
\texttt{setter} (qui doivent maintenant porter le même nom que le pseudo
argument). On a introduit dans les \texttt{setter} des tests de validité
des données.

\begin{Shaded}
\begin{Highlighting}[]
\KeywordTok{class}\NormalTok{ Rectangle:}
    \CommentTok{"""Représente un rectangle"""}

    \KeywordTok{def} \FunctionTok{\_\_init\_\_}\NormalTok{(}\VariableTok{self}\NormalTok{, largeur}\OperatorTok{=}\DecValTok{2}\NormalTok{, hauteur}\OperatorTok{=}\DecValTok{3}\NormalTok{):}
        \VariableTok{self}\NormalTok{.\_\_largeur }\OperatorTok{=}\NormalTok{ largeur}
        \VariableTok{self}\NormalTok{.\_\_hauteur }\OperatorTok{=}\NormalTok{ hauteur}

    \AttributeTok{@property}
    \KeywordTok{def}\NormalTok{ largeur(}\VariableTok{self}\NormalTok{):}
        \ControlFlowTok{return} \VariableTok{self}\NormalTok{.\_\_largeur}

    \AttributeTok{@largeur.setter}
    \KeywordTok{def}\NormalTok{ largeur(}\VariableTok{self}\NormalTok{, largeur):}
        \ControlFlowTok{if} \BuiltInTok{isinstance}\NormalTok{(largeur, (}\BuiltInTok{int}\NormalTok{, }\BuiltInTok{float}\NormalTok{)) }\KeywordTok{and}\NormalTok{ largeur }\OperatorTok{\textgreater{}=} \DecValTok{0}\NormalTok{:}
            \VariableTok{self}\NormalTok{.\_\_largeur }\OperatorTok{=}\NormalTok{ largeur}
        \ControlFlowTok{else}\NormalTok{:}
            \BuiltInTok{print}\NormalTok{(}\StringTok{"Argument invalide, largeur inchangée !"}\NormalTok{)}

    \AttributeTok{@property}
    \KeywordTok{def}\NormalTok{ hauteur(}\VariableTok{self}\NormalTok{):}
        \ControlFlowTok{return} \VariableTok{self}\NormalTok{.\_\_hauteur}

    \AttributeTok{@hauteur.setter}
    \KeywordTok{def}\NormalTok{ hauteur(}\VariableTok{self}\NormalTok{, hauteur):}
        \ControlFlowTok{if} \BuiltInTok{isinstance}\NormalTok{(hauteur, (}\BuiltInTok{int}\NormalTok{, }\BuiltInTok{float}\NormalTok{)) }\KeywordTok{and}\NormalTok{ hauteur }\OperatorTok{\textgreater{}=} \DecValTok{0}\NormalTok{:}
            \VariableTok{self}\NormalTok{.\_\_hauteur }\OperatorTok{=}\NormalTok{ hauteur}
        \ControlFlowTok{else}\NormalTok{:}
            \BuiltInTok{print}\NormalTok{(}\StringTok{"Argument invalide, hauteur inchangée !"}\NormalTok{)}

    \KeywordTok{def}\NormalTok{ perimetre(}\VariableTok{self}\NormalTok{):}
        \CommentTok{"""Retourne le périmètre"""}
        \ControlFlowTok{return} \DecValTok{2} \OperatorTok{*}\NormalTok{ (}\VariableTok{self}\NormalTok{.\_\_largeur }\OperatorTok{+} \VariableTok{self}\NormalTok{.\_\_hauteur)}

    \KeywordTok{def}\NormalTok{ aire(}\VariableTok{self}\NormalTok{):}
        \CommentTok{"""Retourne l\textquotesingle{}aire"""}
        \ControlFlowTok{return} \VariableTok{self}\NormalTok{.\_\_largeur }\OperatorTok{*} \VariableTok{self}\NormalTok{.\_\_hauteur}


\NormalTok{rec }\OperatorTok{=}\NormalTok{ Rectangle(}\DecValTok{10}\NormalTok{, }\DecValTok{25}\NormalTok{)}
\BuiltInTok{print}\NormalTok{(rec.largeur)}
\NormalTok{rec.largeur }\OperatorTok{=} \OperatorTok{{-}}\DecValTok{15}
\BuiltInTok{print}\NormalTok{(rec.largeur)}
\end{Highlighting}
\end{Shaded}

Sortie en console :

\begin{Shaded}
\begin{Highlighting}[]
\DecValTok{10}
\NormalTok{Argument invalide, largeur inchangée }\OperatorTok{!}
\DecValTok{10}
\end{Highlighting}
\end{Shaded}

\hypertarget{lhuxe9ritage}{%
\subsubsection{L'héritage}\label{lhuxe9ritage}}

L'\textbf{héritage} est une relation asymétrique entre deux classes :
l'une est la \textbf{classe mère} (aussi nommée classe parente,
superclasse, classe de base), l'autre la \textbf{classe-fille}.
L'héritage permet une économie d'écriture par la réutilisation
automatique, lors de la définition de la classe-fille, de tous les
membres et autres éléments définis dans la classe mère. Ainsi, les
objets de la classe-fille \emph{héritent de toutes les propriétés} de
leur classe mère.

Par exemple, nous pouvons définir une classe \texttt{carre}, fille de la
classe \texttt{Rectangle}. Les attributs et les méthodes définis pour la
classe \texttt{Rectangle} existent alors automatiquement aussi pour la
classe \texttt{carre}.

Voici la syntaxe Python pour définir une classe fille :

\begin{Shaded}
\begin{Highlighting}[]
\KeywordTok{class}\NormalTok{ Carre(Rectangle):}

    \KeywordTok{def} \FunctionTok{\_\_init\_\_}\NormalTok{(}\VariableTok{self}\NormalTok{, cote}\OperatorTok{=}\DecValTok{2}\NormalTok{):}
\NormalTok{        Rectangle.}\FunctionTok{\_\_init\_\_}\NormalTok{(}\VariableTok{self}\NormalTok{, cote, cote)}
\end{Highlighting}
\end{Shaded}

Utilisation :

\begin{Shaded}
\begin{Highlighting}[]
\OperatorTok{\textgreater{}\textgreater{}\textgreater{}}\NormalTok{ car }\OperatorTok{=}\NormalTok{ Carre(}\DecValTok{5}\NormalTok{)}
\OperatorTok{\textgreater{}\textgreater{}\textgreater{}}\NormalTok{ car.perimetre()}
\DecValTok{20}
\end{Highlighting}
\end{Shaded}

La méthode \texttt{perimetre} est héritée de la classe mère
\texttt{Rectangle}.

\begin{tcolorbox}[enhanced jigsaw, left=2mm, colback=white, coltitle=black, rightrule=.15mm, colbacktitle=quarto-callout-warning-color!10!white, bottomtitle=1mm, title=\textcolor{quarto-callout-warning-color}{\faExclamationTriangle}\hspace{0.5em}{Héritage et initialiseur}, arc=.35mm, colframe=quarto-callout-warning-color-frame, bottomrule=.15mm, breakable, toprule=.15mm, opacityback=0, titlerule=0mm, toptitle=1mm, leftrule=.75mm, opacitybacktitle=0.6]

La méthode initialiseur de la classe \texttt{Carre} fait appel à la
méthode initialiseur de sa classe parente par la commande
\texttt{Rectangle.\_\_init\_\_(self,\ cote,\ cote)}. Cet appel est
nécessaire afin que les membres de la classe \texttt{Carre} soient
définis de la même manière que les membres de la classe
\texttt{Rectangle}. La méthode \texttt{\_\_init\_\_} est un initialiseur
d'instance : elle n'est pas invoquée automatiquement lorsqu'on instancie
des objets d'une classe fille.

\end{tcolorbox}

\hypertarget{le-polymorphisme-et-la-reduxe9finition}{%
\subsubsection{Le polymorphisme et la
redéfinition}\label{le-polymorphisme-et-la-reduxe9finition}}

La \textbf{redéfinition des méthodes} permet à un objet de raffiner une
méthode définie avec la même en-tête dans la classe mère. Une même
méthode pourra ainsi avoir un comportement différent selon qu'elle
s'applique à la classe mère ou à la classe fille : on parle de
\textbf{polymorphisme d'héritage}.

Par exemple, nous pouvons redéfinir la méthode \texttt{aire} de la
classe \texttt{Carre} comme ci-dessous : appliquée à un objet
\texttt{Carre}, la nouvelle définition sera utilisée à la place de la
méthode héritée.

\begin{Shaded}
\begin{Highlighting}[]
\KeywordTok{class}\NormalTok{ Carre(Rectangle):}

    \KeywordTok{def} \FunctionTok{\_\_init\_\_}\NormalTok{(}\VariableTok{self}\NormalTok{, cote}\OperatorTok{=}\DecValTok{2}\NormalTok{):}
        \VariableTok{self}\NormalTok{.\_\_largeur }\OperatorTok{=}\NormalTok{ cote}
        \VariableTok{self}\NormalTok{.\_\_hauteur }\OperatorTok{=}\NormalTok{ cote}
    
    \KeywordTok{def}\NormalTok{ aire(}\VariableTok{self}\NormalTok{):}
        \CommentTok{"""Retourne l\textquotesingle{}aire"""}
        \ControlFlowTok{return} \VariableTok{self}\NormalTok{.\_\_largeur }\OperatorTok{**} \DecValTok{2}
\end{Highlighting}
\end{Shaded}

\hypertarget{les-muxe9thodes-spuxe9ciales}{%
\subsection{Les méthodes spéciales}\label{les-muxe9thodes-spuxe9ciales}}

Un bon exemple de polymorphisme est fourni par la redéfinition des
méthodes spéciales.

Nous savons que la fonction \texttt{dir()} renvoie tous les membres d'un
objet.

Appliquons cette commande à notre objet \texttt{rec}, instance de la
classe \texttt{Rectangle} :

\begin{Shaded}
\begin{Highlighting}[]
\OperatorTok{\textgreater{}\textgreater{}\textgreater{}} \BuiltInTok{dir}\NormalTok{(rec)}
\NormalTok{[}\StringTok{\textquotesingle{}\_\_class\_\_\textquotesingle{}}\NormalTok{,}
 \StringTok{\textquotesingle{}\_\_delattr\_\_\textquotesingle{}}\NormalTok{,}
 \StringTok{\textquotesingle{}\_\_dict\_\_\textquotesingle{}}\NormalTok{,}
 \StringTok{\textquotesingle{}\_\_dir\_\_\textquotesingle{}}\NormalTok{,}
 \StringTok{\textquotesingle{}\_\_doc\_\_\textquotesingle{}}\NormalTok{,}
 \StringTok{\textquotesingle{}\_\_eq\_\_\textquotesingle{}}\NormalTok{,}
 \StringTok{\textquotesingle{}\_\_format\_\_\textquotesingle{}}\NormalTok{,}
 \StringTok{\textquotesingle{}\_\_ge\_\_\textquotesingle{}}\NormalTok{,}
 \StringTok{\textquotesingle{}\_\_getattribute\_\_\textquotesingle{}}\NormalTok{,}
 \StringTok{\textquotesingle{}\_\_gt\_\_\textquotesingle{}}\NormalTok{,}
 \StringTok{\textquotesingle{}\_\_hash\_\_\textquotesingle{}}\NormalTok{,}
 \StringTok{\textquotesingle{}\_\_init\_\_\textquotesingle{}}\NormalTok{,}
 \StringTok{\textquotesingle{}\_\_init\_subclass\_\_\textquotesingle{}}\NormalTok{,}
 \StringTok{\textquotesingle{}\_\_le\_\_\textquotesingle{}}\NormalTok{,}
 \StringTok{\textquotesingle{}\_\_lt\_\_\textquotesingle{}}\NormalTok{,}
 \StringTok{\textquotesingle{}\_\_module\_\_\textquotesingle{}}\NormalTok{,}
 \StringTok{\textquotesingle{}\_\_ne\_\_\textquotesingle{}}\NormalTok{,}
 \StringTok{\textquotesingle{}\_\_new\_\_\textquotesingle{}}\NormalTok{,}
 \StringTok{\textquotesingle{}\_\_reduce\_\_\textquotesingle{}}\NormalTok{,}
 \StringTok{\textquotesingle{}\_\_reduce\_ex\_\_\textquotesingle{}}\NormalTok{,}
 \StringTok{\textquotesingle{}\_\_repr\_\_\textquotesingle{}}\NormalTok{,}
 \StringTok{\textquotesingle{}\_\_setattr\_\_\textquotesingle{}}\NormalTok{,}
 \StringTok{\textquotesingle{}\_\_sizeof\_\_\textquotesingle{}}\NormalTok{,}
 \StringTok{\textquotesingle{}\_\_str\_\_\textquotesingle{}}\NormalTok{,}
 \StringTok{\textquotesingle{}\_\_subclasshook\_\_\textquotesingle{}}\NormalTok{,}
 \StringTok{\textquotesingle{}\_\_weakref\_\_\textquotesingle{}}\NormalTok{,}
 \StringTok{\textquotesingle{}\_hauteur\textquotesingle{}}\NormalTok{,}
 \StringTok{\textquotesingle{}\_largeur\textquotesingle{}}\NormalTok{,}
 \StringTok{\textquotesingle{}aire\textquotesingle{}}\NormalTok{,}
 \StringTok{\textquotesingle{}hauteur\textquotesingle{}}\NormalTok{,}
 \StringTok{\textquotesingle{}largeur\textquotesingle{}}\NormalTok{,}
 \StringTok{\textquotesingle{}perimetre\textquotesingle{}}\NormalTok{]}
\end{Highlighting}
\end{Shaded}

Nous reconnaissons en fin de liste les attributs et méthodes que nous
avons définis, mais nous découvrons l'existence d'un grand nombre de
\textbf{méthodes spéciales} privées (puisque leur nom est entouré de
\texttt{\_\_}) qui sont en fait \textbf{héritées} d'une classe
\texttt{Object} parente de toutes les classes. Parmi celles-ci, nous
avons déjà rencontré \texttt{\_\_init\_\_}, la méthode initialiseur.

Les curieux pourront rechercher le rôle de chacune de ces méthodes
spéciales. Le voici pour certaines d'entre elles :

\begin{longtable}[]{@{}ll@{}}
\toprule\noalign{}
Méthode spéciale & Usage \\
\midrule\noalign{}
\endhead
\bottomrule\noalign{}
\endlastfoot
\textbf{add} & + \\
\textbf{mul} & * \\
\textbf{sub} & - \\
\textbf{eq} & == \\
\textbf{ne} & != \\
\textbf{lt} & \textless{} \\
\textbf{ge} & \textless= \\
\textbf{gt} & \textgreater{} \\
\textbf{ge} & \textgreater= \\
\textbf{repr} & affichage dans la console
\textgreater\textgreater\textgreater{} obj \\
\textbf{str} & str(obj), print(obj) \\
\end{longtable}

La redéfinition de la méthode \texttt{\_\_add\_\_} permettrait par
exemple de donner un sens à l'utilisation du symbole \texttt{+} entre
deux objets (instruction du type \texttt{rec1\ +\ rec\ 2}).

Dans notre exemple, nous allons redéfinir la méthode
\texttt{\_\_str\_\_} pour spécifier ce qui doit s'afficher quand
l'instruction \texttt{print(rec)} est exécutée.

Pour l'instant, on obtient :

\begin{Shaded}
\begin{Highlighting}[]
\OperatorTok{\textgreater{}\textgreater{}\textgreater{}} \BuiltInTok{print}\NormalTok{(rec)}
\OperatorTok{\textless{}}\NormalTok{\_\_main\_\_.Rectangle }\BuiltInTok{object}\NormalTok{ at }\BaseNTok{0x000002386735C730}\OperatorTok{\textgreater{}}
\end{Highlighting}
\end{Shaded}

Ajoutons la méthode ci-dessous \textbf{dans la classe
\texttt{Rectangle}} :

\begin{Shaded}
\begin{Highlighting}[]
\KeywordTok{def} \FunctionTok{\_\_str\_\_}\NormalTok{(}\VariableTok{self}\NormalTok{):}
    \ControlFlowTok{return} \SpecialStringTok{f"Rectangle de largeur }\SpecialCharTok{\{}\VariableTok{self}\SpecialCharTok{.}\NormalTok{\_\_largeur}\SpecialCharTok{\}}\SpecialStringTok{ et de hauteur }\SpecialCharTok{\{}\VariableTok{self}\SpecialCharTok{.}\NormalTok{\_\_hauteur}\SpecialCharTok{\}}\SpecialStringTok{."}
\end{Highlighting}
\end{Shaded}

On obtient maintenant :

\begin{Shaded}
\begin{Highlighting}[]
\OperatorTok{\textgreater{}\textgreater{}\textgreater{}} \BuiltInTok{print}\NormalTok{(rec)}
\NormalTok{Rectangle de largeur }\DecValTok{10}\NormalTok{ et de hauteur }\FloatTok{25.}
\end{Highlighting}
\end{Shaded}




\end{document}
