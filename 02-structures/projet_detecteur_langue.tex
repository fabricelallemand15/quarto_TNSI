% Options for packages loaded elsewhere
\PassOptionsToPackage{unicode}{hyperref}
\PassOptionsToPackage{hyphens}{url}
\PassOptionsToPackage{dvipsnames,svgnames,x11names}{xcolor}
%
\documentclass[
  letterpaper,
  DIV=11,
  numbers=noendperiod]{scrartcl}

\usepackage{amsmath,amssymb}
\usepackage{iftex}
\ifPDFTeX
  \usepackage[T1]{fontenc}
  \usepackage[utf8]{inputenc}
  \usepackage{textcomp} % provide euro and other symbols
\else % if luatex or xetex
  \usepackage{unicode-math}
  \defaultfontfeatures{Scale=MatchLowercase}
  \defaultfontfeatures[\rmfamily]{Ligatures=TeX,Scale=1}
\fi
\usepackage{lmodern}
\ifPDFTeX\else  
    % xetex/luatex font selection
\fi
% Use upquote if available, for straight quotes in verbatim environments
\IfFileExists{upquote.sty}{\usepackage{upquote}}{}
\IfFileExists{microtype.sty}{% use microtype if available
  \usepackage[]{microtype}
  \UseMicrotypeSet[protrusion]{basicmath} % disable protrusion for tt fonts
}{}
\makeatletter
\@ifundefined{KOMAClassName}{% if non-KOMA class
  \IfFileExists{parskip.sty}{%
    \usepackage{parskip}
  }{% else
    \setlength{\parindent}{0pt}
    \setlength{\parskip}{6pt plus 2pt minus 1pt}}
}{% if KOMA class
  \KOMAoptions{parskip=half}}
\makeatother
\usepackage{xcolor}
\usepackage[top=20mm,bottom=20mm,left=20mm,right=20mm,heightrounded]{geometry}
\setlength{\emergencystretch}{3em} % prevent overfull lines
\setcounter{secnumdepth}{-\maxdimen} % remove section numbering
% Make \paragraph and \subparagraph free-standing
\ifx\paragraph\undefined\else
  \let\oldparagraph\paragraph
  \renewcommand{\paragraph}[1]{\oldparagraph{#1}\mbox{}}
\fi
\ifx\subparagraph\undefined\else
  \let\oldsubparagraph\subparagraph
  \renewcommand{\subparagraph}[1]{\oldsubparagraph{#1}\mbox{}}
\fi


\providecommand{\tightlist}{%
  \setlength{\itemsep}{0pt}\setlength{\parskip}{0pt}}\usepackage{longtable,booktabs,array}
\usepackage{calc} % for calculating minipage widths
% Correct order of tables after \paragraph or \subparagraph
\usepackage{etoolbox}
\makeatletter
\patchcmd\longtable{\par}{\if@noskipsec\mbox{}\fi\par}{}{}
\makeatother
% Allow footnotes in longtable head/foot
\IfFileExists{footnotehyper.sty}{\usepackage{footnotehyper}}{\usepackage{footnote}}
\makesavenoteenv{longtable}
\usepackage{graphicx}
\makeatletter
\def\maxwidth{\ifdim\Gin@nat@width>\linewidth\linewidth\else\Gin@nat@width\fi}
\def\maxheight{\ifdim\Gin@nat@height>\textheight\textheight\else\Gin@nat@height\fi}
\makeatother
% Scale images if necessary, so that they will not overflow the page
% margins by default, and it is still possible to overwrite the defaults
% using explicit options in \includegraphics[width, height, ...]{}
\setkeys{Gin}{width=\maxwidth,height=\maxheight,keepaspectratio}
% Set default figure placement to htbp
\makeatletter
\def\fps@figure{htbp}
\makeatother

\usepackage{fancyhdr} \pagestyle{fancy} \usepackage{lastpage}
\KOMAoption{captions}{tablesignature}
\makeatletter
\makeatother
\makeatletter
\makeatother
\makeatletter
\@ifpackageloaded{caption}{}{\usepackage{caption}}
\AtBeginDocument{%
\ifdefined\contentsname
  \renewcommand*\contentsname{Table des matières}
\else
  \newcommand\contentsname{Table des matières}
\fi
\ifdefined\listfigurename
  \renewcommand*\listfigurename{Liste des Figures}
\else
  \newcommand\listfigurename{Liste des Figures}
\fi
\ifdefined\listtablename
  \renewcommand*\listtablename{Liste des Tables}
\else
  \newcommand\listtablename{Liste des Tables}
\fi
\ifdefined\figurename
  \renewcommand*\figurename{Figure}
\else
  \newcommand\figurename{Figure}
\fi
\ifdefined\tablename
  \renewcommand*\tablename{Tableau}
\else
  \newcommand\tablename{Tableau}
\fi
}
\@ifpackageloaded{float}{}{\usepackage{float}}
\floatstyle{ruled}
\@ifundefined{c@chapter}{\newfloat{codelisting}{h}{lop}}{\newfloat{codelisting}{h}{lop}[chapter]}
\floatname{codelisting}{Listing}
\newcommand*\listoflistings{\listof{codelisting}{Liste des Listings}}
\makeatother
\makeatletter
\@ifpackageloaded{caption}{}{\usepackage{caption}}
\@ifpackageloaded{subcaption}{}{\usepackage{subcaption}}
\makeatother
\makeatletter
\@ifpackageloaded{tcolorbox}{}{\usepackage[skins,breakable]{tcolorbox}}
\makeatother
\makeatletter
\@ifundefined{shadecolor}{\definecolor{shadecolor}{rgb}{.97, .97, .97}}
\makeatother
\makeatletter
\makeatother
\makeatletter
\makeatother
\ifLuaTeX
\usepackage[bidi=basic]{babel}
\else
\usepackage[bidi=default]{babel}
\fi
\babelprovide[main,import]{french}
% get rid of language-specific shorthands (see #6817):
\let\LanguageShortHands\languageshorthands
\def\languageshorthands#1{}
\ifLuaTeX
  \usepackage{selnolig}  % disable illegal ligatures
\fi
\IfFileExists{bookmark.sty}{\usepackage{bookmark}}{\usepackage{hyperref}}
\IfFileExists{xurl.sty}{\usepackage{xurl}}{} % add URL line breaks if available
\urlstyle{same} % disable monospaced font for URLs
\hypersetup{
  pdftitle={Projet - Détecteur de langue},
  pdflang={fr},
  colorlinks=true,
  linkcolor={blue},
  filecolor={Maroon},
  citecolor={Blue},
  urlcolor={Blue},
  pdfcreator={LaTeX via pandoc}}

\title{Projet - Détecteur de langue}
\usepackage{etoolbox}
\makeatletter
\providecommand{\subtitle}[1]{% add subtitle to \maketitle
  \apptocmd{\@title}{\par {\large #1 \par}}{}{}
}
\makeatother
\subtitle{S2 - Structures de données}
\author{}
\date{}

\begin{document}
\maketitle
\lhead{Spécialité NSI} \rhead{Terminale} \chead{} \cfoot{} \lfoot{Lycée \'Emile Duclaux} \rfoot{Page \thepage/\pageref{LastPage}} \renewcommand{\headrulewidth}{0pt} \renewcommand{\footrulewidth}{0pt} \thispagestyle{fancy} \vspace{-3cm}

\ifdefined\Shaded\renewenvironment{Shaded}{\begin{tcolorbox}[borderline west={3pt}{0pt}{shadecolor}, boxrule=0pt, interior hidden, sharp corners, breakable, enhanced, frame hidden]}{\end{tcolorbox}}\fi

\hypertarget{projet-duxe9tecteur-de-la-langue-dun-texte}{%
\section{Projet : détecteur de la langue d'un
texte}\label{projet-duxe9tecteur-de-la-langue-dun-texte}}

\begin{figure}

{\centering \includegraphics[width=0.5\textwidth,height=\textheight]{Pieter_Bruegel_the_Elder_-_The_Tower_of_Babel_(Vienna)_-_Google_Art_Project_-_edited.jpg}

}

\caption{Tour de Babel}

\end{figure}

L'objectif final de ce projet est de produire un programme qui détecte
dans quelle langue un texte est écrit, en s'appuyant sur une analyse
fréquentielle des lettres du texte.

Il s'agit d'un projet : les consignes sont donc volontairement très
limitées et seul le résultat attendu est détaillé (cahier des charges).
Il vous revient d'organiser votre travail comme vous le voulez :
création de fichiers, de bibliothèques, recherches documentaires.

Vous pouvez exercer votre liberté et choisir le style de programmation
qui vous plait : impératif, POO, modulaire \ldots{}

Des compléments facultatifs sont proposés pour les plus rapides.

\hypertarget{uxe9tape-1}{%
\subsection{Étape 1}\label{uxe9tape-1}}

Créer un programme qui :

\begin{itemize}
\tightlist
\item
  lit un fichier texte (encodé en UTF-8) ;
\item
  crée un dictionnaire dont les clés sont les lettres de l'alphabet de
  ``a'' à ``z'' ;
\item
  analyse le contenu du fichier caractère par caractère et remplit le
  dictionnaire avec comme valeur le nombre d'apparitions de chacune des
  lettres dans le texte. Les lettres autres que les lettres ``a'' à
  ``z'' sont ignorées ;
\item
  crée un nouveau dictionnaire avec les mêmes clés, mais en valeurs les
  fréquences d'apparition de chaque lettre ;
\item
  affiche un diagramme en barres de cette répartition de fréquences à
  l'aide de la librairie MatPlotLib.
\end{itemize}

\textbf{Remarques} :

\begin{itemize}
\tightlist
\item
  il peut être judicieux de décomposer le programme en plusieurs
  fonctions ;
\item
  vous rechercherez dans la documentation les exemples d'utilisation de
  MatPlotLib ;
\item
  voici un fichier texte pour tester votre programme :
  \href{texteFR.txt}{texte}.
\end{itemize}

\hypertarget{uxe9tape-2}{%
\subsection{Étape 2}\label{uxe9tape-2}}

La fréquence d'apparition des lettres dans les différentes langues qui
utilisent l'alphabet latin est différente. C'est pour cette raison que
les points attribués aux différentes lettres ainsi que le nombre de ces
lettres dans le jeu du Scrabble ne sont pas les mêmes dans tous les
pays.

Le tableau disponible sur Wikipédia à cette adresse :
\href{https://fr.wikipedia.org/wiki/Fr\%C3\%A9quence_d\%27apparition_des_lettres\#Dans_d\textquotesingle{}autres_langues}{-Wikipédia-}
donne cette fréquence dans les langues les plus courantes.

Nous allons choisir comme \emph{signature} d'une langue la liste des dix
lettres les plus utilisées dans cette langue, de la plus utilisée à la
moins utilisée.

Créer un dictionnaire \texttt{signature} dont les clés sont le nom des
langues sous forme de chaîne de caractères et les valeurs sont ces
listes de dix lettres.

Par exemple, l'appel \texttt{signature{[}"Français"{]}{[}0{]}}
retournera \texttt{"e"}.

On se limitera aux langues suivantes : Français, Anglais, Allemand,
Espagnol, Italien, Portugais, Espéranto, Polonais et Néerlandais.

\hypertarget{uxe9tape-3}{%
\subsection{Étape 3}\label{uxe9tape-3}}

À partir du programme réalisé à l'étape 1, programmer une fonction qui,
pour un fichier texte donné, crée une liste \texttt{signature\_texte}
avec pour éléments les dix lettres les plus utilisées dans ce texte, de
la plus utilisée à la moins utilisée.

Écrire une fonction \texttt{detecte\_langue(texte)} qui, à partir des
éléments précédents, retourne le nom de la langue dans laquelle le
\texttt{texte} donné en paramètre est le plus probablement écrit. Il
vous faudra notamment choisir un moyen de comparer la liste
\texttt{signature\_texte} avec l'ensemble des listes du dictionnaire
\texttt{signature} afin de trouver cette qui est la plus ``proche''.

Pour tester votre programme, voici des fichiers texte dans les
différentes langues concernées : \href{textes.zip}{textes} (fichier .zip
à décompresser).

Proposer des améliorations possibles.

\hypertarget{compluxe9ment-facultatif}{%
\subsection{Complément facultatif}\label{compluxe9ment-facultatif}}

Programmer une fonction \texttt{detecte\_langue(url)} qui prend en
argument l'adresse d'une page web et qui retourne le nom de la langue
dans laquelle la page est le plus probablement écrite. Vous pourrez
utiliser la bibliothèque Python BeautifulSoup qui permet d'extraire le
texte présent dans les balises HTML d'une page web (voir la
documentation).



\end{document}
