% Options for packages loaded elsewhere
\PassOptionsToPackage{unicode}{hyperref}
\PassOptionsToPackage{hyphens}{url}
\PassOptionsToPackage{dvipsnames,svgnames,x11names}{xcolor}
%
\documentclass[
  a4paper,
  DIV=11,
  numbers=noendperiod]{scrartcl}

\usepackage{amsmath,amssymb}
\usepackage{iftex}
\ifPDFTeX
  \usepackage[T1]{fontenc}
  \usepackage[utf8]{inputenc}
  \usepackage{textcomp} % provide euro and other symbols
\else % if luatex or xetex
  \usepackage{unicode-math}
  \defaultfontfeatures{Scale=MatchLowercase}
  \defaultfontfeatures[\rmfamily]{Ligatures=TeX,Scale=1}
\fi
\usepackage{lmodern}
\ifPDFTeX\else  
    % xetex/luatex font selection
\fi
% Use upquote if available, for straight quotes in verbatim environments
\IfFileExists{upquote.sty}{\usepackage{upquote}}{}
\IfFileExists{microtype.sty}{% use microtype if available
  \usepackage[]{microtype}
  \UseMicrotypeSet[protrusion]{basicmath} % disable protrusion for tt fonts
}{}
\makeatletter
\@ifundefined{KOMAClassName}{% if non-KOMA class
  \IfFileExists{parskip.sty}{%
    \usepackage{parskip}
  }{% else
    \setlength{\parindent}{0pt}
    \setlength{\parskip}{6pt plus 2pt minus 1pt}}
}{% if KOMA class
  \KOMAoptions{parskip=half}}
\makeatother
\usepackage{xcolor}
\setlength{\emergencystretch}{3em} % prevent overfull lines
\setcounter{secnumdepth}{-\maxdimen} % remove section numbering
% Make \paragraph and \subparagraph free-standing
\ifx\paragraph\undefined\else
  \let\oldparagraph\paragraph
  \renewcommand{\paragraph}[1]{\oldparagraph{#1}\mbox{}}
\fi
\ifx\subparagraph\undefined\else
  \let\oldsubparagraph\subparagraph
  \renewcommand{\subparagraph}[1]{\oldsubparagraph{#1}\mbox{}}
\fi

\usepackage{color}
\usepackage{fancyvrb}
\newcommand{\VerbBar}{|}
\newcommand{\VERB}{\Verb[commandchars=\\\{\}]}
\DefineVerbatimEnvironment{Highlighting}{Verbatim}{commandchars=\\\{\}}
% Add ',fontsize=\small' for more characters per line
\usepackage{framed}
\definecolor{shadecolor}{RGB}{241,243,245}
\newenvironment{Shaded}{\begin{snugshade}}{\end{snugshade}}
\newcommand{\AlertTok}[1]{\textcolor[rgb]{0.68,0.00,0.00}{#1}}
\newcommand{\AnnotationTok}[1]{\textcolor[rgb]{0.37,0.37,0.37}{#1}}
\newcommand{\AttributeTok}[1]{\textcolor[rgb]{0.40,0.45,0.13}{#1}}
\newcommand{\BaseNTok}[1]{\textcolor[rgb]{0.68,0.00,0.00}{#1}}
\newcommand{\BuiltInTok}[1]{\textcolor[rgb]{0.00,0.23,0.31}{#1}}
\newcommand{\CharTok}[1]{\textcolor[rgb]{0.13,0.47,0.30}{#1}}
\newcommand{\CommentTok}[1]{\textcolor[rgb]{0.37,0.37,0.37}{#1}}
\newcommand{\CommentVarTok}[1]{\textcolor[rgb]{0.37,0.37,0.37}{\textit{#1}}}
\newcommand{\ConstantTok}[1]{\textcolor[rgb]{0.56,0.35,0.01}{#1}}
\newcommand{\ControlFlowTok}[1]{\textcolor[rgb]{0.00,0.23,0.31}{#1}}
\newcommand{\DataTypeTok}[1]{\textcolor[rgb]{0.68,0.00,0.00}{#1}}
\newcommand{\DecValTok}[1]{\textcolor[rgb]{0.68,0.00,0.00}{#1}}
\newcommand{\DocumentationTok}[1]{\textcolor[rgb]{0.37,0.37,0.37}{\textit{#1}}}
\newcommand{\ErrorTok}[1]{\textcolor[rgb]{0.68,0.00,0.00}{#1}}
\newcommand{\ExtensionTok}[1]{\textcolor[rgb]{0.00,0.23,0.31}{#1}}
\newcommand{\FloatTok}[1]{\textcolor[rgb]{0.68,0.00,0.00}{#1}}
\newcommand{\FunctionTok}[1]{\textcolor[rgb]{0.28,0.35,0.67}{#1}}
\newcommand{\ImportTok}[1]{\textcolor[rgb]{0.00,0.46,0.62}{#1}}
\newcommand{\InformationTok}[1]{\textcolor[rgb]{0.37,0.37,0.37}{#1}}
\newcommand{\KeywordTok}[1]{\textcolor[rgb]{0.00,0.23,0.31}{#1}}
\newcommand{\NormalTok}[1]{\textcolor[rgb]{0.00,0.23,0.31}{#1}}
\newcommand{\OperatorTok}[1]{\textcolor[rgb]{0.37,0.37,0.37}{#1}}
\newcommand{\OtherTok}[1]{\textcolor[rgb]{0.00,0.23,0.31}{#1}}
\newcommand{\PreprocessorTok}[1]{\textcolor[rgb]{0.68,0.00,0.00}{#1}}
\newcommand{\RegionMarkerTok}[1]{\textcolor[rgb]{0.00,0.23,0.31}{#1}}
\newcommand{\SpecialCharTok}[1]{\textcolor[rgb]{0.37,0.37,0.37}{#1}}
\newcommand{\SpecialStringTok}[1]{\textcolor[rgb]{0.13,0.47,0.30}{#1}}
\newcommand{\StringTok}[1]{\textcolor[rgb]{0.13,0.47,0.30}{#1}}
\newcommand{\VariableTok}[1]{\textcolor[rgb]{0.07,0.07,0.07}{#1}}
\newcommand{\VerbatimStringTok}[1]{\textcolor[rgb]{0.13,0.47,0.30}{#1}}
\newcommand{\WarningTok}[1]{\textcolor[rgb]{0.37,0.37,0.37}{\textit{#1}}}

\providecommand{\tightlist}{%
  \setlength{\itemsep}{0pt}\setlength{\parskip}{0pt}}\usepackage{longtable,booktabs,array}
\usepackage{calc} % for calculating minipage widths
% Correct order of tables after \paragraph or \subparagraph
\usepackage{etoolbox}
\makeatletter
\patchcmd\longtable{\par}{\if@noskipsec\mbox{}\fi\par}{}{}
\makeatother
% Allow footnotes in longtable head/foot
\IfFileExists{footnotehyper.sty}{\usepackage{footnotehyper}}{\usepackage{footnote}}
\makesavenoteenv{longtable}
\usepackage{graphicx}
\makeatletter
\def\maxwidth{\ifdim\Gin@nat@width>\linewidth\linewidth\else\Gin@nat@width\fi}
\def\maxheight{\ifdim\Gin@nat@height>\textheight\textheight\else\Gin@nat@height\fi}
\makeatother
% Scale images if necessary, so that they will not overflow the page
% margins by default, and it is still possible to overwrite the defaults
% using explicit options in \includegraphics[width, height, ...]{}
\setkeys{Gin}{width=\maxwidth,height=\maxheight,keepaspectratio}
% Set default figure placement to htbp
\makeatletter
\def\fps@figure{htbp}
\makeatother

\KOMAoption{captions}{tableheading}
\makeatletter
\makeatother
\makeatletter
\makeatother
\makeatletter
\@ifpackageloaded{caption}{}{\usepackage{caption}}
\AtBeginDocument{%
\ifdefined\contentsname
  \renewcommand*\contentsname{Table of contents}
\else
  \newcommand\contentsname{Table of contents}
\fi
\ifdefined\listfigurename
  \renewcommand*\listfigurename{List of Figures}
\else
  \newcommand\listfigurename{List of Figures}
\fi
\ifdefined\listtablename
  \renewcommand*\listtablename{List of Tables}
\else
  \newcommand\listtablename{List of Tables}
\fi
\ifdefined\figurename
  \renewcommand*\figurename{Figure}
\else
  \newcommand\figurename{Figure}
\fi
\ifdefined\tablename
  \renewcommand*\tablename{Table}
\else
  \newcommand\tablename{Table}
\fi
}
\@ifpackageloaded{float}{}{\usepackage{float}}
\floatstyle{ruled}
\@ifundefined{c@chapter}{\newfloat{codelisting}{h}{lop}}{\newfloat{codelisting}{h}{lop}[chapter]}
\floatname{codelisting}{Listing}
\newcommand*\listoflistings{\listof{codelisting}{List of Listings}}
\makeatother
\makeatletter
\@ifpackageloaded{caption}{}{\usepackage{caption}}
\@ifpackageloaded{subcaption}{}{\usepackage{subcaption}}
\makeatother
\makeatletter
\@ifpackageloaded{tcolorbox}{}{\usepackage[skins,breakable]{tcolorbox}}
\makeatother
\makeatletter
\@ifundefined{shadecolor}{\definecolor{shadecolor}{rgb}{.97, .97, .97}}
\makeatother
\makeatletter
\makeatother
\makeatletter
\makeatother
\ifLuaTeX
  \usepackage{selnolig}  % disable illegal ligatures
\fi
\IfFileExists{bookmark.sty}{\usepackage{bookmark}}{\usepackage{hyperref}}
\IfFileExists{xurl.sty}{\usepackage{xurl}}{} % add URL line breaks if available
\urlstyle{same} % disable monospaced font for URLs
\hypersetup{
  pdftitle={Exercices : Parcours de graphes},
  colorlinks=true,
  linkcolor={blue},
  filecolor={Maroon},
  citecolor={Blue},
  urlcolor={Blue},
  pdfcreator={LaTeX via pandoc}}

\title{Exercices : Parcours de graphes}
\author{}
\date{}

\begin{document}
\maketitle
\ifdefined\Shaded\renewenvironment{Shaded}{\begin{tcolorbox}[interior hidden, breakable, sharp corners, frame hidden, boxrule=0pt, borderline west={3pt}{0pt}{shadecolor}, enhanced]}{\end{tcolorbox}}\fi

\hypertarget{exercice-1}{%
\subsection{Exercice 1}\label{exercice-1}}

Voici quatre animations qui représentent 4 parcours différents sur le
même graphe à partir du sommet `A'.

Repérer le parcours qui correspond bien à un parcours en profondeur du
graphe à partir du sommet `A'.

Expliquer pourquoi les autres parcours ne sont pas un parcours en
profondeur du graphe à partir du sommet `A'.

Parcours 1 :

\begin{figure}

{\centering \includegraphics{exos_graphes_fiche1_files/mediabag/a2_reperer_profondeu.gif}

}

\caption{Parcours 1}

\end{figure}

Parcours 2 :

\begin{figure}

{\centering \includegraphics{exos_graphes_fiche1_files/mediabag/a2_reperer_profondeu1.gif}

}

\caption{Parcours 2}

\end{figure}

Parcours 3 :

\begin{figure}

{\centering \includegraphics{exos_graphes_fiche1_files/mediabag/a2_reperer_profondeu12.gif}

}

\caption{Parcours 3}

\end{figure}

Parcours 4 :

\begin{figure}

{\centering \includegraphics{exos_graphes_fiche1_files/mediabag/a2_reperer_profondeu123.gif}

}

\caption{Parcours 4}

\end{figure}

\begin{Shaded}
\begin{Highlighting}[]
\CommentTok{\# Votre réponse ici}
\end{Highlighting}
\end{Shaded}

\hypertarget{exercice-2}{%
\subsection{Exercice 2}\label{exercice-2}}

Devant tant d'efforts pour comprendre les parcours, vous rêvez de partir
en voyage sur le continent sud-américiain. Mais quel parcours effectuer
pour visiter tous les pays de l'Amérique du Sud ?

Vous allez dans cet exercice :

\begin{itemize}
\tightlist
\item
  utiliser un graphe implémenté comme dictionnaire avec la liste des
  successeurs, représentant les pays frontaliers d'Amérique du Sud,
\item
  programmer un parcours en profondeur,
\item
  l'appliquer au voyage de votre rêve !
\end{itemize}

Voici la carte des pays du continent sud-américain :

\includegraphics{a2_carte_amerique_sud_pays.png}

On considère le dictionnaire suivant qui permet d'associer à chaque pays
d'Amérique du Sud la liste des pays partageant une frontière terrestre.

\begin{Shaded}
\begin{Highlighting}[]
\NormalTok{G }\OperatorTok{=}\NormalTok{ \{\}}
\NormalTok{G[}\StringTok{"France"}\NormalTok{] }\OperatorTok{=}\NormalTok{ [}\StringTok{"Brésil"}\NormalTok{,}\StringTok{"Suriname"}\NormalTok{]}
\NormalTok{G[}\StringTok{"Argentine"}\NormalTok{] }\OperatorTok{=}\NormalTok{ [}\StringTok{"Bolivie"}\NormalTok{,}\StringTok{"Brésil"}\NormalTok{,}\StringTok{"Chili"}\NormalTok{,}\StringTok{"Paraguay"}\NormalTok{,}\StringTok{"Uruguay"}\NormalTok{]}
\NormalTok{G[}\StringTok{"Bolivie"}\NormalTok{] }\OperatorTok{=}\NormalTok{ [}\StringTok{"Argentine"}\NormalTok{,}\StringTok{"Brésil"}\NormalTok{,}\StringTok{"Chili"}\NormalTok{,}\StringTok{"Paraguay"}\NormalTok{,}\StringTok{"Pérou"}\NormalTok{,}\StringTok{"Uruguay"}\NormalTok{]}
\NormalTok{G[}\StringTok{"Brésil"}\NormalTok{] }\OperatorTok{=}\NormalTok{ [}\StringTok{"Argentine"}\NormalTok{,}\StringTok{"Bolivie"}\NormalTok{,}\StringTok{"Colombie"}\NormalTok{,}\StringTok{"France"}\NormalTok{,}\StringTok{"Guyana"}\NormalTok{,}\StringTok{"Paraguay"}\NormalTok{,}\StringTok{"Pérou"}\NormalTok{,}\StringTok{"Suriname"}\NormalTok{,}\StringTok{"Uruguay"}\NormalTok{,}\StringTok{"Venezuela"}\NormalTok{]}
\NormalTok{G[}\StringTok{"Chili"}\NormalTok{]}\OperatorTok{=}\NormalTok{[}\StringTok{"Argentine"}\NormalTok{,}\StringTok{"Bolivie"}\NormalTok{,}\StringTok{"Pérou"}\NormalTok{]}
\NormalTok{G[}\StringTok{"Colombie"}\NormalTok{] }\OperatorTok{=}\NormalTok{ [}\StringTok{"Brésil"}\NormalTok{,}\StringTok{"Équateur"}\NormalTok{,}\StringTok{"Pérou"}\NormalTok{,}\StringTok{"Venezuela"}\NormalTok{]}
\NormalTok{G[}\StringTok{"Équateur"}\NormalTok{] }\OperatorTok{=}\NormalTok{ [}\StringTok{"Colombie"}\NormalTok{,}\StringTok{"Pérou"}\NormalTok{]}
\NormalTok{G[}\StringTok{"Guyana"}\NormalTok{] }\OperatorTok{=}\NormalTok{ [}\StringTok{"Brésil"}\NormalTok{,}\StringTok{"Suriname"}\NormalTok{,}\StringTok{"Venezuela"}\NormalTok{]}
\NormalTok{G[}\StringTok{"Paraguay"}\NormalTok{] }\OperatorTok{=}\NormalTok{ [}\StringTok{"Argentine"}\NormalTok{,}\StringTok{"Bolivie"}\NormalTok{,}\StringTok{"Brésil"}\NormalTok{]}
\NormalTok{G[}\StringTok{"Pérou"}\NormalTok{] }\OperatorTok{=}\NormalTok{ [}\StringTok{"Bolivie"}\NormalTok{,}\StringTok{"Brésil"}\NormalTok{,}\StringTok{"Chili"}\NormalTok{,}\StringTok{"Colombie"}\NormalTok{,}\StringTok{"Équateur"}\NormalTok{]}
\NormalTok{G[}\StringTok{"Suriname"}\NormalTok{] }\OperatorTok{=}\NormalTok{ [}\StringTok{"Brésil"}\NormalTok{,}\StringTok{"France"}\NormalTok{,}\StringTok{"Guyana"}\NormalTok{]}
\NormalTok{G[}\StringTok{"Uruguay"}\NormalTok{] }\OperatorTok{=}\NormalTok{ [}\StringTok{"Argentine"}\NormalTok{,}\StringTok{"Bolivie"}\NormalTok{,}\StringTok{"Brésil"}\NormalTok{]}
\NormalTok{G[}\StringTok{"Venezuela"}\NormalTok{] }\OperatorTok{=}\NormalTok{ [}\StringTok{"Brésil"}\NormalTok{,}\StringTok{"Colombie"}\NormalTok{,}\StringTok{"Guyana"}\NormalTok{]}
\end{Highlighting}
\end{Shaded}

Pour savoir quels sommets ont déjà été visités ou non, vous allez
utiliser un dictionnaire dont les clés seront les sommets du graphe et
les valeurs associées seront les chaînes de caractères soit ``inconnu'',
soit ``visite''

\begin{enumerate}
\def\labelenumi{\arabic{enumi}.}
\tightlist
\item
  Créer une fonction initialiser qui prend comme paramètre un graphe et
  renvoie un dictionnaire dont les clés sont les sommets du graphe et la
  valeur est toujours ``inconnu''.
\end{enumerate}

\begin{Shaded}
\begin{Highlighting}[]
\CommentTok{\# Votre code ci{-}dessous}
\end{Highlighting}
\end{Shaded}

\begin{enumerate}
\def\labelenumi{\arabic{enumi}.}
\setcounter{enumi}{1}
\tightlist
\item
  Créer une fonction visiter qui prend comme paramètre le dictionnaire
  associant à chaque sommet son état de connaissance et un sommet du
  graphe, fonction qui renvoie le dictionnaire de connaissance où la
  valeur associée au sommet entré est mise à `visite'
\end{enumerate}

\begin{Shaded}
\begin{Highlighting}[]
\CommentTok{\# Votre code ci{-}dessous}
\end{Highlighting}
\end{Shaded}

\begin{enumerate}
\def\labelenumi{\arabic{enumi}.}
\setcounter{enumi}{2}
\tightlist
\item
  Écrire en langage Python la procédure récursive
  \texttt{parcours\_longueur\_rec} correspondant à un parcours en
  profondeur en faisant aussi en sorte que chaque sommet qui vient
  d'être marqué soit affiché.
\end{enumerate}

Cette procédure prend en argument un graphe et un sommet de ce graphe,
sommet servant de point de départ au parcours.

Gérer l'affichage à l'aide de l'instruction \texttt{print}.

\begin{Shaded}
\begin{Highlighting}[]
\CommentTok{\# Votre code ci{-}dessous}
\end{Highlighting}
\end{Shaded}

\begin{enumerate}
\def\labelenumi{\arabic{enumi}.}
\setcounter{enumi}{3}
\tightlist
\item
  Appliquer cette procédure au graphe modélisant les pays d'Amérique du
  Sud en prenant comme racine, c'est-à-dire sommet de départ la France
  (pour la Guyane Française). À la lecture de tous les pays que vous
  allez visiter, vous allez être profondément heureux.ses.
\end{enumerate}

\begin{Shaded}
\begin{Highlighting}[]
\CommentTok{\# Votre code ci{-}dessous}
\end{Highlighting}
\end{Shaded}

\hypertarget{exercice-3}{%
\subsection{Exercice 3}\label{exercice-3}}

Dans cet exercice, nous reprenons la situation de l'exercice précédent,
mais nous allons utiliser un parcours en largeur. De plus, le graphe
sera maintenant implémenté avec le module Networkx.

Vous prendrez une liste Python en guise de file en considérant que la
tête de la file correspond au premier élément de la liste.

Le graphe sera supposé être un objet de la classe Graph() de la
bibliothèque Networkx ; ceci permettra d'utiliser les méthodes de cette
bibliothèque, comme :

\begin{itemize}
\item
  \texttt{graphe.neighbors(v)} qui permet d'obtenir les voisins d'un
  sommet \texttt{v} du graphe \texttt{graphe},
\item
  vous pourrez transtyper en liste l'itérable les voisins d'un sommet
  donné grâce à list (comme fait dans le cours :
  \texttt{list(graphe.neighbors(v))}).
\end{itemize}

\begin{enumerate}
\def\labelenumi{\arabic{enumi}.}
\tightlist
\item
  Écrire en langage Python une fonction \texttt{parcours\_largeur} qui
  met en œuvre l'algorithme de parcours en largeur d'un graphe à partir
  d'un sommet donné.
\end{enumerate}

\begin{Shaded}
\begin{Highlighting}[]
\CommentTok{\# Votre code ci{-}dessous}
\end{Highlighting}
\end{Shaded}

\begin{enumerate}
\def\labelenumi{\arabic{enumi}.}
\setcounter{enumi}{1}
\tightlist
\item
  Tester cet algorithme de parcours en profondeur en prenant le graphe
  sur les pays d'Amérique du Sud.
\end{enumerate}

\begin{Shaded}
\begin{Highlighting}[]
\CommentTok{\# Votre code ci{-}dessous}
\end{Highlighting}
\end{Shaded}

\begin{enumerate}
\def\labelenumi{\arabic{enumi}.}
\setcounter{enumi}{2}
\tightlist
\item
  En vous aidant de la carte de l'Amérique du Sud ci-dessus, vérifier
  que la liste renvoyée correspond bien à un parcours en largeur.
\end{enumerate}

\begin{Shaded}
\begin{Highlighting}[]
\CommentTok{\# Votre réponse ci{-}dessous}
\end{Highlighting}
\end{Shaded}




\end{document}
