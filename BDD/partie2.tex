% Options for packages loaded elsewhere
\PassOptionsToPackage{unicode}{hyperref}
\PassOptionsToPackage{hyphens}{url}
\PassOptionsToPackage{dvipsnames,svgnames,x11names}{xcolor}
%
\documentclass[
  letterpaper,
  DIV=11,
  numbers=noendperiod]{scrartcl}

\usepackage{amsmath,amssymb}
\usepackage{iftex}
\ifPDFTeX
  \usepackage[T1]{fontenc}
  \usepackage[utf8]{inputenc}
  \usepackage{textcomp} % provide euro and other symbols
\else % if luatex or xetex
  \usepackage{unicode-math}
  \defaultfontfeatures{Scale=MatchLowercase}
  \defaultfontfeatures[\rmfamily]{Ligatures=TeX,Scale=1}
\fi
\usepackage{lmodern}
\ifPDFTeX\else  
    % xetex/luatex font selection
\fi
% Use upquote if available, for straight quotes in verbatim environments
\IfFileExists{upquote.sty}{\usepackage{upquote}}{}
\IfFileExists{microtype.sty}{% use microtype if available
  \usepackage[]{microtype}
  \UseMicrotypeSet[protrusion]{basicmath} % disable protrusion for tt fonts
}{}
\makeatletter
\@ifundefined{KOMAClassName}{% if non-KOMA class
  \IfFileExists{parskip.sty}{%
    \usepackage{parskip}
  }{% else
    \setlength{\parindent}{0pt}
    \setlength{\parskip}{6pt plus 2pt minus 1pt}}
}{% if KOMA class
  \KOMAoptions{parskip=half}}
\makeatother
\usepackage{xcolor}
\usepackage[top=20mm,bottom=20mm,left=20mm,right=20mm,heightrounded]{geometry}
\setlength{\emergencystretch}{3em} % prevent overfull lines
\setcounter{secnumdepth}{-\maxdimen} % remove section numbering
% Make \paragraph and \subparagraph free-standing
\ifx\paragraph\undefined\else
  \let\oldparagraph\paragraph
  \renewcommand{\paragraph}[1]{\oldparagraph{#1}\mbox{}}
\fi
\ifx\subparagraph\undefined\else
  \let\oldsubparagraph\subparagraph
  \renewcommand{\subparagraph}[1]{\oldsubparagraph{#1}\mbox{}}
\fi


\providecommand{\tightlist}{%
  \setlength{\itemsep}{0pt}\setlength{\parskip}{0pt}}\usepackage{longtable,booktabs,array}
\usepackage{calc} % for calculating minipage widths
% Correct order of tables after \paragraph or \subparagraph
\usepackage{etoolbox}
\makeatletter
\patchcmd\longtable{\par}{\if@noskipsec\mbox{}\fi\par}{}{}
\makeatother
% Allow footnotes in longtable head/foot
\IfFileExists{footnotehyper.sty}{\usepackage{footnotehyper}}{\usepackage{footnote}}
\makesavenoteenv{longtable}
\usepackage{graphicx}
\makeatletter
\def\maxwidth{\ifdim\Gin@nat@width>\linewidth\linewidth\else\Gin@nat@width\fi}
\def\maxheight{\ifdim\Gin@nat@height>\textheight\textheight\else\Gin@nat@height\fi}
\makeatother
% Scale images if necessary, so that they will not overflow the page
% margins by default, and it is still possible to overwrite the defaults
% using explicit options in \includegraphics[width, height, ...]{}
\setkeys{Gin}{width=\maxwidth,height=\maxheight,keepaspectratio}
% Set default figure placement to htbp
\makeatletter
\def\fps@figure{htbp}
\makeatother

\usepackage{fancyhdr} \pagestyle{fancy} \usepackage{lastpage}
\KOMAoption{captions}{tablesignature}
\makeatletter
\@ifpackageloaded{tcolorbox}{}{\usepackage[skins,breakable]{tcolorbox}}
\@ifpackageloaded{fontawesome5}{}{\usepackage{fontawesome5}}
\definecolor{quarto-callout-color}{HTML}{909090}
\definecolor{quarto-callout-note-color}{HTML}{0758E5}
\definecolor{quarto-callout-important-color}{HTML}{CC1914}
\definecolor{quarto-callout-warning-color}{HTML}{EB9113}
\definecolor{quarto-callout-tip-color}{HTML}{00A047}
\definecolor{quarto-callout-caution-color}{HTML}{FC5300}
\definecolor{quarto-callout-color-frame}{HTML}{acacac}
\definecolor{quarto-callout-note-color-frame}{HTML}{4582ec}
\definecolor{quarto-callout-important-color-frame}{HTML}{d9534f}
\definecolor{quarto-callout-warning-color-frame}{HTML}{f0ad4e}
\definecolor{quarto-callout-tip-color-frame}{HTML}{02b875}
\definecolor{quarto-callout-caution-color-frame}{HTML}{fd7e14}
\makeatother
\makeatletter
\makeatother
\makeatletter
\makeatother
\makeatletter
\@ifpackageloaded{caption}{}{\usepackage{caption}}
\AtBeginDocument{%
\ifdefined\contentsname
  \renewcommand*\contentsname{Table des matières}
\else
  \newcommand\contentsname{Table des matières}
\fi
\ifdefined\listfigurename
  \renewcommand*\listfigurename{Liste des Figures}
\else
  \newcommand\listfigurename{Liste des Figures}
\fi
\ifdefined\listtablename
  \renewcommand*\listtablename{Liste des Tables}
\else
  \newcommand\listtablename{Liste des Tables}
\fi
\ifdefined\figurename
  \renewcommand*\figurename{Figure}
\else
  \newcommand\figurename{Figure}
\fi
\ifdefined\tablename
  \renewcommand*\tablename{Tableau}
\else
  \newcommand\tablename{Tableau}
\fi
}
\@ifpackageloaded{float}{}{\usepackage{float}}
\floatstyle{ruled}
\@ifundefined{c@chapter}{\newfloat{codelisting}{h}{lop}}{\newfloat{codelisting}{h}{lop}[chapter]}
\floatname{codelisting}{Listing}
\newcommand*\listoflistings{\listof{codelisting}{Liste des Listings}}
\makeatother
\makeatletter
\@ifpackageloaded{caption}{}{\usepackage{caption}}
\@ifpackageloaded{subcaption}{}{\usepackage{subcaption}}
\makeatother
\makeatletter
\@ifpackageloaded{tcolorbox}{}{\usepackage[skins,breakable]{tcolorbox}}
\makeatother
\makeatletter
\@ifundefined{shadecolor}{\definecolor{shadecolor}{rgb}{.97, .97, .97}}
\makeatother
\makeatletter
\makeatother
\makeatletter
\makeatother
\ifLuaTeX
\usepackage[bidi=basic]{babel}
\else
\usepackage[bidi=default]{babel}
\fi
\babelprovide[main,import]{french}
% get rid of language-specific shorthands (see #6817):
\let\LanguageShortHands\languageshorthands
\def\languageshorthands#1{}
\ifLuaTeX
  \usepackage{selnolig}  % disable illegal ligatures
\fi
\IfFileExists{bookmark.sty}{\usepackage{bookmark}}{\usepackage{hyperref}}
\IfFileExists{xurl.sty}{\usepackage{xurl}}{} % add URL line breaks if available
\urlstyle{same} % disable monospaced font for URLs
\hypersetup{
  pdftitle={Le modèle relationnel (Cours - Partie 2)},
  pdflang={fr},
  colorlinks=true,
  linkcolor={blue},
  filecolor={Maroon},
  citecolor={Blue},
  urlcolor={Blue},
  pdfcreator={LaTeX via pandoc}}

\title{Le modèle relationnel (Cours - Partie 2)}
\usepackage{etoolbox}
\makeatletter
\providecommand{\subtitle}[1]{% add subtitle to \maketitle
  \apptocmd{\@title}{\par {\large #1 \par}}{}{}
}
\makeatother
\subtitle{S3 - Bases de données}
\author{}
\date{}

\begin{document}
\maketitle
\lhead{Spécialité NSI} \rhead{Terminale} \chead{} \cfoot{} \lfoot{Lycée \'Emile Duclaux} \rfoot{Page \thepage/\pageref{LastPage}} \renewcommand{\headrulewidth}{0pt} \renewcommand{\footrulewidth}{0pt} \thispagestyle{fancy} \vspace{-2.5cm}

\ifdefined\Shaded\renewenvironment{Shaded}{\begin{tcolorbox}[frame hidden, enhanced, breakable, boxrule=0pt, interior hidden, borderline west={3pt}{0pt}{shadecolor}, sharp corners]}{\end{tcolorbox}}\fi

\begin{tcolorbox}[enhanced jigsaw, colbacktitle=quarto-callout-tip-color!10!white, opacitybacktitle=0.6, left=2mm, coltitle=black, bottomtitle=1mm, arc=.35mm, opacityback=0, title=\textcolor{quarto-callout-tip-color}{\faLightbulb}\hspace{0.5em}{Définition}, breakable, toprule=.15mm, rightrule=.15mm, bottomrule=.15mm, colback=white, colframe=quarto-callout-tip-color-frame, toptitle=1mm, titlerule=0mm, leftrule=.75mm]

Le modèle relationnel est une manière de modéliser les relations
existantes entre plusieurs informations, et de les ordonner entre elles

\end{tcolorbox}

\hypertarget{relation-attributs-et-domaines}{%
\subsection{1. Relation, attributs et
domaines}\label{relation-attributs-et-domaines}}

Une \textbf{relation} peut être vue comme un tableau composé d'une
en-tête (première ligne) et d'un corps.

Chaque ligne de la relation est un p-uplet et chaque colonne est un
\textbf{attribut} (l'en-tête contient les intitulés des attributs).

\includegraphics{BDD5.png}

Pour la relation ci-dessus, on retrouve les données concernant les
établissements scolaires du second degré dans le Cantal.

Pour chaque attribut d'une relation, il est nécessaire de définir un
\textbf{domaine} : Le domaine d'un attribut donné correspond \textbf{à
un ensemble fini ou infini de valeurs admissibles}.

Par exemple, le domaine de l'attribut ``statut'' correspond à l'ensemble
des deux chaînes \{``Public'', ``Privé''\}. L'attribut ``nom'' a pour
domaine l'ensemble des chaînes de caractères (noté TEXT). L'attribut
``codepostal'' a pour domaine l'ensemble des entiers (noté INT).

Au moment de la création d'une relation, il est nécessaire de renseigner
le domaine de chaque attribut. Le SGBD s'assure qu'un élément ajouté à
une relation respecte bien le domaine de l'attribut correspondant : si
par exemple vous essayez d'ajouter un code postal non entier (par
exemple 8.5), le SGBD signalera cette erreur et n'autorisera pas
l'écriture de cette nouvelle donnée.

\begin{tcolorbox}[enhanced jigsaw, colbacktitle=quarto-callout-important-color!10!white, opacitybacktitle=0.6, left=2mm, coltitle=black, bottomtitle=1mm, arc=.35mm, opacityback=0, title=\textcolor{quarto-callout-important-color}{\faExclamation}\hspace{0.5em}{Règles à respecter}, breakable, toprule=.15mm, rightrule=.15mm, bottomrule=.15mm, colback=white, colframe=quarto-callout-important-color-frame, toptitle=1mm, titlerule=0mm, leftrule=.75mm]

Dans une relation, il est nécessaire de respecter les deux règles
ci-dessous :

\begin{itemize}
\tightlist
\item
  les valeurs des attributs doivent être \textbf{atomiques},
  c'est-à-dire d'un type simple et non d'un type construit (pas de
  listes, de tableaux, de p-uplets, \ldots) ;
\item
  il n'y a pas de doublons : les p-uplets sont tous différents.
\end{itemize}

\end{tcolorbox}

Pour s'assurer qu'il n'y a pas de doublons dans une relation on ajoute
en pratique un critère d'unicité sur l'un des attributs, la
\textbf{clef}.

\hypertarget{clef-primaire-clef-uxe9tranguxe8re}{%
\subsection{2. Clef primaire, clef
étrangère}\label{clef-primaire-clef-uxe9tranguxe8re}}

\begin{tcolorbox}[enhanced jigsaw, colbacktitle=quarto-callout-tip-color!10!white, opacitybacktitle=0.6, left=2mm, coltitle=black, bottomtitle=1mm, arc=.35mm, opacityback=0, title=\textcolor{quarto-callout-tip-color}{\faLightbulb}\hspace{0.5em}{Définition}, breakable, toprule=.15mm, rightrule=.15mm, bottomrule=.15mm, colback=white, colframe=quarto-callout-tip-color-frame, toptitle=1mm, titlerule=0mm, leftrule=.75mm]

Dans une relation, une \textbf{clef primaire} est un attribut (ou un
groupes d'attributs) qui définit de manière unique chacun de p-uplets.

\end{tcolorbox}

En d'autres termes, il s'agit d'un attribut tel que \textbf{deux
p-uplets sont égaux si, et seulement si, ils ont la même clef primaire}.

Par exemple, dans la table des établissements scolaires du Cantal,
l'attribut ``code'' peut être choisi comme clef primaire, car il
identifie de façon unique chaque établissement. L'attribut ``commune''
ne peut pas être choisi comme clef primaire, car plusieurs
établissements existent dans une même commune.

On pourrait naïvement penser qu'il suffit de créer une unique relation
et de tout mettre dedans pour avoir une base de données. En fait, une
telle approche est inapplicable et il est indispensable de créer
plusieurs relations, associées les unes aux autres.

Prenons l'exemple des établissements scolaires, un parcours de la table
nous montre que certaines informations sont répétées plusieurs fois,
comme notamment le nom de la commune :

\begin{figure}

{\centering \includegraphics{BDD6.png}

}

\end{figure}

Cette duplication de l'information n'est pas souhaitable dans une base
de donnée. La solution pour éviter cela est de travailler avec deux
relations (deux tables) au lieu d'une, chacune des relations étant munie
d'une clef primaire.

Nous allons pour cela considérer la table des communes du Cantal qui
contient un attribut nommé ``Code commune'' qui peut être choisi comme
clef primaire. Dans la relation des établissements, nous remplaçons le
nom de la commune par la valeur du \emph{Code commune} correspondante
(attribut \emph{id\_commune}). Voici un extrait des deux relations
obtenues :

\begin{figure}

{\centering \includegraphics{BDD7.png}

}

\end{figure}

\begin{figure}

{\centering \includegraphics{BDD8.png}

}

\end{figure}

L'attribut \emph{id\_commune} permet de lier les deux relations : les
communes sont représentées dans la relation des établissements par leur
\emph{code} dans la relation des communes. On dit que l'attribut
\emph{id\_commune} est une \textbf{clef étrangère}.

\begin{tcolorbox}[enhanced jigsaw, colbacktitle=quarto-callout-tip-color!10!white, opacitybacktitle=0.6, left=2mm, coltitle=black, bottomtitle=1mm, arc=.35mm, opacityback=0, title=\textcolor{quarto-callout-tip-color}{\faLightbulb}\hspace{0.5em}{Définition}, breakable, toprule=.15mm, rightrule=.15mm, bottomrule=.15mm, colback=white, colframe=quarto-callout-tip-color-frame, toptitle=1mm, titlerule=0mm, leftrule=.75mm]

Soient deux relations \(R\) et \(S\) de clefs primaires respectives
\(c_R\) et \(c_S\).

Une \textbf{clef étrangère} de \(S\) dans \(R\) est un attribut \(ce\)
de \(R\) dont la valeur est toujours égale exactement à une des valeurs
de \(c_S\).

Autrement dit, \(ce\) correspond à un et un seul p-uplet de \(S\).

\end{tcolorbox}

Dans notre exemple, l'attribut \emph{id\_commune} est une clef étrangère
de la relation des communes dans la relation des établissements car
c'est un attribut de la relation des établissements dont la valeur est
toujours égale à une des valeurs de la clef primaire \emph{Code commune}
de la relation des communes.

Deux contraintes doivent toujours être vérifiées avec les clefs :

\begin{tcolorbox}[enhanced jigsaw, colbacktitle=quarto-callout-important-color!10!white, opacitybacktitle=0.6, left=2mm, coltitle=black, bottomtitle=1mm, arc=.35mm, opacityback=0, title=\textcolor{quarto-callout-important-color}{\faExclamation}\hspace{0.5em}{Contraintes liées aux clefs}, breakable, toprule=.15mm, rightrule=.15mm, bottomrule=.15mm, colback=white, colframe=quarto-callout-important-color-frame, toptitle=1mm, titlerule=0mm, leftrule=.75mm]

\begin{itemize}
\tightlist
\item
  \textbf{Contrainte d'unicité}: une valeur de clef ne peut apparaître
  qu'une fois dans une relation.
\item
  \textbf{Contrainte d'intégrité référentielle} : la valeur d'une clef
  étrangère doit toujours être également une des valeurs de la clef
  référencée.
\end{itemize}

\end{tcolorbox}

Ces deux contraintes garantissent l'absence totale de redondances et
d'incohérences.

\newpage{}

\hypertarget{schuxe9ma-relationnel}{%
\subsection{3. Schéma relationnel}\label{schuxe9ma-relationnel}}

Le schéma d'une base de données est constitué d'un ensemble de relations
: on parle de \textbf{schéma relationnel}.

Le schéma relationnel d'une base de données contient les informations
suivantes :

\begin{itemize}
\tightlist
\item
  Les noms des différentes relations ;
\item
  pour chaque relation, la liste des attributs avec leur domaine
  respectif ;
\item
  pour chaque relation, la clef primaire et éventuellement les clefs
  étrangères
\end{itemize}

Nommons ETABLISSEMENTS et COMMUNES les deux relations utilisées
ci-dessus. Le schéma relationnel peut s'écrire :

\begin{itemize}
\tightlist
\item
  ETABLISSEMENTS({code}: TEXT, nom: TEXT, statut: TEXT, codepostal: INT,
  \#id\_commune: INT, latitude: FLOAT, longitude: FLOAT)
\item
  COMMUNES({Code commune}: INT, Nom de la commune: TEXT, Population
  totale: INT)
\end{itemize}

Les attributs soulignés sont des clefs primaires, le \# signifie que
l'on a une clef étrangère.

Le schéma relationnel peut être représenté sous forme graphique (image
obtenue ici avec le logiciel
\href{https://dbschema.com/download.html}{DbSchema} à partir de la base
précédente légèrement modifiée : l'attribut Codepostal a été déplacé
dans la relation COMMUNES) :

\begin{figure}

{\centering \includegraphics{BDD11.png}

}

\end{figure}



\end{document}
