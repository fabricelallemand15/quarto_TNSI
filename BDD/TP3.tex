% Options for packages loaded elsewhere
\PassOptionsToPackage{unicode}{hyperref}
\PassOptionsToPackage{hyphens}{url}
\PassOptionsToPackage{dvipsnames,svgnames,x11names}{xcolor}
%
\documentclass[
  a4paper,
  DIV=11,
  numbers=noendperiod]{scrartcl}

\usepackage{amsmath,amssymb}
\usepackage{lmodern}
\usepackage{iftex}
\ifPDFTeX
  \usepackage[T1]{fontenc}
  \usepackage[utf8]{inputenc}
  \usepackage{textcomp} % provide euro and other symbols
\else % if luatex or xetex
  \usepackage{unicode-math}
  \defaultfontfeatures{Scale=MatchLowercase}
  \defaultfontfeatures[\rmfamily]{Ligatures=TeX,Scale=1}
\fi
% Use upquote if available, for straight quotes in verbatim environments
\IfFileExists{upquote.sty}{\usepackage{upquote}}{}
\IfFileExists{microtype.sty}{% use microtype if available
  \usepackage[]{microtype}
  \UseMicrotypeSet[protrusion]{basicmath} % disable protrusion for tt fonts
}{}
\makeatletter
\@ifundefined{KOMAClassName}{% if non-KOMA class
  \IfFileExists{parskip.sty}{%
    \usepackage{parskip}
  }{% else
    \setlength{\parindent}{0pt}
    \setlength{\parskip}{6pt plus 2pt minus 1pt}}
}{% if KOMA class
  \KOMAoptions{parskip=half}}
\makeatother
\usepackage{xcolor}
\usepackage[top=20mm,bottom=20mm,left=20mm,right=20mm,heightrounded]{geometry}
\setlength{\emergencystretch}{3em} % prevent overfull lines
\setcounter{secnumdepth}{-\maxdimen} % remove section numbering
% Make \paragraph and \subparagraph free-standing
\ifx\paragraph\undefined\else
  \let\oldparagraph\paragraph
  \renewcommand{\paragraph}[1]{\oldparagraph{#1}\mbox{}}
\fi
\ifx\subparagraph\undefined\else
  \let\oldsubparagraph\subparagraph
  \renewcommand{\subparagraph}[1]{\oldsubparagraph{#1}\mbox{}}
\fi

\usepackage{color}
\usepackage{fancyvrb}
\newcommand{\VerbBar}{|}
\newcommand{\VERB}{\Verb[commandchars=\\\{\}]}
\DefineVerbatimEnvironment{Highlighting}{Verbatim}{commandchars=\\\{\}}
% Add ',fontsize=\small' for more characters per line
\usepackage{framed}
\definecolor{shadecolor}{RGB}{241,243,245}
\newenvironment{Shaded}{\begin{snugshade}}{\end{snugshade}}
\newcommand{\AlertTok}[1]{\textcolor[rgb]{0.68,0.00,0.00}{#1}}
\newcommand{\AnnotationTok}[1]{\textcolor[rgb]{0.37,0.37,0.37}{#1}}
\newcommand{\AttributeTok}[1]{\textcolor[rgb]{0.40,0.45,0.13}{#1}}
\newcommand{\BaseNTok}[1]{\textcolor[rgb]{0.68,0.00,0.00}{#1}}
\newcommand{\BuiltInTok}[1]{\textcolor[rgb]{0.00,0.23,0.31}{#1}}
\newcommand{\CharTok}[1]{\textcolor[rgb]{0.13,0.47,0.30}{#1}}
\newcommand{\CommentTok}[1]{\textcolor[rgb]{0.37,0.37,0.37}{#1}}
\newcommand{\CommentVarTok}[1]{\textcolor[rgb]{0.37,0.37,0.37}{\textit{#1}}}
\newcommand{\ConstantTok}[1]{\textcolor[rgb]{0.56,0.35,0.01}{#1}}
\newcommand{\ControlFlowTok}[1]{\textcolor[rgb]{0.00,0.23,0.31}{#1}}
\newcommand{\DataTypeTok}[1]{\textcolor[rgb]{0.68,0.00,0.00}{#1}}
\newcommand{\DecValTok}[1]{\textcolor[rgb]{0.68,0.00,0.00}{#1}}
\newcommand{\DocumentationTok}[1]{\textcolor[rgb]{0.37,0.37,0.37}{\textit{#1}}}
\newcommand{\ErrorTok}[1]{\textcolor[rgb]{0.68,0.00,0.00}{#1}}
\newcommand{\ExtensionTok}[1]{\textcolor[rgb]{0.00,0.23,0.31}{#1}}
\newcommand{\FloatTok}[1]{\textcolor[rgb]{0.68,0.00,0.00}{#1}}
\newcommand{\FunctionTok}[1]{\textcolor[rgb]{0.28,0.35,0.67}{#1}}
\newcommand{\ImportTok}[1]{\textcolor[rgb]{0.00,0.46,0.62}{#1}}
\newcommand{\InformationTok}[1]{\textcolor[rgb]{0.37,0.37,0.37}{#1}}
\newcommand{\KeywordTok}[1]{\textcolor[rgb]{0.00,0.23,0.31}{#1}}
\newcommand{\NormalTok}[1]{\textcolor[rgb]{0.00,0.23,0.31}{#1}}
\newcommand{\OperatorTok}[1]{\textcolor[rgb]{0.37,0.37,0.37}{#1}}
\newcommand{\OtherTok}[1]{\textcolor[rgb]{0.00,0.23,0.31}{#1}}
\newcommand{\PreprocessorTok}[1]{\textcolor[rgb]{0.68,0.00,0.00}{#1}}
\newcommand{\RegionMarkerTok}[1]{\textcolor[rgb]{0.00,0.23,0.31}{#1}}
\newcommand{\SpecialCharTok}[1]{\textcolor[rgb]{0.37,0.37,0.37}{#1}}
\newcommand{\SpecialStringTok}[1]{\textcolor[rgb]{0.13,0.47,0.30}{#1}}
\newcommand{\StringTok}[1]{\textcolor[rgb]{0.13,0.47,0.30}{#1}}
\newcommand{\VariableTok}[1]{\textcolor[rgb]{0.07,0.07,0.07}{#1}}
\newcommand{\VerbatimStringTok}[1]{\textcolor[rgb]{0.13,0.47,0.30}{#1}}
\newcommand{\WarningTok}[1]{\textcolor[rgb]{0.37,0.37,0.37}{\textit{#1}}}

\providecommand{\tightlist}{%
  \setlength{\itemsep}{0pt}\setlength{\parskip}{0pt}}\usepackage{longtable,booktabs,array}
\usepackage{calc} % for calculating minipage widths
% Correct order of tables after \paragraph or \subparagraph
\usepackage{etoolbox}
\makeatletter
\patchcmd\longtable{\par}{\if@noskipsec\mbox{}\fi\par}{}{}
\makeatother
% Allow footnotes in longtable head/foot
\IfFileExists{footnotehyper.sty}{\usepackage{footnotehyper}}{\usepackage{footnote}}
\makesavenoteenv{longtable}
\usepackage{graphicx}
\makeatletter
\def\maxwidth{\ifdim\Gin@nat@width>\linewidth\linewidth\else\Gin@nat@width\fi}
\def\maxheight{\ifdim\Gin@nat@height>\textheight\textheight\else\Gin@nat@height\fi}
\makeatother
% Scale images if necessary, so that they will not overflow the page
% margins by default, and it is still possible to overwrite the defaults
% using explicit options in \includegraphics[width, height, ...]{}
\setkeys{Gin}{width=\maxwidth,height=\maxheight,keepaspectratio}
% Set default figure placement to htbp
\makeatletter
\def\fps@figure{htbp}
\makeatother
\newlength{\cslhangindent}
\setlength{\cslhangindent}{1.5em}
\newlength{\csllabelwidth}
\setlength{\csllabelwidth}{3em}
\newlength{\cslentryspacingunit} % times entry-spacing
\setlength{\cslentryspacingunit}{\parskip}
\newenvironment{CSLReferences}[2] % #1 hanging-ident, #2 entry spacing
 {% don't indent paragraphs
  \setlength{\parindent}{0pt}
  % turn on hanging indent if param 1 is 1
  \ifodd #1
  \let\oldpar\par
  \def\par{\hangindent=\cslhangindent\oldpar}
  \fi
  % set entry spacing
  \setlength{\parskip}{#2\cslentryspacingunit}
 }%
 {}
\usepackage{calc}
\newcommand{\CSLBlock}[1]{#1\hfill\break}
\newcommand{\CSLLeftMargin}[1]{\parbox[t]{\csllabelwidth}{#1}}
\newcommand{\CSLRightInline}[1]{\parbox[t]{\linewidth - \csllabelwidth}{#1}\break}
\newcommand{\CSLIndent}[1]{\hspace{\cslhangindent}#1}

\usepackage{fancyhdr} \pagestyle{fancy} \usepackage{lastpage}
\KOMAoption{captions}{tablesignature}
\makeatletter
\makeatother
\makeatletter
\makeatother
\makeatletter
\@ifpackageloaded{caption}{}{\usepackage{caption}}
\AtBeginDocument{%
\ifdefined\contentsname
  \renewcommand*\contentsname{Table des matières}
\else
  \newcommand\contentsname{Table des matières}
\fi
\ifdefined\listfigurename
  \renewcommand*\listfigurename{Liste des Figures}
\else
  \newcommand\listfigurename{Liste des Figures}
\fi
\ifdefined\listtablename
  \renewcommand*\listtablename{Liste des Tables}
\else
  \newcommand\listtablename{Liste des Tables}
\fi
\ifdefined\figurename
  \renewcommand*\figurename{Figure}
\else
  \newcommand\figurename{Figure}
\fi
\ifdefined\tablename
  \renewcommand*\tablename{Tableau}
\else
  \newcommand\tablename{Tableau}
\fi
}
\@ifpackageloaded{float}{}{\usepackage{float}}
\floatstyle{ruled}
\@ifundefined{c@chapter}{\newfloat{codelisting}{h}{lop}}{\newfloat{codelisting}{h}{lop}[chapter]}
\floatname{codelisting}{Listing}
\newcommand*\listoflistings{\listof{codelisting}{Liste des Listings}}
\makeatother
\makeatletter
\@ifpackageloaded{caption}{}{\usepackage{caption}}
\@ifpackageloaded{subcaption}{}{\usepackage{subcaption}}
\makeatother
\makeatletter
\@ifpackageloaded{tcolorbox}{}{\usepackage[many]{tcolorbox}}
\makeatother
\makeatletter
\@ifundefined{shadecolor}{\definecolor{shadecolor}{rgb}{.97, .97, .97}}
\makeatother
\makeatletter
\makeatother
\ifLuaTeX
\usepackage[bidi=basic]{babel}
\else
\usepackage[bidi=default]{babel}
\fi
\babelprovide[main,import]{french}
% get rid of language-specific shorthands (see #6817):
\let\LanguageShortHands\languageshorthands
\def\languageshorthands#1{}
\ifLuaTeX
  \usepackage{selnolig}  % disable illegal ligatures
\fi
\IfFileExists{bookmark.sty}{\usepackage{bookmark}}{\usepackage{hyperref}}
\IfFileExists{xurl.sty}{\usepackage{xurl}}{} % add URL line breaks if available
\urlstyle{same} % disable monospaced font for URLs
\hypersetup{
  pdftitle={TP 3 - BDD et Python},
  pdflang={fr},
  colorlinks=true,
  linkcolor={blue},
  filecolor={Maroon},
  citecolor={Blue},
  urlcolor={Blue},
  pdfcreator={LaTeX via pandoc}}

\title{TP 3 - BDD et Python}
\usepackage{etoolbox}
\makeatletter
\providecommand{\subtitle}[1]{% add subtitle to \maketitle
  \apptocmd{\@title}{\par {\large #1 \par}}{}{}
}
\makeatother
\subtitle{S3 - Bases de données}
\author{}
\date{}

\begin{document}
\maketitle
\lhead{Spécialité NSI} \rhead{Terminale} \chead{} \cfoot{} \lfoot{Lycée \'Emile Duclaux} \rfoot{Page \thepage/\pageref{LastPage}} \renewcommand{\headrulewidth}{0pt} \renewcommand{\footrulewidth}{0pt} \thispagestyle{fancy} \vspace{-2cm}

\ifdefined\Shaded\renewenvironment{Shaded}{\begin{tcolorbox}[borderline west={3pt}{0pt}{shadecolor}, frame hidden, sharp corners, breakable, boxrule=0pt, interior hidden, enhanced]}{\end{tcolorbox}}\fi

Dans ce TP (source : LASSUS (2021)), nous allons créer et interroger une
base de données \texttt{sqlite} avec le module \texttt{sqlite3} de
Python.

\hypertarget{cruxe9ation-dune-table}{%
\subsection{Création d'une table}\label{cruxe9ation-dune-table}}

\begin{Shaded}
\begin{Highlighting}[]
\ImportTok{import}\NormalTok{ sqlite3}

\CommentTok{\#Connexion}
\NormalTok{connexion }\OperatorTok{=}\NormalTok{ sqlite3.}\ExtensionTok{connect}\NormalTok{(}\StringTok{\textquotesingle{}mynewbase.db\textquotesingle{}}\NormalTok{)}

\CommentTok{\#Récupération d\textquotesingle{}un curseur}
\NormalTok{c }\OperatorTok{=}\NormalTok{ connexion.cursor()}

\CommentTok{\# {-}{-}{-}{-} début des instructions SQL}

\CommentTok{\#Création de la table}
\NormalTok{c.execute(}\StringTok{"""}
\StringTok{    CREATE TABLE IF NOT EXISTS bulletin(}
\StringTok{    Nom TEXT,}
\StringTok{    Prénom TEXT,}
\StringTok{    Note INT);}
\StringTok{    """}\NormalTok{)}

\CommentTok{\# {-}{-}{-}{-} fin des instructions SQL}

\CommentTok{\#Validation}
\NormalTok{connexion.commit()}


\CommentTok{\#Déconnexion}
\NormalTok{connexion.close()}
\end{Highlighting}
\end{Shaded}

\begin{itemize}
\tightlist
\item
  Le fichier \texttt{mynewbase.db} sera créé dans le même répertoire que
  le fichier source Python. Si le fichier existe déjà, il est ouvert et
  peut être modifié.
\item
  \texttt{IF\ NOT\ EXISTS} assure de ne pas écraser une table existante
  qui porterait le même nom. Si une telle table existe, elle n'est alors
  pas modifiée.
\item
  La nouvelle table peut être ouverte avec \texttt{DB\ Browser} pour
  vérifier sa structure et ses données.
\end{itemize}

\hypertarget{insertion-denregistrements-dans-la-table}{%
\subsection{Insertion d'enregistrements dans la
table}\label{insertion-denregistrements-dans-la-table}}

Les morceaux de code ci-dessous sont à positionner entre les balises
\texttt{\#\ -\/-\/-\/-\ début\ des\ instructions\ SQL} et
\texttt{\#\ -\/-\/-\/-\ fin\ des\ instructions\ SQL}.

\hypertarget{insertion-dun-enregistrement-unique}{%
\subsubsection{Insertion d'un enregistrement
unique}\label{insertion-dun-enregistrement-unique}}

\begin{Shaded}
\begin{Highlighting}[]
\NormalTok{c.execute(}\StringTok{\textquotesingle{}\textquotesingle{}\textquotesingle{}INSERT INTO bulletin VALUES (\textquotesingle{}Simpson\textquotesingle{}, \textquotesingle{}Bart\textquotesingle{}, 17)\textquotesingle{}\textquotesingle{}\textquotesingle{}}\NormalTok{)}
\end{Highlighting}
\end{Shaded}

Pensez à vérifier avec \texttt{DB\ Browser} si les modifications sont
effectives.

\hypertarget{insertion-dun-enregistrement-unique-avec-variable}{%
\subsubsection{Insertion d'un enregistrement unique avec
variable}\label{insertion-dun-enregistrement-unique-avec-variable}}

\begin{Shaded}
\begin{Highlighting}[]
\NormalTok{data }\OperatorTok{=}\NormalTok{ (}\StringTok{\textquotesingle{}Simpson\textquotesingle{}}\NormalTok{, }\StringTok{\textquotesingle{}Maggie\textquotesingle{}}\NormalTok{, }\DecValTok{2}\NormalTok{)}
\NormalTok{c.execute(}\StringTok{\textquotesingle{}\textquotesingle{}\textquotesingle{}INSERT INTO bulletin VALUES (?,?,?)\textquotesingle{}\textquotesingle{}\textquotesingle{}}\NormalTok{, data)}
\end{Highlighting}
\end{Shaded}

\hypertarget{insertion-de-multiples-enregistrements}{%
\subsubsection{Insertion de multiples
enregistrements}\label{insertion-de-multiples-enregistrements}}

\begin{Shaded}
\begin{Highlighting}[]
\NormalTok{lst\_notes }\OperatorTok{=}\NormalTok{ [ (}\StringTok{\textquotesingle{}Simpson\textquotesingle{}}\NormalTok{, }\StringTok{\textquotesingle{}Lisa\textquotesingle{}}\NormalTok{, }\DecValTok{19}\NormalTok{), (}\StringTok{\textquotesingle{}Muntz\textquotesingle{}}\NormalTok{, }\StringTok{\textquotesingle{}Nelson\textquotesingle{}}\NormalTok{, }\DecValTok{4}\NormalTok{), (}\StringTok{\textquotesingle{}Van Houten\textquotesingle{}}\NormalTok{, }\StringTok{\textquotesingle{}Milhouse\textquotesingle{}}\NormalTok{, }\DecValTok{12}\NormalTok{) ]}

\NormalTok{c.executemany(}\StringTok{\textquotesingle{}\textquotesingle{}\textquotesingle{}INSERT INTO bulletin VALUES (?, ?, ?)\textquotesingle{}\textquotesingle{}\textquotesingle{}}\NormalTok{, lst\_notes)}
\end{Highlighting}
\end{Shaded}

Les différentes valeurs sont stockées au préalable dans une liste de
tuples.

\hypertarget{mini-projet-1}{%
\subsubsection{Mini-projet 1}\label{mini-projet-1}}

Créer un programme qui demande à l'utilisateur un nom et une note, en
boucle. Les résultats sont stockés au fur et à mesure dans une base de
données. Si le nom est égal à «Q» ou «q», le programme s'arrête.

\hypertarget{exemple-dinjection-sql}{%
\subsubsection{Exemple d'injection SQL}\label{exemple-dinjection-sql}}

L'injection SQL est une technique consistant à écrire du code SQL à un
endroit qui n'est pas censé en recevoir.

\begin{itemize}
\tightlist
\item
  Créez un fichier contenant le code suivant :
\end{itemize}

\begin{Shaded}
\begin{Highlighting}[]
\ImportTok{import}\NormalTok{ sqlite3}

\CommentTok{\#Connexion}
\NormalTok{connexion }\OperatorTok{=}\NormalTok{ sqlite3.}\ExtensionTok{connect}\NormalTok{(}\StringTok{\textquotesingle{}mabasecobaye.db\textquotesingle{}}\NormalTok{)}

\CommentTok{\#Récupération d\textquotesingle{}un curseur}
\NormalTok{c }\OperatorTok{=}\NormalTok{ connexion.cursor()}

\NormalTok{c.execute(}\StringTok{"""}
\StringTok{    CREATE TABLE IF NOT EXISTS notes(}
\StringTok{    Nom TEXT,}
\StringTok{    Note INT);}
\StringTok{    """}\NormalTok{)}

\ControlFlowTok{while} \VariableTok{True}\NormalTok{ :}
\NormalTok{    nom }\OperatorTok{=} \BuiltInTok{input}\NormalTok{(}\StringTok{\textquotesingle{}Nom ? \textquotesingle{}}\NormalTok{)}
    \ControlFlowTok{if}\NormalTok{ nom }\KeywordTok{in}\NormalTok{ [}\StringTok{\textquotesingle{}Q\textquotesingle{}}\NormalTok{,}\StringTok{\textquotesingle{}q\textquotesingle{}}\NormalTok{] :}
        \ControlFlowTok{break}
\NormalTok{    note }\OperatorTok{=} \BuiltInTok{input}\NormalTok{(}\StringTok{\textquotesingle{}Note ? \textquotesingle{}}\NormalTok{)}
\NormalTok{    data }\OperatorTok{=}\NormalTok{ (nom, note)}
\NormalTok{    p }\OperatorTok{=} \StringTok{"INSERT INTO notes VALUES (\textquotesingle{}"} \OperatorTok{+}\NormalTok{ nom }\OperatorTok{+} \StringTok{"\textquotesingle{},\textquotesingle{}"} \OperatorTok{+}\NormalTok{ note }\OperatorTok{+} \StringTok{"\textquotesingle{})"}

\NormalTok{    c.executescript(p)}

\CommentTok{\#Validation}
\NormalTok{connexion.commit()}

\CommentTok{\#Déconnexion}
\NormalTok{connexion.close()}
\end{Highlighting}
\end{Shaded}

\begin{itemize}
\tightlist
\item
  Exécutez ce fichier, rentrez quelques valeurs, quittez et ouvrez dans
  \texttt{DB\ Browser} la table \texttt{notes} pour bien vérifier que
  vos valeurs ont bien été stockées.
\item
  Lancez à nouveau le fichier, en donnant ensuite comme nom la chaîne de
  caractères suivante :
  \texttt{g\textquotesingle{},\textquotesingle{}3\textquotesingle{});\ DROP\ TABLE\ notes;-\/-}
\item
  Donnez une note quelconque (par exemple 12), quittez le
  programme\ldots{} et allez observer l'état de la base de données. La
  table \texttt{notes} n'existe plus !
\end{itemize}

\textbf{Explication} :\\
La requête qui a été formulée est
\texttt{INSERT\ INTO\ notes\ VALUES\ (\textquotesingle{}g\textquotesingle{},\textquotesingle{}3\textquotesingle{});\ DROP\ TABLE\ notes;-\/-\textquotesingle{},\textquotesingle{}12\textquotesingle{})}

Dans un premier temps, le couple
\texttt{(\textquotesingle{}g\textquotesingle{},\textquotesingle{}3\textquotesingle{})}
a été inséré.\\
Puis l'ordre a été donné de détruire la table \texttt{notes}.\\
Le reste du code (qui n'est pas correct) est ignoré car \texttt{-\/-}
est le symbole du commentaire en SQL (l'équivalent du \# de Python).

\textbf{Remarques} : Évidemment, ce code a été fait spécifiquement pour
être vulnérable à l'injection SQL. Il suffit d'ailleurs de remplacer le
\texttt{c.executescript(p)} par \texttt{c.execute(p)} pour que le code
reste fonctionnel, mais refuse l'injection SQL. Ceci dit, de nombreux
serveurs sont encore attaqués par cette technique, au prix de
manipulations bien sûr plus complexes que celles que nous venons de
voir.

Rappelons enfin que ce genre de pratiques est interdit sur un serveur
qui ne vous appartient pas.

\hypertarget{lecture-des-enregistrements}{%
\subsection{Lecture des
enregistrements}\label{lecture-des-enregistrements}}

\begin{Shaded}
\begin{Highlighting}[]
\ImportTok{import}\NormalTok{ sqlite3}

\CommentTok{\#Connexion}
\NormalTok{connexion }\OperatorTok{=}\NormalTok{ sqlite3.}\ExtensionTok{connect}\NormalTok{(}\StringTok{\textquotesingle{}mynewbase.db\textquotesingle{}}\NormalTok{)}

\CommentTok{\#Récupération d\textquotesingle{}un curseur}
\NormalTok{c }\OperatorTok{=}\NormalTok{ connexion.cursor()}

\NormalTok{data }\OperatorTok{=}\NormalTok{ (}\StringTok{\textquotesingle{}Simpson\textquotesingle{}}\NormalTok{, )}

\NormalTok{c.execute(}\StringTok{"SELECT Prénom FROM Bulletin WHERE Nom = ?"}\NormalTok{, data)}
\BuiltInTok{print}\NormalTok{(c.fetchall())  }


\CommentTok{\#Déconnexion}
\NormalTok{connexion.close()}
\end{Highlighting}
\end{Shaded}

Ce code renvoie
\texttt{{[}(\textquotesingle{}Homer\textquotesingle{},),\ (\textquotesingle{}Lisa\textquotesingle{},),\ (\textquotesingle{}Maggie\textquotesingle{},){]}},
ou une liste vide s'il n'y a pas de résultat à la requête.

\hypertarget{mini-projet-2}{%
\subsubsection{Mini-projet 2}\label{mini-projet-2}}

Reprendre le mini-projet précédent, en rendant possible à l'utilisateur
de rentrer des notes ou bien de les consulter.

\emph{Exemple :}

\includegraphics[width=0.5\textwidth]{TP3_1-0.png}

\hypertarget{ruxe9fuxe9rences}{%
\subsection*{Références}\label{ruxe9fuxe9rences}}
\addcontentsline{toc}{subsection}{Références}

\hypertarget{refs}{}
\begin{CSLReferences}{1}{0}
\leavevmode\vadjust pre{\hypertarget{ref-lassus}{}}%
LASSUS, Gilles. 2021. {«~Terminale NSI - Lycée François Mauriac -
Bordeaux~»}. \url{https://glassus.github.io/terminale_nsi/}.

\end{CSLReferences}



\end{document}
