% Options for packages loaded elsewhere
\PassOptionsToPackage{unicode}{hyperref}
\PassOptionsToPackage{hyphens}{url}
\PassOptionsToPackage{dvipsnames,svgnames,x11names}{xcolor}
%
\documentclass[
  a4paper,
  DIV=11,
  numbers=noendperiod]{scrartcl}

\usepackage{amsmath,amssymb}
\usepackage{iftex}
\ifPDFTeX
  \usepackage[T1]{fontenc}
  \usepackage[utf8]{inputenc}
  \usepackage{textcomp} % provide euro and other symbols
\else % if luatex or xetex
  \usepackage{unicode-math}
  \defaultfontfeatures{Scale=MatchLowercase}
  \defaultfontfeatures[\rmfamily]{Ligatures=TeX,Scale=1}
\fi
\usepackage{lmodern}
\ifPDFTeX\else  
    % xetex/luatex font selection
\fi
% Use upquote if available, for straight quotes in verbatim environments
\IfFileExists{upquote.sty}{\usepackage{upquote}}{}
\IfFileExists{microtype.sty}{% use microtype if available
  \usepackage[]{microtype}
  \UseMicrotypeSet[protrusion]{basicmath} % disable protrusion for tt fonts
}{}
\makeatletter
\@ifundefined{KOMAClassName}{% if non-KOMA class
  \IfFileExists{parskip.sty}{%
    \usepackage{parskip}
  }{% else
    \setlength{\parindent}{0pt}
    \setlength{\parskip}{6pt plus 2pt minus 1pt}}
}{% if KOMA class
  \KOMAoptions{parskip=half}}
\makeatother
\usepackage{xcolor}
\usepackage[top=20mm,bottom=20mm,left=20mm,right=20mm,heightrounded]{geometry}
\setlength{\emergencystretch}{3em} % prevent overfull lines
\setcounter{secnumdepth}{-\maxdimen} % remove section numbering
% Make \paragraph and \subparagraph free-standing
\ifx\paragraph\undefined\else
  \let\oldparagraph\paragraph
  \renewcommand{\paragraph}[1]{\oldparagraph{#1}\mbox{}}
\fi
\ifx\subparagraph\undefined\else
  \let\oldsubparagraph\subparagraph
  \renewcommand{\subparagraph}[1]{\oldsubparagraph{#1}\mbox{}}
\fi

\usepackage{color}
\usepackage{fancyvrb}
\newcommand{\VerbBar}{|}
\newcommand{\VERB}{\Verb[commandchars=\\\{\}]}
\DefineVerbatimEnvironment{Highlighting}{Verbatim}{commandchars=\\\{\}}
% Add ',fontsize=\small' for more characters per line
\usepackage{framed}
\definecolor{shadecolor}{RGB}{241,243,245}
\newenvironment{Shaded}{\begin{snugshade}}{\end{snugshade}}
\newcommand{\AlertTok}[1]{\textcolor[rgb]{0.68,0.00,0.00}{#1}}
\newcommand{\AnnotationTok}[1]{\textcolor[rgb]{0.37,0.37,0.37}{#1}}
\newcommand{\AttributeTok}[1]{\textcolor[rgb]{0.40,0.45,0.13}{#1}}
\newcommand{\BaseNTok}[1]{\textcolor[rgb]{0.68,0.00,0.00}{#1}}
\newcommand{\BuiltInTok}[1]{\textcolor[rgb]{0.00,0.23,0.31}{#1}}
\newcommand{\CharTok}[1]{\textcolor[rgb]{0.13,0.47,0.30}{#1}}
\newcommand{\CommentTok}[1]{\textcolor[rgb]{0.37,0.37,0.37}{#1}}
\newcommand{\CommentVarTok}[1]{\textcolor[rgb]{0.37,0.37,0.37}{\textit{#1}}}
\newcommand{\ConstantTok}[1]{\textcolor[rgb]{0.56,0.35,0.01}{#1}}
\newcommand{\ControlFlowTok}[1]{\textcolor[rgb]{0.00,0.23,0.31}{#1}}
\newcommand{\DataTypeTok}[1]{\textcolor[rgb]{0.68,0.00,0.00}{#1}}
\newcommand{\DecValTok}[1]{\textcolor[rgb]{0.68,0.00,0.00}{#1}}
\newcommand{\DocumentationTok}[1]{\textcolor[rgb]{0.37,0.37,0.37}{\textit{#1}}}
\newcommand{\ErrorTok}[1]{\textcolor[rgb]{0.68,0.00,0.00}{#1}}
\newcommand{\ExtensionTok}[1]{\textcolor[rgb]{0.00,0.23,0.31}{#1}}
\newcommand{\FloatTok}[1]{\textcolor[rgb]{0.68,0.00,0.00}{#1}}
\newcommand{\FunctionTok}[1]{\textcolor[rgb]{0.28,0.35,0.67}{#1}}
\newcommand{\ImportTok}[1]{\textcolor[rgb]{0.00,0.46,0.62}{#1}}
\newcommand{\InformationTok}[1]{\textcolor[rgb]{0.37,0.37,0.37}{#1}}
\newcommand{\KeywordTok}[1]{\textcolor[rgb]{0.00,0.23,0.31}{#1}}
\newcommand{\NormalTok}[1]{\textcolor[rgb]{0.00,0.23,0.31}{#1}}
\newcommand{\OperatorTok}[1]{\textcolor[rgb]{0.37,0.37,0.37}{#1}}
\newcommand{\OtherTok}[1]{\textcolor[rgb]{0.00,0.23,0.31}{#1}}
\newcommand{\PreprocessorTok}[1]{\textcolor[rgb]{0.68,0.00,0.00}{#1}}
\newcommand{\RegionMarkerTok}[1]{\textcolor[rgb]{0.00,0.23,0.31}{#1}}
\newcommand{\SpecialCharTok}[1]{\textcolor[rgb]{0.37,0.37,0.37}{#1}}
\newcommand{\SpecialStringTok}[1]{\textcolor[rgb]{0.13,0.47,0.30}{#1}}
\newcommand{\StringTok}[1]{\textcolor[rgb]{0.13,0.47,0.30}{#1}}
\newcommand{\VariableTok}[1]{\textcolor[rgb]{0.07,0.07,0.07}{#1}}
\newcommand{\VerbatimStringTok}[1]{\textcolor[rgb]{0.13,0.47,0.30}{#1}}
\newcommand{\WarningTok}[1]{\textcolor[rgb]{0.37,0.37,0.37}{\textit{#1}}}

\providecommand{\tightlist}{%
  \setlength{\itemsep}{0pt}\setlength{\parskip}{0pt}}\usepackage{longtable,booktabs,array}
\usepackage{calc} % for calculating minipage widths
% Correct order of tables after \paragraph or \subparagraph
\usepackage{etoolbox}
\makeatletter
\patchcmd\longtable{\par}{\if@noskipsec\mbox{}\fi\par}{}{}
\makeatother
% Allow footnotes in longtable head/foot
\IfFileExists{footnotehyper.sty}{\usepackage{footnotehyper}}{\usepackage{footnote}}
\makesavenoteenv{longtable}
\usepackage{graphicx}
\makeatletter
\def\maxwidth{\ifdim\Gin@nat@width>\linewidth\linewidth\else\Gin@nat@width\fi}
\def\maxheight{\ifdim\Gin@nat@height>\textheight\textheight\else\Gin@nat@height\fi}
\makeatother
% Scale images if necessary, so that they will not overflow the page
% margins by default, and it is still possible to overwrite the defaults
% using explicit options in \includegraphics[width, height, ...]{}
\setkeys{Gin}{width=\maxwidth,height=\maxheight,keepaspectratio}
% Set default figure placement to htbp
\makeatletter
\def\fps@figure{htbp}
\makeatother

\usepackage{fancyhdr} \pagestyle{fancy} \usepackage{lastpage}
\KOMAoption{captions}{tablesignature}
\makeatletter
\makeatother
\makeatletter
\makeatother
\makeatletter
\@ifpackageloaded{caption}{}{\usepackage{caption}}
\AtBeginDocument{%
\ifdefined\contentsname
  \renewcommand*\contentsname{Table des matières}
\else
  \newcommand\contentsname{Table des matières}
\fi
\ifdefined\listfigurename
  \renewcommand*\listfigurename{Liste des Figures}
\else
  \newcommand\listfigurename{Liste des Figures}
\fi
\ifdefined\listtablename
  \renewcommand*\listtablename{Liste des Tables}
\else
  \newcommand\listtablename{Liste des Tables}
\fi
\ifdefined\figurename
  \renewcommand*\figurename{Figure}
\else
  \newcommand\figurename{Figure}
\fi
\ifdefined\tablename
  \renewcommand*\tablename{Tableau}
\else
  \newcommand\tablename{Tableau}
\fi
}
\@ifpackageloaded{float}{}{\usepackage{float}}
\floatstyle{ruled}
\@ifundefined{c@chapter}{\newfloat{codelisting}{h}{lop}}{\newfloat{codelisting}{h}{lop}[chapter]}
\floatname{codelisting}{Listing}
\newcommand*\listoflistings{\listof{codelisting}{Liste des Listings}}
\makeatother
\makeatletter
\@ifpackageloaded{caption}{}{\usepackage{caption}}
\@ifpackageloaded{subcaption}{}{\usepackage{subcaption}}
\makeatother
\makeatletter
\@ifpackageloaded{tcolorbox}{}{\usepackage[skins,breakable]{tcolorbox}}
\makeatother
\makeatletter
\@ifundefined{shadecolor}{\definecolor{shadecolor}{rgb}{.97, .97, .97}}
\makeatother
\makeatletter
\makeatother
\makeatletter
\makeatother
\makeatletter
\@ifpackageloaded{fontawesome5}{}{\usepackage{fontawesome5}}
\makeatother
\ifLuaTeX
\usepackage[bidi=basic]{babel}
\else
\usepackage[bidi=default]{babel}
\fi
\babelprovide[main,import]{french}
% get rid of language-specific shorthands (see #6817):
\let\LanguageShortHands\languageshorthands
\def\languageshorthands#1{}
\ifLuaTeX
  \usepackage{selnolig}  % disable illegal ligatures
\fi
\IfFileExists{bookmark.sty}{\usepackage{bookmark}}{\usepackage{hyperref}}
\IfFileExists{xurl.sty}{\usepackage{xurl}}{} % add URL line breaks if available
\urlstyle{same} % disable monospaced font for URLs
\hypersetup{
  pdftitle={Bases de données (Exercices)},
  pdflang={fr},
  colorlinks=true,
  linkcolor={blue},
  filecolor={Maroon},
  citecolor={Blue},
  urlcolor={Blue},
  pdfcreator={LaTeX via pandoc}}

\title{Bases de données (Exercices)}
\usepackage{etoolbox}
\makeatletter
\providecommand{\subtitle}[1]{% add subtitle to \maketitle
  \apptocmd{\@title}{\par {\large #1 \par}}{}{}
}
\makeatother
\subtitle{S3 - Bases de données}
\author{}
\date{}

\begin{document}
\maketitle
\lhead{Spécialité NSI} \rhead{Terminale} \chead{} \cfoot{} \lfoot{Lycée \'Emile Duclaux} \rfoot{Page \thepage/\pageref{LastPage}} \renewcommand{\headrulewidth}{0pt} \renewcommand{\footrulewidth}{0pt} \thispagestyle{fancy} \vspace{-3cm}

\ifdefined\Shaded\renewenvironment{Shaded}{\begin{tcolorbox}[borderline west={3pt}{0pt}{shadecolor}, sharp corners, interior hidden, boxrule=0pt, breakable, frame hidden, enhanced]}{\end{tcolorbox}}\fi

\emph{Les exercices précédés du symbole \faIcon{desktop} sont à faire
sur machine, en sauvegardant le fichier si nécessaire.}

\emph{Les exercices précédés du symbole \faIcon{pencil-alt} doivent être
résolus par écrit.}

\hypertarget{fa-solid-pencil-alt-exercice-1}{%
\subsection{\texorpdfstring{\faIcon{pencil-alt} Exercice
1}{ Exercice 1}}\label{fa-solid-pencil-alt-exercice-1}}

Voici un extrait d'une relation référençant des films :

\begin{longtable}[]{@{}lllll@{}}
\toprule\noalign{}
id & titre & realisateur & ann\_sortie & note\_sur\_10 \\
\midrule\noalign{}
\endhead
\bottomrule\noalign{}
\endlastfoot
1 & Alien, le huitième passager & Scott & 1979 & 10 \\
2 & Dune & Lynch & 1985 & 5 \\
3 & 2001 : l'odyssée de l'espace & Kubrick & 1968 & 9 \\
4 & Blade Runner & Scott & 1982 & 10 \\
\end{longtable}

Listez les différents attributs de cette relation. Donnez le domaine de
chaque attribut.

Pour chaque attribut dire si cet attribut peut jouer le rôle de clé
primaire, vous n'oublierez pas de justifier vos réponses.

\hypertarget{fa-solid-pencil-alt-exercice-2}{%
\subsection{\texorpdfstring{\faIcon{pencil-alt} Exercice
2}{ Exercice 2}}\label{fa-solid-pencil-alt-exercice-2}}

Un ski-club utilise une base de données constituée de 2 tables :

\begin{itemize}
\tightlist
\item
  une table ADHERENTS
\item
  une table STATIONS
\end{itemize}

Dans la table ADHERENTS on trouve un attribut ``ref\_station'' qui
permet de connaître les stations de ski préférées des adhérents.

Table ADHERENTS

\begin{longtable}[]{@{}lllll@{}}
\toprule\noalign{}
num\_licence & nom & prenom & annee\_naissance & ref\_station \\
\midrule\noalign{}
\endhead
\bottomrule\noalign{}
\endlastfoot
12558 & Doe & John & 1988 & 5 \\
13668 & Vect & Alice & 1974 & 6 \\
1777 & Dect & Bob & 1967 & 3 \\
13447 & Beau & Tristan & 1999 & 4 \\
1141 & Pabeau & John & 1975 & 3 \\
\end{longtable}

table STATIONS

\begin{longtable}[]{@{}lll@{}}
\toprule\noalign{}
ref & nom & altitude\_max \\
\midrule\noalign{}
\endhead
\bottomrule\noalign{}
\endlastfoot
3 & Le grand Bornand & 2050 \\
4 & La clusaz & 2616 \\
5 & Flaine & 2510 \\
6 & Avoriaz & 2466 \\
\end{longtable}

\begin{enumerate}
\def\labelenumi{\arabic{enumi}.}
\tightlist
\item
  Comment appelle-t-on l'attribut ref\_station de la table ADHERENTS ?
\item
  Écrire la requête SQL permettant d'obtenir le nom des stations ayant
  une altitude maxi strictement supérieure à 2500 m.
\item
  Écrire une requête SQL permettant d'obtenir le numéro de licence des
  adhérents nés après 1980 et ayant pour prénom John.
\item
  Donnez le résultat de la requête SQL suivante :
\end{enumerate}

\begin{Shaded}
\begin{Highlighting}[]
\KeywordTok{SELECT}\NormalTok{ nom }
\KeywordTok{FROM}\NormalTok{ ADHERENTS }
\KeywordTok{WHERE}\NormalTok{ num\_licence }\OperatorTok{\textgreater{}} \DecValTok{2000} \KeywordTok{OR}\NormalTok{  ref\_station }\OperatorTok{=} \DecValTok{3}
\end{Highlighting}
\end{Shaded}

\begin{enumerate}
\def\labelenumi{\arabic{enumi}.}
\setcounter{enumi}{4}
\tightlist
\item
  Donnez le résultat de la requête SQL suivante :
\end{enumerate}

\begin{Shaded}
\begin{Highlighting}[]
\KeywordTok{SELECT}\NormalTok{ STATIONS.nom}
\KeywordTok{FROM}\NormalTok{ STATIONS}
\KeywordTok{INNER} \KeywordTok{JOIN}\NormalTok{ ADHERENTS }\KeywordTok{ON}\NormalTok{ ADHERENTS.ref\_station }\OperatorTok{=}\NormalTok{ STATIONS.}\FunctionTok{ref}
\KeywordTok{WHERE}\NormalTok{ annee\_naissance }\OperatorTok{\textgreater{}} \DecValTok{1975}
\end{Highlighting}
\end{Shaded}

\hypertarget{fa-solid-pencil-alt-exercice-3-exercices-tiruxe9s-des-annales}{%
\subsection{\texorpdfstring{\faIcon{pencil-alt} Exercice 3 : Exercices
tirés des
annales}{ Exercice 3 : Exercices tirés des annales}}\label{fa-solid-pencil-alt-exercice-3-exercices-tiruxe9s-des-annales}}

\begin{enumerate}
\def\labelenumi{\arabic{enumi}.}
\tightlist
\item
  \href{../annales/2022_Metropole_Jour1.pdf}{Métropole 2022 Jour 1} :
  Exercice 2.
\item
  \href{../annales/2022_Metropole_Jour2.pdf}{Métropole 2022 Jour 2} :
  Exercice 4.
\item
  \href{../annales/2022_AmeriqueDuNord_1.pdf}{Amérique du Nord 2022 Jour
  1} : Exercice 1.
\item
  \href{../annales/2022_AmeriqueDuNord_2.pdf}{Amérique du Nord 2022 Jour
  2} : Exercice 3.
\item
  \href{../annales/2022_CentresEtrangers_1.pdf}{Centres étrangers 2022
  Jour 1} : Exercice 4.
\item
  \href{../annales/2022_CentresEtrangers_2.pdf}{Centres étrangers 2022
  Jour 2} : Exercice 3.
\end{enumerate}

\hypertarget{fa-desktop-exercice-4}{%
\subsection{\texorpdfstring{\faIcon{desktop} Exercice
4}{ Exercice 4}}\label{fa-desktop-exercice-4}}

Le CNAM (Conservatoire National des Arts et Métiers) propose en ligne
des travaux pratiques sur une base de données concernant les films de
cinéma. À titre d'entraînement, ouvrir cet exerciseur (cliquer sur
l'image ci-dessous) et essayer de formuler les requêtes correspondant
aux suggestions de la colonne de droite. Les réponses sont disponibles
sur le site, mais prenez le temps de chercher, d'essayer, et de vous
corriger.

\begin{figure}

{\centering 

\href{https://deptfod.cnam.fr/bd/tp/}{\includegraphics{CNAM_SQL.png}}

}

\end{figure}

\hypertarget{fa-desktop-probluxe8me}{%
\subsection{\texorpdfstring{\faIcon{desktop}
Problème}{ Problème}}\label{fa-desktop-probluxe8me}}

Serious game : meurtre à SQL City \ldots{}

\begin{figure}

{\centering 

\href{https://mystery.knightlab.com/}{\includegraphics{sql_city.png}}

}

\end{figure}

\hypertarget{fa-desktop-pour-les-plus-rapides}{%
\subsection{\texorpdfstring{\faIcon{desktop} Pour les plus
rapides}{ Pour les plus rapides}}\label{fa-desktop-pour-les-plus-rapides}}

Serious game : SQL Island \ldots{}

\begin{figure}

{\centering 

\href{https://sql-island.informatik.uni-kl.de/}{\includegraphics{sql_island.png}}

}

\end{figure}



\end{document}
