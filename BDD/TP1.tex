% Options for packages loaded elsewhere
\PassOptionsToPackage{unicode}{hyperref}
\PassOptionsToPackage{hyphens}{url}
\PassOptionsToPackage{dvipsnames,svgnames,x11names}{xcolor}
%
\documentclass[
  a4paper,
  DIV=11,
  numbers=noendperiod]{scrartcl}

\usepackage{amsmath,amssymb}
\usepackage{iftex}
\ifPDFTeX
  \usepackage[T1]{fontenc}
  \usepackage[utf8]{inputenc}
  \usepackage{textcomp} % provide euro and other symbols
\else % if luatex or xetex
  \usepackage{unicode-math}
  \defaultfontfeatures{Scale=MatchLowercase}
  \defaultfontfeatures[\rmfamily]{Ligatures=TeX,Scale=1}
\fi
\usepackage{lmodern}
\ifPDFTeX\else  
    % xetex/luatex font selection
\fi
% Use upquote if available, for straight quotes in verbatim environments
\IfFileExists{upquote.sty}{\usepackage{upquote}}{}
\IfFileExists{microtype.sty}{% use microtype if available
  \usepackage[]{microtype}
  \UseMicrotypeSet[protrusion]{basicmath} % disable protrusion for tt fonts
}{}
\makeatletter
\@ifundefined{KOMAClassName}{% if non-KOMA class
  \IfFileExists{parskip.sty}{%
    \usepackage{parskip}
  }{% else
    \setlength{\parindent}{0pt}
    \setlength{\parskip}{6pt plus 2pt minus 1pt}}
}{% if KOMA class
  \KOMAoptions{parskip=half}}
\makeatother
\usepackage{xcolor}
\usepackage[top=20mm,bottom=20mm,left=20mm,right=20mm,heightrounded]{geometry}
\setlength{\emergencystretch}{3em} % prevent overfull lines
\setcounter{secnumdepth}{-\maxdimen} % remove section numbering
% Make \paragraph and \subparagraph free-standing
\ifx\paragraph\undefined\else
  \let\oldparagraph\paragraph
  \renewcommand{\paragraph}[1]{\oldparagraph{#1}\mbox{}}
\fi
\ifx\subparagraph\undefined\else
  \let\oldsubparagraph\subparagraph
  \renewcommand{\subparagraph}[1]{\oldsubparagraph{#1}\mbox{}}
\fi

\usepackage{color}
\usepackage{fancyvrb}
\newcommand{\VerbBar}{|}
\newcommand{\VERB}{\Verb[commandchars=\\\{\}]}
\DefineVerbatimEnvironment{Highlighting}{Verbatim}{commandchars=\\\{\}}
% Add ',fontsize=\small' for more characters per line
\usepackage{framed}
\definecolor{shadecolor}{RGB}{241,243,245}
\newenvironment{Shaded}{\begin{snugshade}}{\end{snugshade}}
\newcommand{\AlertTok}[1]{\textcolor[rgb]{0.68,0.00,0.00}{#1}}
\newcommand{\AnnotationTok}[1]{\textcolor[rgb]{0.37,0.37,0.37}{#1}}
\newcommand{\AttributeTok}[1]{\textcolor[rgb]{0.40,0.45,0.13}{#1}}
\newcommand{\BaseNTok}[1]{\textcolor[rgb]{0.68,0.00,0.00}{#1}}
\newcommand{\BuiltInTok}[1]{\textcolor[rgb]{0.00,0.23,0.31}{#1}}
\newcommand{\CharTok}[1]{\textcolor[rgb]{0.13,0.47,0.30}{#1}}
\newcommand{\CommentTok}[1]{\textcolor[rgb]{0.37,0.37,0.37}{#1}}
\newcommand{\CommentVarTok}[1]{\textcolor[rgb]{0.37,0.37,0.37}{\textit{#1}}}
\newcommand{\ConstantTok}[1]{\textcolor[rgb]{0.56,0.35,0.01}{#1}}
\newcommand{\ControlFlowTok}[1]{\textcolor[rgb]{0.00,0.23,0.31}{#1}}
\newcommand{\DataTypeTok}[1]{\textcolor[rgb]{0.68,0.00,0.00}{#1}}
\newcommand{\DecValTok}[1]{\textcolor[rgb]{0.68,0.00,0.00}{#1}}
\newcommand{\DocumentationTok}[1]{\textcolor[rgb]{0.37,0.37,0.37}{\textit{#1}}}
\newcommand{\ErrorTok}[1]{\textcolor[rgb]{0.68,0.00,0.00}{#1}}
\newcommand{\ExtensionTok}[1]{\textcolor[rgb]{0.00,0.23,0.31}{#1}}
\newcommand{\FloatTok}[1]{\textcolor[rgb]{0.68,0.00,0.00}{#1}}
\newcommand{\FunctionTok}[1]{\textcolor[rgb]{0.28,0.35,0.67}{#1}}
\newcommand{\ImportTok}[1]{\textcolor[rgb]{0.00,0.46,0.62}{#1}}
\newcommand{\InformationTok}[1]{\textcolor[rgb]{0.37,0.37,0.37}{#1}}
\newcommand{\KeywordTok}[1]{\textcolor[rgb]{0.00,0.23,0.31}{#1}}
\newcommand{\NormalTok}[1]{\textcolor[rgb]{0.00,0.23,0.31}{#1}}
\newcommand{\OperatorTok}[1]{\textcolor[rgb]{0.37,0.37,0.37}{#1}}
\newcommand{\OtherTok}[1]{\textcolor[rgb]{0.00,0.23,0.31}{#1}}
\newcommand{\PreprocessorTok}[1]{\textcolor[rgb]{0.68,0.00,0.00}{#1}}
\newcommand{\RegionMarkerTok}[1]{\textcolor[rgb]{0.00,0.23,0.31}{#1}}
\newcommand{\SpecialCharTok}[1]{\textcolor[rgb]{0.37,0.37,0.37}{#1}}
\newcommand{\SpecialStringTok}[1]{\textcolor[rgb]{0.13,0.47,0.30}{#1}}
\newcommand{\StringTok}[1]{\textcolor[rgb]{0.13,0.47,0.30}{#1}}
\newcommand{\VariableTok}[1]{\textcolor[rgb]{0.07,0.07,0.07}{#1}}
\newcommand{\VerbatimStringTok}[1]{\textcolor[rgb]{0.13,0.47,0.30}{#1}}
\newcommand{\WarningTok}[1]{\textcolor[rgb]{0.37,0.37,0.37}{\textit{#1}}}

\providecommand{\tightlist}{%
  \setlength{\itemsep}{0pt}\setlength{\parskip}{0pt}}\usepackage{longtable,booktabs,array}
\usepackage{calc} % for calculating minipage widths
% Correct order of tables after \paragraph or \subparagraph
\usepackage{etoolbox}
\makeatletter
\patchcmd\longtable{\par}{\if@noskipsec\mbox{}\fi\par}{}{}
\makeatother
% Allow footnotes in longtable head/foot
\IfFileExists{footnotehyper.sty}{\usepackage{footnotehyper}}{\usepackage{footnote}}
\makesavenoteenv{longtable}
\usepackage{graphicx}
\makeatletter
\def\maxwidth{\ifdim\Gin@nat@width>\linewidth\linewidth\else\Gin@nat@width\fi}
\def\maxheight{\ifdim\Gin@nat@height>\textheight\textheight\else\Gin@nat@height\fi}
\makeatother
% Scale images if necessary, so that they will not overflow the page
% margins by default, and it is still possible to overwrite the defaults
% using explicit options in \includegraphics[width, height, ...]{}
\setkeys{Gin}{width=\maxwidth,height=\maxheight,keepaspectratio}
% Set default figure placement to htbp
\makeatletter
\def\fps@figure{htbp}
\makeatother

\usepackage{fancyhdr} \pagestyle{fancy} \usepackage{lastpage}
\KOMAoption{captions}{tablesignature}
\makeatletter
\@ifpackageloaded{tcolorbox}{}{\usepackage[skins,breakable]{tcolorbox}}
\@ifpackageloaded{fontawesome5}{}{\usepackage{fontawesome5}}
\definecolor{quarto-callout-color}{HTML}{909090}
\definecolor{quarto-callout-note-color}{HTML}{0758E5}
\definecolor{quarto-callout-important-color}{HTML}{CC1914}
\definecolor{quarto-callout-warning-color}{HTML}{EB9113}
\definecolor{quarto-callout-tip-color}{HTML}{00A047}
\definecolor{quarto-callout-caution-color}{HTML}{FC5300}
\definecolor{quarto-callout-color-frame}{HTML}{acacac}
\definecolor{quarto-callout-note-color-frame}{HTML}{4582ec}
\definecolor{quarto-callout-important-color-frame}{HTML}{d9534f}
\definecolor{quarto-callout-warning-color-frame}{HTML}{f0ad4e}
\definecolor{quarto-callout-tip-color-frame}{HTML}{02b875}
\definecolor{quarto-callout-caution-color-frame}{HTML}{fd7e14}
\makeatother
\makeatletter
\makeatother
\makeatletter
\makeatother
\makeatletter
\@ifpackageloaded{caption}{}{\usepackage{caption}}
\AtBeginDocument{%
\ifdefined\contentsname
  \renewcommand*\contentsname{Table des matières}
\else
  \newcommand\contentsname{Table des matières}
\fi
\ifdefined\listfigurename
  \renewcommand*\listfigurename{Liste des Figures}
\else
  \newcommand\listfigurename{Liste des Figures}
\fi
\ifdefined\listtablename
  \renewcommand*\listtablename{Liste des Tables}
\else
  \newcommand\listtablename{Liste des Tables}
\fi
\ifdefined\figurename
  \renewcommand*\figurename{Figure}
\else
  \newcommand\figurename{Figure}
\fi
\ifdefined\tablename
  \renewcommand*\tablename{Tableau}
\else
  \newcommand\tablename{Tableau}
\fi
}
\@ifpackageloaded{float}{}{\usepackage{float}}
\floatstyle{ruled}
\@ifundefined{c@chapter}{\newfloat{codelisting}{h}{lop}}{\newfloat{codelisting}{h}{lop}[chapter]}
\floatname{codelisting}{Listing}
\newcommand*\listoflistings{\listof{codelisting}{Liste des Listings}}
\makeatother
\makeatletter
\@ifpackageloaded{caption}{}{\usepackage{caption}}
\@ifpackageloaded{subcaption}{}{\usepackage{subcaption}}
\makeatother
\makeatletter
\@ifpackageloaded{tcolorbox}{}{\usepackage[skins,breakable]{tcolorbox}}
\makeatother
\makeatletter
\@ifundefined{shadecolor}{\definecolor{shadecolor}{rgb}{.97, .97, .97}}
\makeatother
\makeatletter
\makeatother
\makeatletter
\makeatother
\makeatletter
\@ifpackageloaded{fontawesome5}{}{\usepackage{fontawesome5}}
\makeatother
\ifLuaTeX
\usepackage[bidi=basic]{babel}
\else
\usepackage[bidi=default]{babel}
\fi
\babelprovide[main,import]{french}
% get rid of language-specific shorthands (see #6817):
\let\LanguageShortHands\languageshorthands
\def\languageshorthands#1{}
\ifLuaTeX
  \usepackage{selnolig}  % disable illegal ligatures
\fi
\IfFileExists{bookmark.sty}{\usepackage{bookmark}}{\usepackage{hyperref}}
\IfFileExists{xurl.sty}{\usepackage{xurl}}{} % add URL line breaks if available
\urlstyle{same} % disable monospaced font for URLs
\hypersetup{
  pdftitle={TP 1 - Créer une base de données},
  pdflang={fr},
  colorlinks=true,
  linkcolor={blue},
  filecolor={Maroon},
  citecolor={Blue},
  urlcolor={Blue},
  pdfcreator={LaTeX via pandoc}}

\title{TP 1 - Créer une base de données}
\usepackage{etoolbox}
\makeatletter
\providecommand{\subtitle}[1]{% add subtitle to \maketitle
  \apptocmd{\@title}{\par {\large #1 \par}}{}{}
}
\makeatother
\subtitle{S3 - Bases de données}
\author{}
\date{}

\begin{document}
\maketitle
\lhead{Spécialité NSI} \rhead{Terminale} \chead{} \cfoot{} \lfoot{Lycée \'Emile Duclaux} \rfoot{Page \thepage/\pageref{LastPage}} \renewcommand{\headrulewidth}{0pt} \renewcommand{\footrulewidth}{0pt} \thispagestyle{fancy} \vspace{-2cm}

\ifdefined\Shaded\renewenvironment{Shaded}{\begin{tcolorbox}[sharp corners, borderline west={3pt}{0pt}{shadecolor}, boxrule=0pt, interior hidden, breakable, frame hidden, enhanced]}{\end{tcolorbox}}\fi

\begin{tcolorbox}[enhanced jigsaw, colframe=quarto-callout-tip-color-frame, leftrule=.75mm, opacitybacktitle=0.6, breakable, arc=.35mm, titlerule=0mm, opacityback=0, colback=white, left=2mm, coltitle=black, bottomtitle=1mm, colbacktitle=quarto-callout-tip-color!10!white, bottomrule=.15mm, toptitle=1mm, toprule=.15mm, title=\textcolor{quarto-callout-tip-color}{\faLightbulb}\hspace{0.5em}{Objectifs}, rightrule=.15mm]

Utiliser le logiciel ``DB Browser for SqLite'' et le langage SQL pour :

\begin{itemize}
\tightlist
\item
  \faIcon{check} créer une base de données ;
\item
  \faIcon{check} ajouter des données dans une table ;
\item
  \faIcon{check} écrire et tester différentes requêtes.
\end{itemize}

\end{tcolorbox}

\hypertarget{cruxe9ation-dune-bdd-et-insertion-de-valeurs}{%
\subsection{1. Création d'une BDD et insertion de
valeurs}\label{cruxe9ation-dune-bdd-et-insertion-de-valeurs}}

Pour créer une base de données et effectuer des requêtes sur cette
dernière, nous allons utiliser le logiciel ``DB Browser for SQLite'' :
\url{https://sqlitebrowser.org/}. Ce logiciel est intégré dans
EduPython.

\begin{enumerate}
\def\labelenumi{\arabic{enumi}.}
\item
  Ouvrez le logiciel, puis cliquez sur ``Nouvelle base de données''.
  Après avoir choisi un nom pour votre base de données (par exemple
  ``db\_livres.db''), vous devriez avoir la fenêtre suivante :

  \begin{figure}

  {\centering \includegraphics[width=0.5\textwidth,height=\textheight]{TP1_1.png}

  }

  \end{figure}

  Cliquez alors sur ``Annuler''.

  Une nouvelle base de donnée a bien été créée, mais elle ne contient
  encore aucune table.

  \begin{figure}

  {\centering \includegraphics{TP1_2.png}

  }

  \end{figure}
\item
  Pour créer une table, cliquez sur l'onglet ``Exécuter le SQL''. On
  obtient alors :

  \begin{figure}

  {\centering \includegraphics[width=0.5\textwidth,height=\textheight]{TP1_3.png}

  }

  \end{figure}

  Copiez-collez le texte ci-dessous dans la fenêtre ``SQL 1'' :

\begin{Shaded}
\begin{Highlighting}[]
\KeywordTok{CREATE} \KeywordTok{TABLE}\NormalTok{ LIVRES}
\NormalTok{    (}\KeywordTok{id} \DataTypeTok{INT}\NormalTok{, titre TEXT, auteur TEXT, ann\_publi }\DataTypeTok{INT}\NormalTok{, note }\DataTypeTok{INT}\NormalTok{, }\KeywordTok{PRIMARY} \KeywordTok{KEY}\NormalTok{ (}\KeywordTok{id}\NormalTok{));}
\end{Highlighting}
\end{Shaded}

  Cliquez ensuite sur le petit triangle situé au-dessus de la fenêtre
  SQL 1 (ou appuyez sur F5), vous devriez avoir ceci :

  \begin{figure}

  {\centering \includegraphics[width=0.5\textwidth,height=\textheight]{TP1_4.png}

  }

  \end{figure}

  Comme indiqué dans la fenêtre, ``Requête exécutée avec succès'' !

  Quelques explications : la commande \textbf{CREATE TABLE LIVRES}
  permet de créer une nouvelle table nommée ``LIVRES''. Elle est suivie
  d'un p-uplet définissant les noms et les domaines des attributs de la
  nouvelle table :

  \begin{itemize}
  \tightlist
  \item
    \textbf{id} est un entier ;
  \item
    \textbf{titre} est une chaîne de caractères ;
  \item
    \textbf{auteur} est une chaîne de caractères ;
  \item
    \textbf{ann\_pulbi} est un entier ;
  \item
    \textbf{note} est un entier ;
  \end{itemize}

  L'attribut ``id'' va jouer le rôle de \textbf{clé primaire}, nous
  avons donc ajouté dans la requête la mention \textbf{(PRIMARY KEY
  (id))}. Le système de gestion de base de données nous avertira si l'on
  tente d'attribuer 2 fois la même valeur à l'attribut ``id''.
\item
  Nous allons maintenant ajouter des données à la table \textbf{LIVRES}.

  Toujours dans l'onglet ``Exécuter le SQL'', après avoir effacé la
  fenêtre SQL 1, copiez-collez dans cette même fenêtre la requête
  ci-dessous :

\begin{Shaded}
\begin{Highlighting}[]
\KeywordTok{INSERT} \KeywordTok{INTO}\NormalTok{ LIVRES}
\NormalTok{    (}\KeywordTok{id}\NormalTok{,titre,auteur,ann\_publi,note)}
    \KeywordTok{VALUES}
\NormalTok{    (}\DecValTok{1}\NormalTok{,}\OtherTok{"1984"}\NormalTok{,}\OtherTok{"Orwell"}\NormalTok{,}\DecValTok{1949}\NormalTok{,}\DecValTok{10}\NormalTok{),}
\NormalTok{    (}\DecValTok{2}\NormalTok{,}\OtherTok{"Dune"}\NormalTok{,}\OtherTok{"Herbert"}\NormalTok{,}\DecValTok{1965}\NormalTok{,}\DecValTok{8}\NormalTok{),}
\NormalTok{    (}\DecValTok{3}\NormalTok{,}\OtherTok{"Fondation"}\NormalTok{,}\OtherTok{"Asimov"}\NormalTok{,}\DecValTok{1951}\NormalTok{,}\DecValTok{9}\NormalTok{),}
\NormalTok{    (}\DecValTok{4}\NormalTok{,}\OtherTok{"Le meilleur des mondes"}\NormalTok{,}\OtherTok{"Huxley"}\NormalTok{,}\DecValTok{1931}\NormalTok{,}\DecValTok{7}\NormalTok{),}
\NormalTok{    (}\DecValTok{5}\NormalTok{,}\OtherTok{"Fahrenheit 451"}\NormalTok{,}\OtherTok{"Bradbury"}\NormalTok{,}\DecValTok{1953}\NormalTok{,}\DecValTok{7}\NormalTok{),}
\NormalTok{    (}\DecValTok{6}\NormalTok{,}\OtherTok{"Ubik"}\NormalTok{,}\OtherTok{"K.Dick"}\NormalTok{,}\DecValTok{1969}\NormalTok{,}\DecValTok{9}\NormalTok{),}
\NormalTok{    (}\DecValTok{7}\NormalTok{,}\OtherTok{"Chroniques martiennes"}\NormalTok{,}\OtherTok{"Bradbury"}\NormalTok{,}\DecValTok{1950}\NormalTok{,}\DecValTok{8}\NormalTok{),}
\NormalTok{    (}\DecValTok{8}\NormalTok{,}\OtherTok{"La nuit des temps"}\NormalTok{,}\OtherTok{"Barjavel"}\NormalTok{,}\DecValTok{1968}\NormalTok{,}\DecValTok{7}\NormalTok{),}
\NormalTok{    (}\DecValTok{9}\NormalTok{,}\OtherTok{"Blade Runner"}\NormalTok{,}\OtherTok{"K.Dick"}\NormalTok{,}\DecValTok{1968}\NormalTok{,}\DecValTok{8}\NormalTok{),}
\NormalTok{    (}\DecValTok{10}\NormalTok{,}\OtherTok{"Les Robots"}\NormalTok{,}\OtherTok{"Asimov"}\NormalTok{,}\DecValTok{1950}\NormalTok{,}\DecValTok{9}\NormalTok{),}
\NormalTok{    (}\DecValTok{11}\NormalTok{,}\OtherTok{"La Planète des singes"}\NormalTok{,}\OtherTok{"Boulle"}\NormalTok{,}\DecValTok{1963}\NormalTok{,}\DecValTok{8}\NormalTok{),}
\NormalTok{    (}\DecValTok{12}\NormalTok{,}\OtherTok{"Ravage"}\NormalTok{,}\OtherTok{"Barjavel"}\NormalTok{,}\DecValTok{1943}\NormalTok{,}\DecValTok{8}\NormalTok{),}
\NormalTok{    (}\DecValTok{13}\NormalTok{,}\OtherTok{"Le Maître du Haut Château"}\NormalTok{,}\OtherTok{"K.Dick"}\NormalTok{,}\DecValTok{1962}\NormalTok{,}\DecValTok{8}\NormalTok{),}
\NormalTok{    (}\DecValTok{14}\NormalTok{,}\OtherTok{"Le monde des Ā"}\NormalTok{,}\OtherTok{"Van Vogt"}\NormalTok{,}\DecValTok{1945}\NormalTok{,}\DecValTok{7}\NormalTok{),}
\NormalTok{    (}\DecValTok{15}\NormalTok{,}\OtherTok{"La Fin de l\textquotesingle{}éternité"}\NormalTok{,}\OtherTok{"Asimov"}\NormalTok{,}\DecValTok{1955}\NormalTok{,}\DecValTok{8}\NormalTok{),}
\NormalTok{    (}\DecValTok{16}\NormalTok{,}\OtherTok{"De la Terre à la Lune"}\NormalTok{,}\OtherTok{"Verne"}\NormalTok{,}\DecValTok{1865}\NormalTok{,}\DecValTok{10}\NormalTok{);}
\end{Highlighting}
\end{Shaded}

  Un message devrait vous préciser que votre requête a été exécutée avec
  succès :

  \begin{figure}

  {\centering \includegraphics[width=0.5\textwidth,height=\textheight]{TP1_5.png}

  }

  \end{figure}

  La table LIVRES contient maintenant les données souhaitées (onglet
  ``Parcourir les données'') :

  \begin{figure}

  {\centering \includegraphics[width=0.5\textwidth,height=\textheight]{TP1_6.png}

  }

  \end{figure}
\item
  Saisissez et exécutez la requête SQL suivante :

\begin{Shaded}
\begin{Highlighting}[]
\KeywordTok{SELECT} \KeywordTok{id}\NormalTok{, titre, auteur, ann\_publi, note}
\KeywordTok{FROM}\NormalTok{ LIVRES}
\end{Highlighting}
\end{Shaded}

  Après un temps plus ou moins long, vous devriez voir s'afficher ceci :

  \begin{figure}

  {\centering \includegraphics[width=0.5\textwidth,height=\textheight]{TP1_7.png}

  }

  \end{figure}
\item
  Effectuez une requête qui permettra d'obtenir le titre et l'auteur de
  tous les livres présents dans la table LIVRES.
\item
  Saisissez et testez la requête SQL suivante :

\begin{Shaded}
\begin{Highlighting}[]
\KeywordTok{SELECT}\NormalTok{ titre, ann\_publi}
\KeywordTok{FROM}\NormalTok{ LIVRES}
\KeywordTok{WHERE}\NormalTok{ auteur}\OperatorTok{=}\StringTok{\textquotesingle{}Asimov\textquotesingle{}}
\end{Highlighting}
\end{Shaded}

  À quelle question répond-elle ?
\item
  Écrivez et testez une requête permettant d'obtenir uniquement les
  titres des livres écrits par Philip K.Dick.
\item
  Saisissez et testez la requête SQL suivante :

\begin{Shaded}
\begin{Highlighting}[]
\KeywordTok{SELECT}\NormalTok{ titre, ann\_publi}
\KeywordTok{FROM}\NormalTok{ LIVRES}
\KeywordTok{WHERE}\NormalTok{ auteur}\OperatorTok{=}\StringTok{\textquotesingle{}Asimov\textquotesingle{}} \KeywordTok{AND}\NormalTok{ ann\_publi}\OperatorTok{\textgreater{}}\DecValTok{1953}
\end{Highlighting}
\end{Shaded}

  À quelle question répond-elle ?
\item
  Écrivez une requête permettant d'obtenir les titres des livres publiés
  après 1945 qui ont une note supérieure ou égale à 9.
\item
  Écrivez une requête SQL permettant d'obtenir les titres et les années
  de publication des livres de K.Dick classés du plus ancien ou plus
  récent.
\end{enumerate}

\hypertarget{avec-deux-tables}{%
\subsection{2. Avec deux tables}\label{avec-deux-tables}}

Dans la première partie, nous avons une redondance d'information dans
l'attribut \textbf{auteur}, un même auteur étant répété plusieurs fois.
Pour remédier à cela, nous allons maintenant créer une nouvelle base
avec deux tables \textbf{AUTEURS} et \textbf{LIVRES} reliées par une
\textbf{clef étrangère}.

\begin{figure}

{\centering \includegraphics{TP1_8.png}

}

\end{figure}

\begin{enumerate}
\def\labelenumi{\arabic{enumi}.}
\item
  Créez une nouvelle base de données que vous nommerez par exemple
  \textbf{db\_livres\_auteurs.db}, puis créez une table AUTEURS à l'aide
  de la requête SQL suivante :

\begin{Shaded}
\begin{Highlighting}[]
\KeywordTok{CREATE} \KeywordTok{TABLE}\NormalTok{ AUTEURS}
\NormalTok{(}\KeywordTok{id} \DataTypeTok{INT}\NormalTok{, nom TEXT, prenom TEXT, ann\_naissance }\DataTypeTok{INT}\NormalTok{, langue\_ecriture TEXT, }\KeywordTok{PRIMARY} \KeywordTok{KEY}\NormalTok{ (}\KeywordTok{id}\NormalTok{));}
\end{Highlighting}
\end{Shaded}

  Créez ensuite une deuxième table (LIVRES) :

\begin{Shaded}
\begin{Highlighting}[]
\KeywordTok{CREATE} \KeywordTok{TABLE}\NormalTok{ LIVRES}
\NormalTok{(}\KeywordTok{id} \DataTypeTok{INT}\NormalTok{, titre TEXT, id\_auteur }\DataTypeTok{INT}\NormalTok{, ann\_publi }\DataTypeTok{INT}\NormalTok{, note }\DataTypeTok{INT}\NormalTok{, }\KeywordTok{PRIMARY} \KeywordTok{KEY}\NormalTok{ (}\KeywordTok{id}\NormalTok{), }\KeywordTok{FOREIGN} \KeywordTok{KEY}\NormalTok{ (id\_auteur) }\KeywordTok{REFERENCES}\NormalTok{ AUTEURS(}\KeywordTok{id}\NormalTok{));}
\end{Highlighting}
\end{Shaded}

  Dans la création de la table \textbf{LIVRES}, nous avons précisé que
  l'attribut ``id\_auteur'' jouera le rôle de \textbf{clé étrangère} :
  liaison entre ``id\_auteur'' de la table LIVRES et ``id'' de la table
  AUTEURS (FOREIGN KEY (id\_auteur) REFERENCES AUTEURS(id)).
\item
  Ajoutez des données à la table AUTEURS à l'aide de la requête SQL
  suivante :

\begin{Shaded}
\begin{Highlighting}[]
\KeywordTok{INSERT} \KeywordTok{INTO}\NormalTok{ AUTEURS}
\NormalTok{(}\KeywordTok{id}\NormalTok{,nom,prenom,ann\_naissance,langue\_ecriture)}
\KeywordTok{VALUES}
\NormalTok{(}\DecValTok{1}\NormalTok{,}\OtherTok{"Orwell"}\NormalTok{,}\OtherTok{"George"}\NormalTok{,}\DecValTok{1903}\NormalTok{,}\OtherTok{"anglais"}\NormalTok{),}
\NormalTok{(}\DecValTok{2}\NormalTok{,}\OtherTok{"Herbert"}\NormalTok{,}\OtherTok{"Frank"}\NormalTok{,}\DecValTok{1920}\NormalTok{,}\OtherTok{"anglais"}\NormalTok{),}
\NormalTok{(}\DecValTok{3}\NormalTok{,}\OtherTok{"Asimov"}\NormalTok{,}\OtherTok{"Isaac"}\NormalTok{,}\DecValTok{1920}\NormalTok{,}\OtherTok{"anglais"}\NormalTok{),}
\NormalTok{(}\DecValTok{4}\NormalTok{,}\OtherTok{"Huxley"}\NormalTok{,}\OtherTok{"Aldous"}\NormalTok{,}\DecValTok{1894}\NormalTok{,}\OtherTok{"anglais"}\NormalTok{),}
\NormalTok{(}\DecValTok{5}\NormalTok{,}\OtherTok{"Bradbury"}\NormalTok{,}\OtherTok{"Ray"}\NormalTok{,}\DecValTok{1920}\NormalTok{,}\OtherTok{"anglais"}\NormalTok{),}
\NormalTok{(}\DecValTok{6}\NormalTok{,}\OtherTok{"K.Dick"}\NormalTok{,}\OtherTok{"Philip"}\NormalTok{,}\DecValTok{1928}\NormalTok{,}\OtherTok{"anglais"}\NormalTok{),}
\NormalTok{(}\DecValTok{7}\NormalTok{,}\OtherTok{"Barjavel"}\NormalTok{,}\OtherTok{"René"}\NormalTok{,}\DecValTok{1911}\NormalTok{,}\OtherTok{"français"}\NormalTok{),}
\NormalTok{(}\DecValTok{8}\NormalTok{,}\OtherTok{"Boulle"}\NormalTok{,}\OtherTok{"Pierre"}\NormalTok{,}\DecValTok{1912}\NormalTok{,}\OtherTok{"français"}\NormalTok{),}
\NormalTok{(}\DecValTok{9}\NormalTok{,}\OtherTok{"Van Vogt"}\NormalTok{,}\OtherTok{"Alfred Elton"}\NormalTok{,}\DecValTok{1912}\NormalTok{,}\OtherTok{"anglais"}\NormalTok{),}
\NormalTok{(}\DecValTok{10}\NormalTok{,}\OtherTok{"Verne"}\NormalTok{,}\OtherTok{"Jules"}\NormalTok{,}\DecValTok{1828}\NormalTok{,}\OtherTok{"français"}\NormalTok{);}
\end{Highlighting}
\end{Shaded}

  Ajoutez des données à la table LIVRES à l'aide de la requête SQL
  suivante :

\begin{Shaded}
\begin{Highlighting}[]
\KeywordTok{INSERT} \KeywordTok{INTO}\NormalTok{ LIVRES}
\NormalTok{(}\KeywordTok{id}\NormalTok{,titre,id\_auteur,ann\_publi,note)}
\KeywordTok{VALUES}
\NormalTok{(}\DecValTok{1}\NormalTok{,}\OtherTok{"1984"}\NormalTok{,}\DecValTok{1}\NormalTok{,}\DecValTok{1949}\NormalTok{,}\DecValTok{10}\NormalTok{),}
\NormalTok{(}\DecValTok{2}\NormalTok{,}\OtherTok{"Dune"}\NormalTok{,}\DecValTok{2}\NormalTok{,}\DecValTok{1965}\NormalTok{,}\DecValTok{8}\NormalTok{),}
\NormalTok{(}\DecValTok{3}\NormalTok{,}\OtherTok{"Fondation"}\NormalTok{,}\DecValTok{3}\NormalTok{,}\DecValTok{1951}\NormalTok{,}\DecValTok{9}\NormalTok{),}
\NormalTok{(}\DecValTok{4}\NormalTok{,}\OtherTok{"Le meilleur des mondes"}\NormalTok{,}\DecValTok{4}\NormalTok{,}\DecValTok{1931}\NormalTok{,}\DecValTok{7}\NormalTok{),}
\NormalTok{(}\DecValTok{5}\NormalTok{,}\OtherTok{"Fahrenheit 451"}\NormalTok{,}\DecValTok{5}\NormalTok{,}\DecValTok{1953}\NormalTok{,}\DecValTok{7}\NormalTok{),}
\NormalTok{(}\DecValTok{6}\NormalTok{,}\OtherTok{"Ubik"}\NormalTok{,}\DecValTok{6}\NormalTok{,}\DecValTok{1969}\NormalTok{,}\DecValTok{9}\NormalTok{),}
\NormalTok{(}\DecValTok{7}\NormalTok{,}\OtherTok{"Chroniques martiennes"}\NormalTok{,}\DecValTok{5}\NormalTok{,}\DecValTok{1950}\NormalTok{,}\DecValTok{8}\NormalTok{),}
\NormalTok{(}\DecValTok{8}\NormalTok{,}\OtherTok{"La nuit des temps"}\NormalTok{,}\DecValTok{7}\NormalTok{,}\DecValTok{1968}\NormalTok{,}\DecValTok{7}\NormalTok{),}
\NormalTok{(}\DecValTok{9}\NormalTok{,}\OtherTok{"Blade Runner"}\NormalTok{,}\DecValTok{6}\NormalTok{,}\DecValTok{1968}\NormalTok{,}\DecValTok{8}\NormalTok{),}
\NormalTok{(}\DecValTok{10}\NormalTok{,}\OtherTok{"Les Robots"}\NormalTok{,}\DecValTok{3}\NormalTok{,}\DecValTok{1950}\NormalTok{,}\DecValTok{9}\NormalTok{),}
\NormalTok{(}\DecValTok{11}\NormalTok{,}\OtherTok{"La Planète des singes"}\NormalTok{,}\DecValTok{8}\NormalTok{,}\DecValTok{1963}\NormalTok{,}\DecValTok{8}\NormalTok{),}
\NormalTok{(}\DecValTok{12}\NormalTok{,}\OtherTok{"Ravage"}\NormalTok{,}\DecValTok{7}\NormalTok{,}\DecValTok{1943}\NormalTok{,}\DecValTok{8}\NormalTok{),}
\NormalTok{(}\DecValTok{13}\NormalTok{,}\OtherTok{"Le Maître du Haut Château"}\NormalTok{,}\DecValTok{6}\NormalTok{,}\DecValTok{1962}\NormalTok{,}\DecValTok{8}\NormalTok{),}
\NormalTok{(}\DecValTok{14}\NormalTok{,}\OtherTok{"Le monde des Ā"}\NormalTok{,}\DecValTok{9}\NormalTok{,}\DecValTok{1945}\NormalTok{,}\DecValTok{7}\NormalTok{),}
\NormalTok{(}\DecValTok{15}\NormalTok{,}\OtherTok{"La Fin de l\textquotesingle{}éternité"}\NormalTok{,}\DecValTok{3}\NormalTok{,}\DecValTok{1955}\NormalTok{,}\DecValTok{8}\NormalTok{),}
\NormalTok{(}\DecValTok{16}\NormalTok{,}\OtherTok{"De la Terre à la Lune"}\NormalTok{,}\DecValTok{10}\NormalTok{,}\DecValTok{1865}\NormalTok{,}\DecValTok{10}\NormalTok{);}
\end{Highlighting}
\end{Shaded}
\item
  Saisissez et testez la requête SQL suivante :

\begin{Shaded}
\begin{Highlighting}[]
\KeywordTok{SELECT}\NormalTok{ titre,nom, prenom}
\KeywordTok{FROM}\NormalTok{ LIVRES }\KeywordTok{JOIN}\NormalTok{ AUTEURS }
\KeywordTok{ON}\NormalTok{ LIVRES.id\_auteur }\OperatorTok{=}\NormalTok{ AUTEURS.}\KeywordTok{id}
\end{Highlighting}
\end{Shaded}

  \textbf{Remarque} : attention, si un même nom d'attribut est présent
  dans les 2 tables (par exemple ici l'attribut id), il est nécessaire
  d'ajouter le nom de la table devant afin de pouvoir les distinguer
  (AUTEURS.id et LIVRES.id).
\item
  Écrivez une requête SQL permettant d'obtenir les titres des livres
  publiés après 1945 ainsi que le nom de leurs auteurs.
\item
  On souhaite ajouter à la base le livre de \textbf{Arthur C.Clarke}
  intitulé \textbf{2001 : L'Odyssée de l'espace} publié en \textbf{1968}
  et noté \textbf{7}. \textbf{Arthur C.Clarke} est un écrivain
  britannique né en 1917 et mort en 2008.

  Écrivez les requêtes nécessaires à cet ajout. Vous n'oublierez pas de
  définir les clefs primaires pour chacune des nouvelles entrées.
\item
  Écrivez et testez une requête permettant d'attribuer la note de 10 à
  tous les livres écrits par Asimov publiés après 1950.
\item
  Écrivez une requête permettant de supprimer les livres publiés avant
  1945. Testez cette requête.
\end{enumerate}



\end{document}
