% Options for packages loaded elsewhere
\PassOptionsToPackage{unicode}{hyperref}
\PassOptionsToPackage{hyphens}{url}
\PassOptionsToPackage{dvipsnames,svgnames,x11names}{xcolor}
%
\documentclass[
  letterpaper,
  DIV=11,
  numbers=noendperiod]{scrartcl}

\usepackage{amsmath,amssymb}
\usepackage{iftex}
\ifPDFTeX
  \usepackage[T1]{fontenc}
  \usepackage[utf8]{inputenc}
  \usepackage{textcomp} % provide euro and other symbols
\else % if luatex or xetex
  \usepackage{unicode-math}
  \defaultfontfeatures{Scale=MatchLowercase}
  \defaultfontfeatures[\rmfamily]{Ligatures=TeX,Scale=1}
\fi
\usepackage{lmodern}
\ifPDFTeX\else  
    % xetex/luatex font selection
\fi
% Use upquote if available, for straight quotes in verbatim environments
\IfFileExists{upquote.sty}{\usepackage{upquote}}{}
\IfFileExists{microtype.sty}{% use microtype if available
  \usepackage[]{microtype}
  \UseMicrotypeSet[protrusion]{basicmath} % disable protrusion for tt fonts
}{}
\makeatletter
\@ifundefined{KOMAClassName}{% if non-KOMA class
  \IfFileExists{parskip.sty}{%
    \usepackage{parskip}
  }{% else
    \setlength{\parindent}{0pt}
    \setlength{\parskip}{6pt plus 2pt minus 1pt}}
}{% if KOMA class
  \KOMAoptions{parskip=half}}
\makeatother
\usepackage{xcolor}
\usepackage[top=20mm,bottom=20mm,left=20mm,right=20mm,heightrounded]{geometry}
\setlength{\emergencystretch}{3em} % prevent overfull lines
\setcounter{secnumdepth}{-\maxdimen} % remove section numbering
% Make \paragraph and \subparagraph free-standing
\ifx\paragraph\undefined\else
  \let\oldparagraph\paragraph
  \renewcommand{\paragraph}[1]{\oldparagraph{#1}\mbox{}}
\fi
\ifx\subparagraph\undefined\else
  \let\oldsubparagraph\subparagraph
  \renewcommand{\subparagraph}[1]{\oldsubparagraph{#1}\mbox{}}
\fi


\providecommand{\tightlist}{%
  \setlength{\itemsep}{0pt}\setlength{\parskip}{0pt}}\usepackage{longtable,booktabs,array}
\usepackage{calc} % for calculating minipage widths
% Correct order of tables after \paragraph or \subparagraph
\usepackage{etoolbox}
\makeatletter
\patchcmd\longtable{\par}{\if@noskipsec\mbox{}\fi\par}{}{}
\makeatother
% Allow footnotes in longtable head/foot
\IfFileExists{footnotehyper.sty}{\usepackage{footnotehyper}}{\usepackage{footnote}}
\makesavenoteenv{longtable}
\usepackage{graphicx}
\makeatletter
\def\maxwidth{\ifdim\Gin@nat@width>\linewidth\linewidth\else\Gin@nat@width\fi}
\def\maxheight{\ifdim\Gin@nat@height>\textheight\textheight\else\Gin@nat@height\fi}
\makeatother
% Scale images if necessary, so that they will not overflow the page
% margins by default, and it is still possible to overwrite the defaults
% using explicit options in \includegraphics[width, height, ...]{}
\setkeys{Gin}{width=\maxwidth,height=\maxheight,keepaspectratio}
% Set default figure placement to htbp
\makeatletter
\def\fps@figure{htbp}
\makeatother
\newlength{\cslhangindent}
\setlength{\cslhangindent}{1.5em}
\newlength{\csllabelwidth}
\setlength{\csllabelwidth}{3em}
\newlength{\cslentryspacingunit} % times entry-spacing
\setlength{\cslentryspacingunit}{\parskip}
\newenvironment{CSLReferences}[2] % #1 hanging-ident, #2 entry spacing
 {% don't indent paragraphs
  \setlength{\parindent}{0pt}
  % turn on hanging indent if param 1 is 1
  \ifodd #1
  \let\oldpar\par
  \def\par{\hangindent=\cslhangindent\oldpar}
  \fi
  % set entry spacing
  \setlength{\parskip}{#2\cslentryspacingunit}
 }%
 {}
\usepackage{calc}
\newcommand{\CSLBlock}[1]{#1\hfill\break}
\newcommand{\CSLLeftMargin}[1]{\parbox[t]{\csllabelwidth}{#1}}
\newcommand{\CSLRightInline}[1]{\parbox[t]{\linewidth - \csllabelwidth}{#1}\break}
\newcommand{\CSLIndent}[1]{\hspace{\cslhangindent}#1}

\usepackage{fancyhdr} \pagestyle{fancy} \usepackage{lastpage}
\KOMAoption{captions}{tablesignature}
\makeatletter
\@ifpackageloaded{tcolorbox}{}{\usepackage[skins,breakable]{tcolorbox}}
\@ifpackageloaded{fontawesome5}{}{\usepackage{fontawesome5}}
\definecolor{quarto-callout-color}{HTML}{909090}
\definecolor{quarto-callout-note-color}{HTML}{0758E5}
\definecolor{quarto-callout-important-color}{HTML}{CC1914}
\definecolor{quarto-callout-warning-color}{HTML}{EB9113}
\definecolor{quarto-callout-tip-color}{HTML}{00A047}
\definecolor{quarto-callout-caution-color}{HTML}{FC5300}
\definecolor{quarto-callout-color-frame}{HTML}{acacac}
\definecolor{quarto-callout-note-color-frame}{HTML}{4582ec}
\definecolor{quarto-callout-important-color-frame}{HTML}{d9534f}
\definecolor{quarto-callout-warning-color-frame}{HTML}{f0ad4e}
\definecolor{quarto-callout-tip-color-frame}{HTML}{02b875}
\definecolor{quarto-callout-caution-color-frame}{HTML}{fd7e14}
\makeatother
\makeatletter
\makeatother
\makeatletter
\makeatother
\makeatletter
\@ifpackageloaded{caption}{}{\usepackage{caption}}
\AtBeginDocument{%
\ifdefined\contentsname
  \renewcommand*\contentsname{Table des matières}
\else
  \newcommand\contentsname{Table des matières}
\fi
\ifdefined\listfigurename
  \renewcommand*\listfigurename{Liste des Figures}
\else
  \newcommand\listfigurename{Liste des Figures}
\fi
\ifdefined\listtablename
  \renewcommand*\listtablename{Liste des Tables}
\else
  \newcommand\listtablename{Liste des Tables}
\fi
\ifdefined\figurename
  \renewcommand*\figurename{Figure}
\else
  \newcommand\figurename{Figure}
\fi
\ifdefined\tablename
  \renewcommand*\tablename{Tableau}
\else
  \newcommand\tablename{Tableau}
\fi
}
\@ifpackageloaded{float}{}{\usepackage{float}}
\floatstyle{ruled}
\@ifundefined{c@chapter}{\newfloat{codelisting}{h}{lop}}{\newfloat{codelisting}{h}{lop}[chapter]}
\floatname{codelisting}{Listing}
\newcommand*\listoflistings{\listof{codelisting}{Liste des Listings}}
\makeatother
\makeatletter
\@ifpackageloaded{caption}{}{\usepackage{caption}}
\@ifpackageloaded{subcaption}{}{\usepackage{subcaption}}
\makeatother
\makeatletter
\@ifpackageloaded{tcolorbox}{}{\usepackage[skins,breakable]{tcolorbox}}
\makeatother
\makeatletter
\@ifundefined{shadecolor}{\definecolor{shadecolor}{rgb}{.97, .97, .97}}
\makeatother
\makeatletter
\makeatother
\makeatletter
\makeatother
\ifLuaTeX
\usepackage[bidi=basic]{babel}
\else
\usepackage[bidi=default]{babel}
\fi
\babelprovide[main,import]{french}
% get rid of language-specific shorthands (see #6817):
\let\LanguageShortHands\languageshorthands
\def\languageshorthands#1{}
\ifLuaTeX
  \usepackage{selnolig}  % disable illegal ligatures
\fi
\IfFileExists{bookmark.sty}{\usepackage{bookmark}}{\usepackage{hyperref}}
\IfFileExists{xurl.sty}{\usepackage{xurl}}{} % add URL line breaks if available
\urlstyle{same} % disable monospaced font for URLs
\hypersetup{
  pdftitle={Graphes (Cours)},
  pdflang={fr},
  colorlinks=true,
  linkcolor={blue},
  filecolor={Maroon},
  citecolor={Blue},
  urlcolor={Blue},
  pdfcreator={LaTeX via pandoc}}

\title{Graphes (Cours)}
\usepackage{etoolbox}
\makeatletter
\providecommand{\subtitle}[1]{% add subtitle to \maketitle
  \apptocmd{\@title}{\par {\large #1 \par}}{}{}
}
\makeatother
\subtitle{S4 - Arbres et graphes}
\author{}
\date{}

\begin{document}
\maketitle
\lhead{Spécialité NSI} \rhead{Terminale} \chead{} \cfoot{} \lfoot{Lycée \'Emile Duclaux} \rfoot{Page \thepage/\pageref{LastPage}} \renewcommand{\headrulewidth}{0pt} \renewcommand{\footrulewidth}{0pt} \thispagestyle{fancy} \vspace{-2cm}

\ifdefined\Shaded\renewenvironment{Shaded}{\begin{tcolorbox}[boxrule=0pt, frame hidden, sharp corners, interior hidden, enhanced, breakable, borderline west={3pt}{0pt}{shadecolor}]}{\end{tcolorbox}}\fi

\hypertarget{notion-de-graphe}{%
\subsection{1. Notion de graphe}\label{notion-de-graphe}}

\hypertarget{duxe9finition}{%
\subsubsection{1.1. Définition}\label{duxe9finition}}

\begin{tcolorbox}[enhanced jigsaw, title=\textcolor{quarto-callout-tip-color}{\faLightbulb}\hspace{0.5em}{Définition}, titlerule=0mm, rightrule=.15mm, opacitybacktitle=0.6, colframe=quarto-callout-tip-color-frame, left=2mm, coltitle=black, breakable, colbacktitle=quarto-callout-tip-color!10!white, leftrule=.75mm, toptitle=1mm, arc=.35mm, bottomtitle=1mm, colback=white, bottomrule=.15mm, toprule=.15mm, opacityback=0]

Un \textbf{graphe non orienté} est constitué :

\begin{itemize}
\tightlist
\item
  d'un ensemble fini de \textbf{sommets} (représentés par des points) ;
\item
  d'un ensemble d'\textbf{arêtes} (représentées par des traits) qui
  relient des sommets entre eux.
\end{itemize}

\end{tcolorbox}

La figure ci-dessous représente un graphe dont les sommets sont
numérotés de 0 à 6.

\begin{figure}

{\centering \includegraphics[width=0.3\textwidth,height=\textheight]{grpahe1.png}

}

\caption{\label{fig-graphe1}Graphe non orienté}

\end{figure}

\begin{tcolorbox}[enhanced jigsaw, title=\textcolor{quarto-callout-tip-color}{\faLightbulb}\hspace{0.5em}{Définition}, titlerule=0mm, rightrule=.15mm, opacitybacktitle=0.6, colframe=quarto-callout-tip-color-frame, left=2mm, coltitle=black, breakable, colbacktitle=quarto-callout-tip-color!10!white, leftrule=.75mm, toptitle=1mm, arc=.35mm, bottomtitle=1mm, colback=white, bottomrule=.15mm, toprule=.15mm, opacityback=0]

Un \textbf{graphe orienté} est constitué :

\begin{itemize}
\tightlist
\item
  d'un ensemble fini de \textbf{sommets} (représentés par des points) ;
\item
  d'un ensemble d'\textbf{arcs} (représentés par des flèches) qui
  relient des sommets entre eux.
\end{itemize}

\end{tcolorbox}

La figure ci-dessous représente un graphe orienté dont les sommets sont
numérotés de 0 à 6.

\begin{figure}

{\centering \includegraphics[width=0.3\textwidth,height=\textheight]{graphe2.png}

}

\caption{\label{fig-graphe2}Graphe orienté}

\end{figure}

\newpage{}

Un peu de vocabulaire :

\begin{itemize}
\tightlist
\item
  deux sommets reliés par une arête (ou un arc) sont dits
  \textbf{adjacents}, ou \textbf{voisins} ;
\item
  un sommet est dit \textbf{isolé} s'il n'est relié à aucun autre sommet
  ;
\item
  l'\textbf{ordre} d'un graphe est le nombre de ses sommets ;
\item
  le \textbf{degré} d'un sommet \(S\), noté \(deg(S)\), est le nombre
  d'arêtes (ou d'arcs) qui le relient à d'autres sommets ;
\end{itemize}

\hypertarget{lien-avec-les-arbres}{%
\subsubsection{1.2. Lien avec les arbres}\label{lien-avec-les-arbres}}

La structure d'arbre déjà rencontrée dans le cours est un cas
particulier de graphe non orienté.

\begin{tcolorbox}[enhanced jigsaw, title=\textcolor{quarto-callout-tip-color}{\faLightbulb}\hspace{0.5em}{Définition}, titlerule=0mm, rightrule=.15mm, opacitybacktitle=0.6, colframe=quarto-callout-tip-color-frame, left=2mm, coltitle=black, breakable, colbacktitle=quarto-callout-tip-color!10!white, leftrule=.75mm, toptitle=1mm, arc=.35mm, bottomtitle=1mm, colback=white, bottomrule=.15mm, toprule=.15mm, opacityback=0]

Un \textbf{arbre} est un graphe non orienté qui satisfait les conditions
suivantes :

\begin{itemize}
\tightlist
\item
  il est \textbf{connexe} ;
\item
  il n'a pas de cycle.
\end{itemize}

\end{tcolorbox}

\textbf{Explications} : \emph{connexe} signifie que tous les sommets
sont reliés entre eux par un chemin (il n'y a pas de points isolés).
\emph{pas de cycle} signifie que l'on ne peut pas revenir au même sommet
en passant par le même chemin.

En particulier le graphe ci-dessus (Figure~\ref{fig-graphe1}) n'est pas
un arbre, car il n'est pas connexe (le sommet 6 n'est pas relié au
sommet 1) et il contient un cycle (le chemin 1-2-4-1).

\hypertarget{exemples}{%
\subsubsection{1.3. Exemples}\label{exemples}}

Voici quelques situations pouvant être modélisées par un graphe :

\begin{itemize}
\tightlist
\item
  un réseau routier ;
\item
  un réseau de télécommunications ;
\item
  un réseau social ;
\item
  un réseau électrique ;
\end{itemize}

\hypertarget{impluxe9mentations-dun-graphe}{%
\subsection{2. Implémentations d'un
graphe}\label{impluxe9mentations-dun-graphe}}

L'objectif de cette section est de définir une structure de données en
Python permettant de représenter un graphe.

Les opérations suivantes doivent être possibles (cf. FORTIER (2022)):

\begin{itemize}
\tightlist
\item
  créer un graphe vide ;
\item
  ajouter un sommet ;
\item
  ajouter une arête (ou un arc) ;
\item
  supprimer un sommet ;
\item
  supprimer une arête (ou un arc) ;
\item
  vérifier si deux sommets sont adjacents ;
\item
  connaître la liste des sommets adjacents à un sommet donné.
\end{itemize}

Nous étudions deux implémentations : par matrice d'adjacence et par
liste d'adjacence.

\hypertarget{matrice-dadjacence}{%
\subsubsection{2.1. Matrice d'adjacence}\label{matrice-dadjacence}}

\textbf{Rappel} : une \textbf{matrice} est un tableau à deux dimensions,
où chaque élément est identifié par un couple de coordonnées (ligne,
colonne). En Python, on peut représenter une matrice par une liste de
listes.

\begin{tcolorbox}[enhanced jigsaw, title=\textcolor{quarto-callout-tip-color}{\faLightbulb}\hspace{0.5em}{Définition}, titlerule=0mm, rightrule=.15mm, opacitybacktitle=0.6, colframe=quarto-callout-tip-color-frame, left=2mm, coltitle=black, breakable, colbacktitle=quarto-callout-tip-color!10!white, leftrule=.75mm, toptitle=1mm, arc=.35mm, bottomtitle=1mm, colback=white, bottomrule=.15mm, toprule=.15mm, opacityback=0]

La \textbf{matrice d'adjacence} d'un graphe \(G\) (orienté ou non) est
une matrice carrée \(A\) de taille \(n\) telle que :

\begin{itemize}
\tightlist
\item
  \(A_{i,j} = 1\) si les sommets \(i\) et \(j\) sont adjacents ;
\item
  \(A_{i,j} = 0\) si les sommets \(i\) et \(j\) ne sont pas adjacents.
\end{itemize}

\end{tcolorbox}

\textbf{Exemple} : la matrice d'adjacence du graphe non orienté de la
figure ci-dessus (Figure~\ref{fig-graphe1}) est la suivante :

\begin{figure}

\begin{minipage}[t]{0.50\linewidth}

{\centering 

\raisebox{-\height}{

\includegraphics[width=3.125in,height=\textheight]{grpahe1.png}

}

\caption{Graphe Fig.1}

}

\end{minipage}%
%
\begin{minipage}[t]{0.50\linewidth}

{\centering 

\[
\begin{pmatrix}
0 & 0 & 0 & 0 & 0 & 0 & 1 \\
0 & 0 & 1 & 1 & 1 & 0 & 0 \\
0 & 1 & 0 & 0 & 1 & 1 & 0\\
0 & 1 & 0 & 0 & 1 & 0 & 0\\
0 & 1 & 1 & 1 & 0 & 0 & 0\\
0 & 0 & 1 & 0 & 0 & 0 & 0\\
1 & 0 & 0 & 0 & 0 & 0 & 0
\end{pmatrix}
\]

}

\end{minipage}%

\end{figure}

\textbf{Remarques} : dans le cas d'un graphe non orienté, la matrice
d'adjacence est symétrique par rapport à la diagonale principale. Dans
le cas d'un graphe orienté, la matrice d'adjacence n'est pas forcément
symétrique.

\begin{tcolorbox}[enhanced jigsaw, title=\textcolor{quarto-callout-caution-color}{\faFire}\hspace{0.5em}{TP Python}, titlerule=0mm, rightrule=.15mm, opacitybacktitle=0.6, colframe=quarto-callout-caution-color-frame, left=2mm, coltitle=black, breakable, colbacktitle=quarto-callout-caution-color!10!white, leftrule=.75mm, toptitle=1mm, arc=.35mm, bottomtitle=1mm, colback=white, bottomrule=.15mm, toprule=.15mm, opacityback=0]

\href{tp_graphe_matrice.ipynb}{TP Python : implémentation des graphes
par matrice d'adjacence}

\end{tcolorbox}

\newpage{}

\hypertarget{liste-dadjacence}{%
\subsubsection{2.2. Liste d'adjacence}\label{liste-dadjacence}}

\begin{tcolorbox}[enhanced jigsaw, title=\textcolor{quarto-callout-tip-color}{\faLightbulb}\hspace{0.5em}{Définition : liste d'adjacence}, titlerule=0mm, rightrule=.15mm, opacitybacktitle=0.6, colframe=quarto-callout-tip-color-frame, left=2mm, coltitle=black, breakable, colbacktitle=quarto-callout-tip-color!10!white, leftrule=.75mm, toptitle=1mm, arc=.35mm, bottomtitle=1mm, colback=white, bottomrule=.15mm, toprule=.15mm, opacityback=0]

La \textbf{liste d'adjacence} d'un graphe \(G\) (orienté ou non) dont
les sommets sont les entiers compris entre 0 et \(n-1\) est une liste
\(L\) de taille \(n\) telle que :

\begin{itemize}
\tightlist
\item
  \(L[i]\) est la liste des sommets adjacents au sommet \(i\).
\end{itemize}

\end{tcolorbox}

\textbf{Exemple} : la liste d'adjacence du graphe non orienté de la
figure ci-dessus (Figure~\ref{fig-graphe1}) est la suivante :

\begin{figure}

\begin{minipage}[t]{0.50\linewidth}

{\centering 

\raisebox{-\height}{

\includegraphics[width=3.125in,height=\textheight]{grpahe1.png}

}

\caption{Graphe Fig.1}

}

\end{minipage}%
%
\begin{minipage}[t]{0.50\linewidth}

{\centering 

\[
\begin{array}{|c|c|}
\hline
0 & [6] \\
\hline
1 & [2,3,4] \\
\hline
2 & [1,4,5] \\
\hline
3 & [1,4] \\
\hline
4 & [1,2,3] \\
\hline
5 & [2] \\
\hline
6 & [0] \\
\hline
\end{array}
\]

}

\end{minipage}%

\end{figure}

Dans le cas d'un graphe comportant peu d'arêtes, la liste d'adjacence
occupe moins de mémoire que la matrice d'adjacence. Mais certaines
opérations, comme la vérification de l'adjacence de deux sommets ou la
suppression d'une arête, sont plus coûteuses en temps d'exécution.

\begin{tcolorbox}[enhanced jigsaw, title=\textcolor{quarto-callout-caution-color}{\faFire}\hspace{0.5em}{TP Python}, titlerule=0mm, rightrule=.15mm, opacitybacktitle=0.6, colframe=quarto-callout-caution-color-frame, left=2mm, coltitle=black, breakable, colbacktitle=quarto-callout-caution-color!10!white, leftrule=.75mm, toptitle=1mm, arc=.35mm, bottomtitle=1mm, colback=white, bottomrule=.15mm, toprule=.15mm, opacityback=0]

Dans le TP ci-dessous, vous devez implémenter la structure de graphe par
liste d'adjacence. Vous devez également implémenter les fonctions
permettant de passer d'une représentation à l'autre.

\href{tp_graphe_liste.ipynb}{TP Python : implémentation des graphes par
liste d'adjacence}

\end{tcolorbox}

\hrulefill

\textbf{Sources utilisées pour la rédaction de ce chapitre}

\hypertarget{refs}{}
\begin{CSLReferences}{1}{0}
\leavevmode\vadjust pre{\hypertarget{ref-fortier}{}}%
FORTIER, Quentin. 2022. {«~Informatique commune en 1ère année en
CPGE~»}. \emph{CPGE-ITC}.
\href{https://cpge-itc.github.io/itc1/\%20\%20\%20}{https://cpge-itc.github.io/itc1/
}.

\end{CSLReferences}



\end{document}
