% Options for packages loaded elsewhere
\PassOptionsToPackage{unicode}{hyperref}
\PassOptionsToPackage{hyphens}{url}
\PassOptionsToPackage{dvipsnames,svgnames,x11names}{xcolor}
%
\documentclass[
  a4paper,
  DIV=11,
  numbers=noendperiod]{scrartcl}

\usepackage{amsmath,amssymb}
\usepackage{lmodern}
\usepackage{iftex}
\ifPDFTeX
  \usepackage[T1]{fontenc}
  \usepackage[utf8]{inputenc}
  \usepackage{textcomp} % provide euro and other symbols
\else % if luatex or xetex
  \usepackage{unicode-math}
  \defaultfontfeatures{Scale=MatchLowercase}
  \defaultfontfeatures[\rmfamily]{Ligatures=TeX,Scale=1}
\fi
% Use upquote if available, for straight quotes in verbatim environments
\IfFileExists{upquote.sty}{\usepackage{upquote}}{}
\IfFileExists{microtype.sty}{% use microtype if available
  \usepackage[]{microtype}
  \UseMicrotypeSet[protrusion]{basicmath} % disable protrusion for tt fonts
}{}
\makeatletter
\@ifundefined{KOMAClassName}{% if non-KOMA class
  \IfFileExists{parskip.sty}{%
    \usepackage{parskip}
  }{% else
    \setlength{\parindent}{0pt}
    \setlength{\parskip}{6pt plus 2pt minus 1pt}}
}{% if KOMA class
  \KOMAoptions{parskip=half}}
\makeatother
\usepackage{xcolor}
\usepackage[top=20mm,bottom=20mm,left=20mm,right=20mm,heightrounded]{geometry}
\setlength{\emergencystretch}{3em} % prevent overfull lines
\setcounter{secnumdepth}{-\maxdimen} % remove section numbering
% Make \paragraph and \subparagraph free-standing
\ifx\paragraph\undefined\else
  \let\oldparagraph\paragraph
  \renewcommand{\paragraph}[1]{\oldparagraph{#1}\mbox{}}
\fi
\ifx\subparagraph\undefined\else
  \let\oldsubparagraph\subparagraph
  \renewcommand{\subparagraph}[1]{\oldsubparagraph{#1}\mbox{}}
\fi

\usepackage{color}
\usepackage{fancyvrb}
\newcommand{\VerbBar}{|}
\newcommand{\VERB}{\Verb[commandchars=\\\{\}]}
\DefineVerbatimEnvironment{Highlighting}{Verbatim}{commandchars=\\\{\}}
% Add ',fontsize=\small' for more characters per line
\usepackage{framed}
\definecolor{shadecolor}{RGB}{241,243,245}
\newenvironment{Shaded}{\begin{snugshade}}{\end{snugshade}}
\newcommand{\AlertTok}[1]{\textcolor[rgb]{0.68,0.00,0.00}{#1}}
\newcommand{\AnnotationTok}[1]{\textcolor[rgb]{0.37,0.37,0.37}{#1}}
\newcommand{\AttributeTok}[1]{\textcolor[rgb]{0.40,0.45,0.13}{#1}}
\newcommand{\BaseNTok}[1]{\textcolor[rgb]{0.68,0.00,0.00}{#1}}
\newcommand{\BuiltInTok}[1]{\textcolor[rgb]{0.00,0.23,0.31}{#1}}
\newcommand{\CharTok}[1]{\textcolor[rgb]{0.13,0.47,0.30}{#1}}
\newcommand{\CommentTok}[1]{\textcolor[rgb]{0.37,0.37,0.37}{#1}}
\newcommand{\CommentVarTok}[1]{\textcolor[rgb]{0.37,0.37,0.37}{\textit{#1}}}
\newcommand{\ConstantTok}[1]{\textcolor[rgb]{0.56,0.35,0.01}{#1}}
\newcommand{\ControlFlowTok}[1]{\textcolor[rgb]{0.00,0.23,0.31}{#1}}
\newcommand{\DataTypeTok}[1]{\textcolor[rgb]{0.68,0.00,0.00}{#1}}
\newcommand{\DecValTok}[1]{\textcolor[rgb]{0.68,0.00,0.00}{#1}}
\newcommand{\DocumentationTok}[1]{\textcolor[rgb]{0.37,0.37,0.37}{\textit{#1}}}
\newcommand{\ErrorTok}[1]{\textcolor[rgb]{0.68,0.00,0.00}{#1}}
\newcommand{\ExtensionTok}[1]{\textcolor[rgb]{0.00,0.23,0.31}{#1}}
\newcommand{\FloatTok}[1]{\textcolor[rgb]{0.68,0.00,0.00}{#1}}
\newcommand{\FunctionTok}[1]{\textcolor[rgb]{0.28,0.35,0.67}{#1}}
\newcommand{\ImportTok}[1]{\textcolor[rgb]{0.00,0.46,0.62}{#1}}
\newcommand{\InformationTok}[1]{\textcolor[rgb]{0.37,0.37,0.37}{#1}}
\newcommand{\KeywordTok}[1]{\textcolor[rgb]{0.00,0.23,0.31}{#1}}
\newcommand{\NormalTok}[1]{\textcolor[rgb]{0.00,0.23,0.31}{#1}}
\newcommand{\OperatorTok}[1]{\textcolor[rgb]{0.37,0.37,0.37}{#1}}
\newcommand{\OtherTok}[1]{\textcolor[rgb]{0.00,0.23,0.31}{#1}}
\newcommand{\PreprocessorTok}[1]{\textcolor[rgb]{0.68,0.00,0.00}{#1}}
\newcommand{\RegionMarkerTok}[1]{\textcolor[rgb]{0.00,0.23,0.31}{#1}}
\newcommand{\SpecialCharTok}[1]{\textcolor[rgb]{0.37,0.37,0.37}{#1}}
\newcommand{\SpecialStringTok}[1]{\textcolor[rgb]{0.13,0.47,0.30}{#1}}
\newcommand{\StringTok}[1]{\textcolor[rgb]{0.13,0.47,0.30}{#1}}
\newcommand{\VariableTok}[1]{\textcolor[rgb]{0.07,0.07,0.07}{#1}}
\newcommand{\VerbatimStringTok}[1]{\textcolor[rgb]{0.13,0.47,0.30}{#1}}
\newcommand{\WarningTok}[1]{\textcolor[rgb]{0.37,0.37,0.37}{\textit{#1}}}

\providecommand{\tightlist}{%
  \setlength{\itemsep}{0pt}\setlength{\parskip}{0pt}}\usepackage{longtable,booktabs,array}
\usepackage{calc} % for calculating minipage widths
% Correct order of tables after \paragraph or \subparagraph
\usepackage{etoolbox}
\makeatletter
\patchcmd\longtable{\par}{\if@noskipsec\mbox{}\fi\par}{}{}
\makeatother
% Allow footnotes in longtable head/foot
\IfFileExists{footnotehyper.sty}{\usepackage{footnotehyper}}{\usepackage{footnote}}
\makesavenoteenv{longtable}
\usepackage{graphicx}
\makeatletter
\def\maxwidth{\ifdim\Gin@nat@width>\linewidth\linewidth\else\Gin@nat@width\fi}
\def\maxheight{\ifdim\Gin@nat@height>\textheight\textheight\else\Gin@nat@height\fi}
\makeatother
% Scale images if necessary, so that they will not overflow the page
% margins by default, and it is still possible to overwrite the defaults
% using explicit options in \includegraphics[width, height, ...]{}
\setkeys{Gin}{width=\maxwidth,height=\maxheight,keepaspectratio}
% Set default figure placement to htbp
\makeatletter
\def\fps@figure{htbp}
\makeatother

\usepackage{fancyhdr} \pagestyle{fancy} \usepackage{lastpage}
\KOMAoption{captions}{tablesignature}
\makeatletter
\@ifpackageloaded{tcolorbox}{}{\usepackage[many]{tcolorbox}}
\@ifpackageloaded{fontawesome5}{}{\usepackage{fontawesome5}}
\definecolor{quarto-callout-color}{HTML}{909090}
\definecolor{quarto-callout-note-color}{HTML}{0758E5}
\definecolor{quarto-callout-important-color}{HTML}{CC1914}
\definecolor{quarto-callout-warning-color}{HTML}{EB9113}
\definecolor{quarto-callout-tip-color}{HTML}{00A047}
\definecolor{quarto-callout-caution-color}{HTML}{FC5300}
\definecolor{quarto-callout-color-frame}{HTML}{acacac}
\definecolor{quarto-callout-note-color-frame}{HTML}{4582ec}
\definecolor{quarto-callout-important-color-frame}{HTML}{d9534f}
\definecolor{quarto-callout-warning-color-frame}{HTML}{f0ad4e}
\definecolor{quarto-callout-tip-color-frame}{HTML}{02b875}
\definecolor{quarto-callout-caution-color-frame}{HTML}{fd7e14}
\makeatother
\makeatletter
\makeatother
\makeatletter
\makeatother
\makeatletter
\@ifpackageloaded{caption}{}{\usepackage{caption}}
\AtBeginDocument{%
\ifdefined\contentsname
  \renewcommand*\contentsname{Table des matières}
\else
  \newcommand\contentsname{Table des matières}
\fi
\ifdefined\listfigurename
  \renewcommand*\listfigurename{Liste des Figures}
\else
  \newcommand\listfigurename{Liste des Figures}
\fi
\ifdefined\listtablename
  \renewcommand*\listtablename{Liste des Tables}
\else
  \newcommand\listtablename{Liste des Tables}
\fi
\ifdefined\figurename
  \renewcommand*\figurename{Figure}
\else
  \newcommand\figurename{Figure}
\fi
\ifdefined\tablename
  \renewcommand*\tablename{Tableau}
\else
  \newcommand\tablename{Tableau}
\fi
}
\@ifpackageloaded{float}{}{\usepackage{float}}
\floatstyle{ruled}
\@ifundefined{c@chapter}{\newfloat{codelisting}{h}{lop}}{\newfloat{codelisting}{h}{lop}[chapter]}
\floatname{codelisting}{Listing}
\newcommand*\listoflistings{\listof{codelisting}{Liste des Listings}}
\makeatother
\makeatletter
\@ifpackageloaded{caption}{}{\usepackage{caption}}
\@ifpackageloaded{subcaption}{}{\usepackage{subcaption}}
\makeatother
\makeatletter
\@ifpackageloaded{tcolorbox}{}{\usepackage[many]{tcolorbox}}
\makeatother
\makeatletter
\@ifundefined{shadecolor}{\definecolor{shadecolor}{rgb}{.97, .97, .97}}
\makeatother
\makeatletter
\makeatother
\ifLuaTeX
\usepackage[bidi=basic]{babel}
\else
\usepackage[bidi=default]{babel}
\fi
\babelprovide[main,import]{french}
% get rid of language-specific shorthands (see #6817):
\let\LanguageShortHands\languageshorthands
\def\languageshorthands#1{}
\ifLuaTeX
  \usepackage{selnolig}  % disable illegal ligatures
\fi
\IfFileExists{bookmark.sty}{\usepackage{bookmark}}{\usepackage{hyperref}}
\IfFileExists{xurl.sty}{\usepackage{xurl}}{} % add URL line breaks if available
\urlstyle{same} % disable monospaced font for URLs
\hypersetup{
  pdftitle={Implémentation des arbres en Python},
  pdflang={fr},
  colorlinks=true,
  linkcolor={blue},
  filecolor={Maroon},
  citecolor={Blue},
  urlcolor={Blue},
  pdfcreator={LaTeX via pandoc}}

\title{Implémentation des arbres en Python}
\usepackage{etoolbox}
\makeatletter
\providecommand{\subtitle}[1]{% add subtitle to \maketitle
  \apptocmd{\@title}{\par {\large #1 \par}}{}{}
}
\makeatother
\subtitle{S4 - Arbres et graphes}
\author{}
\date{}

\begin{document}
\maketitle
\lhead{Spécialité NSI} \rhead{Terminale} \chead{} \cfoot{} \lfoot{Lycée \'Emile Duclaux} \rfoot{Page \thepage/\pageref{LastPage}} \renewcommand{\headrulewidth}{0pt} \renewcommand{\footrulewidth}{0pt} \thispagestyle{fancy} \vspace{-2cm}

\ifdefined\Shaded\renewenvironment{Shaded}{\begin{tcolorbox}[boxrule=0pt, enhanced, interior hidden, borderline west={3pt}{0pt}{shadecolor}, breakable, sharp corners, frame hidden]}{\end{tcolorbox}}\fi

L'objectif de cette partie est d'\textbf{implémenter} la structure
d'arbre binaire en Python. Nous allons pour cela utiliser la
Programmation Orientée Objet et construire un module réutilisable
proposant à l'utilisateur une interface (\textbf{API}) permettant de
travailler avec les arbres binaires.

\hypertarget{arbres-binaires}{%
\subsection{1. Arbres binaires}\label{arbres-binaires}}

Une interface souhaitable devrait permettre de :

\begin{itemize}
\tightlist
\item
  Créer un arbre vide ;
\item
  Accéder au sous-arbre gauche et au sous-arbre droit d'un nœud ;
\item
  Accéder à une clef ;
\item
  Tester si un nœud est une feuille ;
\item
  Tester si un arbre est vide ;
\item
  Retourner la taille ;
\item
  Retourner la hauteur.
\end{itemize}

De plus, il serait souhaitable de parvenir à afficher un arbre de façon
visuelle.

Nous avons vu que la structure d'arbre binaire est une structure
\textbf{récursive} : cette propriété est exploitée dans l'implémentation
que nous allons présenter. Pour définir un arbre, il suffit de définir
un nœud racine ainsi que les deux sous-arbres gauche et droite qui sont
eux-même des arbres binaires. Cela revient à assimiler un arbre à sa
racine associée à un lien vers ses deux fils.

Nous définissons ci-dessous un objet \texttt{ArbreBinaire} possédant
trois attributs \texttt{clef}, \texttt{gauche}, \texttt{droit}. Pour
respecter les principes de la POO, et notamment la notion
d'\textbf{encapsulation}, nous avons défini des méthodes d'accès aux
attributs (elles commencent par \texttt{get}) et des méthodes de
modification des attributs (elles commencent par \texttt{set}) et on
s'interdira tout accès ou affectation direct(e) du type
\texttt{arbre.racine\ =\ ...}.

La méthode \texttt{setRacine}, qui permet de définir la clef d'un nœud
assure que chaque nœud a toujours un sous-arbre gauche \textbf{et} un
sous-arbre droit, éventuellement vides, ce qui facilite le traitement
des arbres dans les algorithmes suivants. On matérialise ici l'aspect
récursif de la structure.

\begin{Shaded}
\begin{Highlighting}[]
\KeywordTok{class}\NormalTok{ ArbreBinaire:}
    \CommentTok{""" Implémentation de la structure d\textquotesingle{}arbre binaire """}

    \KeywordTok{def} \FunctionTok{\_\_init\_\_}\NormalTok{(}\VariableTok{self}\NormalTok{):}
        \VariableTok{self}\NormalTok{.racine }\OperatorTok{=} \VariableTok{None}
        \CommentTok{\# les sous{-}arbres gauche et droit doivent être des }
        \CommentTok{\# instances de l\textquotesingle{}objet ArbreBinaire}
        \VariableTok{self}\NormalTok{.gauche }\OperatorTok{=} \VariableTok{None}
        \VariableTok{self}\NormalTok{.droit }\OperatorTok{=} \VariableTok{None}

    \KeywordTok{def}\NormalTok{ setRacine(}\VariableTok{self}\NormalTok{, racine):}
        \CommentTok{"""définit la clef de la racine de l\textquotesingle{}instance}
\CommentTok{         et crée les sous arbres vides gauches et droits"""}
        \VariableTok{self}\NormalTok{.racine }\OperatorTok{=}\NormalTok{ racine}
        \ControlFlowTok{if} \VariableTok{self}\NormalTok{.gauche }\KeywordTok{is} \VariableTok{None}\NormalTok{:}
            \VariableTok{self}\NormalTok{.gauche }\OperatorTok{=}\NormalTok{ ArbreBinaire()}
        \ControlFlowTok{if} \VariableTok{self}\NormalTok{.droit }\KeywordTok{is} \VariableTok{None}\NormalTok{:}
            \VariableTok{self}\NormalTok{.droit }\OperatorTok{=}\NormalTok{ ArbreBinaire()}
    
    \KeywordTok{def}\NormalTok{ getRacine(}\VariableTok{self}\NormalTok{):}
        \CommentTok{"""retourne la clef de la racine de l\textquotesingle{}arbre"""}
        \ControlFlowTok{return} \VariableTok{self}\NormalTok{.racine}

    \KeywordTok{def}\NormalTok{ getSousArbreGauche(}\VariableTok{self}\NormalTok{):}
        \ControlFlowTok{return} \VariableTok{self}\NormalTok{.gauche}

    \KeywordTok{def}\NormalTok{ setSousArbreGauche(}\VariableTok{self}\NormalTok{, arbre):}
        \ControlFlowTok{if} \BuiltInTok{isinstance}\NormalTok{(arbre, ArbreBinaire):}
            \VariableTok{self}\NormalTok{.gauche }\OperatorTok{=}\NormalTok{ arbre}

    \KeywordTok{def}\NormalTok{ getSousArbreDroit(}\VariableTok{self}\NormalTok{):}
        \ControlFlowTok{return} \VariableTok{self}\NormalTok{.droit}

    \KeywordTok{def}\NormalTok{ setSousArbreDroit(}\VariableTok{self}\NormalTok{, arbre):}
        \ControlFlowTok{if} \BuiltInTok{isinstance}\NormalTok{(arbre, ArbreBinaire):}
            \VariableTok{self}\NormalTok{.droit }\OperatorTok{=}\NormalTok{ arbre}

    \KeywordTok{def}\NormalTok{ estVide(}\VariableTok{self}\NormalTok{) }\OperatorTok{{-}\textgreater{}} \BuiltInTok{bool}\NormalTok{:}
        \ControlFlowTok{return} \VariableTok{self}\NormalTok{.racine }\KeywordTok{is} \VariableTok{None}

    \KeywordTok{def}\NormalTok{ estFeuille(}\VariableTok{self}\NormalTok{) }\OperatorTok{{-}\textgreater{}} \BuiltInTok{bool}\NormalTok{:}
        \ControlFlowTok{if} \VariableTok{self}\NormalTok{.estVide():}
            \ControlFlowTok{return} \VariableTok{False}
        \ControlFlowTok{else}\NormalTok{:}
            \ControlFlowTok{return} \VariableTok{self}\NormalTok{.gauche.estVide() }\KeywordTok{and} \VariableTok{self}\NormalTok{.droit.estVide()}

    \KeywordTok{def} \FunctionTok{\_\_str\_\_}\NormalTok{(}\VariableTok{self}\NormalTok{):}
        \ControlFlowTok{if} \VariableTok{self}\NormalTok{.estVide():}
            \ControlFlowTok{return} \StringTok{"()"}
        \ControlFlowTok{elif} \VariableTok{self}\NormalTok{.estFeuille():}
            \ControlFlowTok{return} \SpecialStringTok{f"(\textquotesingle{}}\SpecialCharTok{\{}\VariableTok{self}\SpecialCharTok{.}\NormalTok{racine}\SpecialCharTok{\}}\SpecialStringTok{\textquotesingle{}, (), ())"}
        \ControlFlowTok{else}\NormalTok{:}
            \ControlFlowTok{return} \SpecialStringTok{f"(\textquotesingle{}}\SpecialCharTok{\{}\VariableTok{self}\SpecialCharTok{.}\NormalTok{racine}\SpecialCharTok{\}}\SpecialStringTok{\textquotesingle{}, }\SpecialCharTok{\{}\VariableTok{self}\SpecialCharTok{.}\NormalTok{gauche}\SpecialCharTok{.}\FunctionTok{\_\_str\_\_}\NormalTok{()}\SpecialCharTok{\}}\SpecialStringTok{, }\SpecialCharTok{\{}\VariableTok{self}\SpecialCharTok{.}\NormalTok{droit}\SpecialCharTok{.}\FunctionTok{\_\_str\_\_}\NormalTok{()}\SpecialCharTok{\}}\SpecialStringTok{)"}
\end{Highlighting}
\end{Shaded}

La classe est complétée par une méthode \texttt{estVide} permettant de
tester si un arbre est vide ou non et une méthode \texttt{estFeuille}
permettant de tester si un nœud est une feuille ou non (on confond un
nœud avec un arbre de hauteur 1).

La dernière méthode est la méthode spéciale \texttt{\_\_str\_\_} qui
définit la façon dont un arbre va être affiché par la fonction
\texttt{print}. Ici, on a choisi un affichage sous forme de tuple du
type \texttt{(clef,\ sous-arbre\ gauche,\ sous-arbre\ droit)}.

Pour créer un module, on enregistre le code ci-dessus dans un fichier
nommé par exemple \texttt{structures.py}.

On peut ensuite utiliser notre nouvelle structure dans un autre fichier
Python (dans le même dossier), ou dans la console interactive, en
important le module :

\begin{Shaded}
\begin{Highlighting}[]
\ImportTok{from}\NormalTok{ structures }\ImportTok{import} \OperatorTok{*}

\NormalTok{a }\OperatorTok{=}\NormalTok{ ArbreBinaire()}
\NormalTok{a.setRacine(}\DecValTok{8}\NormalTok{)}
\NormalTok{a.getSousArbreGauche().setRacine(}\DecValTok{3}\NormalTok{)}
\NormalTok{a.getSousArbreDroit().setRacine(}\DecValTok{9}\NormalTok{)}
\NormalTok{b }\OperatorTok{=}\NormalTok{ a.getSousArbreGauche()}
\NormalTok{c }\OperatorTok{=}\NormalTok{ a.getSousArbreDroit()}
\NormalTok{b.getSousArbreGauche().setRacine(}\DecValTok{7}\NormalTok{)}
\NormalTok{b.getSousArbreDroit().setRacine(}\DecValTok{5}\NormalTok{)}
\NormalTok{c.getSousArbreDroit().setRacine(}\DecValTok{1}\NormalTok{)}
\BuiltInTok{print}\NormalTok{(a)}
\end{Highlighting}
\end{Shaded}

On obtient en sortie :

\begin{Shaded}
\begin{Highlighting}[]
\OperatorTok{\textgreater{}\textgreater{}\textgreater{}}\NormalTok{ (}\DecValTok{8}\NormalTok{, (}\DecValTok{3}\NormalTok{, (}\DecValTok{7}\NormalTok{, (), ()), (}\DecValTok{5}\NormalTok{, (), ())), (}\DecValTok{9}\NormalTok{, (), (}\DecValTok{1}\NormalTok{, (), ())))}
\end{Highlighting}
\end{Shaded}

Cela correspond à l'arbre représenté ci-dessous :

\begin{figure}

{\centering \includegraphics[width=0.3\textwidth,height=\textheight]{arbre_bin3.png}

}

\end{figure}

On peut tester les autres méthodes dans la console :

\begin{Shaded}
\begin{Highlighting}[]
\BuiltInTok{print}\NormalTok{(c)}
\OperatorTok{\textgreater{}\textgreater{}\textgreater{}}\NormalTok{ (}\DecValTok{9}\NormalTok{, (), (}\DecValTok{1}\NormalTok{, (), ()))}
\NormalTok{c.getSousArbreGauche().estVide()}
\OperatorTok{\textgreater{}\textgreater{}\textgreater{}} \VariableTok{True}
\NormalTok{c.estFeuille()}
\OperatorTok{\textgreater{}\textgreater{}\textgreater{}} \VariableTok{False}
\NormalTok{c.getSousArbreDroit().estFeuille()}
\OperatorTok{\textgreater{}\textgreater{}\textgreater{}} \VariableTok{True}
\end{Highlighting}
\end{Shaded}

Nous pouvons maintenant ajouter au fichier \texttt{structures.py} les
deux fonctions suivantes (en dehors de la classe \texttt{ArbreBinaire}
car ce ne sont pas des méthodes) qui retournent respectivement la taille
et la hauteur d'un arbre binaire.

\begin{Shaded}
\begin{Highlighting}[]
\KeywordTok{def}\NormalTok{ taille(arbre) }\OperatorTok{{-}\textgreater{}} \BuiltInTok{int}\NormalTok{:}
    \CommentTok{"""Retourne la taille de l\textquotesingle{}arbre, càd son nombre de noeuds"""}
    \ControlFlowTok{if}\NormalTok{ arbre.racine }\KeywordTok{is} \VariableTok{None}\NormalTok{:}
        \ControlFlowTok{return} \DecValTok{0}
    \ControlFlowTok{else}\NormalTok{:}
        \ControlFlowTok{return} \DecValTok{1} \OperatorTok{+}\NormalTok{ taille(arbre.gauche) }\OperatorTok{+}\NormalTok{ taille(arbre.droit)}

\KeywordTok{def}\NormalTok{ hauteur(arbre) }\OperatorTok{{-}\textgreater{}} \BuiltInTok{int}\NormalTok{:}
    \CommentTok{"""Retourne la hauteur de l\textquotesingle{}arbre"""}
    \ControlFlowTok{if}\NormalTok{ arbre.racine }\KeywordTok{is} \VariableTok{None}\NormalTok{:}
        \ControlFlowTok{return} \DecValTok{0}
    \ControlFlowTok{else}\NormalTok{:}
        \ControlFlowTok{return} \DecValTok{1} \OperatorTok{+} \BuiltInTok{max}\NormalTok{(hauteur(arbre.gauche), hauteur(arbre.droit))}
\end{Highlighting}
\end{Shaded}

Prendre le temps de bien comprendre comment fonctionnent ces deux
fonctions \ldots{}

\begin{Shaded}
\begin{Highlighting}[]
\NormalTok{taille(a)}
\OperatorTok{\textgreater{}\textgreater{}\textgreater{}} \DecValTok{6}
\NormalTok{hauteur(a)}
\OperatorTok{\textgreater{}\textgreater{}\textgreater{}} \DecValTok{3}
\end{Highlighting}
\end{Shaded}

Ce module \texttt{structures} sera utilisé en exercices et plus tard
dans l'année lorsque nous étudierons les algorithmes sur les arbres.

\hypertarget{arbres-binaires-de-recherche-abr}{%
\subsection{2. Arbres binaires de recherche
(ABR)}\label{arbres-binaires-de-recherche-abr}}

Les ABR sont des arbres binaires. Nous pouvons donc créer une classe
\texttt{ABR} fille de la classe \texttt{ArbreBinaire} en utilisant la
notion d'\textbf{héritage} et de \textbf{polymorphisme} de la POO (voir
les compléments de cours à ce sujet). Nous définissons une méthode
spécifique : l'insertion d'une clef. Cette méthode ajoute une clef à un
ABR existant en s'assurant que l'arbre obtenu est toujours un ABR (le
nouveau nœud est toujours une feuille).

\begin{Shaded}
\begin{Highlighting}[]
\KeywordTok{class}\NormalTok{ ABR(ArbreBinaire):}
    \CommentTok{""" Implémentation de la structure d\textquotesingle{}arbre binaire de recherche """}

    \KeywordTok{def} \FunctionTok{\_\_init\_\_}\NormalTok{(}\VariableTok{self}\NormalTok{):}
        \BuiltInTok{super}\NormalTok{().}\FunctionTok{\_\_init\_\_}\NormalTok{()}

    \KeywordTok{def}\NormalTok{ setRacine(}\VariableTok{self}\NormalTok{, racine):}
        \CommentTok{"""définit la clef de la racine de l\textquotesingle{}instance}
\CommentTok{         et crée les sous arbres vides gauches et droits}
\CommentTok{         Provoque une erreur si la racine casse la structure d\textquotesingle{}ABR"""}
        \VariableTok{self}\NormalTok{.racine }\OperatorTok{=}\NormalTok{ racine}
        \ControlFlowTok{if} \VariableTok{self}\NormalTok{.gauche }\KeywordTok{is} \VariableTok{None}\NormalTok{:}
            \VariableTok{self}\NormalTok{.gauche }\OperatorTok{=}\NormalTok{ ABR()}
        \ControlFlowTok{if} \VariableTok{self}\NormalTok{.droit }\KeywordTok{is} \VariableTok{None}\NormalTok{:}
            \VariableTok{self}\NormalTok{.droit }\OperatorTok{=}\NormalTok{ ABR()}
        \ControlFlowTok{if} \KeywordTok{not}\NormalTok{ estABR(}\VariableTok{self}\NormalTok{):}
            \ControlFlowTok{raise} \PreprocessorTok{Exception}\NormalTok{(}\StringTok{"Cette affectation de clef casse la structure ABR !!!"}\NormalTok{)}

    \KeywordTok{def}\NormalTok{ insere(}\VariableTok{self}\NormalTok{, racine):}
        \CommentTok{"""insère une clef dans l\textquotesingle{}arbre en préservant la structure ABR"""}
        \ControlFlowTok{if} \VariableTok{self}\NormalTok{.racine }\KeywordTok{is} \VariableTok{None}\NormalTok{:}
            \VariableTok{self}\NormalTok{.racine }\OperatorTok{=}\NormalTok{ racine}
            \VariableTok{self}\NormalTok{.gauche }\OperatorTok{=}\NormalTok{ ABR()}
            \VariableTok{self}\NormalTok{.droit }\OperatorTok{=}\NormalTok{ ABR()}
        \ControlFlowTok{else}\NormalTok{:}
            \ControlFlowTok{if}\NormalTok{ racine }\OperatorTok{\textless{}} \VariableTok{self}\NormalTok{.racine:}
                \VariableTok{self}\NormalTok{.gauche.insere(racine)}
            \ControlFlowTok{else}\NormalTok{:}
                \VariableTok{self}\NormalTok{.droit.insere(racine)}
\end{Highlighting}
\end{Shaded}

Pour définir un arbre binaire de recherche valide, on utilisera toujours
la méthode \texttt{insere} car elle permet de s'assurer de toujours
conserver un ABR.

Pour faciliter la vérification, nous définissons une fonction
\texttt{estABR} qui peut s'appliquer aussi bien à un arbre binaire
quelconque qu'à un ABR et qui retourne \texttt{True} si l'arbre est un
ABR et \texttt{False} sinon.

\begin{Shaded}
\begin{Highlighting}[]
\KeywordTok{def}\NormalTok{ estABR(arbre, mini}\OperatorTok{={-}}\BuiltInTok{float}\NormalTok{(}\StringTok{"inf"}\NormalTok{), maxi}\OperatorTok{=+}\BuiltInTok{float}\NormalTok{(}\StringTok{"inf"}\NormalTok{)) }\OperatorTok{{-}\textgreater{}} \BuiltInTok{bool}\NormalTok{:}
    \ControlFlowTok{if}\NormalTok{ arbre.getRacine() }\KeywordTok{is} \VariableTok{None}\NormalTok{:}
        \ControlFlowTok{return} \VariableTok{True}
    \ControlFlowTok{else}\NormalTok{:}
        \ControlFlowTok{return}\NormalTok{ estABR(arbre.getSousArbreGauche(), mini, arbre.getRacine()) }\KeywordTok{and} 
\NormalTok{               estABR(arbre.getSousArbreDroit(), arbre.getRacine(), maxi) }\KeywordTok{and} 
\NormalTok{               mini }\OperatorTok{\textless{}}\NormalTok{ arbre.racine }\OperatorTok{\textless{}}\NormalTok{ maxi}
\end{Highlighting}
\end{Shaded}

Prendre le temps de bien comprendre cette fonction \ldots{}

Utilisation :

\begin{Shaded}
\begin{Highlighting}[]
\ImportTok{from}\NormalTok{ structures }\ImportTok{import} \OperatorTok{*}

\NormalTok{a }\OperatorTok{=}\NormalTok{ ABR()}
\NormalTok{a.setRacine(}\DecValTok{8}\NormalTok{)}
\NormalTok{a.insere(}\DecValTok{5}\NormalTok{)}
\NormalTok{a.insere(}\DecValTok{3}\NormalTok{)}
\NormalTok{a.insere(}\DecValTok{12}\NormalTok{)}
\NormalTok{a.insere(}\DecValTok{10}\NormalTok{)}
\NormalTok{a.insere(}\DecValTok{15}\NormalTok{)}
\BuiltInTok{print}\NormalTok{(a)}
\BuiltInTok{print}\NormalTok{(estABR(a))}
\CommentTok{\# Affectation directe à proscrire :}
\CommentTok{\# a.getSousArbreDroit().setRacine(1) \#\# provoque une erreur}
\end{Highlighting}
\end{Shaded}

Sortie :

\begin{Shaded}
\begin{Highlighting}[]
\NormalTok{(}\DecValTok{8}\NormalTok{, (}\DecValTok{5}\NormalTok{, (}\DecValTok{3}\NormalTok{, (), ()), ()), (}\DecValTok{12}\NormalTok{, (}\DecValTok{10}\NormalTok{, (), ()), (}\DecValTok{15}\NormalTok{, (), ())))}
\VariableTok{True}
\end{Highlighting}
\end{Shaded}

L'arbre correspond à :

\begin{figure}

{\centering \includegraphics[width=0.3\textwidth,height=\textheight]{arbre.gv.png}

}

\end{figure}

Le module \texttt{structure.py} est à conserver : il sera utilisé en
exercices et dans les chapitres suivants.

\begin{tcolorbox}[enhanced jigsaw, colback=white, opacitybacktitle=0.6, toptitle=1mm, titlerule=0mm, coltitle=black, toprule=.15mm, bottomrule=.15mm, leftrule=.75mm, breakable, colbacktitle=quarto-callout-note-color!10!white, opacityback=0, bottomtitle=1mm, rightrule=.15mm, title=\textcolor{quarto-callout-note-color}{\faInfo}\hspace{0.5em}{Complément}, arc=.35mm, left=2mm]

On peut ajouter une fonctionnalité de représentation graphique d'un
arbre en utilisant les bibliothèques \texttt{networkx} et
\texttt{matplotlib}. Ajouter la fonction ci-dessous au fichier
\texttt{structures.py} :

\begin{Shaded}
\begin{Highlighting}[]
\ImportTok{import}\NormalTok{ networkx }\ImportTok{as}\NormalTok{ nx}
\ImportTok{import}\NormalTok{ matplotlib.pyplot }\ImportTok{as}\NormalTok{ plt}

\KeywordTok{def}\NormalTok{ afficheArbre(arbre, size}\OperatorTok{=}\NormalTok{(}\DecValTok{4}\NormalTok{,}\DecValTok{4}\NormalTok{), null\_node}\OperatorTok{=}\VariableTok{False}\NormalTok{):}
\CommentTok{"""}
\CommentTok{size : tuple de 2 entiers. Si size est int {-}\textgreater{} (size, size)}
\CommentTok{null\_node : si True, trace les liaisons vers les sous{-}arbres vides}
\CommentTok{"""}
\NormalTok{arbreAsTuple }\OperatorTok{=} \BuiltInTok{eval}\NormalTok{(arbre.}\FunctionTok{\_\_str\_\_}\NormalTok{())}
\KeywordTok{def}\NormalTok{ parkour(arbre, noeuds, branches, labels, positions, profondeur, }
\NormalTok{            pos\_courante, pos\_parent, null\_node):}
    \ControlFlowTok{if}\NormalTok{ arbre }\OperatorTok{!=}\NormalTok{ ():}
\NormalTok{        noeuds[}\DecValTok{0}\NormalTok{].append(pos\_courante)}
\NormalTok{        positions[pos\_courante] }\OperatorTok{=}\NormalTok{ (pos\_courante, profondeur)}
\NormalTok{        profondeur }\OperatorTok{{-}=} \DecValTok{1}
\NormalTok{        labels[pos\_courante] }\OperatorTok{=} \BuiltInTok{str}\NormalTok{(arbre[}\DecValTok{0}\NormalTok{])}
\NormalTok{        branches[}\DecValTok{0}\NormalTok{].append((pos\_courante, pos\_parent))}
\NormalTok{        pos\_gauche }\OperatorTok{=}\NormalTok{ pos\_courante }\OperatorTok{{-}} \DecValTok{2} \OperatorTok{**}\NormalTok{ profondeur}
\NormalTok{        parkour(arbre[}\DecValTok{1}\NormalTok{], noeuds, branches, labels, positions, profondeur, }
\NormalTok{                pos\_gauche, pos\_courante, null\_node)}
\NormalTok{        pos\_droit }\OperatorTok{=}\NormalTok{ pos\_courante }\OperatorTok{+} \DecValTok{2} \OperatorTok{**}\NormalTok{ profondeur}
\NormalTok{        parkour(arbre[}\DecValTok{2}\NormalTok{], noeuds, branches, labels, positions, profondeur, }
\NormalTok{                pos\_droit, pos\_courante, null\_node)}
    \ControlFlowTok{elif}\NormalTok{ null\_node:}
\NormalTok{        noeuds[}\DecValTok{1}\NormalTok{].append(pos\_courante)}
\NormalTok{        positions[pos\_courante] }\OperatorTok{=}\NormalTok{ (pos\_courante, profondeur)}
\NormalTok{        branches[}\DecValTok{1}\NormalTok{].append((pos\_courante, pos\_parent))}

\ControlFlowTok{if}\NormalTok{ arbreAsTuple }\OperatorTok{==}\NormalTok{ ():}
    \ControlFlowTok{return}

\NormalTok{branches }\OperatorTok{=}\NormalTok{ [[]]}
\NormalTok{profondeur }\OperatorTok{=}\NormalTok{ hauteur(arbre)}
\NormalTok{pos\_courante }\OperatorTok{=} \DecValTok{2} \OperatorTok{**}\NormalTok{ profondeur}
\NormalTok{noeuds }\OperatorTok{=}\NormalTok{ [[pos\_courante]]}
\NormalTok{positions }\OperatorTok{=}\NormalTok{ \{pos\_courante: (pos\_courante, profondeur)\}}
\NormalTok{labels }\OperatorTok{=}\NormalTok{ \{pos\_courante: }\BuiltInTok{str}\NormalTok{(arbreAsTuple[}\DecValTok{0}\NormalTok{])\}}

\ControlFlowTok{if}\NormalTok{ null\_node:}
\NormalTok{    branches.append([])}
\NormalTok{    noeuds.append([])}

\NormalTok{profondeur }\OperatorTok{{-}=} \DecValTok{1}
\NormalTok{parkour(arbreAsTuple[}\DecValTok{1}\NormalTok{], noeuds, branches, labels, positions, profondeur, }
\NormalTok{        pos\_courante }\OperatorTok{{-}} \DecValTok{2} \OperatorTok{**}\NormalTok{ profondeur, pos\_courante, null\_node)}
\NormalTok{parkour(arbreAsTuple[}\DecValTok{2}\NormalTok{], noeuds, branches, labels, positions, profondeur, }
\NormalTok{        pos\_courante }\OperatorTok{+} \DecValTok{2} \OperatorTok{**}\NormalTok{ profondeur, pos\_courante, null\_node)}

\NormalTok{mon\_arbre }\OperatorTok{=}\NormalTok{ nx.Graph()}

\ControlFlowTok{if} \BuiltInTok{type}\NormalTok{(size) }\OperatorTok{==} \BuiltInTok{int}\NormalTok{:}
\NormalTok{    size }\OperatorTok{=}\NormalTok{ (size, size)}
\NormalTok{plt.figure(figsize}\OperatorTok{=}\NormalTok{size)}

\NormalTok{nx.draw\_networkx\_nodes(mon\_arbre, positions, nodelist}\OperatorTok{=}\NormalTok{noeuds[}\DecValTok{0}\NormalTok{], }
\NormalTok{                       node\_color}\OperatorTok{=}\StringTok{"white"}\NormalTok{, node\_size}\OperatorTok{=}\DecValTok{550}\NormalTok{, edgecolors}\OperatorTok{=}\StringTok{"blue"}\NormalTok{)}
\NormalTok{nx.draw\_networkx\_edges(mon\_arbre, positions, edgelist}\OperatorTok{=}\NormalTok{branches[}\DecValTok{0}\NormalTok{], }
\NormalTok{                       edge\_color}\OperatorTok{=}\StringTok{"black"}\NormalTok{, width}\OperatorTok{=}\DecValTok{2}\NormalTok{)}
\NormalTok{nx.draw\_networkx\_labels(mon\_arbre, positions, labels)}

\ControlFlowTok{if}\NormalTok{ null\_node:}
\NormalTok{    nx.draw\_networkx\_nodes(mon\_arbre, positions, nodelist}\OperatorTok{=}\NormalTok{noeuds[}\DecValTok{1}\NormalTok{], }
\NormalTok{                           node\_color}\OperatorTok{=}\StringTok{"white"}\NormalTok{, node\_size}\OperatorTok{=}\DecValTok{50}\NormalTok{, edgecolors}\OperatorTok{=}\StringTok{"grey"}\NormalTok{)}
\NormalTok{    nx.draw\_networkx\_edges(mon\_arbre, positions, edgelist}\OperatorTok{=}\NormalTok{branches[}\DecValTok{1}\NormalTok{], }
\NormalTok{                           edge\_color}\OperatorTok{=}\StringTok{"grey"}\NormalTok{, width}\OperatorTok{=}\DecValTok{1}\NormalTok{)}

\NormalTok{ax }\OperatorTok{=}\NormalTok{ plt.gca()}
\NormalTok{ax.margins(}\FloatTok{0.1}\NormalTok{)}
\NormalTok{plt.axis(}\StringTok{"off"}\NormalTok{)}
\NormalTok{plt.show()}
\NormalTok{plt.close()}
\end{Highlighting}
\end{Shaded}

Utilisation :

\begin{Shaded}
\begin{Highlighting}[]
\ImportTok{from}\NormalTok{ structures }\ImportTok{import} \OperatorTok{*}

\NormalTok{a }\OperatorTok{=}\NormalTok{ ABR()}
\NormalTok{a.setRacine(}\DecValTok{8}\NormalTok{)}
\NormalTok{a.insere(}\DecValTok{5}\NormalTok{)}
\NormalTok{a.insere(}\DecValTok{3}\NormalTok{)}
\NormalTok{a.insere(}\DecValTok{12}\NormalTok{)}
\NormalTok{a.insere(}\DecValTok{10}\NormalTok{)}
\NormalTok{a.insere(}\DecValTok{15}\NormalTok{)}

\NormalTok{afficheArbre(a)}
\end{Highlighting}
\end{Shaded}

Sortie :

\begin{figure}[H]

{\centering \includegraphics[width=0.3\textwidth,height=\textheight]{arbre.gv.png}

}

\end{figure}

\end{tcolorbox}



\end{document}
