% Options for packages loaded elsewhere
\PassOptionsToPackage{unicode}{hyperref}
\PassOptionsToPackage{hyphens}{url}
\PassOptionsToPackage{dvipsnames,svgnames,x11names}{xcolor}
%
\documentclass[
  a4paper,
  DIV=11,
  numbers=noendperiod]{scrartcl}

\usepackage{amsmath,amssymb}
\usepackage{iftex}
\ifPDFTeX
  \usepackage[T1]{fontenc}
  \usepackage[utf8]{inputenc}
  \usepackage{textcomp} % provide euro and other symbols
\else % if luatex or xetex
  \usepackage{unicode-math}
  \defaultfontfeatures{Scale=MatchLowercase}
  \defaultfontfeatures[\rmfamily]{Ligatures=TeX,Scale=1}
\fi
\usepackage{lmodern}
\ifPDFTeX\else  
    % xetex/luatex font selection
\fi
% Use upquote if available, for straight quotes in verbatim environments
\IfFileExists{upquote.sty}{\usepackage{upquote}}{}
\IfFileExists{microtype.sty}{% use microtype if available
  \usepackage[]{microtype}
  \UseMicrotypeSet[protrusion]{basicmath} % disable protrusion for tt fonts
}{}
\makeatletter
\@ifundefined{KOMAClassName}{% if non-KOMA class
  \IfFileExists{parskip.sty}{%
    \usepackage{parskip}
  }{% else
    \setlength{\parindent}{0pt}
    \setlength{\parskip}{6pt plus 2pt minus 1pt}}
}{% if KOMA class
  \KOMAoptions{parskip=half}}
\makeatother
\usepackage{xcolor}
\usepackage[top=20mm,bottom=20mm,left=20mm,right=20mm,heightrounded]{geometry}
\setlength{\emergencystretch}{3em} % prevent overfull lines
\setcounter{secnumdepth}{-\maxdimen} % remove section numbering
% Make \paragraph and \subparagraph free-standing
\ifx\paragraph\undefined\else
  \let\oldparagraph\paragraph
  \renewcommand{\paragraph}[1]{\oldparagraph{#1}\mbox{}}
\fi
\ifx\subparagraph\undefined\else
  \let\oldsubparagraph\subparagraph
  \renewcommand{\subparagraph}[1]{\oldsubparagraph{#1}\mbox{}}
\fi

\usepackage{color}
\usepackage{fancyvrb}
\newcommand{\VerbBar}{|}
\newcommand{\VERB}{\Verb[commandchars=\\\{\}]}
\DefineVerbatimEnvironment{Highlighting}{Verbatim}{commandchars=\\\{\}}
% Add ',fontsize=\small' for more characters per line
\usepackage{framed}
\definecolor{shadecolor}{RGB}{241,243,245}
\newenvironment{Shaded}{\begin{snugshade}}{\end{snugshade}}
\newcommand{\AlertTok}[1]{\textcolor[rgb]{0.68,0.00,0.00}{#1}}
\newcommand{\AnnotationTok}[1]{\textcolor[rgb]{0.37,0.37,0.37}{#1}}
\newcommand{\AttributeTok}[1]{\textcolor[rgb]{0.40,0.45,0.13}{#1}}
\newcommand{\BaseNTok}[1]{\textcolor[rgb]{0.68,0.00,0.00}{#1}}
\newcommand{\BuiltInTok}[1]{\textcolor[rgb]{0.00,0.23,0.31}{#1}}
\newcommand{\CharTok}[1]{\textcolor[rgb]{0.13,0.47,0.30}{#1}}
\newcommand{\CommentTok}[1]{\textcolor[rgb]{0.37,0.37,0.37}{#1}}
\newcommand{\CommentVarTok}[1]{\textcolor[rgb]{0.37,0.37,0.37}{\textit{#1}}}
\newcommand{\ConstantTok}[1]{\textcolor[rgb]{0.56,0.35,0.01}{#1}}
\newcommand{\ControlFlowTok}[1]{\textcolor[rgb]{0.00,0.23,0.31}{#1}}
\newcommand{\DataTypeTok}[1]{\textcolor[rgb]{0.68,0.00,0.00}{#1}}
\newcommand{\DecValTok}[1]{\textcolor[rgb]{0.68,0.00,0.00}{#1}}
\newcommand{\DocumentationTok}[1]{\textcolor[rgb]{0.37,0.37,0.37}{\textit{#1}}}
\newcommand{\ErrorTok}[1]{\textcolor[rgb]{0.68,0.00,0.00}{#1}}
\newcommand{\ExtensionTok}[1]{\textcolor[rgb]{0.00,0.23,0.31}{#1}}
\newcommand{\FloatTok}[1]{\textcolor[rgb]{0.68,0.00,0.00}{#1}}
\newcommand{\FunctionTok}[1]{\textcolor[rgb]{0.28,0.35,0.67}{#1}}
\newcommand{\ImportTok}[1]{\textcolor[rgb]{0.00,0.46,0.62}{#1}}
\newcommand{\InformationTok}[1]{\textcolor[rgb]{0.37,0.37,0.37}{#1}}
\newcommand{\KeywordTok}[1]{\textcolor[rgb]{0.00,0.23,0.31}{#1}}
\newcommand{\NormalTok}[1]{\textcolor[rgb]{0.00,0.23,0.31}{#1}}
\newcommand{\OperatorTok}[1]{\textcolor[rgb]{0.37,0.37,0.37}{#1}}
\newcommand{\OtherTok}[1]{\textcolor[rgb]{0.00,0.23,0.31}{#1}}
\newcommand{\PreprocessorTok}[1]{\textcolor[rgb]{0.68,0.00,0.00}{#1}}
\newcommand{\RegionMarkerTok}[1]{\textcolor[rgb]{0.00,0.23,0.31}{#1}}
\newcommand{\SpecialCharTok}[1]{\textcolor[rgb]{0.37,0.37,0.37}{#1}}
\newcommand{\SpecialStringTok}[1]{\textcolor[rgb]{0.13,0.47,0.30}{#1}}
\newcommand{\StringTok}[1]{\textcolor[rgb]{0.13,0.47,0.30}{#1}}
\newcommand{\VariableTok}[1]{\textcolor[rgb]{0.07,0.07,0.07}{#1}}
\newcommand{\VerbatimStringTok}[1]{\textcolor[rgb]{0.13,0.47,0.30}{#1}}
\newcommand{\WarningTok}[1]{\textcolor[rgb]{0.37,0.37,0.37}{\textit{#1}}}

\providecommand{\tightlist}{%
  \setlength{\itemsep}{0pt}\setlength{\parskip}{0pt}}\usepackage{longtable,booktabs,array}
\usepackage{calc} % for calculating minipage widths
% Correct order of tables after \paragraph or \subparagraph
\usepackage{etoolbox}
\makeatletter
\patchcmd\longtable{\par}{\if@noskipsec\mbox{}\fi\par}{}{}
\makeatother
% Allow footnotes in longtable head/foot
\IfFileExists{footnotehyper.sty}{\usepackage{footnotehyper}}{\usepackage{footnote}}
\makesavenoteenv{longtable}
\usepackage{graphicx}
\makeatletter
\def\maxwidth{\ifdim\Gin@nat@width>\linewidth\linewidth\else\Gin@nat@width\fi}
\def\maxheight{\ifdim\Gin@nat@height>\textheight\textheight\else\Gin@nat@height\fi}
\makeatother
% Scale images if necessary, so that they will not overflow the page
% margins by default, and it is still possible to overwrite the defaults
% using explicit options in \includegraphics[width, height, ...]{}
\setkeys{Gin}{width=\maxwidth,height=\maxheight,keepaspectratio}
% Set default figure placement to htbp
\makeatletter
\def\fps@figure{htbp}
\makeatother

\usepackage{fancyhdr} \pagestyle{fancy} \usepackage{lastpage}
\KOMAoption{captions}{tablesignature}
\makeatletter
\@ifpackageloaded{tcolorbox}{}{\usepackage[skins,breakable]{tcolorbox}}
\@ifpackageloaded{fontawesome5}{}{\usepackage{fontawesome5}}
\definecolor{quarto-callout-color}{HTML}{909090}
\definecolor{quarto-callout-note-color}{HTML}{0758E5}
\definecolor{quarto-callout-important-color}{HTML}{CC1914}
\definecolor{quarto-callout-warning-color}{HTML}{EB9113}
\definecolor{quarto-callout-tip-color}{HTML}{00A047}
\definecolor{quarto-callout-caution-color}{HTML}{FC5300}
\definecolor{quarto-callout-color-frame}{HTML}{acacac}
\definecolor{quarto-callout-note-color-frame}{HTML}{4582ec}
\definecolor{quarto-callout-important-color-frame}{HTML}{d9534f}
\definecolor{quarto-callout-warning-color-frame}{HTML}{f0ad4e}
\definecolor{quarto-callout-tip-color-frame}{HTML}{02b875}
\definecolor{quarto-callout-caution-color-frame}{HTML}{fd7e14}
\makeatother
\makeatletter
\makeatother
\makeatletter
\makeatother
\makeatletter
\@ifpackageloaded{caption}{}{\usepackage{caption}}
\AtBeginDocument{%
\ifdefined\contentsname
  \renewcommand*\contentsname{Table des matières}
\else
  \newcommand\contentsname{Table des matières}
\fi
\ifdefined\listfigurename
  \renewcommand*\listfigurename{Liste des Figures}
\else
  \newcommand\listfigurename{Liste des Figures}
\fi
\ifdefined\listtablename
  \renewcommand*\listtablename{Liste des Tables}
\else
  \newcommand\listtablename{Liste des Tables}
\fi
\ifdefined\figurename
  \renewcommand*\figurename{Figure}
\else
  \newcommand\figurename{Figure}
\fi
\ifdefined\tablename
  \renewcommand*\tablename{Tableau}
\else
  \newcommand\tablename{Tableau}
\fi
}
\@ifpackageloaded{float}{}{\usepackage{float}}
\floatstyle{ruled}
\@ifundefined{c@chapter}{\newfloat{codelisting}{h}{lop}}{\newfloat{codelisting}{h}{lop}[chapter]}
\floatname{codelisting}{Listing}
\newcommand*\listoflistings{\listof{codelisting}{Liste des Listings}}
\makeatother
\makeatletter
\@ifpackageloaded{caption}{}{\usepackage{caption}}
\@ifpackageloaded{subcaption}{}{\usepackage{subcaption}}
\makeatother
\makeatletter
\@ifpackageloaded{tcolorbox}{}{\usepackage[skins,breakable]{tcolorbox}}
\makeatother
\makeatletter
\@ifundefined{shadecolor}{\definecolor{shadecolor}{rgb}{.97, .97, .97}}
\makeatother
\makeatletter
\makeatother
\makeatletter
\makeatother
\ifLuaTeX
\usepackage[bidi=basic]{babel}
\else
\usepackage[bidi=default]{babel}
\fi
\babelprovide[main,import]{french}
% get rid of language-specific shorthands (see #6817):
\let\LanguageShortHands\languageshorthands
\def\languageshorthands#1{}
\ifLuaTeX
  \usepackage{selnolig}  % disable illegal ligatures
\fi
\IfFileExists{bookmark.sty}{\usepackage{bookmark}}{\usepackage{hyperref}}
\IfFileExists{xurl.sty}{\usepackage{xurl}}{} % add URL line breaks if available
\urlstyle{same} % disable monospaced font for URLs
\hypersetup{
  pdftitle={Arbres (Cours)},
  pdflang={fr},
  colorlinks=true,
  linkcolor={blue},
  filecolor={Maroon},
  citecolor={Blue},
  urlcolor={Blue},
  pdfcreator={LaTeX via pandoc}}

\title{Arbres (Cours)}
\usepackage{etoolbox}
\makeatletter
\providecommand{\subtitle}[1]{% add subtitle to \maketitle
  \apptocmd{\@title}{\par {\large #1 \par}}{}{}
}
\makeatother
\subtitle{S4 - Arbres et graphes}
\author{}
\date{}

\begin{document}
\maketitle
\lhead{Spécialité NSI} \rhead{Terminale} \chead{} \cfoot{} \lfoot{Lycée \'Emile Duclaux} \rfoot{Page \thepage/\pageref{LastPage}} \renewcommand{\headrulewidth}{0pt} \renewcommand{\footrulewidth}{0pt} \thispagestyle{fancy} \vspace{-2cm}

\ifdefined\Shaded\renewenvironment{Shaded}{\begin{tcolorbox}[borderline west={3pt}{0pt}{shadecolor}, interior hidden, enhanced, sharp corners, breakable, frame hidden, boxrule=0pt]}{\end{tcolorbox}}\fi

\hypertarget{introduction}{%
\subsection{1. Introduction}\label{introduction}}

\hypertarget{vocabulaire}{%
\subsubsection{1.1. Vocabulaire}\label{vocabulaire}}

\begin{tcolorbox}[enhanced jigsaw, colback=white, bottomrule=.15mm, colframe=quarto-callout-tip-color-frame, left=2mm, rightrule=.15mm, title=\textcolor{quarto-callout-tip-color}{\faLightbulb}\hspace{0.5em}{Définition}, opacitybacktitle=0.6, breakable, leftrule=.75mm, toptitle=1mm, colbacktitle=quarto-callout-tip-color!10!white, arc=.35mm, opacityback=0, coltitle=black, bottomtitle=1mm, titlerule=0mm, toprule=.15mm]

En informatique, un arbre est une structure de données qui peut se
représenter sous forme d'une \textbf{hiérarchie} dont chaque élément est
appelé \textbf{nœud}, le nœud initial étant appelé \textbf{racine}.

\end{tcolorbox}

\begin{figure}

{\centering \includegraphics[width=0.3\textwidth,height=\textheight]{arbre.png}

}

\end{figure}

\begin{tcolorbox}[enhanced jigsaw, colback=white, bottomrule=.15mm, colframe=quarto-callout-note-color-frame, left=2mm, rightrule=.15mm, title=\textcolor{quarto-callout-note-color}{\faInfo}\hspace{0.5em}{Vocabulaire}, opacitybacktitle=0.6, breakable, leftrule=.75mm, toptitle=1mm, colbacktitle=quarto-callout-note-color!10!white, arc=.35mm, opacityback=0, coltitle=black, bottomtitle=1mm, titlerule=0mm, toprule=.15mm]

\begin{itemize}
\tightlist
\item
  Chaque nœud a exactement un nœud \textbf{père}, sauf le nœud
  \textbf{racine} (situé en haut) qui n'a pas de père.
\item
  Un nœud peut avoir une nombre quelconque de \textbf{fils}.
\item
  Les nœuds qui n'ont pas de fils sont appelés des \textbf{feuilles}
  (situées aux extrémités des branches !).
\item
  Les nœuds possèdent une valeur, ou \textbf{clef}, ou encore
  \textbf{étiquette}.
\end{itemize}

\end{tcolorbox}

Par exemple, dans l'arbre représenté ci-dessus :

\begin{tcolorbox}[enhanced jigsaw, colback=white, bottomrule=.15mm, colframe=quarto-callout-caution-color-frame, left=2mm, rightrule=.15mm, title=\textcolor{quarto-callout-caution-color}{\faFire}\hspace{0.5em}{Exemple}, opacitybacktitle=0.6, breakable, leftrule=.75mm, toptitle=1mm, colbacktitle=quarto-callout-caution-color!10!white, arc=.35mm, opacityback=0, coltitle=black, bottomtitle=1mm, titlerule=0mm, toprule=.15mm]

\begin{itemize}
\tightlist
\item
  La racine possède l'étiquette \textbf{D}.
\item
  Le nœud père \textbf{D} a trois fils \textbf{U}, \textbf{L} et
  \textbf{A}.
\item
  Le nœud \textbf{V} a pour père le nœud \textbf{A}.
\item
  Les nœuds \textbf{C}, \textbf{L} et \textbf{X} sont des feuilles.
\end{itemize}

\end{tcolorbox}

\newpage{}

\hypertarget{exemples}{%
\subsubsection{1.2. Exemples}\label{exemples}}

Voici quelques exemples de situations dans lesquelles une structure de
données arborescente est utile.

\hypertarget{larborescence-dun-disque-dur}{%
\paragraph{L'arborescence d'un disque
dur}\label{larborescence-dun-disque-dur}}

Déjà rencontrée dans le chapitre sur les systèmes d'exploitation en
première, une arborescence de dossiers dans un disque dur peut être
modélisée par un arbre.

\begin{figure}

{\centering \includegraphics[width=0.3\textwidth,height=\textheight]{arborescence.png}

}

\end{figure}

\hypertarget{le-dom-dune-page-web}{%
\paragraph{Le DOM d'une page web}\label{le-dom-dune-page-web}}

LE DOM (Document Object Model) est une interface de programmation pour
les pages web dans laquelle une page HTML est modélisée sous la forme
d'un arbre.

Par exemple, le code HTML ci-dessous sera modélisé par l'arbre
en-dessous (source :
\href{https://www.w3.org/TR/WD-DOM/introduction.html}{w3.org}).

\begin{Shaded}
\begin{Highlighting}[]
\KeywordTok{\textless{}TABLE\textgreater{}}
    \KeywordTok{\textless{}ROWS\textgreater{}} 
      \KeywordTok{\textless{}TR\textgreater{}} 
        \KeywordTok{\textless{}TD\textgreater{}}\NormalTok{Shady Grove}\KeywordTok{\textless{}/TD\textgreater{}}
        \KeywordTok{\textless{}TD\textgreater{}}\NormalTok{Aeolian}\KeywordTok{\textless{}/TD\textgreater{}} 
      \KeywordTok{\textless{}/TR\textgreater{}} 
      \KeywordTok{\textless{}TR\textgreater{}}
        \KeywordTok{\textless{}TD\textgreater{}}\NormalTok{Over the River, Charlie}\KeywordTok{\textless{}/TD\textgreater{}}
        \KeywordTok{\textless{}TD\textgreater{}}\NormalTok{Dorian}\KeywordTok{\textless{}/TD\textgreater{}} 
      \KeywordTok{\textless{}/TR\textgreater{}} 
    \KeywordTok{\textless{}/ROWS\textgreater{}}
\KeywordTok{\textless{}/TABLE\textgreater{}}
\end{Highlighting}
\end{Shaded}

\begin{figure}

{\centering \includegraphics[width=0.3\textwidth,height=\textheight]{DOM.png}

}

\end{figure}

\hypertarget{un-arbre-guxe9nuxe9alogique}{%
\paragraph{Un arbre généalogique}\label{un-arbre-guxe9nuxe9alogique}}

\begin{figure}

{\centering \includegraphics[width=0.3\textwidth,height=\textheight]{arbre_genea.png}

}

\end{figure}

\hypertarget{caractuxe9ristiques-dun-arbre}{%
\subsubsection{1.3. Caractéristiques d'un
arbre}\label{caractuxe9ristiques-dun-arbre}}

Différents paramètres numériques peuvent être définis concernant un
arbre.

\begin{tcolorbox}[enhanced jigsaw, colback=white, bottomrule=.15mm, colframe=quarto-callout-tip-color-frame, left=2mm, rightrule=.15mm, title=\textcolor{quarto-callout-tip-color}{\faLightbulb}\hspace{0.5em}{Définitions}, opacitybacktitle=0.6, breakable, leftrule=.75mm, toptitle=1mm, colbacktitle=quarto-callout-tip-color!10!white, arc=.35mm, opacityback=0, coltitle=black, bottomtitle=1mm, titlerule=0mm, toprule=.15mm]

\begin{itemize}
\tightlist
\item
  La \textbf{taille} d'un arbre est son nombre total de nœuds.
\item
  La \textbf{profondeur} d'un nœud est le nombre de nœuds de la branche
  allant de la racine à ce nœud, en comptant les extrémités.
\item
  La \textbf{hauteur} d'un arbre est la profondeur de son nœud le plus
  profond. Par convention, si l'arbre est vide, sa hauteur vaut 0, si
  l'arbre n'est composé que d'un nœud racine, sa hauteur vaut 1.
\end{itemize}

\end{tcolorbox}

\begin{tcolorbox}[enhanced jigsaw, colback=white, bottomrule=.15mm, colframe=quarto-callout-caution-color-frame, left=2mm, rightrule=.15mm, title=\textcolor{quarto-callout-caution-color}{\faFire}\hspace{0.5em}{Exemple}, opacitybacktitle=0.6, breakable, leftrule=.75mm, toptitle=1mm, colbacktitle=quarto-callout-caution-color!10!white, arc=.35mm, opacityback=0, coltitle=black, bottomtitle=1mm, titlerule=0mm, toprule=.15mm]

Considérons l'arbre ci-dessous :

\begin{figure}[H]

{\centering \includegraphics[width=0.3\textwidth,height=\textheight]{arbre.png}

}

\end{figure}

\begin{itemize}
\tightlist
\item
  La taille de cet arbre est égale à 7.
\item
  La profondeur du nœud \textbf{C} est égale à 3, celle de \textbf{X}
  est égale à 4.
\item
  La hauteur de cet arbre est égale à 4 : \textbf{X} est le nœud le plus
  profond.
\end{itemize}

\end{tcolorbox}

\hypertarget{arbres-binaires}{%
\subsection{2. Arbres binaires}\label{arbres-binaires}}

\hypertarget{duxe9finition-1}{%
\subsubsection{2.1. Définition}\label{duxe9finition-1}}

\begin{tcolorbox}[enhanced jigsaw, colback=white, bottomrule=.15mm, colframe=quarto-callout-tip-color-frame, left=2mm, rightrule=.15mm, title=\textcolor{quarto-callout-tip-color}{\faLightbulb}\hspace{0.5em}{Définition}, opacitybacktitle=0.6, breakable, leftrule=.75mm, toptitle=1mm, colbacktitle=quarto-callout-tip-color!10!white, arc=.35mm, opacityback=0, coltitle=black, bottomtitle=1mm, titlerule=0mm, toprule=.15mm]

Un \textbf{arbre binaire} est un arbre dans lequel chaque nœud possède
\textbf{au plus} deux fils au niveau inférieur, appelés \emph{gauche} et
\emph{droite}.

\end{tcolorbox}

L'arbre donné en exemple ci-dessus n'est pas un arbre binaire car le
nœud \textbf{D} possède 3 fils.

L'arbre ci-dessous est un arbre binaire :

\begin{figure}

{\centering \includegraphics[width=0.3\textwidth,height=\textheight]{arbre_bin.png}

}

\end{figure}

\hypertarget{sous-arbres}{%
\subsubsection{2.2. Sous-arbres}\label{sous-arbres}}

Chaque nœud n'ayant que deux fils (au maximum), nous pouvons définir un
sous-arbre gauche et un sous-arbre droit qui sont tous les deux
également des arbres binaires (éventuellement vides).

\begin{figure}

{\centering \includegraphics[width=0.3\textwidth,height=\textheight]{arbre_bin2.png}

}

\end{figure}

\begin{tcolorbox}[enhanced jigsaw, colback=white, bottomrule=.15mm, colframe=quarto-callout-caution-color-frame, left=2mm, rightrule=.15mm, title=\textcolor{quarto-callout-caution-color}{\faFire}\hspace{0.5em}{Exemple}, opacitybacktitle=0.6, breakable, leftrule=.75mm, toptitle=1mm, colbacktitle=quarto-callout-caution-color!10!white, arc=.35mm, opacityback=0, coltitle=black, bottomtitle=1mm, titlerule=0mm, toprule=.15mm]

Pour l'arbre représenté ci-dessous, nous avons mis en évidence le
sous-arbre gauche et le sous-arbre droit du nœud racine \textbf{C}.

Le nœud \textbf{Q} admet comme sous-arbre gauche le nœud \textbf{H} et
comme sous-arbre droit, l'arbre vide.

\end{tcolorbox}

Cette notion de sous arbre permet de mettre en évidence la
\textbf{structure récursive} d'un arbre binaire : un arbre binaire est
un arbre dans lequel chaque nœud possède un arbre fils gauche et un
arbre fils droit qui sont tous deux des arbres binaires.

\hypertarget{hauteur-dun-arbre-binaire}{%
\subsubsection{2.3. Hauteur d'un arbre
binaire}\label{hauteur-dun-arbre-binaire}}

Nous avons défini plus haut les notions de \textbf{taille} et de
\textbf{hauteur} d'un arbre.

Notons ici \(n\) la taille d'un arbre binaire et \(h\) sa hauteur. Ces
deux entiers ne sont pas indépendants l'un de l'autre.

Un cas extrême est le cas de l'arbre filiforme dans lequel chaque nœud
n'a qu'un fils. La figure ci-dessous est réalisée avec \(n=7\).

\begin{figure}

{\centering \includegraphics[width=0.25\textwidth,height=\textheight]{arbre_fili.png}

}

\end{figure}

Dans ce cas, nous avons \(n=h\) : la hauteur de l'arbre est égale au
nombre de nœuds de l'arbre.

Un autre cas ``extrême'' est le cas d'un \textbf{arbre complet} : il
s'agit d'un arbre binaire dans lequel tous les nœuds (sauf les feuilles)
ont exactement deux fils et toutes les feuilles ont la même profondeur.
La figure ci-dessous représente un arbre complet à 7 nœuds.

\begin{figure}

{\centering \includegraphics[width=0.3\textwidth,height=\textheight]{arbre_complet.png}

}

\end{figure}

Dans un tel arbre, le nombre de nœuds est \textbf{doublé à chaque
niveau}. Si la hauteur est \(h\), le nombre total de nœuds est donc égal
à :

\[n=1+2^1+2^2+2^3+\ldots+2^{h-1}\]

(on numérote les niveaux de \(0\) à \(h-1\))

On obtient donc le résultat suivant :

\begin{tcolorbox}[enhanced jigsaw, colback=white, bottomrule=.15mm, colframe=quarto-callout-important-color-frame, left=2mm, rightrule=.15mm, title=\textcolor{quarto-callout-important-color}{\faExclamation}\hspace{0.5em}{Lien entre hauteur et taille d'un arbre binaire complet}, opacitybacktitle=0.6, breakable, leftrule=.75mm, toptitle=1mm, colbacktitle=quarto-callout-important-color!10!white, arc=.35mm, opacityback=0, coltitle=black, bottomtitle=1mm, titlerule=0mm, toprule=.15mm]

Soit un arbre binaire complet de taille \(n\) et de hauteur \(h\).

On a alors :

\[n=2^h-1\]

\end{tcolorbox}

\begin{tcolorbox}[enhanced jigsaw, colback=white, bottomrule=.15mm, colframe=quarto-callout-caution-color-frame, left=2mm, rightrule=.15mm, title=\textcolor{quarto-callout-caution-color}{\faFire}\hspace{0.5em}{Preuve}, opacitybacktitle=0.6, breakable, leftrule=.75mm, toptitle=1mm, colbacktitle=quarto-callout-caution-color!10!white, arc=.35mm, opacityback=0, coltitle=black, bottomtitle=1mm, titlerule=0mm, toprule=.15mm]

Nous avons vu que \(n=1+2^1+2^2+2^3+\ldots+2^{h-1}\). Cette somme est la
somme des premiers termes d'une suite géométrique de raison 2.

On peut donc utiliser la formule de calcul vue en première en
Mathématiques :

\[1+2^1+2^2+2^3+\ldots+2^{h-1} = \frac{2^h-1}{2-1}=2^h-1\]

Pour ceux qui n'auraient jamais vu cette formule, sa preuve n'est pas
difficile : il suffit de calculer le produit
\((2-1)(1+2^1+2^2+2^3+\ldots+2^{h-1})\) en développant :

\[(2-1)(1+2^1+2^2+2^3+\ldots+2^{h-1})= 2^1+2^2+2^3+\ldots+2^{h} -1-2^1-2^2-2^3-\ldots-2^{h-1}\]

Tous les termes s'annulent sauf \(2^{h}\) et \(-1\), d'où le résultat.

\end{tcolorbox}

Tout arbre étant situé entre ces deux cas extrêmes, nous pouvons écrire
un encadrement du nombre de nœuds \(n\) en fonction de la hauteur \(h\),
valable pour tout arbre binaire :

\[h\leqslant n\leqslant 2^h-1\]

À partir de cet encadrement de \(n\), on peut déduire un encadrement de
\(h\). Nous avons déjà de façon évidente \(h\leqslant n\). La seconde
inégalité \(n\leqslant 2^h-1\) est équivalente à \(n+1\leqslant 2^h\).

Pour isoler \(h\) dans cette inégalité, nous avons besoin de la fonction
\textbf{logarithme binaire}. Le logarithme binaire d'un entier positif
est son exposant quand on l'écrit sous la forme d'une puissance de 2.
Par exemple \(\log_2(8)=3\) car \(2^3=8\) et \(\log_2(2^h)=h\). Cette
fonction \(\log_2\) étant intuitivement croissante, nous obtenons, en
l'appliquant à l'inégalité \(n+1\leqslant 2^h\) :
\(\log_2(n+1)\leqslant h\).

Nous avons finalement l'encadrement suivant :

\begin{tcolorbox}[enhanced jigsaw, colback=white, bottomrule=.15mm, colframe=quarto-callout-important-color-frame, left=2mm, rightrule=.15mm, title=\textcolor{quarto-callout-important-color}{\faExclamation}\hspace{0.5em}{Encadrement de la hauteur d'un arbre binaire}, opacitybacktitle=0.6, breakable, leftrule=.75mm, toptitle=1mm, colbacktitle=quarto-callout-important-color!10!white, arc=.35mm, opacityback=0, coltitle=black, bottomtitle=1mm, titlerule=0mm, toprule=.15mm]

Soit un arbre binaire de taille \(n\) et de hauteur \(h\).

On a alors :

\[\log_2(n+1)\leqslant h\leqslant n\]

\end{tcolorbox}

Cet encadrement nous sera utile lors du calcul du coût d'exécution des
algorithmes sur les arbres.

\hypertarget{arbres-binaires-de-recherche}{%
\subsection{3. Arbres binaires de
recherche}\label{arbres-binaires-de-recherche}}

\begin{tcolorbox}[enhanced jigsaw, colback=white, bottomrule=.15mm, colframe=quarto-callout-tip-color-frame, left=2mm, rightrule=.15mm, title=\textcolor{quarto-callout-tip-color}{\faLightbulb}\hspace{0.5em}{Définition}, opacitybacktitle=0.6, breakable, leftrule=.75mm, toptitle=1mm, colbacktitle=quarto-callout-tip-color!10!white, arc=.35mm, opacityback=0, coltitle=black, bottomtitle=1mm, titlerule=0mm, toprule=.15mm]

Un \textbf{arbre binaire de recherche} (ABR) est un arbre binaire dont
les clefs des nœuds (leur valeur) vérifient les propriétés suivantes :

\begin{itemize}
\tightlist
\item
  la clef d'un nœud est \textbf{supérieure} à celle de chaque nœud de
  son \textbf{sous-arbre gauche}.
\item
  la clef d'un nœud est \textbf{inférieure} à celle du chaque nœud de
  son \textbf{sous-arbre droit}.
\end{itemize}

\end{tcolorbox}

Cette définition n'a de sens que dans le cas où les clefs des nœuds sont
toujours \textbf{comparables} entre elles. Dans la pratique, nous
travaillerons toujours avec des clefs numériques ou alphanumériques
(ordre alphabétique).

Nous supposerons toujours que toutes les clefs sont
\textbf{différentes}.

\begin{figure}

{\centering \includegraphics[width=0.3\textwidth,height=\textheight]{ABR.png}

}

\end{figure}

\begin{tcolorbox}[enhanced jigsaw, colback=white, bottomrule=.15mm, colframe=quarto-callout-note-color-frame, left=2mm, rightrule=.15mm, title=\textcolor{quarto-callout-note-color}{\faInfo}\hspace{0.5em}{Remarque}, opacitybacktitle=0.6, breakable, leftrule=.75mm, toptitle=1mm, colbacktitle=quarto-callout-note-color!10!white, arc=.35mm, opacityback=0, coltitle=black, bottomtitle=1mm, titlerule=0mm, toprule=.15mm]

Dans un ABR, pour un nœud donné, \textbf{tous} les nœuds de son
sous-arbre gauche ont des clefs inférieures et \textbf{tous} les nœuds
de son sous-arbre droit ont des clefs supérieures.

\end{tcolorbox}



\end{document}
