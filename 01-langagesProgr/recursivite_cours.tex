% Options for packages loaded elsewhere
\PassOptionsToPackage{unicode}{hyperref}
\PassOptionsToPackage{hyphens}{url}
\PassOptionsToPackage{dvipsnames,svgnames,x11names}{xcolor}
%
\documentclass[
  letterpaper,
  DIV=11,
  numbers=noendperiod]{scrartcl}

\usepackage{amsmath,amssymb}
\usepackage{iftex}
\ifPDFTeX
  \usepackage[T1]{fontenc}
  \usepackage[utf8]{inputenc}
  \usepackage{textcomp} % provide euro and other symbols
\else % if luatex or xetex
  \usepackage{unicode-math}
  \defaultfontfeatures{Scale=MatchLowercase}
  \defaultfontfeatures[\rmfamily]{Ligatures=TeX,Scale=1}
\fi
\usepackage{lmodern}
\ifPDFTeX\else  
    % xetex/luatex font selection
\fi
% Use upquote if available, for straight quotes in verbatim environments
\IfFileExists{upquote.sty}{\usepackage{upquote}}{}
\IfFileExists{microtype.sty}{% use microtype if available
  \usepackage[]{microtype}
  \UseMicrotypeSet[protrusion]{basicmath} % disable protrusion for tt fonts
}{}
\makeatletter
\@ifundefined{KOMAClassName}{% if non-KOMA class
  \IfFileExists{parskip.sty}{%
    \usepackage{parskip}
  }{% else
    \setlength{\parindent}{0pt}
    \setlength{\parskip}{6pt plus 2pt minus 1pt}}
}{% if KOMA class
  \KOMAoptions{parskip=half}}
\makeatother
\usepackage{xcolor}
\usepackage[top=20mm,bottom=20mm,left=20mm,right=20mm,heightrounded]{geometry}
\setlength{\emergencystretch}{3em} % prevent overfull lines
\setcounter{secnumdepth}{-\maxdimen} % remove section numbering
% Make \paragraph and \subparagraph free-standing
\ifx\paragraph\undefined\else
  \let\oldparagraph\paragraph
  \renewcommand{\paragraph}[1]{\oldparagraph{#1}\mbox{}}
\fi
\ifx\subparagraph\undefined\else
  \let\oldsubparagraph\subparagraph
  \renewcommand{\subparagraph}[1]{\oldsubparagraph{#1}\mbox{}}
\fi

\usepackage{color}
\usepackage{fancyvrb}
\newcommand{\VerbBar}{|}
\newcommand{\VERB}{\Verb[commandchars=\\\{\}]}
\DefineVerbatimEnvironment{Highlighting}{Verbatim}{commandchars=\\\{\}}
% Add ',fontsize=\small' for more characters per line
\usepackage{framed}
\definecolor{shadecolor}{RGB}{241,243,245}
\newenvironment{Shaded}{\begin{snugshade}}{\end{snugshade}}
\newcommand{\AlertTok}[1]{\textcolor[rgb]{0.68,0.00,0.00}{#1}}
\newcommand{\AnnotationTok}[1]{\textcolor[rgb]{0.37,0.37,0.37}{#1}}
\newcommand{\AttributeTok}[1]{\textcolor[rgb]{0.40,0.45,0.13}{#1}}
\newcommand{\BaseNTok}[1]{\textcolor[rgb]{0.68,0.00,0.00}{#1}}
\newcommand{\BuiltInTok}[1]{\textcolor[rgb]{0.00,0.23,0.31}{#1}}
\newcommand{\CharTok}[1]{\textcolor[rgb]{0.13,0.47,0.30}{#1}}
\newcommand{\CommentTok}[1]{\textcolor[rgb]{0.37,0.37,0.37}{#1}}
\newcommand{\CommentVarTok}[1]{\textcolor[rgb]{0.37,0.37,0.37}{\textit{#1}}}
\newcommand{\ConstantTok}[1]{\textcolor[rgb]{0.56,0.35,0.01}{#1}}
\newcommand{\ControlFlowTok}[1]{\textcolor[rgb]{0.00,0.23,0.31}{#1}}
\newcommand{\DataTypeTok}[1]{\textcolor[rgb]{0.68,0.00,0.00}{#1}}
\newcommand{\DecValTok}[1]{\textcolor[rgb]{0.68,0.00,0.00}{#1}}
\newcommand{\DocumentationTok}[1]{\textcolor[rgb]{0.37,0.37,0.37}{\textit{#1}}}
\newcommand{\ErrorTok}[1]{\textcolor[rgb]{0.68,0.00,0.00}{#1}}
\newcommand{\ExtensionTok}[1]{\textcolor[rgb]{0.00,0.23,0.31}{#1}}
\newcommand{\FloatTok}[1]{\textcolor[rgb]{0.68,0.00,0.00}{#1}}
\newcommand{\FunctionTok}[1]{\textcolor[rgb]{0.28,0.35,0.67}{#1}}
\newcommand{\ImportTok}[1]{\textcolor[rgb]{0.00,0.46,0.62}{#1}}
\newcommand{\InformationTok}[1]{\textcolor[rgb]{0.37,0.37,0.37}{#1}}
\newcommand{\KeywordTok}[1]{\textcolor[rgb]{0.00,0.23,0.31}{#1}}
\newcommand{\NormalTok}[1]{\textcolor[rgb]{0.00,0.23,0.31}{#1}}
\newcommand{\OperatorTok}[1]{\textcolor[rgb]{0.37,0.37,0.37}{#1}}
\newcommand{\OtherTok}[1]{\textcolor[rgb]{0.00,0.23,0.31}{#1}}
\newcommand{\PreprocessorTok}[1]{\textcolor[rgb]{0.68,0.00,0.00}{#1}}
\newcommand{\RegionMarkerTok}[1]{\textcolor[rgb]{0.00,0.23,0.31}{#1}}
\newcommand{\SpecialCharTok}[1]{\textcolor[rgb]{0.37,0.37,0.37}{#1}}
\newcommand{\SpecialStringTok}[1]{\textcolor[rgb]{0.13,0.47,0.30}{#1}}
\newcommand{\StringTok}[1]{\textcolor[rgb]{0.13,0.47,0.30}{#1}}
\newcommand{\VariableTok}[1]{\textcolor[rgb]{0.07,0.07,0.07}{#1}}
\newcommand{\VerbatimStringTok}[1]{\textcolor[rgb]{0.13,0.47,0.30}{#1}}
\newcommand{\WarningTok}[1]{\textcolor[rgb]{0.37,0.37,0.37}{\textit{#1}}}

\providecommand{\tightlist}{%
  \setlength{\itemsep}{0pt}\setlength{\parskip}{0pt}}\usepackage{longtable,booktabs,array}
\usepackage{calc} % for calculating minipage widths
% Correct order of tables after \paragraph or \subparagraph
\usepackage{etoolbox}
\makeatletter
\patchcmd\longtable{\par}{\if@noskipsec\mbox{}\fi\par}{}{}
\makeatother
% Allow footnotes in longtable head/foot
\IfFileExists{footnotehyper.sty}{\usepackage{footnotehyper}}{\usepackage{footnote}}
\makesavenoteenv{longtable}
\usepackage{graphicx}
\makeatletter
\def\maxwidth{\ifdim\Gin@nat@width>\linewidth\linewidth\else\Gin@nat@width\fi}
\def\maxheight{\ifdim\Gin@nat@height>\textheight\textheight\else\Gin@nat@height\fi}
\makeatother
% Scale images if necessary, so that they will not overflow the page
% margins by default, and it is still possible to overwrite the defaults
% using explicit options in \includegraphics[width, height, ...]{}
\setkeys{Gin}{width=\maxwidth,height=\maxheight,keepaspectratio}
% Set default figure placement to htbp
\makeatletter
\def\fps@figure{htbp}
\makeatother

\usepackage{fancyhdr} \pagestyle{fancy} \usepackage{lastpage}
\KOMAoption{captions}{tablesignature}
\makeatletter
\@ifpackageloaded{tcolorbox}{}{\usepackage[skins,breakable]{tcolorbox}}
\@ifpackageloaded{fontawesome5}{}{\usepackage{fontawesome5}}
\definecolor{quarto-callout-color}{HTML}{909090}
\definecolor{quarto-callout-note-color}{HTML}{0758E5}
\definecolor{quarto-callout-important-color}{HTML}{CC1914}
\definecolor{quarto-callout-warning-color}{HTML}{EB9113}
\definecolor{quarto-callout-tip-color}{HTML}{00A047}
\definecolor{quarto-callout-caution-color}{HTML}{FC5300}
\definecolor{quarto-callout-color-frame}{HTML}{acacac}
\definecolor{quarto-callout-note-color-frame}{HTML}{4582ec}
\definecolor{quarto-callout-important-color-frame}{HTML}{d9534f}
\definecolor{quarto-callout-warning-color-frame}{HTML}{f0ad4e}
\definecolor{quarto-callout-tip-color-frame}{HTML}{02b875}
\definecolor{quarto-callout-caution-color-frame}{HTML}{fd7e14}
\makeatother
\makeatletter
\makeatother
\makeatletter
\makeatother
\makeatletter
\@ifpackageloaded{caption}{}{\usepackage{caption}}
\AtBeginDocument{%
\ifdefined\contentsname
  \renewcommand*\contentsname{Table des matières}
\else
  \newcommand\contentsname{Table des matières}
\fi
\ifdefined\listfigurename
  \renewcommand*\listfigurename{Liste des Figures}
\else
  \newcommand\listfigurename{Liste des Figures}
\fi
\ifdefined\listtablename
  \renewcommand*\listtablename{Liste des Tables}
\else
  \newcommand\listtablename{Liste des Tables}
\fi
\ifdefined\figurename
  \renewcommand*\figurename{Figure}
\else
  \newcommand\figurename{Figure}
\fi
\ifdefined\tablename
  \renewcommand*\tablename{Tableau}
\else
  \newcommand\tablename{Tableau}
\fi
}
\@ifpackageloaded{float}{}{\usepackage{float}}
\floatstyle{ruled}
\@ifundefined{c@chapter}{\newfloat{codelisting}{h}{lop}}{\newfloat{codelisting}{h}{lop}[chapter]}
\floatname{codelisting}{Listing}
\newcommand*\listoflistings{\listof{codelisting}{Liste des Listings}}
\makeatother
\makeatletter
\@ifpackageloaded{caption}{}{\usepackage{caption}}
\@ifpackageloaded{subcaption}{}{\usepackage{subcaption}}
\makeatother
\makeatletter
\@ifpackageloaded{tcolorbox}{}{\usepackage[skins,breakable]{tcolorbox}}
\makeatother
\makeatletter
\@ifundefined{shadecolor}{\definecolor{shadecolor}{rgb}{.97, .97, .97}}
\makeatother
\makeatletter
\makeatother
\makeatletter
\makeatother
\ifLuaTeX
\usepackage[bidi=basic]{babel}
\else
\usepackage[bidi=default]{babel}
\fi
\babelprovide[main,import]{french}
% get rid of language-specific shorthands (see #6817):
\let\LanguageShortHands\languageshorthands
\def\languageshorthands#1{}
\ifLuaTeX
  \usepackage{selnolig}  % disable illegal ligatures
\fi
\IfFileExists{bookmark.sty}{\usepackage{bookmark}}{\usepackage{hyperref}}
\IfFileExists{xurl.sty}{\usepackage{xurl}}{} % add URL line breaks if available
\urlstyle{same} % disable monospaced font for URLs
\hypersetup{
  pdftitle={Récursivité (Cours)},
  pdflang={fr},
  colorlinks=true,
  linkcolor={blue},
  filecolor={Maroon},
  citecolor={Blue},
  urlcolor={Blue},
  pdfcreator={LaTeX via pandoc}}

\title{Récursivité (Cours)}
\usepackage{etoolbox}
\makeatletter
\providecommand{\subtitle}[1]{% add subtitle to \maketitle
  \apptocmd{\@title}{\par {\large #1 \par}}{}{}
}
\makeatother
\subtitle{S1 - Langages et programmation}
\author{}
\date{}

\begin{document}
\maketitle
\lhead{Spécialité NSI} \rhead{Terminale} \chead{} \cfoot{} \lfoot{Lycée \'Emile Duclaux} \rfoot{Page \thepage/\pageref{LastPage}} \renewcommand{\headrulewidth}{0pt} \renewcommand{\footrulewidth}{0pt} \thispagestyle{fancy} \vspace{-2cm}

\ifdefined\Shaded\renewenvironment{Shaded}{\begin{tcolorbox}[interior hidden, frame hidden, breakable, enhanced, sharp corners, boxrule=0pt, borderline west={3pt}{0pt}{shadecolor}]}{\end{tcolorbox}}\fi

\hypertarget{motivation-et-introduction-du-concept}{%
\subsection{1. Motivation et introduction du
concept}\label{motivation-et-introduction-du-concept}}

Un algorithme est dit récursif s'il \textbf{s'appelle lui-même}
directement ou indirectement via l'appel d'une ou de plusieurs autres
fonctions qui elles-mêmes finissent par l'appeler.

La récursivité est un concept fondamental en informatique qui met
naturellement en pratique un mode de pensée puissant qui consiste à
pouvoir découper la tâche à réaliser en sous-tâches de mêmes natures
mais plus petites qui finalement sont simples à résoudre.

Prenons par exemple le calcul de la factorielle d'un nombre entier
\(n\). Par définition pour un \(n\) entier strictement positif, \(n!\)
est égale au produit des entiers strictement positifs inférieurs à
\(n\). Par convention on a aussi \(0! = 1\).

Par exemple, on a : \(5!=1\times 2\times 3\times 4\times 5 = 120\).

Donnons le code itératif d'une fonction calculant la factorielle:

\begin{Shaded}
\begin{Highlighting}[]
  \KeywordTok{def}\NormalTok{ fact(n):}
    \CommentTok{"""Renvoie la factorielle de n."""}
\NormalTok{    res }\OperatorTok{=} \DecValTok{1}
    \ControlFlowTok{for}\NormalTok{ i }\KeywordTok{in} \BuiltInTok{range}\NormalTok{(}\DecValTok{1}\NormalTok{,n}\OperatorTok{+}\DecValTok{1}\NormalTok{):}
\NormalTok{      res }\OperatorTok{=}\NormalTok{ res }\OperatorTok{*}\NormalTok{ i  }
    \ControlFlowTok{return}\NormalTok{ res}
\end{Highlighting}
\end{Shaded}

La définition récursive se base sur le fait que \(n! = n\times (n-1)!\)
pour tout \(n>0\).

On obtient le code:

\begin{Shaded}
\begin{Highlighting}[]
  \KeywordTok{def}\NormalTok{ fact(n):}
    \CommentTok{"""Renvoie la factorielle de n (méthode récursive)."""}
    \ControlFlowTok{if}\NormalTok{ n }\OperatorTok{==} \DecValTok{0}\NormalTok{:}
\NormalTok{      res }\OperatorTok{=} \DecValTok{1}
    \ControlFlowTok{else}\NormalTok{:}
\NormalTok{      res }\OperatorTok{=}\NormalTok{ n}\OperatorTok{*}\NormalTok{fact(n}\OperatorTok{{-}}\DecValTok{1}\NormalTok{)}
    \ControlFlowTok{return}\NormalTok{ res}
\end{Highlighting}
\end{Shaded}

Pour commencer à comprendre comment fonctionne cette fonction récursive,
nous pouvons visualiser le calcul de \texttt{fact(4)} grâce à l'outil
Python Tutor ci-dessous.

\hypertarget{muxe9canisme}{%
\subsection{2. Mécanisme}\label{muxe9canisme}}

Considérons la fonction \texttt{foo} ci-dessous :

\begin{Shaded}
\begin{Highlighting}[]
\KeywordTok{def}\NormalTok{ foo(n):}
    \ControlFlowTok{if}\NormalTok{ n }\OperatorTok{==} \DecValTok{0}\NormalTok{:}
        \BuiltInTok{print}\NormalTok{(}\StringTok{"Cas de base : "}\NormalTok{, n)}
    \ControlFlowTok{else}\NormalTok{:}
        \BuiltInTok{print}\NormalTok{(}\StringTok{"Début avec n = "}\NormalTok{ , n)}
\NormalTok{        foo(n}\OperatorTok{{-}}\DecValTok{1}\NormalTok{)}
        \BuiltInTok{print}\NormalTok{(}\StringTok{"Fin avec n = "}\NormalTok{ , n)}


\NormalTok{foo(}\DecValTok{3}\NormalTok{)}
\end{Highlighting}
\end{Shaded}

Ce programme génère la sortie suivante :

\begin{Shaded}
\begin{Highlighting}[]
\NormalTok{Début avec n =  3}
\NormalTok{Début avec n =  2}
\NormalTok{Début avec n =  1}
\NormalTok{Cas de base :  0}
\NormalTok{Fin avec n =  1}
\NormalTok{Fin avec n =  2}
\NormalTok{Fin avec n =  3}
\end{Highlighting}
\end{Shaded}

L'observation de ces résultats permet de comprendre que le système, lors
de l'exécution de ce programme, utilise une \textbf{pile d'exécution}.
Une pile d'exécution permet d'enregistrer des informations sur les
fonctions en cours d'exécution dans un programme. On parle de pile, car
les exécutions successives ``s'empilent'' les unes sur les autres, comme
une pile d'assiettes, ou de crêpes. Si nous nous intéressons à la pile
d'exécution du programme étudié ci-dessus, nous obtenons le schéma
suivant :

\begin{figure}

{\centering \includegraphics{pile_recursive.png}

}

\caption{Pile d'exécution}

\end{figure}

Il est important de bien comprendre que la fonction située au sommet de
la pile d'exécution est en cours d'exécution. Toutes les fonctions
situées ``en dessous'' sont mises en pause jusqu'au moment où elles se
retrouveront au sommet de la pile. Quand une fonction termine son
exécution, elle est automatiquement retirée du sommet de la pile (on dit
que la fonction est dépilée).

La pile d'exécution permet de retenir la prochaine instruction à
exécuter au moment où une fonction sera sortie de son ``état de pause''
(qu'elle se retrouvera au sommet de la pile d'exécution). Elle
enregistre aussi \textbf{le contexte}, c'est-à-dire par exemple ici la
valeur de la variable locale \(n\) associée à chaque appel de la
fonction.

Nous pouvons comprendre que ce sont plusieurs copies (on dira plutôt des
\textbf{instances}) de la fonction \texttt{foo} qui sont présentent dans
la pile, chacune ayant son propre \textbf{espace de noms} : la variable
\(n\) de la fonction située en haut de la pile n'est pas la même que la
variable \(n\) de la fonction située en-dessous.

\begin{tcolorbox}[enhanced jigsaw, rightrule=.15mm, arc=.35mm, breakable, left=2mm, opacityback=0, coltitle=black, bottomrule=.15mm, toprule=.15mm, colframe=quarto-callout-warning-color-frame, leftrule=.75mm, bottomtitle=1mm, opacitybacktitle=0.6, toptitle=1mm, titlerule=0mm, colback=white, colbacktitle=quarto-callout-warning-color!10!white, title=\textcolor{quarto-callout-warning-color}{\faExclamationTriangle}\hspace{0.5em}{Limitation propre à Python}]

Le langage Python limite à 1000 le nombre d'appels récursifs d'une
fonction, autrement dit la hauteur de la pile.

\begin{Shaded}
\begin{Highlighting}[]
\PreprocessorTok{RecursionError}\NormalTok{: maximum recursion depth exceeded }\ControlFlowTok{while}\NormalTok{ calling a Python }\BuiltInTok{object}
\end{Highlighting}
\end{Shaded}

\end{tcolorbox}

\hypertarget{uxe9crire-un-algorithme-ruxe9cursif}{%
\subsection{3. Écrire un algorithme
récursif}\label{uxe9crire-un-algorithme-ruxe9cursif}}

Lors de l'écriture d'un algorithme récursif, trois règles doivent
toujours être vérifiées :

\begin{tcolorbox}[enhanced jigsaw, rightrule=.15mm, arc=.35mm, breakable, left=2mm, opacityback=0, coltitle=black, bottomrule=.15mm, toprule=.15mm, colframe=quarto-callout-important-color-frame, leftrule=.75mm, bottomtitle=1mm, opacitybacktitle=0.6, toptitle=1mm, titlerule=0mm, colback=white, colbacktitle=quarto-callout-important-color!10!white, title=\textcolor{quarto-callout-important-color}{\faExclamation}\hspace{0.5em}{Les trois règles de récursivité}]

\begin{enumerate}
\def\labelenumi{\arabic{enumi}.}
\tightlist
\item
  La fonction s'appelle elle-même !
\item
  La fonction comporte un ``cas de base'' qui correspond à une condition
  d'arrêt.
\item
  L'algorithme conduit vers le cas de base : il n'y a pas une infinité
  d'appels récursifs.
\end{enumerate}

\end{tcolorbox}

La troisième règle est assurée par la \textbf{preuve de terminaison} qui
se fait souvent en identifiant la construction d'une suite strictement
décroissante d'entiers positifs ou nuls.

\hypertarget{fonction-ruxe9cursive-et-fonction-ituxe9rative}{%
\subsection{4. Fonction récursive et fonction
itérative}\label{fonction-ruxe9cursive-et-fonction-ituxe9rative}}

La programmation récursive n'est ni meilleure, ni pire que, la
programmation itérative. Toute fonction récursive peut aussi être
programmée de façon itérative. Cependant, en cas de nombreux appels
récursifs, la mémoire de la machine sera trop fortement sollicitée et
l'exécution ralentie, voire impossible.

Le choix entre une solution récursive ou une solution itérative est donc
guidé par le type de problème à résoudre car certains problèmes
s'écrivent \emph{naturellement} de façon récursive.



\end{document}
