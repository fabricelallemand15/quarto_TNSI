% Options for packages loaded elsewhere
\PassOptionsToPackage{unicode}{hyperref}
\PassOptionsToPackage{hyphens}{url}
\PassOptionsToPackage{dvipsnames,svgnames,x11names}{xcolor}
%
\documentclass[
  letterpaper,
  DIV=11,
  numbers=noendperiod]{scrartcl}

\usepackage{amsmath,amssymb}
\usepackage{iftex}
\ifPDFTeX
  \usepackage[T1]{fontenc}
  \usepackage[utf8]{inputenc}
  \usepackage{textcomp} % provide euro and other symbols
\else % if luatex or xetex
  \usepackage{unicode-math}
  \defaultfontfeatures{Scale=MatchLowercase}
  \defaultfontfeatures[\rmfamily]{Ligatures=TeX,Scale=1}
\fi
\usepackage{lmodern}
\ifPDFTeX\else  
    % xetex/luatex font selection
\fi
% Use upquote if available, for straight quotes in verbatim environments
\IfFileExists{upquote.sty}{\usepackage{upquote}}{}
\IfFileExists{microtype.sty}{% use microtype if available
  \usepackage[]{microtype}
  \UseMicrotypeSet[protrusion]{basicmath} % disable protrusion for tt fonts
}{}
\makeatletter
\@ifundefined{KOMAClassName}{% if non-KOMA class
  \IfFileExists{parskip.sty}{%
    \usepackage{parskip}
  }{% else
    \setlength{\parindent}{0pt}
    \setlength{\parskip}{6pt plus 2pt minus 1pt}}
}{% if KOMA class
  \KOMAoptions{parskip=half}}
\makeatother
\usepackage{xcolor}
\setlength{\emergencystretch}{3em} % prevent overfull lines
\setcounter{secnumdepth}{-\maxdimen} % remove section numbering
% Make \paragraph and \subparagraph free-standing
\ifx\paragraph\undefined\else
  \let\oldparagraph\paragraph
  \renewcommand{\paragraph}[1]{\oldparagraph{#1}\mbox{}}
\fi
\ifx\subparagraph\undefined\else
  \let\oldsubparagraph\subparagraph
  \renewcommand{\subparagraph}[1]{\oldsubparagraph{#1}\mbox{}}
\fi

\usepackage{color}
\usepackage{fancyvrb}
\newcommand{\VerbBar}{|}
\newcommand{\VERB}{\Verb[commandchars=\\\{\}]}
\DefineVerbatimEnvironment{Highlighting}{Verbatim}{commandchars=\\\{\}}
% Add ',fontsize=\small' for more characters per line
\usepackage{framed}
\definecolor{shadecolor}{RGB}{241,243,245}
\newenvironment{Shaded}{\begin{snugshade}}{\end{snugshade}}
\newcommand{\AlertTok}[1]{\textcolor[rgb]{0.68,0.00,0.00}{#1}}
\newcommand{\AnnotationTok}[1]{\textcolor[rgb]{0.37,0.37,0.37}{#1}}
\newcommand{\AttributeTok}[1]{\textcolor[rgb]{0.40,0.45,0.13}{#1}}
\newcommand{\BaseNTok}[1]{\textcolor[rgb]{0.68,0.00,0.00}{#1}}
\newcommand{\BuiltInTok}[1]{\textcolor[rgb]{0.00,0.23,0.31}{#1}}
\newcommand{\CharTok}[1]{\textcolor[rgb]{0.13,0.47,0.30}{#1}}
\newcommand{\CommentTok}[1]{\textcolor[rgb]{0.37,0.37,0.37}{#1}}
\newcommand{\CommentVarTok}[1]{\textcolor[rgb]{0.37,0.37,0.37}{\textit{#1}}}
\newcommand{\ConstantTok}[1]{\textcolor[rgb]{0.56,0.35,0.01}{#1}}
\newcommand{\ControlFlowTok}[1]{\textcolor[rgb]{0.00,0.23,0.31}{#1}}
\newcommand{\DataTypeTok}[1]{\textcolor[rgb]{0.68,0.00,0.00}{#1}}
\newcommand{\DecValTok}[1]{\textcolor[rgb]{0.68,0.00,0.00}{#1}}
\newcommand{\DocumentationTok}[1]{\textcolor[rgb]{0.37,0.37,0.37}{\textit{#1}}}
\newcommand{\ErrorTok}[1]{\textcolor[rgb]{0.68,0.00,0.00}{#1}}
\newcommand{\ExtensionTok}[1]{\textcolor[rgb]{0.00,0.23,0.31}{#1}}
\newcommand{\FloatTok}[1]{\textcolor[rgb]{0.68,0.00,0.00}{#1}}
\newcommand{\FunctionTok}[1]{\textcolor[rgb]{0.28,0.35,0.67}{#1}}
\newcommand{\ImportTok}[1]{\textcolor[rgb]{0.00,0.46,0.62}{#1}}
\newcommand{\InformationTok}[1]{\textcolor[rgb]{0.37,0.37,0.37}{#1}}
\newcommand{\KeywordTok}[1]{\textcolor[rgb]{0.00,0.23,0.31}{#1}}
\newcommand{\NormalTok}[1]{\textcolor[rgb]{0.00,0.23,0.31}{#1}}
\newcommand{\OperatorTok}[1]{\textcolor[rgb]{0.37,0.37,0.37}{#1}}
\newcommand{\OtherTok}[1]{\textcolor[rgb]{0.00,0.23,0.31}{#1}}
\newcommand{\PreprocessorTok}[1]{\textcolor[rgb]{0.68,0.00,0.00}{#1}}
\newcommand{\RegionMarkerTok}[1]{\textcolor[rgb]{0.00,0.23,0.31}{#1}}
\newcommand{\SpecialCharTok}[1]{\textcolor[rgb]{0.37,0.37,0.37}{#1}}
\newcommand{\SpecialStringTok}[1]{\textcolor[rgb]{0.13,0.47,0.30}{#1}}
\newcommand{\StringTok}[1]{\textcolor[rgb]{0.13,0.47,0.30}{#1}}
\newcommand{\VariableTok}[1]{\textcolor[rgb]{0.07,0.07,0.07}{#1}}
\newcommand{\VerbatimStringTok}[1]{\textcolor[rgb]{0.13,0.47,0.30}{#1}}
\newcommand{\WarningTok}[1]{\textcolor[rgb]{0.37,0.37,0.37}{\textit{#1}}}

\providecommand{\tightlist}{%
  \setlength{\itemsep}{0pt}\setlength{\parskip}{0pt}}\usepackage{longtable,booktabs,array}
\usepackage{calc} % for calculating minipage widths
% Correct order of tables after \paragraph or \subparagraph
\usepackage{etoolbox}
\makeatletter
\patchcmd\longtable{\par}{\if@noskipsec\mbox{}\fi\par}{}{}
\makeatother
% Allow footnotes in longtable head/foot
\IfFileExists{footnotehyper.sty}{\usepackage{footnotehyper}}{\usepackage{footnote}}
\makesavenoteenv{longtable}
\usepackage{graphicx}
\makeatletter
\def\maxwidth{\ifdim\Gin@nat@width>\linewidth\linewidth\else\Gin@nat@width\fi}
\def\maxheight{\ifdim\Gin@nat@height>\textheight\textheight\else\Gin@nat@height\fi}
\makeatother
% Scale images if necessary, so that they will not overflow the page
% margins by default, and it is still possible to overwrite the defaults
% using explicit options in \includegraphics[width, height, ...]{}
\setkeys{Gin}{width=\maxwidth,height=\maxheight,keepaspectratio}
% Set default figure placement to htbp
\makeatletter
\def\fps@figure{htbp}
\makeatother

\KOMAoption{captions}{tablesignature}
\makeatletter
\makeatother
\makeatletter
\makeatother
\makeatletter
\@ifpackageloaded{caption}{}{\usepackage{caption}}
\AtBeginDocument{%
\ifdefined\contentsname
  \renewcommand*\contentsname{Table des matières}
\else
  \newcommand\contentsname{Table des matières}
\fi
\ifdefined\listfigurename
  \renewcommand*\listfigurename{Liste des Figures}
\else
  \newcommand\listfigurename{Liste des Figures}
\fi
\ifdefined\listtablename
  \renewcommand*\listtablename{Liste des Tables}
\else
  \newcommand\listtablename{Liste des Tables}
\fi
\ifdefined\figurename
  \renewcommand*\figurename{Figure}
\else
  \newcommand\figurename{Figure}
\fi
\ifdefined\tablename
  \renewcommand*\tablename{Tableau}
\else
  \newcommand\tablename{Tableau}
\fi
}
\@ifpackageloaded{float}{}{\usepackage{float}}
\floatstyle{ruled}
\@ifundefined{c@chapter}{\newfloat{codelisting}{h}{lop}}{\newfloat{codelisting}{h}{lop}[chapter]}
\floatname{codelisting}{Listing}
\newcommand*\listoflistings{\listof{codelisting}{Liste des Listings}}
\makeatother
\makeatletter
\@ifpackageloaded{caption}{}{\usepackage{caption}}
\@ifpackageloaded{subcaption}{}{\usepackage{subcaption}}
\makeatother
\makeatletter
\@ifpackageloaded{tcolorbox}{}{\usepackage[skins,breakable]{tcolorbox}}
\makeatother
\makeatletter
\@ifundefined{shadecolor}{\definecolor{shadecolor}{rgb}{.97, .97, .97}}
\makeatother
\makeatletter
\makeatother
\makeatletter
\makeatother
\ifLuaTeX
\usepackage[bidi=basic]{babel}
\else
\usepackage[bidi=default]{babel}
\fi
\babelprovide[main,import]{french}
% get rid of language-specific shorthands (see #6817):
\let\LanguageShortHands\languageshorthands
\def\languageshorthands#1{}
\ifLuaTeX
  \usepackage{selnolig}  % disable illegal ligatures
\fi
\IfFileExists{bookmark.sty}{\usepackage{bookmark}}{\usepackage{hyperref}}
\IfFileExists{xurl.sty}{\usepackage{xurl}}{} % add URL line breaks if available
\urlstyle{same} % disable monospaced font for URLs
\hypersetup{
  pdftitle={Centres étrangers 2023 Jour 2},
  pdflang={fr},
  colorlinks=true,
  linkcolor={blue},
  filecolor={Maroon},
  citecolor={Blue},
  urlcolor={Blue},
  pdfcreator={LaTeX via pandoc}}

\title{Centres étrangers 2023 Jour 2}
\author{}
\date{}

\begin{document}
\maketitle
\ifdefined\Shaded\renewenvironment{Shaded}{\begin{tcolorbox}[sharp corners, enhanced, breakable, frame hidden, boxrule=0pt, interior hidden, borderline west={3pt}{0pt}{shadecolor}]}{\end{tcolorbox}}\fi

\hypertarget{exercice-1}{%
\subsection{Exercice 1}\label{exercice-1}}

\hypertarget{partie-a-ladressage-ip}{%
\subsubsection{Partie A : L'adressage IP}\label{partie-a-ladressage-ip}}

\begin{enumerate}
\def\labelenumi{\arabic{enumi}.}
\item
  \begin{enumerate}
  \def\labelenumii{\alph{enumii}.}
  \item
    Toute adresse IP du type \texttt{192.168.5.XYZ} avec \texttt{XYZ}
    différente de 000, 255 et 003 est valide pour le routeur F. En
    effet, l'adresse \texttt{192.168.5.0} est celle du réseau lui-même,
    l'adresse \texttt{192.168.5.255} correspond en général à l'adresse
    de diffusion et l'adresse \texttt{192.168.5.3} est déjà utilisée par
    une machine. On peut donc par exemple affecter l'adresse
    \texttt{192.168.5.1} au routeur F.
  \item
    En tenant compte des remarques précédentes, \texttt{XYZ} peut
    prendre les valeurs de 1 à 254. Il y a donc 254 adresses IP valides
    pour le réseau F.
  \end{enumerate}
\item
  \begin{enumerate}
  \def\labelenumii{\alph{enumii}.}
  \item
    Le masque de sous-réseau du réseau B est \texttt{255.255.240.0}.
  \item
    Une des machine du réseau B a pour adresse IP \texttt{192.168.2.2}.
    Pour déterminer le masque de sous-réseau, on convertit ces adresses
    en binaire. On obtient alors
    \texttt{11000000.10101000.00000010.00000010} pour la machine et
    \texttt{11111111.11111111.11110000.00000000} pour le masque. En
    effectuant un ET logique bit à bit, on obtient
    \texttt{11000000.10101000.00000000.00000000} qui correspond à
    l'adresse du réseau B. On peut donc conclure que le masque de
    sous-réseau du réseau B est \texttt{192.168.0.0}.
  \item
    L'interconnexion entre les routeurs A, B, E et F permet, en cas de
    défaillance de l'un d'entre eux, de maintenir la liaison entre
    toutes les machines représentées sur le schéma.
  \end{enumerate}
\end{enumerate}

\hypertarget{partie-b-le-routage}{%
\subsubsection{Partie B : Le routage}\label{partie-b-le-routage}}

\begin{enumerate}
\def\labelenumi{\arabic{enumi}.}
\item
  \begin{enumerate}
  \def\labelenumii{\alph{enumii}.}
  \item
    Il existe un chemin de longueur 2 entre le routeur A et le routeur E
    : A - B - E. Il s'agit du plus cours chemin possible en terme de
    nombre de sauts. Pour aller de F vers B, il existe plusieurs chemins
    optimaux en termes de nombre de sauts. Ce sont tous les chemins de
    longueur 3 : F - D - A - B, F - H - G - B et F - H - E - B.
  \item
    \begin{longtable}[]{@{}ccc@{}}
    \caption{Table de routage du routeur E}\tabularnewline
    \toprule\noalign{}
    Destination & Routeur suivant & Distance \\
    \midrule\noalign{}
    \endfirsthead
    \toprule\noalign{}
    Destination & Routeur suivant & Distance \\
    \midrule\noalign{}
    \endhead
    \bottomrule\noalign{}
    \endlastfoot
    A & B & 2 \\
    B & B & 1 \\
    C & H & 2 \\
    D & G & 2 \\
    E & E & 0 \\
    F & H & 2 \\
    G & G & 1 \\
    H & H & 1 \\
    \end{longtable}

    \begin{longtable}[]{@{}ccc@{}}
    \caption{Table de routage du routeur G}\tabularnewline
    \toprule\noalign{}
    Destination & Routeur suivant & Distance \\
    \midrule\noalign{}
    \endfirsthead
    \toprule\noalign{}
    Destination & Routeur suivant & Distance \\
    \midrule\noalign{}
    \endhead
    \bottomrule\noalign{}
    \endlastfoot
    A & B & 2 \\
    B & B & 1 \\
    C & D & 1 \\
    D & D & 1 \\
    E & E & 1 \\
    F & D & 2 \\
    G & G & 0 \\
    H & H & 1 \\
    \end{longtable}
  \end{enumerate}
\item
  \begin{enumerate}
  \def\labelenumii{\alph{enumii}.}
  \item
    \begin{longtable}[]{@{}ccc@{}}
    \caption{Table de routage du routeur F}\tabularnewline
    \toprule\noalign{}
    Destination & Routeur suivant & Coût total \\
    \midrule\noalign{}
    \endfirsthead
    \toprule\noalign{}
    Destination & Routeur suivant & Coût total \\
    \midrule\noalign{}
    \endhead
    \bottomrule\noalign{}
    \endlastfoot
    A & D & 1.1 \\
    B & H & 10.11 \\
    C & D & 1.1 \\
    D & D & 0.1 \\
    E & H & 10.1 \\
    G & D & 1.1 \\
    H & H & 0.1 \\
    \end{longtable}
  \item
    Entre le routeur E et le routeur D, le chemin optimal est E - H - F
    - D, dont le coût total est de 10.2.
  \end{enumerate}
\end{enumerate}

\hypertarget{exercice-2}{%
\subsection{Exercice 2}\label{exercice-2}}

\begin{enumerate}
\def\labelenumi{\arabic{enumi}.}
\item
  \begin{enumerate}
  \def\labelenumii{\alph{enumii}.}
  \tightlist
  \item
    Le résultat de la requête est le suivant :
  \end{enumerate}

  \begin{longtable}[]{@{}ccc@{}}
  \toprule\noalign{}
  age & taille & poids \\
  \midrule\noalign{}
  \endhead
  \bottomrule\noalign{}
  \endlastfoot
  6 & 1.70 & 100 \\
  \end{longtable}

  \begin{enumerate}
  \def\labelenumii{\alph{enumii}.}
  \setcounter{enumii}{1}
  \tightlist
  \item
    La requête est la suivante :
  \end{enumerate}

\begin{Shaded}
\begin{Highlighting}[]
\KeywordTok{SELECT}\NormalTok{ nom, age }
\KeywordTok{FROM}\NormalTok{ animal}
\KeywordTok{WHERE}\NormalTok{ nom\_espece }\OperatorTok{=} \StringTok{\textquotesingle{}bonobo\textquotesingle{}}
\KeywordTok{ORDER} \KeywordTok{BY}\NormalTok{ age}
\end{Highlighting}
\end{Shaded}
\item
  \begin{enumerate}
  \def\labelenumii{\alph{enumii}.}
  \item
    L'attribut \texttt{nom\_espece} peut vraisemblablement servir de clé
    primaire pour la relation \texttt{espece} car deux espèces
    différentes doivent avoir des noms différents. L'attribut
    \texttt{num\_enclos} est une clé étrangère relative à la clé
    primaire \texttt{num\_enclos} de la relation \texttt{enclos}.
  \item
    Schéma relationnel de la base de données :

    \begin{itemize}
    \tightlist
    \item
      animal({id\_animal} : INT, nom : VARCHAR, age : INT, taille :
      FLOAT, poids : INT, \#nom\_espece : VARCHAR)
    \item
      enclos({num\_enclos} : INT, ecosysteme : VARCHAR, surface : INT,
      struct : VARCHAR, date\_entretien : DATE)
    \item
      espece({nom\_espece} : VARCHAR, classe : VARCHAR, alimentation :
      VARCHAR, \#num\_enclos : INT)
    \end{itemize}
  \end{enumerate}
\item
  \begin{enumerate}
  \def\labelenumii{\alph{enumii}.}
  \tightlist
  \item
    La requête suivante corrige l'erreur signalée :
  \end{enumerate}

\begin{Shaded}
\begin{Highlighting}[]
\KeywordTok{UPDATE}\NormalTok{ espece}
\KeywordTok{SET}\NormalTok{ classe}\OperatorTok{=}\StringTok{\textquotesingle{}mammifères\textquotesingle{}}
\KeywordTok{WHERE}\NormalTok{ nom\_espece}\OperatorTok{=}\StringTok{\textquotesingle{}ornithorynque\textquotesingle{}}
\end{Highlighting}
\end{Shaded}

  \begin{enumerate}
  \def\labelenumii{\alph{enumii}.}
  \setcounter{enumii}{1}
  \tightlist
  \item
    La requête suivante permet d'intégrer le nouveau venu dans la base
    de données :
  \end{enumerate}

\begin{Shaded}
\begin{Highlighting}[]
\KeywordTok{INSERT} \KeywordTok{INTO}\NormalTok{ animal }\KeywordTok{VALUES}\NormalTok{ (}\DecValTok{179}\NormalTok{, }\StringTok{\textquotesingle{}Serge\textquotesingle{}}\NormalTok{, }\DecValTok{0}\NormalTok{, }\FloatTok{0.8}\NormalTok{, }\DecValTok{30}\NormalTok{, }\StringTok{\textquotesingle{}lama\textquotesingle{}}\NormalTok{)}
\end{Highlighting}
\end{Shaded}
\item
  \begin{enumerate}
  \def\labelenumii{\alph{enumii}.}
  \tightlist
  \item
    Requête permettant de recenser le nom et l'espèce de tous les
    animaux carnivores vivant en vivarium dans le zoo :
  \end{enumerate}

\begin{Shaded}
\begin{Highlighting}[]
\KeywordTok{SELECT}\NormalTok{ nom, nom\_espece}
\KeywordTok{FROM}\NormalTok{ animal}
\KeywordTok{JOIN}\NormalTok{ espece }\KeywordTok{ON}\NormalTok{ animal.nom\_espece }\OperatorTok{=}\NormalTok{ espece.nom\_espece}
\KeywordTok{JOIN}\NormalTok{ enclos }\KeywordTok{ON}\NormalTok{ espece.num\_enclos }\OperatorTok{=}\NormalTok{ enclos.num\_enclos}
\KeywordTok{WHERE}\NormalTok{ enclos.struct }\OperatorTok{=} \StringTok{\textquotesingle{}vivarium\textquotesingle{}} \KeywordTok{and}\NormalTok{ espece.alimentation }\OperatorTok{=} \StringTok{\textquotesingle{}carnivore\textquotesingle{}}
\end{Highlighting}
\end{Shaded}

  \begin{enumerate}
  \def\labelenumii{\alph{enumii}.}
  \setcounter{enumii}{1}
  \tightlist
  \item
    Requête permettant de connaître le nombre d'oiseaux dans tout le zoo
    :
  \end{enumerate}

\begin{Shaded}
\begin{Highlighting}[]
\KeywordTok{SELECT} \FunctionTok{COUNT}\NormalTok{(}\OperatorTok{*}\NormalTok{)}
\KeywordTok{FROM}\NormalTok{ animal}
\KeywordTok{JOIN}\NormalTok{ espece }\KeywordTok{ON}\NormalTok{ animal.nom\_espece }\OperatorTok{=}\NormalTok{ espece.nom\_espece}
\KeywordTok{WHERE}\NormalTok{ espece.classe }\OperatorTok{=} \StringTok{\textquotesingle{}oiseau\textquotesingle{}}
\end{Highlighting}
\end{Shaded}
\end{enumerate}

\hypertarget{exercice-3}{%
\subsection{Exercice 3}\label{exercice-3}}

\begin{enumerate}
\def\labelenumi{\arabic{enumi}.}
\item
  \begin{enumerate}
  \def\labelenumii{\alph{enumii}.}
  \item
    La fonction retourne : \texttt{Bonjour\ Alan\ !}.
  \item
    \texttt{x} et \texttt{y} sont deux variable booléennes. \texttt{x}
    est la valeur de vérité de la comparaison entre les caractères
    \texttt{n} et \texttt{j}, elle prend donc la valeur \texttt{False}.
    \texttt{y} est la valeur de vérité de la comparaison entre les
    caractères \texttt{o} et \texttt{o}, elle prend donc la valeur
    \texttt{True}.
  \item
    La fonction suivante prend en paramètre une chaîne
    \texttt{une\_chaine} et une lettre \texttt{une\_lettre} et retourne
    le nombre de fois où la lettre \texttt{une\_lettre} apparaît dans la
    chaîne \texttt{une\_chaine} :
  \end{enumerate}

\begin{Shaded}
\begin{Highlighting}[]
\KeywordTok{def}\NormalTok{ occurrences\_lettre(une\_chaine, une\_lettre):}
    \CommentTok{"""Retourne le nombre d\textquotesingle{}occurrences de la lettre une\_lettre dans la chaîne une\_chaine."""}
\NormalTok{    compteur }\OperatorTok{=} \DecValTok{0}
    \ControlFlowTok{for}\NormalTok{ lettre }\KeywordTok{in}\NormalTok{ une\_chaine:}
        \ControlFlowTok{if}\NormalTok{ lettre }\OperatorTok{==}\NormalTok{ une\_lettre:}
\NormalTok{            compteur }\OperatorTok{+=} \DecValTok{1}
    \ControlFlowTok{return}\NormalTok{ compteur}
\end{Highlighting}
\end{Shaded}
\item
  \begin{enumerate}
  \def\labelenumii{\alph{enumii}.}
  \item
    Pour obtenir un arbre binaire de hauteur minimale, on range les mots
    dans l'ordre alphabétique et on place le mot du milieu à la racine.
    On répète l'opération sur les deux sous-arbres de gauche et de
    droite.

    Liste dans l'ordre alphabétique :
    \texttt{{[}\textquotesingle{}chameau\textquotesingle{},\ \textquotesingle{}gnou\textquotesingle{},\ \textquotesingle{}pingouin\textquotesingle{},\ \textquotesingle{}python\textquotesingle{},\ \textquotesingle{}renard\textquotesingle{}{]}}.

    On obtient l'arbre :

    \includegraphics{2023_CE_J2_fig1.png}
  \item
    Pour obtenir un arbre de hauteur maximale, on peut placer à la
    racine le premier mot de la liste classée dans l'ordre alphabétique,
    puis placer le mot suivant en sous-arbre droit et chaque mot suivant
    en sous-arbre droit du précédent. On obtient un arbre filiforme.

    \includegraphics{2023_CE_J2_fig2.png}
  \end{enumerate}
\item
  \begin{enumerate}
  \def\labelenumii{\alph{enumii}.}
  \item
    \texttt{mystere(abr\_mots\_francais)} retourne 336 531. Cette
    fonction calcule en effet de façon récursive le nombre d'éléments de
    l'arbre binaire donné en paramètre, égal à un (on compte la racine,
    si l'arbre n'est pas vide) plus le nombre d'éléments de l'arbre
    binaire de gauche plus le nombre d'éléments de l'arbre binaire de
    droite.
  \item
    Fonction permettant de calculer la hauteur d'un arbre binaire :
  \end{enumerate}

\begin{Shaded}
\begin{Highlighting}[]
\KeywordTok{def}\NormalTok{ hauteur(un\_abr):}
    \CommentTok{"""Retourne la hauteur de l\textquotesingle{}arbre binaire un\_abr."""}
    \ControlFlowTok{if}\NormalTok{ un\_abr.est\_vide():}
        \ControlFlowTok{return} \DecValTok{0}
    \ControlFlowTok{else}\NormalTok{:}
        \ControlFlowTok{return} \DecValTok{1} \OperatorTok{+} \BuiltInTok{max}\NormalTok{(hauteur(un\_abr.sous\_arbre\_gauche), hauteur(un\_abr.sous\_arbre\_droit))}
\end{Highlighting}
\end{Shaded}
\item
  \begin{enumerate}
  \def\labelenumii{\alph{enumii}.}
  \tightlist
  \item
    Code de la fonction complétée :
  \end{enumerate}

\begin{Shaded}
\begin{Highlighting}[]
\KeywordTok{def}\NormalTok{ chercher\_mots(liste\_mots, longueur, lettre, position):}
\NormalTok{    res }\OperatorTok{=}\NormalTok{ []}
    \ControlFlowTok{for}\NormalTok{ i }\KeywordTok{in} \BuiltInTok{range}\NormalTok{(}\BuiltInTok{len}\NormalTok{(liste\_mots)):}
        \ControlFlowTok{if} \BuiltInTok{len}\NormalTok{(liste\_mots[i]) }\OperatorTok{==}\NormalTok{ longueur }\KeywordTok{and}\NormalTok{ liste\_mots[i][position] }\OperatorTok{==}\NormalTok{ lettre:}
\NormalTok{            res.append(liste\_mots[i])}
    \ControlFlowTok{return}\NormalTok{ res}
\end{Highlighting}
\end{Shaded}

  \begin{enumerate}
  \def\labelenumii{\alph{enumii}.}
  \setcounter{enumii}{1}
  \item
    La commande
    \texttt{chercher\_mots(liste\_mots\_francais,\ 3,\ \textquotesingle{}x\textquotesingle{},\ 2)}
    retourne la liste des mots français de longueur 3 contenant la
    lettre \texttt{x} à la troisième position. La commande
    \texttt{chercher\_mots(chercher\_mots(liste\_mots\_francais,\ 3,\ \textquotesingle{}x\textquotesingle{},\ 2),\ 3,\ \textquotesingle{}a\textquotesingle{},\ 1)}
    retourne, parmi ceux-ci, les mots qui possèdent un `a' à la deuxième
    position, soit à partir de l'exemple donné dans l'énoncé : {[}`fax',
    `max'{]}.
  \item
    Code permettant de trouver les mots de 5 lettres qui se terminent
    par `ter' :
  \end{enumerate}

\begin{Shaded}
\begin{Highlighting}[]
\NormalTok{chercher\_mots(chercher\_mots(chercher\_mots(liste\_mots\_francais, }\DecValTok{5}\NormalTok{, }\StringTok{\textquotesingle{}t\textquotesingle{}}\NormalTok{, }\DecValTok{2}\NormalTok{), }\DecValTok{5}\NormalTok{, }\StringTok{\textquotesingle{}e\textquotesingle{}}\NormalTok{, }\DecValTok{3}\NormalTok{), }\DecValTok{5}\NormalTok{, }\StringTok{\textquotesingle{}r\textquotesingle{}}\NormalTok{, }\DecValTok{4}\NormalTok{)}
\end{Highlighting}
\end{Shaded}
\end{enumerate}



\end{document}
