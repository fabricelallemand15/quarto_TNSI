% Options for packages loaded elsewhere
\PassOptionsToPackage{unicode}{hyperref}
\PassOptionsToPackage{hyphens}{url}
\PassOptionsToPackage{dvipsnames,svgnames,x11names}{xcolor}
%
\documentclass[
  letterpaper,
  DIV=11,
  numbers=noendperiod]{scrartcl}

\usepackage{amsmath,amssymb}
\usepackage{iftex}
\ifPDFTeX
  \usepackage[T1]{fontenc}
  \usepackage[utf8]{inputenc}
  \usepackage{textcomp} % provide euro and other symbols
\else % if luatex or xetex
  \usepackage{unicode-math}
  \defaultfontfeatures{Scale=MatchLowercase}
  \defaultfontfeatures[\rmfamily]{Ligatures=TeX,Scale=1}
\fi
\usepackage{lmodern}
\ifPDFTeX\else  
    % xetex/luatex font selection
\fi
% Use upquote if available, for straight quotes in verbatim environments
\IfFileExists{upquote.sty}{\usepackage{upquote}}{}
\IfFileExists{microtype.sty}{% use microtype if available
  \usepackage[]{microtype}
  \UseMicrotypeSet[protrusion]{basicmath} % disable protrusion for tt fonts
}{}
\makeatletter
\@ifundefined{KOMAClassName}{% if non-KOMA class
  \IfFileExists{parskip.sty}{%
    \usepackage{parskip}
  }{% else
    \setlength{\parindent}{0pt}
    \setlength{\parskip}{6pt plus 2pt minus 1pt}}
}{% if KOMA class
  \KOMAoptions{parskip=half}}
\makeatother
\usepackage{xcolor}
\usepackage[top=20mm,bottom=20mm,left=20mm,right=20mm,heightrounded]{geometry}
\setlength{\emergencystretch}{3em} % prevent overfull lines
\setcounter{secnumdepth}{-\maxdimen} % remove section numbering
% Make \paragraph and \subparagraph free-standing
\ifx\paragraph\undefined\else
  \let\oldparagraph\paragraph
  \renewcommand{\paragraph}[1]{\oldparagraph{#1}\mbox{}}
\fi
\ifx\subparagraph\undefined\else
  \let\oldsubparagraph\subparagraph
  \renewcommand{\subparagraph}[1]{\oldsubparagraph{#1}\mbox{}}
\fi

\usepackage{color}
\usepackage{fancyvrb}
\newcommand{\VerbBar}{|}
\newcommand{\VERB}{\Verb[commandchars=\\\{\}]}
\DefineVerbatimEnvironment{Highlighting}{Verbatim}{commandchars=\\\{\}}
% Add ',fontsize=\small' for more characters per line
\usepackage{framed}
\definecolor{shadecolor}{RGB}{241,243,245}
\newenvironment{Shaded}{\begin{snugshade}}{\end{snugshade}}
\newcommand{\AlertTok}[1]{\textcolor[rgb]{0.68,0.00,0.00}{#1}}
\newcommand{\AnnotationTok}[1]{\textcolor[rgb]{0.37,0.37,0.37}{#1}}
\newcommand{\AttributeTok}[1]{\textcolor[rgb]{0.40,0.45,0.13}{#1}}
\newcommand{\BaseNTok}[1]{\textcolor[rgb]{0.68,0.00,0.00}{#1}}
\newcommand{\BuiltInTok}[1]{\textcolor[rgb]{0.00,0.23,0.31}{#1}}
\newcommand{\CharTok}[1]{\textcolor[rgb]{0.13,0.47,0.30}{#1}}
\newcommand{\CommentTok}[1]{\textcolor[rgb]{0.37,0.37,0.37}{#1}}
\newcommand{\CommentVarTok}[1]{\textcolor[rgb]{0.37,0.37,0.37}{\textit{#1}}}
\newcommand{\ConstantTok}[1]{\textcolor[rgb]{0.56,0.35,0.01}{#1}}
\newcommand{\ControlFlowTok}[1]{\textcolor[rgb]{0.00,0.23,0.31}{#1}}
\newcommand{\DataTypeTok}[1]{\textcolor[rgb]{0.68,0.00,0.00}{#1}}
\newcommand{\DecValTok}[1]{\textcolor[rgb]{0.68,0.00,0.00}{#1}}
\newcommand{\DocumentationTok}[1]{\textcolor[rgb]{0.37,0.37,0.37}{\textit{#1}}}
\newcommand{\ErrorTok}[1]{\textcolor[rgb]{0.68,0.00,0.00}{#1}}
\newcommand{\ExtensionTok}[1]{\textcolor[rgb]{0.00,0.23,0.31}{#1}}
\newcommand{\FloatTok}[1]{\textcolor[rgb]{0.68,0.00,0.00}{#1}}
\newcommand{\FunctionTok}[1]{\textcolor[rgb]{0.28,0.35,0.67}{#1}}
\newcommand{\ImportTok}[1]{\textcolor[rgb]{0.00,0.46,0.62}{#1}}
\newcommand{\InformationTok}[1]{\textcolor[rgb]{0.37,0.37,0.37}{#1}}
\newcommand{\KeywordTok}[1]{\textcolor[rgb]{0.00,0.23,0.31}{#1}}
\newcommand{\NormalTok}[1]{\textcolor[rgb]{0.00,0.23,0.31}{#1}}
\newcommand{\OperatorTok}[1]{\textcolor[rgb]{0.37,0.37,0.37}{#1}}
\newcommand{\OtherTok}[1]{\textcolor[rgb]{0.00,0.23,0.31}{#1}}
\newcommand{\PreprocessorTok}[1]{\textcolor[rgb]{0.68,0.00,0.00}{#1}}
\newcommand{\RegionMarkerTok}[1]{\textcolor[rgb]{0.00,0.23,0.31}{#1}}
\newcommand{\SpecialCharTok}[1]{\textcolor[rgb]{0.37,0.37,0.37}{#1}}
\newcommand{\SpecialStringTok}[1]{\textcolor[rgb]{0.13,0.47,0.30}{#1}}
\newcommand{\StringTok}[1]{\textcolor[rgb]{0.13,0.47,0.30}{#1}}
\newcommand{\VariableTok}[1]{\textcolor[rgb]{0.07,0.07,0.07}{#1}}
\newcommand{\VerbatimStringTok}[1]{\textcolor[rgb]{0.13,0.47,0.30}{#1}}
\newcommand{\WarningTok}[1]{\textcolor[rgb]{0.37,0.37,0.37}{\textit{#1}}}

\providecommand{\tightlist}{%
  \setlength{\itemsep}{0pt}\setlength{\parskip}{0pt}}\usepackage{longtable,booktabs,array}
\usepackage{calc} % for calculating minipage widths
% Correct order of tables after \paragraph or \subparagraph
\usepackage{etoolbox}
\makeatletter
\patchcmd\longtable{\par}{\if@noskipsec\mbox{}\fi\par}{}{}
\makeatother
% Allow footnotes in longtable head/foot
\IfFileExists{footnotehyper.sty}{\usepackage{footnotehyper}}{\usepackage{footnote}}
\makesavenoteenv{longtable}
\usepackage{graphicx}
\makeatletter
\def\maxwidth{\ifdim\Gin@nat@width>\linewidth\linewidth\else\Gin@nat@width\fi}
\def\maxheight{\ifdim\Gin@nat@height>\textheight\textheight\else\Gin@nat@height\fi}
\makeatother
% Scale images if necessary, so that they will not overflow the page
% margins by default, and it is still possible to overwrite the defaults
% using explicit options in \includegraphics[width, height, ...]{}
\setkeys{Gin}{width=\maxwidth,height=\maxheight,keepaspectratio}
% Set default figure placement to htbp
\makeatletter
\def\fps@figure{htbp}
\makeatother

\usepackage{fancyhdr} \pagestyle{fancy} \usepackage{lastpage}
\KOMAoption{captions}{tablesignature}
\makeatletter
\@ifpackageloaded{tcolorbox}{}{\usepackage[skins,breakable]{tcolorbox}}
\@ifpackageloaded{fontawesome5}{}{\usepackage{fontawesome5}}
\definecolor{quarto-callout-color}{HTML}{909090}
\definecolor{quarto-callout-note-color}{HTML}{0758E5}
\definecolor{quarto-callout-important-color}{HTML}{CC1914}
\definecolor{quarto-callout-warning-color}{HTML}{EB9113}
\definecolor{quarto-callout-tip-color}{HTML}{00A047}
\definecolor{quarto-callout-caution-color}{HTML}{FC5300}
\definecolor{quarto-callout-color-frame}{HTML}{acacac}
\definecolor{quarto-callout-note-color-frame}{HTML}{4582ec}
\definecolor{quarto-callout-important-color-frame}{HTML}{d9534f}
\definecolor{quarto-callout-warning-color-frame}{HTML}{f0ad4e}
\definecolor{quarto-callout-tip-color-frame}{HTML}{02b875}
\definecolor{quarto-callout-caution-color-frame}{HTML}{fd7e14}
\makeatother
\makeatletter
\makeatother
\makeatletter
\makeatother
\makeatletter
\makeatother
\makeatletter
\@ifpackageloaded{caption}{}{\usepackage{caption}}
\AtBeginDocument{%
\ifdefined\contentsname
  \renewcommand*\contentsname{Table des matières}
\else
  \newcommand\contentsname{Table des matières}
\fi
\ifdefined\listfigurename
  \renewcommand*\listfigurename{Liste des Figures}
\else
  \newcommand\listfigurename{Liste des Figures}
\fi
\ifdefined\listtablename
  \renewcommand*\listtablename{Liste des Tables}
\else
  \newcommand\listtablename{Liste des Tables}
\fi
\ifdefined\figurename
  \renewcommand*\figurename{Figure}
\else
  \newcommand\figurename{Figure}
\fi
\ifdefined\tablename
  \renewcommand*\tablename{Tableau}
\else
  \newcommand\tablename{Tableau}
\fi
}
\@ifpackageloaded{float}{}{\usepackage{float}}
\floatstyle{ruled}
\@ifundefined{c@chapter}{\newfloat{codelisting}{h}{lop}}{\newfloat{codelisting}{h}{lop}[chapter]}
\floatname{codelisting}{Listing}
\newcommand*\listoflistings{\listof{codelisting}{Liste des Listings}}
\makeatother
\makeatletter
\@ifpackageloaded{caption}{}{\usepackage{caption}}
\@ifpackageloaded{subcaption}{}{\usepackage{subcaption}}
\makeatother
\makeatletter
\@ifpackageloaded{tcolorbox}{}{\usepackage[skins,breakable]{tcolorbox}}
\makeatother
\makeatletter
\@ifundefined{shadecolor}{\definecolor{shadecolor}{rgb}{.97, .97, .97}}
\makeatother
\makeatletter
\makeatother
\makeatletter
\makeatother
\ifLuaTeX
\usepackage[bidi=basic]{babel}
\else
\usepackage[bidi=default]{babel}
\fi
\babelprovide[main,import]{french}
% get rid of language-specific shorthands (see #6817):
\let\LanguageShortHands\languageshorthands
\def\languageshorthands#1{}
\ifLuaTeX
  \usepackage{selnolig}  % disable illegal ligatures
\fi
\IfFileExists{bookmark.sty}{\usepackage{bookmark}}{\usepackage{hyperref}}
\IfFileExists{xurl.sty}{\usepackage{xurl}}{} % add URL line breaks if available
\urlstyle{same} % disable monospaced font for URLs
\hypersetup{
  pdftitle={2. Le langage SQL (Cours)},
  pdflang={fr},
  colorlinks=true,
  linkcolor={blue},
  filecolor={Maroon},
  citecolor={Blue},
  urlcolor={Blue},
  pdfcreator={LaTeX via pandoc}}

\title{2. Le langage SQL (Cours)}
\usepackage{etoolbox}
\makeatletter
\providecommand{\subtitle}[1]{% add subtitle to \maketitle
  \apptocmd{\@title}{\par {\large #1 \par}}{}{}
}
\makeatother
\subtitle{S3 - Bases de données}
\author{}
\date{}

\begin{document}
\maketitle
\lhead{Spécialité NSI} \rhead{Terminale} \chead{} \cfoot{} \lfoot{Lycée \'Emile Duclaux} \rfoot{Page \thepage/\pageref{LastPage}} \renewcommand{\headrulewidth}{0pt} \renewcommand{\footrulewidth}{0pt} \thispagestyle{fancy} \vspace{-3cm}

\ifdefined\Shaded\renewenvironment{Shaded}{\begin{tcolorbox}[boxrule=0pt, enhanced, sharp corners, interior hidden, breakable, borderline west={3pt}{0pt}{shadecolor}, frame hidden]}{\end{tcolorbox}}\fi

\href{https://sitelf.fr/notebook/?kernel=sql\&extensions=admonitions\&module=https://sitelf.fr/nsi/assets/BDD/Cantal.db\&from=https://sitelf.fr/nsi/assets/notebooks/cours_SQL.ipynb}{Version
Notebook de ce cours}

Nous avons étudié la structure d'une base de données relationnelle, nous
allons maintenant apprendre à réaliser des \textbf{requêtes},
c'est-à-dire que nous allons apprendre à créer une base des données,
créer des attributs, ajouter des données, modifier des données et enfin,
nous allons surtout apprendre à interroger une base de données afin
d'obtenir des informations.

Pour réaliser toutes ces requêtes, nous allons devoir apprendre un
langage de requêtes : SQL (Structured Query Language). SQL est propre
aux bases de données relationnelles.

Dans ce cours nous allons travailler avec SQLite. SQLite est un système
de gestion de base de données relationnelle très répandu. Noter qu'il
existe d'autres systèmes de gestion de base de données relationnelle
comme MySQL ou PostgreSQL. Dans tous les cas, le langage de requête
utilisé est le SQL (même si parfois on peut noter quelques petites
différences). Ce qui sera vu ici avec SQLite pourra, à quelques petites
modifications près, être utilisé avec, par exemple, MySQL.

Nous allons illustrer chacune des instructions SQL du programme avec la
base de donnée \url{Cantal.db} contenant les deux tables
``etablissements'' et ``communes'' déjà rencontrées dans la partie
précédente, dont voici un extrait (les noms des attributs ont été un peu
modifiés (pas d'espaces) ; la copie d'écran est faite à partir de DB
Browser for SqLite). De plus, nous avons déplacé l'attribut
``Codepostal'' de la table ``etablissement'' vers la table ``communes'',
ce qui est plus cohérent.

\begin{figure}

{\centering \includegraphics[width=0.5\textwidth,height=\textheight]{BDD9.png}

}

\end{figure}

\begin{figure}

{\centering \includegraphics[width=0.5\textwidth,height=\textheight]{BDD10.png}

}

\end{figure}

\textbf{Schéma relationnel} :

\begin{itemize}
\tightlist
\item
  communes({Codecommune}: INT, Nomdelacommune: TEXT, Populationtotale:
  INT, Codepostal: INT)
\item
  etablissements({code}: TEXT, nom: TEXT, statut: TEXT, \#id\_commune:
  INT, latitude: FLOAT, longitude: FLOAT)
\end{itemize}

\begin{figure}

{\centering \includegraphics[width=0.5\textwidth,height=\textheight]{BDD11.png}

}

\end{figure}

Pour se connecter à cette base de donnée, vous pouvez utiliser le
logiciel DB Browser (SqLite) ou bien Edupython (qui propose une version
portable du même logiciel)

\hypertarget{requuxeates-dinterrogation}{%
\subsection{1. Requêtes
d'interrogation}\label{requuxeates-dinterrogation}}

\hypertarget{requuxeates-simples}{%
\subsubsection{Requêtes simples}\label{requuxeates-simples}}

Quand on désire extraire des informations d'une table, on effectue une
\textbf{requête d'interrogation} à l'aide du mot clé \textbf{SELECT}.
Voici un exemple de requête d'interrogation :

\begin{Shaded}
\begin{Highlighting}[]
\KeywordTok{SELECT}\NormalTok{ Nomdelacommune, Populationtotale}
\KeywordTok{FROM}\NormalTok{ communes}
\end{Highlighting}
\end{Shaded}

Cette requête va nous permettre d'obtenir le nom de la commune et sa
population pour toutes les communes présentes dans la table
``communes''.

Voici le résultat de cette requête dans le logiciel DB Browser for
SqLite :

\begin{figure}

{\centering \includegraphics[width=0.5\textwidth,height=\textheight]{SQL1.png}

}

\end{figure}

D'une façon générale, le mot clé \textbf{SELECT} est suivi par les
attributs que l'on désire obtenir. Le mot clé \textbf{FROM} est suivi
par la table concernée.

Noter qu'il est possible d'obtenir tous les attributs sans être obligé
de tous les noter grâce au caractère étoile * :

\begin{Shaded}
\begin{Highlighting}[]
\KeywordTok{SELECT} \OperatorTok{*}
\KeywordTok{FROM}\NormalTok{ communes}
\end{Highlighting}
\end{Shaded}

est équivalent à :

\begin{Shaded}
\begin{Highlighting}[]
\KeywordTok{SELECT}\NormalTok{  Codecommune, Nomdelacommune, Populationtotale, Codepostal}
\KeywordTok{FROM}\NormalTok{ communes}
\end{Highlighting}
\end{Shaded}

\hypertarget{requuxeates-conditionnelles}{%
\subsubsection{Requêtes
conditionnelles}\label{requuxeates-conditionnelles}}

La clause \textbf{WHERE} permet d'imposer une (ou des) condition(s)
permettant de sélectionner uniquement certaines lignes.

La condition doit suivre le mot-clé \textbf{WHERE}.

\begin{Shaded}
\begin{Highlighting}[]
\KeywordTok{SELECT}\NormalTok{ Nomdelacommune}
\KeywordTok{FROM}\NormalTok{ communes}
\KeywordTok{WHERE}\NormalTok{ Populationtotale }\OperatorTok{\textgreater{}} \DecValTok{2500}
\end{Highlighting}
\end{Shaded}

La requête ci-dessus permettra d'afficher le nom des communes dont la
population est strictement supérieure à 2500 habitants.

\begin{figure}

{\centering \includegraphics[width=0.5\textwidth,height=\textheight]{SQL2.png}

}

\end{figure}

Il est possible de combiner les conditions à l'aide d'un OR ou d'un AND
:

\begin{Shaded}
\begin{Highlighting}[]
\KeywordTok{SELECT}\NormalTok{ Nomdelacommune}
\KeywordTok{FROM}\NormalTok{ communes}
\KeywordTok{WHERE}\NormalTok{ Populationtotale }\OperatorTok{\textgreater{}} \DecValTok{2500} \KeywordTok{AND}\NormalTok{ Populationtotale }\OperatorTok{\textless{}} \DecValTok{10000}
\end{Highlighting}
\end{Shaded}

Cette requête permet d'obtenir le nom des communes dont la population
est comprise entre 2500 et 10000 habitants.

\begin{figure}

{\centering \includegraphics[width=0.5\textwidth,height=\textheight]{SQL3.png}

}

\end{figure}

La requête ci-dessous permet d'afficher le nom des communes dont la
population est supérieure à 5000 habitants OU dont le nom contient la
lettre ``Z'' (noter le symbole ``\%'' qui remplace n'importe quelle
séquence de caractères en SQL).

\begin{Shaded}
\begin{Highlighting}[]
\KeywordTok{SELECT}\NormalTok{ Nomdelacommune}
\KeywordTok{FROM}\NormalTok{ communes}
\KeywordTok{WHERE}\NormalTok{ Populationtotale }\OperatorTok{\textgreater{}} \DecValTok{5000} \KeywordTok{OR}\NormalTok{ Nomdelacommune }\KeywordTok{LIKE} \OtherTok{"\%Z\%"}
\end{Highlighting}
\end{Shaded}

\begin{figure}

{\centering \includegraphics[width=0.5\textwidth,height=\textheight]{SQL4.png}

}

\end{figure}

\hypertarget{ordonner-les-ruxe9sultats}{%
\subsubsection{Ordonner les résultats}\label{ordonner-les-ruxe9sultats}}

La clause \textbf{ORDER BY} permet d'ordonner les résultats dans l'ordre
croissant.

\begin{Shaded}
\begin{Highlighting}[]
\KeywordTok{SELECT}\NormalTok{ Nomdelacommune, Populationtotale}
\KeywordTok{FROM}\NormalTok{ communes}
\KeywordTok{WHERE}\NormalTok{ Populationtotale }\OperatorTok{\textgreater{}} \DecValTok{5000} \KeywordTok{ORDER} \KeywordTok{BY}\NormalTok{ Populationtotale}
\end{Highlighting}
\end{Shaded}

Cette requête affiche le nom et la population des communes de plus de
5000 habitants dans l'ordre croissant de leur population.

\begin{figure}

{\centering \includegraphics[width=0.5\textwidth,height=\textheight]{SQL5.png}

}

\end{figure}

Pour ordonner les résultats dans l'ordre décroissant, on ajoute
\textbf{DESC}.

Si la clause \textbf{ORDER BY} porte sur un attribut de type TEXT, on
aura un rangement dans l'ordre alphabétique.

\begin{Shaded}
\begin{Highlighting}[]
\KeywordTok{SELECT}\NormalTok{ Nomdelacommune, Populationtotale}
\KeywordTok{FROM}\NormalTok{ communes}
\KeywordTok{WHERE}\NormalTok{ Populationtotale }\OperatorTok{\textgreater{}} \DecValTok{5000} \KeywordTok{ORDER} \KeywordTok{BY}\NormalTok{ Nomdelacommune }\KeywordTok{DESC}
\end{Highlighting}
\end{Shaded}

Cette requête affiche le nom et la population des communes de plus de
5000 habitants dans l'ordre inverse de l'ordre alphabétique de leur nom.

\begin{figure}

{\centering \includegraphics[width=0.5\textwidth,height=\textheight]{SQL6.png}

}

\end{figure}

\hypertarget{uxe9viter-les-doublons}{%
\subsubsection{Éviter les doublons}\label{uxe9viter-les-doublons}}

Pour éviter les doublons dans les résultats d'une requête, on peut
ajouter la clause DISTINCT juste après SELECT.

Considérons par exemple la relation ``communes'', la requête suivante a
pour objectif d'afficher, dans l'ordre croissant, la liste des codes
postaux des communes du Cantal :

\begin{Shaded}
\begin{Highlighting}[]
\KeywordTok{SELECT}\NormalTok{ codepostal}
\KeywordTok{FROM}\NormalTok{ communes}
\KeywordTok{ORDER} \KeywordTok{BY}\NormalTok{ codepostal}
\end{Highlighting}
\end{Shaded}

\begin{figure}

{\centering \includegraphics[width=1.04167in,height=\textheight]{SQL7.png}

}

\end{figure}

Nous voyons que les codes postaux sont répétés autant de fois qu'il y a
de communes desservies par ce code postal.

La requête suivante, avec la clause DISTINCT, permet de n'afficher
qu'une fois chacun des codes postaux considérés :

\begin{Shaded}
\begin{Highlighting}[]
\KeywordTok{SELECT} \KeywordTok{DISTINCT}\NormalTok{ codepostal}
\KeywordTok{FROM}\NormalTok{ communes}
\KeywordTok{ORDER} \KeywordTok{BY}\NormalTok{ codepostal}
\end{Highlighting}
\end{Shaded}

\begin{figure}

{\centering \includegraphics[width=1.04167in,height=\textheight]{SQL8.png}

}

\end{figure}

\hypertarget{les-jointures}{%
\subsubsection{Les jointures}\label{les-jointures}}

Une requête dans une base de donnée peut nécessiter de regrouper des
données provenant de différentes tables.

\begin{tcolorbox}[enhanced jigsaw, opacityback=0, titlerule=0mm, leftrule=.75mm, arc=.35mm, colframe=quarto-callout-tip-color-frame, toprule=.15mm, coltitle=black, toptitle=1mm, left=2mm, breakable, opacitybacktitle=0.6, rightrule=.15mm, title=\textcolor{quarto-callout-tip-color}{\faLightbulb}\hspace{0.5em}{Définition}, bottomtitle=1mm, colbacktitle=quarto-callout-tip-color!10!white, bottomrule=.15mm, colback=white]

Une requête combinant les données de plusieurs relations (tables) est
appelée une \textbf{jointure}.

\end{tcolorbox}

Poursuivons avec l'exemple de la base de données des établissements
scolaires du Cantal qui comporte deux tables, dont on rappelle
ci-dessous le schéma relationnel :

\begin{itemize}
\tightlist
\item
  communes({Codecommune}: INT, Nomdelacommune: TEXT, Populationtotale:
  INT, Codepostal: INT)
\item
  etablissements({code}: TEXT, nom: TEXT, statut: TEXT, \#id\_commune:
  INT, latitude: FLOAT, longitude: FLOAT)
\end{itemize}

\begin{figure}

{\centering \includegraphics[width=0.5\textwidth,height=\textheight]{BDD11.png}

}

\end{figure}

La requête suivante permet d'obtenir la table des noms d'établissements
suivis du nom de leur commune. Il faut pour cela \textbf{joindre} les
informations de la table ``etablissements'' (pour le nom de
l'établissement) avec celles de la table ``communes'' (pour le nom de la
commune). L'élément qui permet cette \textbf{jointure} est la
\textbf{clef étrangère} ``id\_commune'' de la table ``etablissements''
qui fait référence à la \textbf{clef primaire} ``Codecommune'' de la
table ``communes''.

\begin{Shaded}
\begin{Highlighting}[]
\KeywordTok{SELECT}\NormalTok{ etablissements.nom, communes.Nomdelacommune }
\KeywordTok{FROM}\NormalTok{ etablissements }\KeywordTok{JOIN}\NormalTok{ communes }
\KeywordTok{ON}\NormalTok{ etablissements.id\_commune }\OperatorTok{=}\NormalTok{ communes.Codecommune}
\end{Highlighting}
\end{Shaded}

Résultat :

\begin{figure}

{\centering \includegraphics[width=0.5\textwidth,height=\textheight]{SQL9.png}

}

\end{figure}

\begin{tcolorbox}[enhanced jigsaw, opacityback=0, titlerule=0mm, leftrule=.75mm, arc=.35mm, colframe=quarto-callout-note-color-frame, toprule=.15mm, coltitle=black, toptitle=1mm, left=2mm, breakable, opacitybacktitle=0.6, rightrule=.15mm, title=\textcolor{quarto-callout-note-color}{\faInfo}\hspace{0.5em}{Remarque}, bottomtitle=1mm, colbacktitle=quarto-callout-note-color!10!white, bottomrule=.15mm, colback=white]

Lorsqu'on effectue une jointure, plusieurs tables sont en jeu. Pour
davantage de clarté, il est recommandé de préfixer chaque attribut par
le nom de la table dont il provient. On utilise pour cela un point :
``etablissements.nom'' est l'attribut nommé ``nom'' de la relation
``etablissements''.

\end{tcolorbox}

Il est possible d'ajouter à la suite de la jointure une clause WHERE
afin de ne sélectionner que quelques lignes de la table obtenue :

\begin{Shaded}
\begin{Highlighting}[]
\KeywordTok{SELECT}\NormalTok{ etablissements.nom, communes.Nomdelacommune }
\KeywordTok{FROM}\NormalTok{ etablissements }\KeywordTok{JOIN}\NormalTok{ communes }
\KeywordTok{ON}\NormalTok{ etablissements.id\_commune }\OperatorTok{=}\NormalTok{ communes.Codecommune}
\KeywordTok{WHERE}\NormalTok{ etablissements.nom }\KeywordTok{LIKE} \OtherTok{"Collège\%"}
\end{Highlighting}
\end{Shaded}

Résultat la table de tous les collèges avec leur commune :

\begin{figure}

{\centering \includegraphics[width=0.5\textwidth,height=\textheight]{SQL10.png}

}

\end{figure}

\hypertarget{requuxeates-de-mise-uxe0-jour}{%
\subsection{2.2. Requêtes de mise à
jour}\label{requuxeates-de-mise-uxe0-jour}}

\hypertarget{ajouter-une-entruxe9e}{%
\subsubsection{Ajouter une entrée}\label{ajouter-une-entruxe9e}}

Pour ajouter une entrée, nous utilisons la clause INSERT.

Supposons par exemple qu'un nouvel établissement soit ouvert à Labrousse
(code 85): il s'agit d'un lycée hôtelier public. La requête suivante
permet de créer cette nouvelle entrée. Attention, l'ordre des valeurs
données doit être strictement le même que l'ordre des attributs cités.

\begin{Shaded}
\begin{Highlighting}[]
\KeywordTok{INSERT} \KeywordTok{INTO}\NormalTok{ etablissements}
\NormalTok{(code, nom, statut, id\_commune, latitude, longitude)}
\KeywordTok{VALUES}
\NormalTok{(}\OtherTok{"0159999Z"}\NormalTok{, }\OtherTok{"Lycée hôtelier du Cantal"}\NormalTok{, }\OtherTok{"Public"}\NormalTok{, }\DecValTok{85}\NormalTok{, }\FloatTok{44.8572222}\NormalTok{, }\FloatTok{2.5427778}\NormalTok{)}
\end{Highlighting}
\end{Shaded}

\hypertarget{modifier-une-entruxe9e-existante}{%
\subsubsection{Modifier une entrée
existante}\label{modifier-une-entruxe9e-existante}}

Pour modifier un ou plusieurs attributs d'un p-uplet existant, on
utilise la clause UPDATE.

Supposons par exemple que le nouveau lycée soit en fait un lycée Privé
et que son code soit ``0158888Z'' :

\begin{Shaded}
\begin{Highlighting}[]
\KeywordTok{UPDATE}\NormalTok{ etablissements}
\KeywordTok{SET}\NormalTok{ code}\OperatorTok{=}\OtherTok{"0158888Z"}\NormalTok{, statut}\OperatorTok{=}\OtherTok{"Privé"}
\KeywordTok{WHERE}\NormalTok{ code}\OperatorTok{=}\OtherTok{"0159999Z"}
\end{Highlighting}
\end{Shaded}

La clause WHERE permet de spécifier le ou les p-uplets à modifier.

\hypertarget{supprimer-une-entruxe9e}{%
\subsubsection{Supprimer une entrée}\label{supprimer-une-entruxe9e}}

Pour supprimer un p-uplet, on utilise la clause DELETE.

Finalement, le projet de lycée hôtelier est abandonné :

\begin{Shaded}
\begin{Highlighting}[]
\KeywordTok{DELETE} \KeywordTok{FROM}\NormalTok{ etablissements}
\KeywordTok{WHERE}\NormalTok{ code}\OperatorTok{=}\OtherTok{"0158888Z"}
\end{Highlighting}
\end{Shaded}

\begin{tcolorbox}[enhanced jigsaw, opacityback=0, titlerule=0mm, leftrule=.75mm, arc=.35mm, colframe=quarto-callout-warning-color-frame, toprule=.15mm, coltitle=black, toptitle=1mm, left=2mm, breakable, opacitybacktitle=0.6, rightrule=.15mm, title=\textcolor{quarto-callout-warning-color}{\faExclamationTriangle}\hspace{0.5em}{Attention !}, bottomtitle=1mm, colbacktitle=quarto-callout-warning-color!10!white, bottomrule=.15mm, colback=white]

La requête DELETE, sans clause WHERE, supprimera tous les p-uplets de la
relation !

\end{tcolorbox}

\hypertarget{quelques-compluxe9ments}{%
\subsection{2.3. Quelques compléments}\label{quelques-compluxe9ments}}

Le langage SQL propose aussi des \textbf{fonctions d'agrégation}
permettant de faire quelques calculs à partir des données d'une table.
En voici quelques exemples.

\hypertarget{calculer-une-somme}{%
\subsubsection{Calculer une somme}\label{calculer-une-somme}}

Par exemple, la somme des populations de toutes les communes du Cantal :

\begin{Shaded}
\begin{Highlighting}[]
\KeywordTok{SELECT} \FunctionTok{SUM}\NormalTok{(Populationtotale)}
\KeywordTok{FROM}\NormalTok{ communes}
\end{Highlighting}
\end{Shaded}

Résultat : 149 664.

\hypertarget{calculer-une-moyenne}{%
\subsubsection{Calculer une moyenne}\label{calculer-une-moyenne}}

Par exemple, la population moyenne des communes du Cantal dont le code
postal est 15250 (AVG = average):

\begin{Shaded}
\begin{Highlighting}[]
\KeywordTok{SELECT} \FunctionTok{AVG}\NormalTok{(Populationtotale)}
\KeywordTok{FROM}\NormalTok{ communes}
\KeywordTok{where}\NormalTok{ Codepostal}\OperatorTok{=}\DecValTok{15250}
\end{Highlighting}
\end{Shaded}

Résultat : 1 099, 67

\hypertarget{calculer-un-minimum-ou-un-maximum}{%
\subsubsection{Calculer un minimum ou un
maximum}\label{calculer-un-minimum-ou-un-maximum}}

Les fonctions MIN et MAX fonctionnent de la même façon.

Quel est le nom et la population de la commune du Cantal la moins
peuplée ?

\begin{Shaded}
\begin{Highlighting}[]
\KeywordTok{SELECT}\NormalTok{ Nomdelacommune, Populationtotale}
\KeywordTok{FROM}\NormalTok{ communes}
\KeywordTok{WHERE}\NormalTok{ Populationtotale }\OperatorTok{=}\NormalTok{ (}\KeywordTok{SELECT} \FunctionTok{MIN}\NormalTok{(Populationtotale) }\KeywordTok{FROM}\NormalTok{ communes)}
\end{Highlighting}
\end{Shaded}

Résultat : VALJOUZE, 23 habitants.

\begin{tcolorbox}[enhanced jigsaw, opacityback=0, titlerule=0mm, leftrule=.75mm, arc=.35mm, colframe=quarto-callout-note-color-frame, toprule=.15mm, coltitle=black, toptitle=1mm, left=2mm, breakable, opacitybacktitle=0.6, rightrule=.15mm, title=\textcolor{quarto-callout-note-color}{\faInfo}\hspace{0.5em}{Remarque}, bottomtitle=1mm, colbacktitle=quarto-callout-note-color!10!white, bottomrule=.15mm, colback=white]

Ce dernier exemple est un peu plus compliqué que les précédents : on a
en effet imbriqué deux requêtes l'une dans l'autre : on parle de
requêtes \textbf{composées}. La requête ``SELECT MIN(Populationtotale)
FROM communes'' située entre parenthèses retourne la valeur minimale des
populations de toutes les communes. On demande ensuite le nom de la ou
des communes dont la population est cette valeur minimale. Prenez le
temps de bien comprendre cet exemple.

\end{tcolorbox}

\hypertarget{compter-des-donnuxe9es}{%
\subsubsection{Compter des données}\label{compter-des-donnuxe9es}}

La fonction COUNT permet de compter des données.

Combien y a-t-il d'établissements scolaires dans le Cantal ?

\begin{Shaded}
\begin{Highlighting}[]
\KeywordTok{SELECT} \FunctionTok{COUNT}\NormalTok{(}\OperatorTok{*}\NormalTok{)}
\KeywordTok{FROM}\NormalTok{ etablissements}
\end{Highlighting}
\end{Shaded}

Résultat : 203

Combien de communes possèdent le code postal 15250 ?

\begin{Shaded}
\begin{Highlighting}[]
\KeywordTok{SELECT} \FunctionTok{COUNT}\NormalTok{(}\OperatorTok{*}\NormalTok{)}
\KeywordTok{FROM}\NormalTok{ communes}
\KeywordTok{WHERE}\NormalTok{ Codepostal}\OperatorTok{=}\DecValTok{15250}
\end{Highlighting}
\end{Shaded}

Résultat : 9

Combien de noms d'établissements différents parmi les établissements
scolaires ?

\begin{Shaded}
\begin{Highlighting}[]
\KeywordTok{SELECT} \FunctionTok{COUNT}\NormalTok{(}\KeywordTok{DISTINCT}\NormalTok{ nom)}
\KeywordTok{FROM}\NormalTok{ etablissements}
\end{Highlighting}
\end{Shaded}

Résultat : 88

À quelle question répond la requête suivante ?

\begin{Shaded}
\begin{Highlighting}[]
\KeywordTok{SELECT} \FunctionTok{COUNT}\NormalTok{(}\OperatorTok{*}\NormalTok{)}
\KeywordTok{FROM}\NormalTok{ etablissements }\KeywordTok{JOIN}\NormalTok{ communes }
\KeywordTok{ON}\NormalTok{ etablissements.id\_commune}\OperatorTok{=}\NormalTok{communes.Codecommune}
\KeywordTok{WHERE}\NormalTok{ (etablissements.nom }\KeywordTok{LIKE} \OtherTok{"Collège\%"}\NormalTok{) }\KeywordTok{AND}\NormalTok{ communes.Nomdelacommune}\OperatorTok{=}\OtherTok{"AURILLAC"}
\end{Highlighting}
\end{Shaded}

\begin{tcolorbox}[enhanced jigsaw, opacityback=0, titlerule=0mm, leftrule=.75mm, arc=.35mm, colframe=quarto-callout-tip-color-frame, toprule=.15mm, coltitle=black, toptitle=1mm, left=2mm, breakable, opacitybacktitle=0.6, rightrule=.15mm, title=\textcolor{quarto-callout-tip-color}{\faLightbulb}\hspace{0.5em}{Réponse}, bottomtitle=1mm, colbacktitle=quarto-callout-tip-color!10!white, bottomrule=.15mm, colback=white]

Combien y a-t-il de collèges à Aurillac ? Réponse : 5.

\end{tcolorbox}

\begin{tcolorbox}[enhanced jigsaw, opacityback=0, titlerule=0mm, leftrule=.75mm, arc=.35mm, colframe=quarto-callout-caution-color-frame, toprule=.15mm, coltitle=black, toptitle=1mm, left=2mm, breakable, opacitybacktitle=0.6, rightrule=.15mm, title=\textcolor{quarto-callout-caution-color}{\faFire}\hspace{0.5em}{Pour compléter \ldots{}}, bottomtitle=1mm, colbacktitle=quarto-callout-caution-color!10!white, bottomrule=.15mm, colback=white]

\begin{itemize}
\item
  Excellentes vidéos sur Lumni :

  \begin{itemize}
  \tightlist
  \item
    \href{http://www.lumni.fr/video/qu-est-ce-qu-une-base-de-donnees-relationnelle}{Qu'est-ce
    qu'une base de données relationnelle ?}
  \item
    \href{http://www.lumni.fr/video/interrogation-d-une-base-de-donnees-relationnelle}{Interrogation
    d'une base de données relationnelle}
  \end{itemize}
\end{itemize}

\end{tcolorbox}



\end{document}
