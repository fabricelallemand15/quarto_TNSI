% Options for packages loaded elsewhere
\PassOptionsToPackage{unicode}{hyperref}
\PassOptionsToPackage{hyphens}{url}
\PassOptionsToPackage{dvipsnames,svgnames,x11names}{xcolor}
%
\documentclass[
  letterpaper,
  DIV=11,
  numbers=noendperiod]{scrartcl}

\usepackage{amsmath,amssymb}
\usepackage{iftex}
\ifPDFTeX
  \usepackage[T1]{fontenc}
  \usepackage[utf8]{inputenc}
  \usepackage{textcomp} % provide euro and other symbols
\else % if luatex or xetex
  \usepackage{unicode-math}
  \defaultfontfeatures{Scale=MatchLowercase}
  \defaultfontfeatures[\rmfamily]{Ligatures=TeX,Scale=1}
\fi
\usepackage{lmodern}
\ifPDFTeX\else  
    % xetex/luatex font selection
\fi
% Use upquote if available, for straight quotes in verbatim environments
\IfFileExists{upquote.sty}{\usepackage{upquote}}{}
\IfFileExists{microtype.sty}{% use microtype if available
  \usepackage[]{microtype}
  \UseMicrotypeSet[protrusion]{basicmath} % disable protrusion for tt fonts
}{}
\makeatletter
\@ifundefined{KOMAClassName}{% if non-KOMA class
  \IfFileExists{parskip.sty}{%
    \usepackage{parskip}
  }{% else
    \setlength{\parindent}{0pt}
    \setlength{\parskip}{6pt plus 2pt minus 1pt}}
}{% if KOMA class
  \KOMAoptions{parskip=half}}
\makeatother
\usepackage{xcolor}
\usepackage[top=20mm,bottom=20mm,left=20mm,right=20mm,heightrounded]{geometry}
\setlength{\emergencystretch}{3em} % prevent overfull lines
\setcounter{secnumdepth}{-\maxdimen} % remove section numbering
% Make \paragraph and \subparagraph free-standing
\ifx\paragraph\undefined\else
  \let\oldparagraph\paragraph
  \renewcommand{\paragraph}[1]{\oldparagraph{#1}\mbox{}}
\fi
\ifx\subparagraph\undefined\else
  \let\oldsubparagraph\subparagraph
  \renewcommand{\subparagraph}[1]{\oldsubparagraph{#1}\mbox{}}
\fi

\usepackage{color}
\usepackage{fancyvrb}
\newcommand{\VerbBar}{|}
\newcommand{\VERB}{\Verb[commandchars=\\\{\}]}
\DefineVerbatimEnvironment{Highlighting}{Verbatim}{commandchars=\\\{\}}
% Add ',fontsize=\small' for more characters per line
\usepackage{framed}
\definecolor{shadecolor}{RGB}{241,243,245}
\newenvironment{Shaded}{\begin{snugshade}}{\end{snugshade}}
\newcommand{\AlertTok}[1]{\textcolor[rgb]{0.68,0.00,0.00}{#1}}
\newcommand{\AnnotationTok}[1]{\textcolor[rgb]{0.37,0.37,0.37}{#1}}
\newcommand{\AttributeTok}[1]{\textcolor[rgb]{0.40,0.45,0.13}{#1}}
\newcommand{\BaseNTok}[1]{\textcolor[rgb]{0.68,0.00,0.00}{#1}}
\newcommand{\BuiltInTok}[1]{\textcolor[rgb]{0.00,0.23,0.31}{#1}}
\newcommand{\CharTok}[1]{\textcolor[rgb]{0.13,0.47,0.30}{#1}}
\newcommand{\CommentTok}[1]{\textcolor[rgb]{0.37,0.37,0.37}{#1}}
\newcommand{\CommentVarTok}[1]{\textcolor[rgb]{0.37,0.37,0.37}{\textit{#1}}}
\newcommand{\ConstantTok}[1]{\textcolor[rgb]{0.56,0.35,0.01}{#1}}
\newcommand{\ControlFlowTok}[1]{\textcolor[rgb]{0.00,0.23,0.31}{#1}}
\newcommand{\DataTypeTok}[1]{\textcolor[rgb]{0.68,0.00,0.00}{#1}}
\newcommand{\DecValTok}[1]{\textcolor[rgb]{0.68,0.00,0.00}{#1}}
\newcommand{\DocumentationTok}[1]{\textcolor[rgb]{0.37,0.37,0.37}{\textit{#1}}}
\newcommand{\ErrorTok}[1]{\textcolor[rgb]{0.68,0.00,0.00}{#1}}
\newcommand{\ExtensionTok}[1]{\textcolor[rgb]{0.00,0.23,0.31}{#1}}
\newcommand{\FloatTok}[1]{\textcolor[rgb]{0.68,0.00,0.00}{#1}}
\newcommand{\FunctionTok}[1]{\textcolor[rgb]{0.28,0.35,0.67}{#1}}
\newcommand{\ImportTok}[1]{\textcolor[rgb]{0.00,0.46,0.62}{#1}}
\newcommand{\InformationTok}[1]{\textcolor[rgb]{0.37,0.37,0.37}{#1}}
\newcommand{\KeywordTok}[1]{\textcolor[rgb]{0.00,0.23,0.31}{#1}}
\newcommand{\NormalTok}[1]{\textcolor[rgb]{0.00,0.23,0.31}{#1}}
\newcommand{\OperatorTok}[1]{\textcolor[rgb]{0.37,0.37,0.37}{#1}}
\newcommand{\OtherTok}[1]{\textcolor[rgb]{0.00,0.23,0.31}{#1}}
\newcommand{\PreprocessorTok}[1]{\textcolor[rgb]{0.68,0.00,0.00}{#1}}
\newcommand{\RegionMarkerTok}[1]{\textcolor[rgb]{0.00,0.23,0.31}{#1}}
\newcommand{\SpecialCharTok}[1]{\textcolor[rgb]{0.37,0.37,0.37}{#1}}
\newcommand{\SpecialStringTok}[1]{\textcolor[rgb]{0.13,0.47,0.30}{#1}}
\newcommand{\StringTok}[1]{\textcolor[rgb]{0.13,0.47,0.30}{#1}}
\newcommand{\VariableTok}[1]{\textcolor[rgb]{0.07,0.07,0.07}{#1}}
\newcommand{\VerbatimStringTok}[1]{\textcolor[rgb]{0.13,0.47,0.30}{#1}}
\newcommand{\WarningTok}[1]{\textcolor[rgb]{0.37,0.37,0.37}{\textit{#1}}}

\providecommand{\tightlist}{%
  \setlength{\itemsep}{0pt}\setlength{\parskip}{0pt}}\usepackage{longtable,booktabs,array}
\usepackage{calc} % for calculating minipage widths
% Correct order of tables after \paragraph or \subparagraph
\usepackage{etoolbox}
\makeatletter
\patchcmd\longtable{\par}{\if@noskipsec\mbox{}\fi\par}{}{}
\makeatother
% Allow footnotes in longtable head/foot
\IfFileExists{footnotehyper.sty}{\usepackage{footnotehyper}}{\usepackage{footnote}}
\makesavenoteenv{longtable}
\usepackage{graphicx}
\makeatletter
\def\maxwidth{\ifdim\Gin@nat@width>\linewidth\linewidth\else\Gin@nat@width\fi}
\def\maxheight{\ifdim\Gin@nat@height>\textheight\textheight\else\Gin@nat@height\fi}
\makeatother
% Scale images if necessary, so that they will not overflow the page
% margins by default, and it is still possible to overwrite the defaults
% using explicit options in \includegraphics[width, height, ...]{}
\setkeys{Gin}{width=\maxwidth,height=\maxheight,keepaspectratio}
% Set default figure placement to htbp
\makeatletter
\def\fps@figure{htbp}
\makeatother

\usepackage{fancyhdr} \pagestyle{fancy} \usepackage{lastpage}
\KOMAoption{captions}{tablesignature}
\makeatletter
\@ifpackageloaded{tcolorbox}{}{\usepackage[skins,breakable]{tcolorbox}}
\@ifpackageloaded{fontawesome5}{}{\usepackage{fontawesome5}}
\definecolor{quarto-callout-color}{HTML}{909090}
\definecolor{quarto-callout-note-color}{HTML}{0758E5}
\definecolor{quarto-callout-important-color}{HTML}{CC1914}
\definecolor{quarto-callout-warning-color}{HTML}{EB9113}
\definecolor{quarto-callout-tip-color}{HTML}{00A047}
\definecolor{quarto-callout-caution-color}{HTML}{FC5300}
\definecolor{quarto-callout-color-frame}{HTML}{acacac}
\definecolor{quarto-callout-note-color-frame}{HTML}{4582ec}
\definecolor{quarto-callout-important-color-frame}{HTML}{d9534f}
\definecolor{quarto-callout-warning-color-frame}{HTML}{f0ad4e}
\definecolor{quarto-callout-tip-color-frame}{HTML}{02b875}
\definecolor{quarto-callout-caution-color-frame}{HTML}{fd7e14}
\makeatother
\makeatletter
\makeatother
\makeatletter
\makeatother
\makeatletter
\@ifpackageloaded{caption}{}{\usepackage{caption}}
\AtBeginDocument{%
\ifdefined\contentsname
  \renewcommand*\contentsname{Table des matières}
\else
  \newcommand\contentsname{Table des matières}
\fi
\ifdefined\listfigurename
  \renewcommand*\listfigurename{Liste des Figures}
\else
  \newcommand\listfigurename{Liste des Figures}
\fi
\ifdefined\listtablename
  \renewcommand*\listtablename{Liste des Tables}
\else
  \newcommand\listtablename{Liste des Tables}
\fi
\ifdefined\figurename
  \renewcommand*\figurename{Figure}
\else
  \newcommand\figurename{Figure}
\fi
\ifdefined\tablename
  \renewcommand*\tablename{Tableau}
\else
  \newcommand\tablename{Tableau}
\fi
}
\@ifpackageloaded{float}{}{\usepackage{float}}
\floatstyle{ruled}
\@ifundefined{c@chapter}{\newfloat{codelisting}{h}{lop}}{\newfloat{codelisting}{h}{lop}[chapter]}
\floatname{codelisting}{Listing}
\newcommand*\listoflistings{\listof{codelisting}{Liste des Listings}}
\makeatother
\makeatletter
\@ifpackageloaded{caption}{}{\usepackage{caption}}
\@ifpackageloaded{subcaption}{}{\usepackage{subcaption}}
\makeatother
\makeatletter
\@ifpackageloaded{tcolorbox}{}{\usepackage[skins,breakable]{tcolorbox}}
\makeatother
\makeatletter
\@ifundefined{shadecolor}{\definecolor{shadecolor}{rgb}{.97, .97, .97}}
\makeatother
\makeatletter
\makeatother
\makeatletter
\makeatother
\ifLuaTeX
\usepackage[bidi=basic]{babel}
\else
\usepackage[bidi=default]{babel}
\fi
\babelprovide[main,import]{french}
% get rid of language-specific shorthands (see #6817):
\let\LanguageShortHands\languageshorthands
\def\languageshorthands#1{}
\ifLuaTeX
  \usepackage{selnolig}  % disable illegal ligatures
\fi
\IfFileExists{bookmark.sty}{\usepackage{bookmark}}{\usepackage{hyperref}}
\IfFileExists{xurl.sty}{\usepackage{xurl}}{} % add URL line breaks if available
\urlstyle{same} % disable monospaced font for URLs
\hypersetup{
  pdftitle={Le modèle relationnel (Cours - Partie 1)},
  pdflang={fr},
  colorlinks=true,
  linkcolor={blue},
  filecolor={Maroon},
  citecolor={Blue},
  urlcolor={Blue},
  pdfcreator={LaTeX via pandoc}}

\title{Le modèle relationnel (Cours - Partie 1)}
\usepackage{etoolbox}
\makeatletter
\providecommand{\subtitle}[1]{% add subtitle to \maketitle
  \apptocmd{\@title}{\par {\large #1 \par}}{}{}
}
\makeatother
\subtitle{S3 - Bases de données}
\author{}
\date{}

\begin{document}
\maketitle
\lhead{Spécialité NSI} \rhead{Terminale} \chead{} \cfoot{} \lfoot{Lycée \'Emile Duclaux} \rfoot{Page \thepage/\pageref{LastPage}} \renewcommand{\headrulewidth}{0pt} \renewcommand{\footrulewidth}{0pt} \thispagestyle{fancy} \vspace{-3cm}

\ifdefined\Shaded\renewenvironment{Shaded}{\begin{tcolorbox}[enhanced, sharp corners, frame hidden, interior hidden, boxrule=0pt, breakable, borderline west={3pt}{0pt}{shadecolor}]}{\end{tcolorbox}}\fi

\hypertarget{introduction}{%
\subsection{Introduction}\label{introduction}}

Une \textbf{donnée} est valeur numérisée décrivant de manière
élémentaire un fait, une mesure, une réalité

\emph{Exemple} : le nom de l'auteur, l'âge du capitaine, le titre du
livre \ldots{}

Les données décrivent des entités du monde réel, elles-mêmes
\textbf{associées} les unes aux autres.

\emph{Exemple} : Nicolas Bouvier est un écrivain suisse auteur de récit
de voyage culte ``l'usage du monde'' paru en 1963 : deux entités, liées
par la notion d'auteur.

Une base de données est un ensemble (potentiellement volumineux, mais
pas forcément) de telles informations conformes à une structure
prédéfinie au moment de la conception, avec, de plus, une
caractéristique essentielle : on souhaite les mémoriser de manière
persistante. La persistance désigne la capacité d'une base à exister
indépendamment des applications qui la manipulent, ou du système qui
l'héberge. On peut arrêter toutes les machines un soir et retrouver la
base de données le lendemain. Cela implique qu'une base est toujours
stockée sur un support comme les disques magnétiques qui préservent leur
contenu même en l'absence d'alimentation électrique.

On arrive donc à la définition suivante :

\begin{tcolorbox}[enhanced jigsaw, titlerule=0mm, left=2mm, rightrule=.15mm, colframe=quarto-callout-tip-color-frame, opacityback=0, leftrule=.75mm, bottomtitle=1mm, toptitle=1mm, bottomrule=.15mm, opacitybacktitle=0.6, colbacktitle=quarto-callout-tip-color!10!white, breakable, arc=.35mm, toprule=.15mm, title=\textcolor{quarto-callout-tip-color}{\faLightbulb}\hspace{0.5em}{Définition}, colback=white, coltitle=black]

Une base de données est ensemble d'informations structurées mémorisées
sur un support persistant.

\end{tcolorbox}

Un fichier de base de données a nécessairement une structure qui permet
d'une part de distinguer les données les unes des autres, et d'autre
part de représenter leurs liens.

Prenons l'exemple des fichiers CSV, l'une des structures les plus
simples et les plus répandues, sur lesquels nous avons
\href{https://sitelf.fr/nsi/premiere/07_tables/tables_cours/}{travaillé
en première}. Dans un fichier CSV, les données élémentaires sont
représentées par des « \textbf{champs} » délimités par des virgules ou
des points-virgule. Les champs sont associés les uns aux autres par le
simple fait d'être placés dans une même ligne. Les lignes en revanche
sont indépendantes les unes des autres. On peut placer autant de lignes
que l'on veut dans un fichier, et même changer leur ordre sans que cela
modifie en quoi que ce soit l'information représentée.

Voici l'exemple de nos données, représentées en CSV.

\begin{Shaded}
\begin{Highlighting}[]
\CommentTok{"Bouvier"} \OperatorTok{;} \StringTok{"Nicolas"}\OperatorTok{;} \StringTok{"L\textquotesingle{}usage du monde"} \OperatorTok{;} \DecValTok{1963}
\end{Highlighting}
\end{Shaded}

On comprend bien que le premier champ est le nom, le second le prénom,
etc. Il paraît donc cohérent d'ajouter de nouvelles lignes comme:

\begin{Shaded}
\begin{Highlighting}[]
\CommentTok{"Bouvier"}   \OperatorTok{;} \StringTok{"Nicolas"}\OperatorTok{;} \StringTok{"L\textquotesingle{}usage du monde"} \OperatorTok{;} \DecValTok{1963}
\CommentTok{"Stevenson"} \OperatorTok{;} \StringTok{"Robert{-}Louis"}  \OperatorTok{;} \StringTok{"Voyage dans les Cévennes avec un âne"} \OperatorTok{;} \DecValTok{1879}
\end{Highlighting}
\end{Shaded}

On a donné une structure régulière à nos informations, ce qui va
permettre de les interroger et de les manipuler avec précision. On les
stocke dans un fichier sur disque, et nous sommes donc en cours de
constitution d'une véritable base de données. On peut en fait
généraliser ce constat~: une base de données est toujours un ensemble de
fichiers, stockés sur une mémoire externe comme un disque, dont le
contenu obéit à certaines règles de structuration.

Peut-on se satisfaire de cette solution et imaginer que nous pouvons
construire des applications en nous appuyant directement sur des
fichiers structurés, par exemple des fichiers CSV ? C'est la méthode
illustrée par la figure ci-dessous. Dans une telle situation, chaque
utilisateur applique des programmes au fichier, pour en extraire des
données, pour les modifier, pour les créer.

\begin{figure}

{\centering \includegraphics[width=0.5\textwidth,height=\textheight]{BDD1.png}

}

\end{figure}

Cette approche soulève de nombreuses difficultés, parmi lesquelles :

\begin{itemize}
\tightlist
\item
  \emph{Lourdeur d'accès aux données}. En pratique, pour chaque accès,
  même le plus simple, il faudrait écrire un programme adapté à la
  structure du fichier. La production et la maintenance de tels
  programmes seraient extrêmement coûteuses.
\item
  \emph{Risques élevés pour l'intégrité et la sécurité}. Si tout
  programmeur peut accéder directement aux fichiers, il est impossible
  de garantir la sécurité et l'intégrité des données. Quelqu'un peut
  très bien par exemple, en toute bonne foi, faire une fausse manœuvre
  qui rend le fichier illisible.
\item
  \emph{Pas de contrôle de concurrence}. Dans un environnement où
  plusieurs utilisateurs accèdent aux mêmes fichiers, comme sur la Fig.
  1, des problèmes de concurrence d'accès se posent, notamment pour les
  mises à jour. Comment gérer par exemple la situation où deux
  utilisateurs souhaitent en même temps ajouter une ligne au fichier\,?
\item
  \emph{Performances}. Tant qu'un fichier ne contient que quelques
  centaines de lignes, on peut supposer que les performances ne posent
  pas de problème, mais que faire quand on atteint les Gigaoctets (1,000
  Mégaoctets), ou même le Téraoctet (1,000 Gigaoctets)\,? Maintenir des
  performances acceptables suppose la mise en œuvre d'algorithmes ou de
  structures de données demandant des compétences très avancées,
  probablement hors de portée du développeur d'application qui a, de
  toute façon, mieux à faire.
\end{itemize}

Pour surmonter ces problèmes des systèmes complexes capables d'offrir à
la fois un accès simple, sécurisé, performant au contenu d'une base, et
d'accomplir le tour de force de satisfaire de tels accès pour des
dizaines, centaines ou même milliers d'utilisateurs simultanés, le tout
en garantissant l'intégrité de la base même en cas de panne sont mis en
place. De tels systèmes sont appelés \textbf{Systèmes de Gestion de
Bases de Données}, \textbf{SGBD} en bref.

\begin{tcolorbox}[enhanced jigsaw, titlerule=0mm, left=2mm, rightrule=.15mm, colframe=quarto-callout-tip-color-frame, opacityback=0, leftrule=.75mm, bottomtitle=1mm, toptitle=1mm, bottomrule=.15mm, opacitybacktitle=0.6, colbacktitle=quarto-callout-tip-color!10!white, breakable, arc=.35mm, toprule=.15mm, title=\textcolor{quarto-callout-tip-color}{\faLightbulb}\hspace{0.5em}{Définition}, colback=white, coltitle=black]

Un \textbf{Système de Gestion de Bases de Données} (\textbf{SGBD}) est
un système informatique qui assure la gestion de l'ensemble des
informations stockées dans une base de données. Il prend en charge,
notamment, les deux grandes fonctionnalités suivantes~:

\begin{itemize}
\tightlist
\item
  Accès aux fichiers de la base, garantissant leur intégrité, contrôlant
  les opérations concurrentes, optimisant les recherches et mises à
  jour.
\item
  Interactions avec les applications et utilisateurs, grâce à des
  langages d'interrogation et de manipulation à haut niveau
  d'abstraction.
\end{itemize}

\end{tcolorbox}

Avec un SGBD, les applications n'ont plus jamais accès directement aux
fichiers, et ne savent d'ailleurs même pas qu'ils existent, quelle est
leur structure et où ils sont situés. L'architecture classique est celle
illustrée par la figure ci-dessous. Le SGBD apparaît sous la forme d'un
\textbf{serveur}, c'est-à-dire d'un processus informatique prêt à
communiquer avec d'autres (les « \textbf{clients} ») via le réseau. Ce
serveur est hébergé sur une machine (la « machine serveur ») et est le
seul à pouvoir accéder aux fichiers contenant les données, ces fichiers
étant le plus souvent stockés sur le disque de la machine serveur.

\begin{figure}

{\centering \includegraphics[width=0.5\textwidth,height=\textheight]{BDD2.png}

}

\end{figure}

Les applications utilisateurs, maintenant, accèdent à la base via le
programme serveur auquel elles sont connectés. Elles transmettent des
commandes (d'où le nom « d'applications clientes ») que le serveur se
charge d'appliquer. Ces applications bénéficient donc des puissants
algorithmes implantés par le SGBD dans son serveur, comme la capacité à
gérer les accès concurrents, où à satisfaire avec efficacité des
recherches portant sur de très grosses bases.

Cette architecture est à peu près universellement adoptée par tous les
SGBD. Les notions suivantes et le vocabulaire associé, sont donc très
importantes à retenir.

\begin{tcolorbox}[enhanced jigsaw, titlerule=0mm, left=2mm, rightrule=.15mm, colframe=quarto-callout-note-color-frame, opacityback=0, leftrule=.75mm, bottomtitle=1mm, toptitle=1mm, bottomrule=.15mm, opacitybacktitle=0.6, colbacktitle=quarto-callout-note-color!10!white, breakable, arc=.35mm, toprule=.15mm, title=\textcolor{quarto-callout-note-color}{\faInfo}\hspace{0.5em}{Définitions}, colback=white, coltitle=black]

\textbf{Programme serveur}. Un SGBD est instancié sur une machine sous
la forme d'un programme serveur qui gère une ou plusieurs bases de
données, chacune constituée de fichiers stockés sur disque. Le programme
serveur est seul responsable de tous les accès à une base, et de
l'utilisation des ressources (mémoire, disques) qui servent de support à
ces accès.

\textbf{Clients (programmes)}. Les programmes (ou applications) clients
se connectent au programme serveur via le réseau, lui transmettent des
\textbf{requêtes} et reçoivent des données en retour. Ils ne disposent
d'aucune information directe sur la base.

\end{tcolorbox}

Le fait que le serveur de données s'interpose entre les fichiers et les
programmes clients a une conséquence extrêmement importante~: ces
clients, n'ayant pas accès aux fichiers, ne voient les données que sous
la forme que veut bien leur présenter le serveur. Ce dernier peut donc
choisir le mode de représentation qui lui semble le plus approprié :
pour nous, ce sera sous forme de \textbf{tables} et nous parlerons alors
de \textbf{modèle relationnel}.

Une des propriétés les plus importantes des SGBD est donc la distinction
entre plusieurs niveaux d'abstraction pour la représentation des données
: le niveau logique et le niveau physique.

\begin{tcolorbox}[enhanced jigsaw, titlerule=0mm, left=2mm, rightrule=.15mm, colframe=quarto-callout-note-color-frame, opacityback=0, leftrule=.75mm, bottomtitle=1mm, toptitle=1mm, bottomrule=.15mm, opacitybacktitle=0.6, colbacktitle=quarto-callout-note-color!10!white, breakable, arc=.35mm, toprule=.15mm, title=\textcolor{quarto-callout-note-color}{\faInfo}\hspace{0.5em}{Définition}, colback=white, coltitle=black]

\begin{itemize}
\tightlist
\item
  Le \textbf{niveau physique} est celui du codage des données dans des
  fichiers stockés sur disque.
\item
  Le \textbf{niveau logique} est celui de la représentation les données
  dans des structures abstraites, proposées aux applications clientes,
  obtenues par conversion du niveau physique (pour nous ce sont des
  structures en tables)
\end{itemize}

\end{tcolorbox}

La figure ci-dessous illustre les niveaux d'abstraction dans
l'architecture d'un système de gestion de données. Les programmes
clients ne voient que le niveau logique, c'est-à-dire des tables. Le
serveur est chargé du niveau physique, de la conversion des données vers
le niveau logique, et de toute la machinerie qui permet de faire
fonctionner le système~: mémoire, disques, algorithmes et structures de
données. Tout cela est, encore une fois, invisible (et c'est tant mieux)
pour les programmes clients qui peuvent se concentrer sur l'accès à des
données présentées le plus simplement possible.

\begin{figure}

{\centering \includegraphics[width=0.5\textwidth,height=\textheight]{BDD3.png}

}

\end{figure}

Précisions que les niveaux sont en grande partie indépendants, dans le
sens où l'on peut modifier complètement l'organisation du niveau
physique sans avoir besoin de changer qui que ce soit aux applications
qui accèdent à la base. Cette indépendance logique-physique est très
précieuse pour l'administration des bases de données.

Un langage est nécessaire pour interagir avec les données (insérer,
modifier, détruire, déplacer, protéger, etc.). Le langage permet de
construire les commandes transmises au serveur.

Le modèle relationnel s'est construit sur des bases formelles
(mathématiques) rigoureuses, ce qui explique en grande partie sa
robustesse et sa stabilité depuis l'essentiel des travaux qui l'ont
élaboré, dans les années 70-80.

Le \textbf{langage SQL} est utilisé depuis les années 1970 dans tous les
systèmes relationnels.

Le terme SQL désigne plus qu'un langage d'interrogation, même s'il
s'agit de son principal aspect. La norme couvre également les mises à
jour, la définition des tables, les contraintes portant sur les données,
les droits d'accès. SQL est donc le langage à connaître pour interagir
avec un système relationnel.

\begin{figure}

{\centering \includegraphics{BDD4.png}

}

\end{figure}



\end{document}
