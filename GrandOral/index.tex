% Options for packages loaded elsewhere
\PassOptionsToPackage{unicode}{hyperref}
\PassOptionsToPackage{hyphens}{url}
\PassOptionsToPackage{dvipsnames,svgnames,x11names}{xcolor}
%
\documentclass[
  letterpaper,
  DIV=11,
  numbers=noendperiod]{scrartcl}

\usepackage{amsmath,amssymb}
\usepackage{iftex}
\ifPDFTeX
  \usepackage[T1]{fontenc}
  \usepackage[utf8]{inputenc}
  \usepackage{textcomp} % provide euro and other symbols
\else % if luatex or xetex
  \usepackage{unicode-math}
  \defaultfontfeatures{Scale=MatchLowercase}
  \defaultfontfeatures[\rmfamily]{Ligatures=TeX,Scale=1}
\fi
\usepackage{lmodern}
\ifPDFTeX\else  
    % xetex/luatex font selection
\fi
% Use upquote if available, for straight quotes in verbatim environments
\IfFileExists{upquote.sty}{\usepackage{upquote}}{}
\IfFileExists{microtype.sty}{% use microtype if available
  \usepackage[]{microtype}
  \UseMicrotypeSet[protrusion]{basicmath} % disable protrusion for tt fonts
}{}
\makeatletter
\@ifundefined{KOMAClassName}{% if non-KOMA class
  \IfFileExists{parskip.sty}{%
    \usepackage{parskip}
  }{% else
    \setlength{\parindent}{0pt}
    \setlength{\parskip}{6pt plus 2pt minus 1pt}}
}{% if KOMA class
  \KOMAoptions{parskip=half}}
\makeatother
\usepackage{xcolor}
\usepackage[top=20mm,bottom=20mm,left=20mm,right=20mm,heightrounded]{geometry}
\setlength{\emergencystretch}{3em} % prevent overfull lines
\setcounter{secnumdepth}{-\maxdimen} % remove section numbering
% Make \paragraph and \subparagraph free-standing
\ifx\paragraph\undefined\else
  \let\oldparagraph\paragraph
  \renewcommand{\paragraph}[1]{\oldparagraph{#1}\mbox{}}
\fi
\ifx\subparagraph\undefined\else
  \let\oldsubparagraph\subparagraph
  \renewcommand{\subparagraph}[1]{\oldsubparagraph{#1}\mbox{}}
\fi


\providecommand{\tightlist}{%
  \setlength{\itemsep}{0pt}\setlength{\parskip}{0pt}}\usepackage{longtable,booktabs,array}
\usepackage{calc} % for calculating minipage widths
% Correct order of tables after \paragraph or \subparagraph
\usepackage{etoolbox}
\makeatletter
\patchcmd\longtable{\par}{\if@noskipsec\mbox{}\fi\par}{}{}
\makeatother
% Allow footnotes in longtable head/foot
\IfFileExists{footnotehyper.sty}{\usepackage{footnotehyper}}{\usepackage{footnote}}
\makesavenoteenv{longtable}
\usepackage{graphicx}
\makeatletter
\def\maxwidth{\ifdim\Gin@nat@width>\linewidth\linewidth\else\Gin@nat@width\fi}
\def\maxheight{\ifdim\Gin@nat@height>\textheight\textheight\else\Gin@nat@height\fi}
\makeatother
% Scale images if necessary, so that they will not overflow the page
% margins by default, and it is still possible to overwrite the defaults
% using explicit options in \includegraphics[width, height, ...]{}
\setkeys{Gin}{width=\maxwidth,height=\maxheight,keepaspectratio}
% Set default figure placement to htbp
\makeatletter
\def\fps@figure{htbp}
\makeatother

\usepackage{fancyhdr} \pagestyle{fancy} \usepackage{lastpage}
\KOMAoption{captions}{tablesignature}
\makeatletter
\@ifpackageloaded{tcolorbox}{}{\usepackage[skins,breakable]{tcolorbox}}
\@ifpackageloaded{fontawesome5}{}{\usepackage{fontawesome5}}
\definecolor{quarto-callout-color}{HTML}{909090}
\definecolor{quarto-callout-note-color}{HTML}{0758E5}
\definecolor{quarto-callout-important-color}{HTML}{CC1914}
\definecolor{quarto-callout-warning-color}{HTML}{EB9113}
\definecolor{quarto-callout-tip-color}{HTML}{00A047}
\definecolor{quarto-callout-caution-color}{HTML}{FC5300}
\definecolor{quarto-callout-color-frame}{HTML}{acacac}
\definecolor{quarto-callout-note-color-frame}{HTML}{4582ec}
\definecolor{quarto-callout-important-color-frame}{HTML}{d9534f}
\definecolor{quarto-callout-warning-color-frame}{HTML}{f0ad4e}
\definecolor{quarto-callout-tip-color-frame}{HTML}{02b875}
\definecolor{quarto-callout-caution-color-frame}{HTML}{fd7e14}
\makeatother
\makeatletter
\makeatother
\makeatletter
\makeatother
\makeatletter
\@ifpackageloaded{caption}{}{\usepackage{caption}}
\AtBeginDocument{%
\ifdefined\contentsname
  \renewcommand*\contentsname{Table des matières}
\else
  \newcommand\contentsname{Table des matières}
\fi
\ifdefined\listfigurename
  \renewcommand*\listfigurename{Liste des Figures}
\else
  \newcommand\listfigurename{Liste des Figures}
\fi
\ifdefined\listtablename
  \renewcommand*\listtablename{Liste des Tables}
\else
  \newcommand\listtablename{Liste des Tables}
\fi
\ifdefined\figurename
  \renewcommand*\figurename{Figure}
\else
  \newcommand\figurename{Figure}
\fi
\ifdefined\tablename
  \renewcommand*\tablename{Tableau}
\else
  \newcommand\tablename{Tableau}
\fi
}
\@ifpackageloaded{float}{}{\usepackage{float}}
\floatstyle{ruled}
\@ifundefined{c@chapter}{\newfloat{codelisting}{h}{lop}}{\newfloat{codelisting}{h}{lop}[chapter]}
\floatname{codelisting}{Listing}
\newcommand*\listoflistings{\listof{codelisting}{Liste des Listings}}
\makeatother
\makeatletter
\@ifpackageloaded{caption}{}{\usepackage{caption}}
\@ifpackageloaded{subcaption}{}{\usepackage{subcaption}}
\makeatother
\makeatletter
\@ifpackageloaded{tcolorbox}{}{\usepackage[skins,breakable]{tcolorbox}}
\makeatother
\makeatletter
\@ifundefined{shadecolor}{\definecolor{shadecolor}{rgb}{.97, .97, .97}}
\makeatother
\makeatletter
\makeatother
\makeatletter
\makeatother
\ifLuaTeX
\usepackage[bidi=basic]{babel}
\else
\usepackage[bidi=default]{babel}
\fi
\babelprovide[main,import]{french}
% get rid of language-specific shorthands (see #6817):
\let\LanguageShortHands\languageshorthands
\def\languageshorthands#1{}
\ifLuaTeX
  \usepackage{selnolig}  % disable illegal ligatures
\fi
\IfFileExists{bookmark.sty}{\usepackage{bookmark}}{\usepackage{hyperref}}
\IfFileExists{xurl.sty}{\usepackage{xurl}}{} % add URL line breaks if available
\urlstyle{same} % disable monospaced font for URLs
\hypersetup{
  pdftitle={Grand Oral et NSI},
  pdflang={fr},
  colorlinks=true,
  linkcolor={blue},
  filecolor={Maroon},
  citecolor={Blue},
  urlcolor={Blue},
  pdfcreator={LaTeX via pandoc}}

\title{Grand Oral et NSI}
\author{}
\date{}

\begin{document}
\maketitle
\lhead{Spécialité NSI} \rhead{Terminale} \chead{} \cfoot{} \lfoot{Lycée \'Emile Duclaux} \rfoot{Page \thepage/\pageref{LastPage}} \renewcommand{\headrulewidth}{0pt} \renewcommand{\footrulewidth}{0pt} \thispagestyle{fancy} \vspace{-2cm}

\ifdefined\Shaded\renewenvironment{Shaded}{\begin{tcolorbox}[breakable, interior hidden, boxrule=0pt, sharp corners, borderline west={3pt}{0pt}{shadecolor}, enhanced, frame hidden]}{\end{tcolorbox}}\fi

\hypertarget{pruxe9sentation}{%
\subsection{Présentation}\label{pruxe9sentation}}

Le grand oral est une épreuve orale qui se dérouler en juin. Son
coefficient est de 10 (sur un total de 100). Vous devez préparer en
première et surtout en terminale deux questions en rapport avec vos
spécialités de terminale.
\href{https://www.education.gouv.fr/baccalaureat-2021-epreuve-du-grand-oral-permettre-aux-eleves-de-travailler-une-competence-89576}{Page
de présentation officielle du grand oral}
\href{https://www.education.gouv.fr/bo/20/Special2/MENE2002780N.htm}{Article
du bulletin officiel (BO) définissant le grand oral}.

\hypertarget{les-deux-questions}{%
\subsection{Les deux questions}\label{les-deux-questions}}

\hypertarget{ruxe9partition-des-spuxe9cialituxe9s}{%
\subsubsection{Répartition des
spécialités}\label{ruxe9partition-des-spuxe9cialituxe9s}}

Le BO dit : ``ces questions portent sur les deux enseignements de
spécialité soit pris isolément, soit abordés de manière transversale''.
On comprend qu'il est nécessaire d'aborder les deux enseignements de
spécialité, et qu'il est possible de les mélanger. Néanmoins il
semblerait qu'il soit nécessaire (pour des raisons d'organisation, voir
plus bas) d'avoir une spécialité majeure différente dans chaque
question. Voici donc le schéma des répartitions possibles des
spécialités (A et B) dans les deux questions (1 et 2) :

\begin{longtable}[]{@{}cc@{}}
\toprule\noalign{}
Question 1 & Question 2 \\
\midrule\noalign{}
\endhead
\bottomrule\noalign{}
\endlastfoot
A & B \\
A & majeure B et mineure A \\
majeure A et mineure B & B \\
majeure A et mineure B & majeure B et mineure A \\
\end{longtable}

Bien sûr, le caractère majeur ou mineur d'une spécialité dans une
question peut être sujet à discussion. Cela donne néanmoins un cadre de
réflexion.

\hypertarget{sujet-des-questions}{%
\subsubsection{Sujet des questions}\label{sujet-des-questions}}

Encore une fois le BO est assez concis : Elles mettent en lumière un des
grands enjeux du ou des programmes de ces enseignements {[}de
spécialité{]}. Elles sont adossées à tout ou partie du programme du
cycle terminal. Les questions doivent donc avoir un lien avec le
programme de terminale ou éventuellement de première. La question de NSI
(ou de majeure NSI) doit donc s'inscrire dans les thèmes suivants :

Pour la terminale :

\begin{itemize}
\tightlist
\item
  histoire de l'informatique ;
\item
  structures de données ;
\item
  bases de données ;
\item
  architectures matérielles, systèmes d'exploitation et réseaux ;
\item
  langages et programmation
\item
  algorithmique
\end{itemize}

Pour la première :

\begin{itemize}
\tightlist
\item
  histoire de l'informatique ;
\item
  représentation des données: types et valeurs de base ;
\item
  représentation des données: types construits ;
\item
  traitement de données en tables ;
\item
  interactions entre l'homme et la machine sur le Web
\item
  architectures matérielles et systèmes d'exploitation
\item
  langages et programmation
\item
  algorithmique
\end{itemize}

Dans \href{https://eduscol.education.fr/media/3925/download}{ce document
offiicel}, on peut lire :

\begin{quote}
L'entrée choisie par l'élève peut être variée : le choix du champ
disciplinaire dans un parcours d'orientation ; des exemples de notions
mathématiques qui ont changé son regard ou lui ont apporté des clés de
lecture ; des obstacles didactiques auxquels il a été confronté ; une
notion du programme ; un point de l'histoire des sciences ; une
démonstration ; un lien avec une autre spécialité, une attention portée
à une notion pour ses enjeux sociétaux ou dans un parcours d'orientation
comme l'éducation à la santé, au développement durable, aux médias et à
l'information, aux problèmes bioéthiques.
\end{quote}

Si la question rentre dans ces critères, elle peut donc être valide.

\hypertarget{duxe9roulement}{%
\subsubsection{Déroulement}\label{duxe9roulement}}

\hypertarget{choix-de-la-question}{%
\paragraph{Choix de la question}\label{choix-de-la-question}}

Vous serez évalué par un jury composé d'un professeur d'une de vos
spécialités et d'un professeur d'une autre matière (votre autre
spécialité ou non). Le jury choisira donc la question pour laquelle il
est compétent.

\hypertarget{minutes-de-pruxe9paration}{%
\paragraph{20 minutes de préparation}\label{minutes-de-pruxe9paration}}

Une fois que vous connaissez la question choisie par le jury, vous avez
20 minutes pour mettre vos idées au clair. Vous avez la possibilité de
réaliser un support sur ue feuille fournie. Vous pouvez donner ce
support au jury au début de l'entretien.

\hypertarget{minutes-dentretien}{%
\paragraph{20 minutes d'entretien}\label{minutes-dentretien}}

L'entretien se déroule en trois temps. Tout l'entretien se fait sans
notes et sans support excepté, éventuellement, celui que vous avez créé
pendant votre préparation.

\begin{tcolorbox}[enhanced jigsaw, title=\textcolor{quarto-callout-important-color}{\faExclamation}\hspace{0.5em}{Précisions au sujet du support papier}, breakable, bottomrule=.15mm, rightrule=.15mm, leftrule=.75mm, left=2mm, colbacktitle=quarto-callout-important-color!10!white, toprule=.15mm, colback=white, coltitle=black, toptitle=1mm, titlerule=0mm, opacityback=0, opacitybacktitle=0.6, arc=.35mm, bottomtitle=1mm, colframe=quarto-callout-important-color-frame]

Ce support est une aide pour la parole du candidat ; il n'a pas vocation
à être donné à lire au jury. Il s'agit de notes, d'un plan d'exposé, de
trame de prise de parole, de mots-clefs ou d'idées directrices. Ces
notes peuvent aussi servir de document d'appui à l'argumentation
(schéma, courbe, diagramme, tableau, formule mathématique\ldots).

\textbf{Source} :
\href{https://eduscol.education.fr/729/presentation-du-grand-oral}{Eduscol}

\end{tcolorbox}

\hypertarget{minutes-de-pruxe9sentation}{%
\subparagraph{5 minutes de
présentation}\label{minutes-de-pruxe9sentation}}

Vous disposez de 5 minutes pour :

\begin{itemize}
\tightlist
\item
  expliquer pourquoi vous avez choisi cette question ;
\item
  développer le question ;
\item
  y répondre.
\end{itemize}

Le jury ne vous interrompt pas sauf si vous dépassez du temps imparti.
Tout cela se fait sans notes et sans support.

\hypertarget{minutes-duxe9change-avec-le-jury}{%
\subparagraph{10 minutes d'échange avec le
jury}\label{minutes-duxe9change-avec-le-jury}}

Pendant ce temps, le jury vous interroge sur votre question, il demande
des précisions. Il peut élargir les questions au thème abordé puis à
tout le programme de terminale et première.

\hypertarget{minutes-duxe9change-sur-votre-projet-dorientation}{%
\subparagraph{5 minutes d'échange sur votre projet
d'orientation}\label{minutes-duxe9change-sur-votre-projet-dorientation}}

Il est conseillé de faire un lien entre la question traitée et votre
projet d'orientation. Vous devez expliquer les étapes de la maturation
de votre projet d'orientation et détailler votre projet après le bac.

\hypertarget{uxe9valuation}{%
\subsubsection{Évaluation}\label{uxe9valuation}}

Lors de votre présentation, le jury évalue les capacités argumentatives
et les qualités oratoires du candidat. Vous n'êtes pas évalué ici sur le
fond, mais plutôt sur la forme (je rappelle qu'il y aura
vraisemblablement un membre du jury qui n'aura rien compris à ce que
vous avez dit).

Lors de l'échange avec le jury, il évalue la solidité des connaissances
et les capacités argumentatives du candidat. Vous êtes donc évalué ici
sur vos connaissances.

Lors de l'échange sur votre projet d'orientation, le jury mesure la
capacité du candidat à conduire et exprimer une réflexion personnelle
témoignant de sa curiosité intellectuelle et de son aptitude à exprimer
ses motivations. Il n'évalue surtout pas votre projet d'orientation en
lui-même.

\hypertarget{exemples-de-sujets}{%
\subsection{Exemples de sujets}\label{exemples-de-sujets}}

\hypertarget{exemples-issus-du-document-officiel}{%
\subsubsection{\texorpdfstring{Exemples issus du
\href{https://eduscol.education.fr/document/3919/download}{document
officiel}}{Exemples issus du document officiel}}\label{exemples-issus-du-document-officiel}}

\hypertarget{lhistoire-de-linformatique}{%
\paragraph{L'histoire de
l'informatique}\label{lhistoire-de-linformatique}}

\begin{itemize}
\tightlist
\item
  Femmes et numérique : quelle histoire ? quel avenir ?
\item
  Ada Lovelace, pionnière du langage informatique.
\item
  Alan Turing, et l'informatique fut.
\item
  Quelle est la différence entre le web 1.0 et le web 2.0 ?
\end{itemize}

\hypertarget{langages-et-programmation}{%
\paragraph{Langages et programmation}\label{langages-et-programmation}}

\begin{itemize}
\tightlist
\item
  P = NP, un problème à un million de dollars ?
\item
  Tours de Hanoï : plus qu'un jeu d'enfants ?
\item
  Les fractales : informatique et mathématiques imitent-elles la nature
  ?
\item
  De la récurrence à la récursivité.
\item
  Les bugs : bête noire des développeurs ?
\item
  Comment rendre l'informatique plus sûre ?
\end{itemize}

\hypertarget{donnuxe9es-structuruxe9es-et-structures-de-donnuxe9es}{%
\paragraph{Données structurées et structures de
données}\label{donnuxe9es-structuruxe9es-et-structures-de-donnuxe9es}}

\begin{itemize}
\tightlist
\item
  L'informatisation des métros : progrès ou outil de surveillance ?
\item
  Musique et informatique : une alliance possible de l'art et de la
  science ?
\end{itemize}

\hypertarget{algorithmique}{%
\paragraph{Algorithmique}\label{algorithmique}}

\begin{itemize}
\tightlist
\item
  Comment créer une machine intelligente ?
\item
  Comment lutter contre les biais algorithmiques ?
\item
  Quels sont les enjeux de la reconnaissance faciale (notamment
  éthiques) ?
\item
  Quels sont les enjeux de l'intelligence artificielle ?
\item
  Transformation d'images : Deep Fakes, une arme de désinformation
  massive ? La fin de la preuve par l'image ?
\item
  Qu'apporte la récursivité dans un algorithme ?
\item
  Quel est l'impact de la complexité d'un algorithme sur son efficacité
  ?
\end{itemize}

\hypertarget{bases-de-donnuxe9es}{%
\paragraph{Bases de données}\label{bases-de-donnuxe9es}}

\begin{itemize}
\tightlist
\item
  Données personnelles : la vie privée en voie d'extinction ?
\item
  Comment optimiser les données ?
\end{itemize}

\hypertarget{architectures-matuxe9rielles-systuxe8mes-dexploitation-et-ruxe9seaux}{%
\paragraph{Architectures matérielles, systèmes d'exploitation et
réseaux}\label{architectures-matuxe9rielles-systuxe8mes-dexploitation-et-ruxe9seaux}}

\begin{itemize}
\tightlist
\item
  L'ordinateur quantique : nouvelle révolution informatique ?
\item
  La course à l'infiniment petit : jusqu'où ?
\item
  Peut-on vraiment sécuriser les communications ?
\item
  Quelle est l'utilité des protocoles pour l'internet ?
\item
  Cyberguerre : la 3e guerre mondiale ?
\end{itemize}

\hypertarget{interfaces-hommes-machines-ihm}{%
\paragraph{Interfaces Hommes-Machines
(IHM)}\label{interfaces-hommes-machines-ihm}}

\begin{itemize}
\tightlist
\item
  Smart cities, smart control ?
\item
  La réalité virtuelle : un nouveau monde ?
\item
  La voiture autonome, quels enjeux ?
\end{itemize}

\hypertarget{impact-sociuxe9tal-et-uxe9thique-de-linformatique}{%
\paragraph{Impact sociétal et éthique de
l'informatique}\label{impact-sociuxe9tal-et-uxe9thique-de-linformatique}}

\begin{itemize}
\tightlist
\item
  Comment protéger les données numériques sur les réseaux sociaux ?
\item
  Quelle est l'empreinte carbone du numérique en termes de consommation
  ?
\item
  Pourquoi chiffrer ses communications ?
\item
  Les réseaux sociaux sont-ils compatibles avec la politique ?
\item
  Les réseaux sociaux sont-ils compatibles avec le journalisme ?
\item
  Les réseaux sociaux permettent-ils de lutter contre les infox ?
\item
  L'informatique va-t-elle révolutionner le dessin animé ?
\item
  L'informatique va-t-elle révolutionner la composition musicale ?
\item
  L'informatique va-t-elle révolutionner l'art ?
\item
  L'informatique va-t-elle révolutionner le cinéma ?
\item
  L'informatique va-t-elle révolutionner la médecine ?
\item
  L'informatique va-t-elle révolutionner la physique ?
\item
  L'informatique va-t-elle révolutionner l'entreprise ?
\item
  Le numérique : facteur de démocratisation ou de fractures sociales ?
\item
  Informatique : quel impact sur le climat ?
\end{itemize}

\hypertarget{exemples-issus-des-sujets-choisis-les-annuxe9es-pruxe9cuxe9dentes}{%
\subsubsection{Exemples issus des sujets choisis les années
précédentes}\label{exemples-issus-des-sujets-choisis-les-annuxe9es-pruxe9cuxe9dentes}}

\begin{itemize}
\tightlist
\item
  Pourquoi Javascript est-il devenu le langage le plus utilisé au monde
  ?
\item
  Pourquoi chiffrer les communications ?
\item
  L'informatique va-t-elle révolutionner le dessin animé ?
\item
  Comment l'informatique a-t-elle révolutionné la vie des gens dans le
  domaine du divertissement au fil des années ?
\item
  Comment fonctionnent les bases de données distribuées ?
\item
  Nos informations personnelles sont-elles en sécurité dans les bases de
  données ?
\item
  Comment l'informatique s'inspire du vivant ?
\item
  Enigma et décryptage. En quoi l'informatique a-t-elle fait évoluer les
  maths ?
\item
  Comment évolue la cyber-sécurité ?
\item
  Modélisation 3D et graphes.
\item
  L'art algorithmique est-il vraiment un art ?
\item
  Comment protéger les données numériques sur les réseaux
\item
  Quels sont les enjeux des cyberguerres ?
\item
  Fractales, l'info et les maths imitent-elles la nature ?
\item
  Les réseaux sociaux sont-ils compatibles avec la politique ?
\item
  Avenir de la réalité virtuelle ?
\item
  Les réseaux sociaux sont-ils compatibles avec le journalisme ?
\item
  En quoi la cryptologie a-t-elle permis de raccourcir la deuxième
  guerre mondiale ?
\item
  Comment fonctionne Tor ?
\item
  Comment flash est-il passé du statu d'utilisable à mort en 2021 ?
\item
  Pourquoi Python a-t-il été inventé ?
\item
  Comment l'automate cellulaire peut-il imiter les êtres vivants par de
  simples règles informatiques ?
\item
  Comment HTML 5 a révolutionné le développement web ?
\item
  Quelles ont été les évolutions de l'utilisation d'internet ?
\item
  (Maths/Info) Quel est l'impact de la complexité d'un algorithme sur
  son efficacité ?
\item
  Comment optimiser le traitement des données ?
\item
  La machine de Turing est-elle obsolète ?
\item
  (Maths/Info) De la récurrence à la récursivité.
\item
  (Maths/Info) La course à l'infiniment petit, jusqu'où ?
\item
  (Info/Phy?) En quoi l'ordinateur quantique est-il révolutionnaire ?
\item
  Qu'apporte la récursivité dans un algorithme ?
\item
  Peut-on vraiment sécuriser les communications ?
\item
  Quels sont les enjeux de l'intelligence artificielle ?
\item
  (Maths/info) P=NP, un problème à un million de dollars.
\item
  Comment rendre l'informatique plus sûre ?
\item
  (Info/ Maths) En quoi la machine Enigma a t-elle révolutionnée
  l'informatique ?
\item
  (Info/ Maths) Un ordinateur peut-il écrire une démonstration
  mathématique ?
\end{itemize}

\hypertarget{sources}{%
\paragraph{Sources}\label{sources}}

\begin{itemize}
\tightlist
\item
  \url{https://kxs.fr/cours/grand-oral/exemples}
\item
  \url{https://bfourlegnie.com/Tnsi_2020/cours/Gd_oral/grand_oral_NSI.pdf}
\end{itemize}



\end{document}
