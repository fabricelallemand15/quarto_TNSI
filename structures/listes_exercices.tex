% Options for packages loaded elsewhere
\PassOptionsToPackage{unicode}{hyperref}
\PassOptionsToPackage{hyphens}{url}
\PassOptionsToPackage{dvipsnames,svgnames,x11names}{xcolor}
%
\documentclass[
  a4paper,
  DIV=11,
  numbers=noendperiod]{scrartcl}

\usepackage{amsmath,amssymb}
\usepackage{iftex}
\ifPDFTeX
  \usepackage[T1]{fontenc}
  \usepackage[utf8]{inputenc}
  \usepackage{textcomp} % provide euro and other symbols
\else % if luatex or xetex
  \usepackage{unicode-math}
  \defaultfontfeatures{Scale=MatchLowercase}
  \defaultfontfeatures[\rmfamily]{Ligatures=TeX,Scale=1}
\fi
\usepackage{lmodern}
\ifPDFTeX\else  
    % xetex/luatex font selection
\fi
% Use upquote if available, for straight quotes in verbatim environments
\IfFileExists{upquote.sty}{\usepackage{upquote}}{}
\IfFileExists{microtype.sty}{% use microtype if available
  \usepackage[]{microtype}
  \UseMicrotypeSet[protrusion]{basicmath} % disable protrusion for tt fonts
}{}
\makeatletter
\@ifundefined{KOMAClassName}{% if non-KOMA class
  \IfFileExists{parskip.sty}{%
    \usepackage{parskip}
  }{% else
    \setlength{\parindent}{0pt}
    \setlength{\parskip}{6pt plus 2pt minus 1pt}}
}{% if KOMA class
  \KOMAoptions{parskip=half}}
\makeatother
\usepackage{xcolor}
\usepackage[top=20mm,bottom=20mm,left=20mm,right=20mm,heightrounded]{geometry}
\setlength{\emergencystretch}{3em} % prevent overfull lines
\setcounter{secnumdepth}{-\maxdimen} % remove section numbering
% Make \paragraph and \subparagraph free-standing
\ifx\paragraph\undefined\else
  \let\oldparagraph\paragraph
  \renewcommand{\paragraph}[1]{\oldparagraph{#1}\mbox{}}
\fi
\ifx\subparagraph\undefined\else
  \let\oldsubparagraph\subparagraph
  \renewcommand{\subparagraph}[1]{\oldsubparagraph{#1}\mbox{}}
\fi

\usepackage{color}
\usepackage{fancyvrb}
\newcommand{\VerbBar}{|}
\newcommand{\VERB}{\Verb[commandchars=\\\{\}]}
\DefineVerbatimEnvironment{Highlighting}{Verbatim}{commandchars=\\\{\}}
% Add ',fontsize=\small' for more characters per line
\usepackage{framed}
\definecolor{shadecolor}{RGB}{241,243,245}
\newenvironment{Shaded}{\begin{snugshade}}{\end{snugshade}}
\newcommand{\AlertTok}[1]{\textcolor[rgb]{0.68,0.00,0.00}{#1}}
\newcommand{\AnnotationTok}[1]{\textcolor[rgb]{0.37,0.37,0.37}{#1}}
\newcommand{\AttributeTok}[1]{\textcolor[rgb]{0.40,0.45,0.13}{#1}}
\newcommand{\BaseNTok}[1]{\textcolor[rgb]{0.68,0.00,0.00}{#1}}
\newcommand{\BuiltInTok}[1]{\textcolor[rgb]{0.00,0.23,0.31}{#1}}
\newcommand{\CharTok}[1]{\textcolor[rgb]{0.13,0.47,0.30}{#1}}
\newcommand{\CommentTok}[1]{\textcolor[rgb]{0.37,0.37,0.37}{#1}}
\newcommand{\CommentVarTok}[1]{\textcolor[rgb]{0.37,0.37,0.37}{\textit{#1}}}
\newcommand{\ConstantTok}[1]{\textcolor[rgb]{0.56,0.35,0.01}{#1}}
\newcommand{\ControlFlowTok}[1]{\textcolor[rgb]{0.00,0.23,0.31}{#1}}
\newcommand{\DataTypeTok}[1]{\textcolor[rgb]{0.68,0.00,0.00}{#1}}
\newcommand{\DecValTok}[1]{\textcolor[rgb]{0.68,0.00,0.00}{#1}}
\newcommand{\DocumentationTok}[1]{\textcolor[rgb]{0.37,0.37,0.37}{\textit{#1}}}
\newcommand{\ErrorTok}[1]{\textcolor[rgb]{0.68,0.00,0.00}{#1}}
\newcommand{\ExtensionTok}[1]{\textcolor[rgb]{0.00,0.23,0.31}{#1}}
\newcommand{\FloatTok}[1]{\textcolor[rgb]{0.68,0.00,0.00}{#1}}
\newcommand{\FunctionTok}[1]{\textcolor[rgb]{0.28,0.35,0.67}{#1}}
\newcommand{\ImportTok}[1]{\textcolor[rgb]{0.00,0.46,0.62}{#1}}
\newcommand{\InformationTok}[1]{\textcolor[rgb]{0.37,0.37,0.37}{#1}}
\newcommand{\KeywordTok}[1]{\textcolor[rgb]{0.00,0.23,0.31}{#1}}
\newcommand{\NormalTok}[1]{\textcolor[rgb]{0.00,0.23,0.31}{#1}}
\newcommand{\OperatorTok}[1]{\textcolor[rgb]{0.37,0.37,0.37}{#1}}
\newcommand{\OtherTok}[1]{\textcolor[rgb]{0.00,0.23,0.31}{#1}}
\newcommand{\PreprocessorTok}[1]{\textcolor[rgb]{0.68,0.00,0.00}{#1}}
\newcommand{\RegionMarkerTok}[1]{\textcolor[rgb]{0.00,0.23,0.31}{#1}}
\newcommand{\SpecialCharTok}[1]{\textcolor[rgb]{0.37,0.37,0.37}{#1}}
\newcommand{\SpecialStringTok}[1]{\textcolor[rgb]{0.13,0.47,0.30}{#1}}
\newcommand{\StringTok}[1]{\textcolor[rgb]{0.13,0.47,0.30}{#1}}
\newcommand{\VariableTok}[1]{\textcolor[rgb]{0.07,0.07,0.07}{#1}}
\newcommand{\VerbatimStringTok}[1]{\textcolor[rgb]{0.13,0.47,0.30}{#1}}
\newcommand{\WarningTok}[1]{\textcolor[rgb]{0.37,0.37,0.37}{\textit{#1}}}

\providecommand{\tightlist}{%
  \setlength{\itemsep}{0pt}\setlength{\parskip}{0pt}}\usepackage{longtable,booktabs,array}
\usepackage{calc} % for calculating minipage widths
% Correct order of tables after \paragraph or \subparagraph
\usepackage{etoolbox}
\makeatletter
\patchcmd\longtable{\par}{\if@noskipsec\mbox{}\fi\par}{}{}
\makeatother
% Allow footnotes in longtable head/foot
\IfFileExists{footnotehyper.sty}{\usepackage{footnotehyper}}{\usepackage{footnote}}
\makesavenoteenv{longtable}
\usepackage{graphicx}
\makeatletter
\def\maxwidth{\ifdim\Gin@nat@width>\linewidth\linewidth\else\Gin@nat@width\fi}
\def\maxheight{\ifdim\Gin@nat@height>\textheight\textheight\else\Gin@nat@height\fi}
\makeatother
% Scale images if necessary, so that they will not overflow the page
% margins by default, and it is still possible to overwrite the defaults
% using explicit options in \includegraphics[width, height, ...]{}
\setkeys{Gin}{width=\maxwidth,height=\maxheight,keepaspectratio}
% Set default figure placement to htbp
\makeatletter
\def\fps@figure{htbp}
\makeatother

\usepackage{fancyhdr} \pagestyle{fancy} \usepackage{lastpage}
\KOMAoption{captions}{tablesignature}
\makeatletter
\makeatother
\makeatletter
\makeatother
\makeatletter
\@ifpackageloaded{caption}{}{\usepackage{caption}}
\AtBeginDocument{%
\ifdefined\contentsname
  \renewcommand*\contentsname{Table des matières}
\else
  \newcommand\contentsname{Table des matières}
\fi
\ifdefined\listfigurename
  \renewcommand*\listfigurename{Liste des Figures}
\else
  \newcommand\listfigurename{Liste des Figures}
\fi
\ifdefined\listtablename
  \renewcommand*\listtablename{Liste des Tables}
\else
  \newcommand\listtablename{Liste des Tables}
\fi
\ifdefined\figurename
  \renewcommand*\figurename{Figure}
\else
  \newcommand\figurename{Figure}
\fi
\ifdefined\tablename
  \renewcommand*\tablename{Tableau}
\else
  \newcommand\tablename{Tableau}
\fi
}
\@ifpackageloaded{float}{}{\usepackage{float}}
\floatstyle{ruled}
\@ifundefined{c@chapter}{\newfloat{codelisting}{h}{lop}}{\newfloat{codelisting}{h}{lop}[chapter]}
\floatname{codelisting}{Listing}
\newcommand*\listoflistings{\listof{codelisting}{Liste des Listings}}
\makeatother
\makeatletter
\@ifpackageloaded{caption}{}{\usepackage{caption}}
\@ifpackageloaded{subcaption}{}{\usepackage{subcaption}}
\makeatother
\makeatletter
\@ifpackageloaded{tcolorbox}{}{\usepackage[skins,breakable]{tcolorbox}}
\makeatother
\makeatletter
\@ifundefined{shadecolor}{\definecolor{shadecolor}{rgb}{.97, .97, .97}}
\makeatother
\makeatletter
\makeatother
\makeatletter
\makeatother
\makeatletter
\@ifpackageloaded{fontawesome5}{}{\usepackage{fontawesome5}}
\makeatother
\ifLuaTeX
\usepackage[bidi=basic]{babel}
\else
\usepackage[bidi=default]{babel}
\fi
\babelprovide[main,import]{french}
% get rid of language-specific shorthands (see #6817):
\let\LanguageShortHands\languageshorthands
\def\languageshorthands#1{}
\ifLuaTeX
  \usepackage{selnolig}  % disable illegal ligatures
\fi
\IfFileExists{bookmark.sty}{\usepackage{bookmark}}{\usepackage{hyperref}}
\IfFileExists{xurl.sty}{\usepackage{xurl}}{} % add URL line breaks if available
\urlstyle{same} % disable monospaced font for URLs
\hypersetup{
  pdftitle={Listes, Piles et Files (Exercices)},
  pdflang={fr},
  colorlinks=true,
  linkcolor={blue},
  filecolor={Maroon},
  citecolor={Blue},
  urlcolor={Blue},
  pdfcreator={LaTeX via pandoc}}

\title{Listes, Piles et Files (Exercices)}
\usepackage{etoolbox}
\makeatletter
\providecommand{\subtitle}[1]{% add subtitle to \maketitle
  \apptocmd{\@title}{\par {\large #1 \par}}{}{}
}
\makeatother
\subtitle{S2 - Structures de données}
\author{}
\date{}

\begin{document}
\maketitle
\lhead{Spécialité NSI} \rhead{Terminale} \chead{} \cfoot{} \lfoot{Lycée \'Emile Duclaux} \rfoot{Page \thepage/\pageref{LastPage}} \renewcommand{\headrulewidth}{0pt} \renewcommand{\footrulewidth}{0pt} \thispagestyle{fancy} \vspace{-2cm}

\ifdefined\Shaded\renewenvironment{Shaded}{\begin{tcolorbox}[interior hidden, sharp corners, enhanced, breakable, borderline west={3pt}{0pt}{shadecolor}, boxrule=0pt, frame hidden]}{\end{tcolorbox}}\fi

\emph{Les exercices précédés du symbole \faIcon{desktop} sont à faire
sur machine, en sauvegardant le fichier si nécessaire.}

\emph{Les exercices précédés du symbole \faIcon{pencil-alt} doivent être
résolus par écrit.}

\hypertarget{fa-desktop-exercice-1-listes}{%
\subsection{\texorpdfstring{\faIcon{desktop} Exercice 1
(listes)}{ Exercice 1 (listes)}}\label{fa-desktop-exercice-1-listes}}

On reprend l'implémentation des listes avec des tuples présentée dans le
cours.

Prévoir l'effet et l'affichage en console des instructions suivantes,
puis vérifier en exécutant ces instructions dans la console interactive
après avoir créé et importé un fichier contenant le code du cours :

\begin{Shaded}
\begin{Highlighting}[]
\OperatorTok{\textgreater{}\textgreater{}\textgreater{}}\NormalTok{ L }\OperatorTok{=}\NormalTok{ creer()}
\OperatorTok{\textgreater{}\textgreater{}\textgreater{}}\NormalTok{ est\_vide(L)}
\OperatorTok{\textgreater{}\textgreater{}\textgreater{}}\NormalTok{ L }\OperatorTok{=}\NormalTok{ ajouter(}\DecValTok{5}\NormalTok{, ajouter(}\DecValTok{4}\NormalTok{, ajouter(}\DecValTok{3}\NormalTok{, ajouter(}\DecValTok{2}\NormalTok{, ajouter(}\DecValTok{1}\NormalTok{, ajouter(}\DecValTok{0}\NormalTok{,()))))))}
\OperatorTok{\textgreater{}\textgreater{}\textgreater{}}\NormalTok{ est\_vide(L)}
\OperatorTok{\textgreater{}\textgreater{}\textgreater{}}\NormalTok{ longueur(L)}
\OperatorTok{\textgreater{}\textgreater{}\textgreater{}}\NormalTok{ L }\OperatorTok{=}\NormalTok{ ajouter(}\DecValTok{6}\NormalTok{,L)}
\OperatorTok{\textgreater{}\textgreater{}\textgreater{}}\NormalTok{ longueur(L)}
\OperatorTok{\textgreater{}\textgreater{}\textgreater{}}\NormalTok{ tete(L)}
\OperatorTok{\textgreater{}\textgreater{}\textgreater{}}\NormalTok{ queue(L)}
\OperatorTok{\textgreater{}\textgreater{}\textgreater{}}\NormalTok{ longueur(queue(L))}
\end{Highlighting}
\end{Shaded}

\hypertarget{fa-solid-pencil-alt-exercice-2-listes}{%
\subsection{\texorpdfstring{\faIcon{pencil-alt} Exercice 2
(listes)}{ Exercice 2 (listes)}}\label{fa-solid-pencil-alt-exercice-2-listes}}

Soit la suite d'instructions suivantes :

\begin{Shaded}
\begin{Highlighting}[]
\NormalTok{L }\OperatorTok{=}\NormalTok{ creer()}
\NormalTok{L }\OperatorTok{=}\NormalTok{ ajouter(}\DecValTok{2}\NormalTok{, ajouter(}\DecValTok{15}\NormalTok{, ajouter (}\DecValTok{23}\NormalTok{, L)))}
\NormalTok{L1 }\OperatorTok{=}\NormalTok{ queue(L)}
\NormalTok{a }\OperatorTok{=}\NormalTok{ tete(L1)}
\NormalTok{L1 }\OperatorTok{=}\NormalTok{ ajouter(}\DecValTok{4}\NormalTok{, ajouter(}\DecValTok{3}\NormalTok{, L1))}
\end{Highlighting}
\end{Shaded}

Donnez le contenu des listes \texttt{L} et \texttt{L1} et la valeur de
\texttt{a}.

\hypertarget{fa-solid-pencil-alt-exercice-3-piles}{%
\subsection{\texorpdfstring{\faIcon{pencil-alt} Exercice 3
(piles)}{ Exercice 3 (piles)}}\label{fa-solid-pencil-alt-exercice-3-piles}}

Soit une pile P initialement vide. Soit les instructions suivantes
(implémentation des piles avec des listes Python) :

\begin{Shaded}
\begin{Highlighting}[]
\OperatorTok{\textgreater{}\textgreater{}\textgreater{}}\NormalTok{ empiler(P,}\DecValTok{4}\NormalTok{)}
\OperatorTok{\textgreater{}\textgreater{}\textgreater{}}\NormalTok{ empiler(P,}\DecValTok{7}\NormalTok{)}
\OperatorTok{\textgreater{}\textgreater{}\textgreater{}}\NormalTok{ a }\OperatorTok{=}\NormalTok{ depiler(P)}
\OperatorTok{\textgreater{}\textgreater{}\textgreater{}}\NormalTok{ b }\OperatorTok{=}\NormalTok{ taille(P)}
\OperatorTok{\textgreater{}\textgreater{}\textgreater{}}\NormalTok{ c }\OperatorTok{=}\NormalTok{ depiler(P)}
\OperatorTok{\textgreater{}\textgreater{}\textgreater{}}\NormalTok{ empiler(P,}\DecValTok{3}\NormalTok{)}
\OperatorTok{\textgreater{}\textgreater{}\textgreater{}}\NormalTok{ empiler(P,}\DecValTok{2}\NormalTok{)}
\OperatorTok{\textgreater{}\textgreater{}\textgreater{}}\NormalTok{ d }\OperatorTok{=}\NormalTok{ taille(P)}
\end{Highlighting}
\end{Shaded}

Donnez le contenu de la pile \texttt{P}, la valeur de \texttt{a}, la
valeur de \texttt{b}, la valeur de \texttt{c} et la valeur de
\texttt{d}.

\hypertarget{fa-solid-pencil-alt-exercice-4-piles}{%
\subsection{\texorpdfstring{\faIcon{pencil-alt} Exercice 4
(piles)}{ Exercice 4 (piles)}}\label{fa-solid-pencil-alt-exercice-4-piles}}

Soit le programme Python suivant (on utilise l'implémentation des piles
en POO) :

\begin{Shaded}
\begin{Highlighting}[]
\NormalTok{pile }\OperatorTok{=}\NormalTok{ Pile()}
\NormalTok{tab }\OperatorTok{=}\NormalTok{ [}\DecValTok{5}\NormalTok{,}\DecValTok{8}\NormalTok{,}\DecValTok{6}\NormalTok{,}\DecValTok{1}\NormalTok{,}\DecValTok{3}\NormalTok{,}\DecValTok{7}\NormalTok{]}
\ControlFlowTok{for}\NormalTok{ k }\KeywordTok{in}\NormalTok{ tab:}
\NormalTok{    pile.empiler(k)}
\NormalTok{pile.empiler(}\DecValTok{5}\NormalTok{)}
\NormalTok{pile.empiler(}\DecValTok{10}\NormalTok{)}
\NormalTok{pile.empiler(}\DecValTok{8}\NormalTok{)}
\NormalTok{pile.empiler(}\DecValTok{15}\NormalTok{)}
\ControlFlowTok{for}\NormalTok{ k }\KeywordTok{in}\NormalTok{ tab:}
    \ControlFlowTok{if}\NormalTok{ k }\OperatorTok{\textgreater{}} \DecValTok{5}\NormalTok{:}
\NormalTok{        pile.depiler()}
\end{Highlighting}
\end{Shaded}

Donnez l'état de la pile \texttt{pile} après l'exécution de ce
programme.

\hypertarget{fa-solid-pencil-alt-fa-desktop-exercice-5-piles}{%
\subsection{\texorpdfstring{\faIcon{pencil-alt} \faIcon{desktop}
Exercice 5
(piles)}{  Exercice 5 (piles)}}\label{fa-solid-pencil-alt-fa-desktop-exercice-5-piles}}

Ce problème propose une application concrète des piles. Il s'agit
d'écrire une fonction qui contrôle si une expression mathématique,
donnée sous forme d'une chaîne de caractères, est bien parenthésée,
c'est- à-dire s'il y a autant de parenthèses ouvrantes que de fermantes,
et qu'elles sont bien placées

Par exemple :

\begin{itemize}
\tightlist
\item
  (..(..)..) est bien parenthésée
\item
  (\ldots(..(..)\ldots) ne l'est pas
\end{itemize}

L'algorithme :

\begin{itemize}
\tightlist
\item
  On crée une pile
\item
  On parcourt l'expression de gauche à droite.
\item
  À chaque fois que l'on rencontre une parenthèse ouvrante ``('' on
  l'empile
\item
  Si on rencontre une parenthèse fermante '' ) '' et que la pile n'est
  pas vide on dépile ( sinon on retourne faux )
\item
  À la fin la pile doit être vide\ldots{}
\end{itemize}

\begin{enumerate}
\def\labelenumi{\arabic{enumi}.}
\tightlist
\item
  En utilisant l'une des structures pile du cours, écrire une fonction
  \texttt{verification(expr)} qui vérifie si une expression mathématique
  passée en paramètre est correctement parenthésée.
\item
  Proposer un jeu de tests unitaires vérifiant le bon fonctionnement de
  la fonction.
\item
  Faire en sorte que le programme tienne compte également des {[}.
\end{enumerate}

\hypertarget{fa-solid-pencil-alt-exercice-6-files}{%
\subsection{\texorpdfstring{\faIcon{pencil-alt} Exercice 6
(files)}{ Exercice 6 (files)}}\label{fa-solid-pencil-alt-exercice-6-files}}

Soit une file F initialement vide. Soit les instructions suivantes :

\begin{Shaded}
\begin{Highlighting}[]
\NormalTok{enfiler(F,}\DecValTok{6}\NormalTok{)}
\NormalTok{enfiler(F,}\DecValTok{3}\NormalTok{)}
\NormalTok{a }\OperatorTok{=}\NormalTok{ defiler(F)}
\NormalTok{enfiler(F,}\DecValTok{9}\NormalTok{)}
\NormalTok{b }\OperatorTok{=}\NormalTok{ taille\_file(F)}
\NormalTok{enfiler(F,}\DecValTok{17}\NormalTok{)}
\NormalTok{c }\OperatorTok{=}\NormalTok{ defiler(F)}
\NormalTok{enfiler(F,}\DecValTok{2}\NormalTok{)}
\NormalTok{d }\OperatorTok{=}\NormalTok{ taille\_file(F)}
\end{Highlighting}
\end{Shaded}

Donnez le contenu de la file \texttt{F}, la valeur de \texttt{a}, la
valeur de \texttt{b}, la valeur de \texttt{c} et la valeur de
\texttt{d}.

\hypertarget{fa-solid-pencil-alt-exercice-7-files}{%
\subsection{\texorpdfstring{\faIcon{pencil-alt} Exercice 7
(files)}{ Exercice 7 (files)}}\label{fa-solid-pencil-alt-exercice-7-files}}

Soit le programme Python suivant :

\begin{Shaded}
\begin{Highlighting}[]
\BuiltInTok{file} \OperatorTok{=}\NormalTok{ File()}
\NormalTok{tab }\OperatorTok{=}\NormalTok{ [}\DecValTok{2}\NormalTok{,}\DecValTok{78}\NormalTok{,}\DecValTok{6}\NormalTok{,}\DecValTok{89}\NormalTok{,}\DecValTok{3}\NormalTok{,}\DecValTok{17}\NormalTok{]}
\BuiltInTok{file}\NormalTok{.enfiler(}\DecValTok{5}\NormalTok{)}
\BuiltInTok{file}\NormalTok{.enfiler(}\DecValTok{10}\NormalTok{)}
\BuiltInTok{file}\NormalTok{.enfiler(}\DecValTok{8}\NormalTok{)}
\BuiltInTok{file}\NormalTok{.enfiler(}\DecValTok{15}\NormalTok{)}
\ControlFlowTok{for}\NormalTok{ i }\KeywordTok{in}\NormalTok{ tab:}
    \ControlFlowTok{if}\NormalTok{ i }\OperatorTok{\textgreater{}} \DecValTok{50}\NormalTok{:}
        \BuiltInTok{file}\NormalTok{.defiler()}
\end{Highlighting}
\end{Shaded}

Donnez l'état de la file \texttt{file} après l'exécution de ce programme

\hypertarget{fa-solid-pencil-alt-exercices-tombuxe9s-au-bac}{%
\subsection{\texorpdfstring{\faIcon{pencil-alt} Exercices tombés au
bac}{ Exercices tombés au bac}}\label{fa-solid-pencil-alt-exercices-tombuxe9s-au-bac}}

\begin{itemize}
\tightlist
\item
  \href{../annales//2021_Metropole_CL_1.pdf}{2021 Métropole Jour 1 : exo
  2}.
\item
  \href{../annales//2021_CentresEtrangers_2.pdf}{2021 Centres étrangers
  Jour 2 : exo 1}.
\item
  \href{../annales//2021_CentresEtrangers_1.pdf}{2021 Centres étrangers
  Jour 1 : exo 5}.
\item
  \href{../annales//2021_AmeriqueDuNord_1.pdf}{2021 Amérique du Nord
  Jour1 : exo 5}.
\item
  \href{../annales//2021_Sujet_0.pdf}{2021 Sujet zéro : exo 1}.
\item
  \href{../annales//2022_CentresEtrangers_1.pdf}{2022 Centres étrangers
  Jour 1 : exo 2}.
\item
  \href{../annales//2022_Metropole_Jour1.pdf}{2022 Métropole Jour 1 :
  exo 1}.
\item
  \href{../annales//2022_Metropole_Jour2.pdf}{2022 Métropole Jour 2 :
  exo 2}.
\item
  \href{../annales//2022_Mayotte_Liban_1.pdf}{2022 Mayotte Liban Jour 1
  : exo 1}.
\item
  \href{../annales//2022_Mayotte_Liban_2.pdf}{2022 Mayotte Liban Jour 2
  : exo 1}.
\item
  \href{../annales//2022_AmeriqueDuNord_1.pdf}{2022 Amérique du Nord
  Jour 1 : exo 5}.
\end{itemize}



\end{document}
