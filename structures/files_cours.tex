% Options for packages loaded elsewhere
\PassOptionsToPackage{unicode}{hyperref}
\PassOptionsToPackage{hyphens}{url}
\PassOptionsToPackage{dvipsnames,svgnames,x11names}{xcolor}
%
\documentclass[
  a4paper,
  DIV=11,
  numbers=noendperiod]{scrartcl}

\usepackage{amsmath,amssymb}
\usepackage{lmodern}
\usepackage{iftex}
\ifPDFTeX
  \usepackage[T1]{fontenc}
  \usepackage[utf8]{inputenc}
  \usepackage{textcomp} % provide euro and other symbols
\else % if luatex or xetex
  \usepackage{unicode-math}
  \defaultfontfeatures{Scale=MatchLowercase}
  \defaultfontfeatures[\rmfamily]{Ligatures=TeX,Scale=1}
\fi
% Use upquote if available, for straight quotes in verbatim environments
\IfFileExists{upquote.sty}{\usepackage{upquote}}{}
\IfFileExists{microtype.sty}{% use microtype if available
  \usepackage[]{microtype}
  \UseMicrotypeSet[protrusion]{basicmath} % disable protrusion for tt fonts
}{}
\makeatletter
\@ifundefined{KOMAClassName}{% if non-KOMA class
  \IfFileExists{parskip.sty}{%
    \usepackage{parskip}
  }{% else
    \setlength{\parindent}{0pt}
    \setlength{\parskip}{6pt plus 2pt minus 1pt}}
}{% if KOMA class
  \KOMAoptions{parskip=half}}
\makeatother
\usepackage{xcolor}
\usepackage[top=20mm,bottom=20mm,left=20mm,right=20mm,heightrounded]{geometry}
\setlength{\emergencystretch}{3em} % prevent overfull lines
\setcounter{secnumdepth}{-\maxdimen} % remove section numbering
% Make \paragraph and \subparagraph free-standing
\ifx\paragraph\undefined\else
  \let\oldparagraph\paragraph
  \renewcommand{\paragraph}[1]{\oldparagraph{#1}\mbox{}}
\fi
\ifx\subparagraph\undefined\else
  \let\oldsubparagraph\subparagraph
  \renewcommand{\subparagraph}[1]{\oldsubparagraph{#1}\mbox{}}
\fi

\usepackage{color}
\usepackage{fancyvrb}
\newcommand{\VerbBar}{|}
\newcommand{\VERB}{\Verb[commandchars=\\\{\}]}
\DefineVerbatimEnvironment{Highlighting}{Verbatim}{commandchars=\\\{\}}
% Add ',fontsize=\small' for more characters per line
\usepackage{framed}
\definecolor{shadecolor}{RGB}{241,243,245}
\newenvironment{Shaded}{\begin{snugshade}}{\end{snugshade}}
\newcommand{\AlertTok}[1]{\textcolor[rgb]{0.68,0.00,0.00}{#1}}
\newcommand{\AnnotationTok}[1]{\textcolor[rgb]{0.37,0.37,0.37}{#1}}
\newcommand{\AttributeTok}[1]{\textcolor[rgb]{0.40,0.45,0.13}{#1}}
\newcommand{\BaseNTok}[1]{\textcolor[rgb]{0.68,0.00,0.00}{#1}}
\newcommand{\BuiltInTok}[1]{\textcolor[rgb]{0.00,0.23,0.31}{#1}}
\newcommand{\CharTok}[1]{\textcolor[rgb]{0.13,0.47,0.30}{#1}}
\newcommand{\CommentTok}[1]{\textcolor[rgb]{0.37,0.37,0.37}{#1}}
\newcommand{\CommentVarTok}[1]{\textcolor[rgb]{0.37,0.37,0.37}{\textit{#1}}}
\newcommand{\ConstantTok}[1]{\textcolor[rgb]{0.56,0.35,0.01}{#1}}
\newcommand{\ControlFlowTok}[1]{\textcolor[rgb]{0.00,0.23,0.31}{#1}}
\newcommand{\DataTypeTok}[1]{\textcolor[rgb]{0.68,0.00,0.00}{#1}}
\newcommand{\DecValTok}[1]{\textcolor[rgb]{0.68,0.00,0.00}{#1}}
\newcommand{\DocumentationTok}[1]{\textcolor[rgb]{0.37,0.37,0.37}{\textit{#1}}}
\newcommand{\ErrorTok}[1]{\textcolor[rgb]{0.68,0.00,0.00}{#1}}
\newcommand{\ExtensionTok}[1]{\textcolor[rgb]{0.00,0.23,0.31}{#1}}
\newcommand{\FloatTok}[1]{\textcolor[rgb]{0.68,0.00,0.00}{#1}}
\newcommand{\FunctionTok}[1]{\textcolor[rgb]{0.28,0.35,0.67}{#1}}
\newcommand{\ImportTok}[1]{\textcolor[rgb]{0.00,0.46,0.62}{#1}}
\newcommand{\InformationTok}[1]{\textcolor[rgb]{0.37,0.37,0.37}{#1}}
\newcommand{\KeywordTok}[1]{\textcolor[rgb]{0.00,0.23,0.31}{#1}}
\newcommand{\NormalTok}[1]{\textcolor[rgb]{0.00,0.23,0.31}{#1}}
\newcommand{\OperatorTok}[1]{\textcolor[rgb]{0.37,0.37,0.37}{#1}}
\newcommand{\OtherTok}[1]{\textcolor[rgb]{0.00,0.23,0.31}{#1}}
\newcommand{\PreprocessorTok}[1]{\textcolor[rgb]{0.68,0.00,0.00}{#1}}
\newcommand{\RegionMarkerTok}[1]{\textcolor[rgb]{0.00,0.23,0.31}{#1}}
\newcommand{\SpecialCharTok}[1]{\textcolor[rgb]{0.37,0.37,0.37}{#1}}
\newcommand{\SpecialStringTok}[1]{\textcolor[rgb]{0.13,0.47,0.30}{#1}}
\newcommand{\StringTok}[1]{\textcolor[rgb]{0.13,0.47,0.30}{#1}}
\newcommand{\VariableTok}[1]{\textcolor[rgb]{0.07,0.07,0.07}{#1}}
\newcommand{\VerbatimStringTok}[1]{\textcolor[rgb]{0.13,0.47,0.30}{#1}}
\newcommand{\WarningTok}[1]{\textcolor[rgb]{0.37,0.37,0.37}{\textit{#1}}}

\providecommand{\tightlist}{%
  \setlength{\itemsep}{0pt}\setlength{\parskip}{0pt}}\usepackage{longtable,booktabs,array}
\usepackage{calc} % for calculating minipage widths
% Correct order of tables after \paragraph or \subparagraph
\usepackage{etoolbox}
\makeatletter
\patchcmd\longtable{\par}{\if@noskipsec\mbox{}\fi\par}{}{}
\makeatother
% Allow footnotes in longtable head/foot
\IfFileExists{footnotehyper.sty}{\usepackage{footnotehyper}}{\usepackage{footnote}}
\makesavenoteenv{longtable}
\usepackage{graphicx}
\makeatletter
\def\maxwidth{\ifdim\Gin@nat@width>\linewidth\linewidth\else\Gin@nat@width\fi}
\def\maxheight{\ifdim\Gin@nat@height>\textheight\textheight\else\Gin@nat@height\fi}
\makeatother
% Scale images if necessary, so that they will not overflow the page
% margins by default, and it is still possible to overwrite the defaults
% using explicit options in \includegraphics[width, height, ...]{}
\setkeys{Gin}{width=\maxwidth,height=\maxheight,keepaspectratio}
% Set default figure placement to htbp
\makeatletter
\def\fps@figure{htbp}
\makeatother

\usepackage{fancyhdr} \pagestyle{fancy} \usepackage{lastpage}
\KOMAoption{captions}{tablesignature}
\makeatletter
\@ifpackageloaded{tcolorbox}{}{\usepackage[many]{tcolorbox}}
\@ifpackageloaded{fontawesome5}{}{\usepackage{fontawesome5}}
\definecolor{quarto-callout-color}{HTML}{909090}
\definecolor{quarto-callout-note-color}{HTML}{0758E5}
\definecolor{quarto-callout-important-color}{HTML}{CC1914}
\definecolor{quarto-callout-warning-color}{HTML}{EB9113}
\definecolor{quarto-callout-tip-color}{HTML}{00A047}
\definecolor{quarto-callout-caution-color}{HTML}{FC5300}
\definecolor{quarto-callout-color-frame}{HTML}{acacac}
\definecolor{quarto-callout-note-color-frame}{HTML}{4582ec}
\definecolor{quarto-callout-important-color-frame}{HTML}{d9534f}
\definecolor{quarto-callout-warning-color-frame}{HTML}{f0ad4e}
\definecolor{quarto-callout-tip-color-frame}{HTML}{02b875}
\definecolor{quarto-callout-caution-color-frame}{HTML}{fd7e14}
\makeatother
\makeatletter
\makeatother
\makeatletter
\makeatother
\makeatletter
\@ifpackageloaded{caption}{}{\usepackage{caption}}
\AtBeginDocument{%
\ifdefined\contentsname
  \renewcommand*\contentsname{Table des matières}
\else
  \newcommand\contentsname{Table des matières}
\fi
\ifdefined\listfigurename
  \renewcommand*\listfigurename{Liste des Figures}
\else
  \newcommand\listfigurename{Liste des Figures}
\fi
\ifdefined\listtablename
  \renewcommand*\listtablename{Liste des Tables}
\else
  \newcommand\listtablename{Liste des Tables}
\fi
\ifdefined\figurename
  \renewcommand*\figurename{Figure}
\else
  \newcommand\figurename{Figure}
\fi
\ifdefined\tablename
  \renewcommand*\tablename{Tableau}
\else
  \newcommand\tablename{Tableau}
\fi
}
\@ifpackageloaded{float}{}{\usepackage{float}}
\floatstyle{ruled}
\@ifundefined{c@chapter}{\newfloat{codelisting}{h}{lop}}{\newfloat{codelisting}{h}{lop}[chapter]}
\floatname{codelisting}{Listing}
\newcommand*\listoflistings{\listof{codelisting}{Liste des Listings}}
\makeatother
\makeatletter
\@ifpackageloaded{caption}{}{\usepackage{caption}}
\@ifpackageloaded{subcaption}{}{\usepackage{subcaption}}
\makeatother
\makeatletter
\@ifpackageloaded{tcolorbox}{}{\usepackage[many]{tcolorbox}}
\makeatother
\makeatletter
\@ifundefined{shadecolor}{\definecolor{shadecolor}{rgb}{.97, .97, .97}}
\makeatother
\makeatletter
\makeatother
\ifLuaTeX
\usepackage[bidi=basic]{babel}
\else
\usepackage[bidi=default]{babel}
\fi
\babelprovide[main,import]{french}
% get rid of language-specific shorthands (see #6817):
\let\LanguageShortHands\languageshorthands
\def\languageshorthands#1{}
\ifLuaTeX
  \usepackage{selnolig}  % disable illegal ligatures
\fi
\IfFileExists{bookmark.sty}{\usepackage{bookmark}}{\usepackage{hyperref}}
\IfFileExists{xurl.sty}{\usepackage{xurl}}{} % add URL line breaks if available
\urlstyle{same} % disable monospaced font for URLs
\hypersetup{
  pdftitle={Files (Cours)},
  pdflang={fr},
  colorlinks=true,
  linkcolor={blue},
  filecolor={Maroon},
  citecolor={Blue},
  urlcolor={Blue},
  pdfcreator={LaTeX via pandoc}}

\title{Files (Cours)}
\usepackage{etoolbox}
\makeatletter
\providecommand{\subtitle}[1]{% add subtitle to \maketitle
  \apptocmd{\@title}{\par {\large #1 \par}}{}{}
}
\makeatother
\subtitle{S2 - Structures de données}
\author{}
\date{}

\begin{document}
\maketitle
\lhead{Spécialité NSI} \rhead{Terminale} \chead{} \cfoot{} \lfoot{Lycée \'Emile Duclaux} \rfoot{Page \thepage/\pageref{LastPage}} \renewcommand{\headrulewidth}{0pt} \renewcommand{\footrulewidth}{0pt} \thispagestyle{fancy} \vspace{-2cm}

\ifdefined\Shaded\renewenvironment{Shaded}{\begin{tcolorbox}[frame hidden, borderline west={3pt}{0pt}{shadecolor}, interior hidden, breakable, boxrule=0pt, enhanced, sharp corners]}{\end{tcolorbox}}\fi

\hypertarget{du-point-de-vue-utilisateur-interface}{%
\subsection{1. Du point de vue utilisateur :
interface}\label{du-point-de-vue-utilisateur-interface}}

\begin{tcolorbox}[enhanced jigsaw, bottomrule=.15mm, bottomtitle=1mm, title=\textcolor{quarto-callout-tip-color}{\faLightbulb}\hspace{0.5em}{Définition}, toptitle=1mm, opacityback=0, colbacktitle=quarto-callout-tip-color!10!white, coltitle=black, leftrule=.75mm, colback=white, toprule=.15mm, titlerule=0mm, arc=.35mm, breakable, rightrule=.15mm, left=2mm, opacitybacktitle=0.6]

La \textbf{file}, comme la liste et la pile, permet de stocker des
données et d'y accéder. La différence se situe au niveau de l'ajout et
du retrait d'éléments.

\begin{itemize}
\tightlist
\item
  Le prochain élément auquel on peut accéder est le premier élément
  ajouté à la structure ;
\item
  Les nouveaux éléments viennent en bout de file : on ne pourra y
  accéder que lorsque tous les éléments ayant été ajoutés avant eux
  seront sortis de la file.
\end{itemize}

\end{tcolorbox}

On parle de mode \textbf{FIFO} (First in, First out, en anglais, premier
arrivé, premier sorti), c'est-à-dire que le premier élément ayant été
ajouté à la structure sera le prochain élément auquel on accédera. Les
derniers éléments ajoutés devront « attendre » que tous les éléments
ayant été ajoutés avant eux soient sortis de la file. Contrairement aux
listes, on ne peut donc pas accéder à n'importe quelle valeur de la
structure (pas d'index).

Pour gérer cette contrainte, la pile est caractérisée par deux «
emplacements » :

\begin{itemize}
\tightlist
\item
  la \textbf{tête} de file, sortie de la file (début de la structure),
  où les éléments sont retirés ;
\item
  le \textbf{bout} de file, entrée de la file (fin de la structure), où
  les éléments sont ajoutés.
\end{itemize}

On peut s'imaginer une \textbf{file d'attente}, dans un cinéma par
exemple. Les premières personnes à pouvoir acheter leur place sont les
premières arrivées, et les nouveaux arrivants se placent au bout de la
file.

\includegraphics{file.jpg}

Une file est une collection de données. On appelle tête de file le
premier élément de la structure et bout de file le dernier élément.
Quand un élément est ajouté à la file, on l'ajoute en bout de file et il
devient le nouveau bout de file c'est-à-dire l'élément « suivant »
l'élément situé précédemment en bout de file. Quand un élément est
retiré de la file, on le sélectionne à la tête de la file et la nouvelle
tête est l'élément qui suivait l'ancienne tête. Lorsqu'on ajoute un
élément à une file vide, celui-ci est donc à la fois la tête et le bout
de la file.

\includegraphics{FIFO_queue.png}

6 primitives constituent l'interface permettant de définir le type
abstrait de données ``file'' :

\begin{itemize}
\tightlist
\item
  \texttt{creer()}, qui crée une file vide ;
\item
  \texttt{taille(file)}, qui permet de connaître le nombre d'éléments
  contenus dans la file ;
\item
  \texttt{est\_vide(file)}, qui renvoie vrai si la file est vide, faux
  sinon ;
\item
  \texttt{enfiler(file,\ element)}, qui ajoute un élément au bout de la
  file (et devient le nouveau bout de file) ;
\item
  \texttt{defiler(file)}, qui retire et renvoie l'élément situé à la
  tête de la file (la nouvelle tête devient l'élément qui suivait
  l'ancienne tête) ;
\item
  \texttt{tete(file)}, qui renvoie l'élément situé à la tête de la file
  (sans le retirer).
\end{itemize}

Enfiler se dit \emph{enqueue} en anglais et défiler se dit
\emph{dequeue}.

\begin{tcolorbox}[enhanced jigsaw, bottomrule=.15mm, bottomtitle=1mm, title=\textcolor{quarto-callout-note-color}{\faInfo}\hspace{0.5em}{La file est utile dans différents types de problèmes}, toptitle=1mm, opacityback=0, colbacktitle=quarto-callout-note-color!10!white, coltitle=black, leftrule=.75mm, colback=white, toprule=.15mm, titlerule=0mm, arc=.35mm, breakable, rightrule=.15mm, left=2mm, opacitybacktitle=0.6]

\begin{itemize}
\tightlist
\item
  pour une imprimante, gestion de la file d'attente des documents à
  imprimer ;
\item
  modélisation du jeu de la bataille (on révèle la carte au-dessus du
  paquet et on place celles gagnées en dessous\ldots) ;
\item
  gestion de mémoires tampon, pour gérer les flux de lecture et
  d'écriture dans un fichier, par exemple ;
\item
  matérialisation d'une file d'attente, pour un logiciel
  (visioconférence par exemple) ou un jeu (gestion des connexions des
  utilisateurs, des tours de jeu\ldots),\ldots{}
\item
  algorithme du parcours en largeur pour les arbres et les graphes, par
  exemple, pour trouver le plus court trajet sur une carte, ou récupérer
  les valeurs d'une structure dans l'ordre croissant.. (voir séquence
  6).
\end{itemize}

\end{tcolorbox}

\begin{tcolorbox}[enhanced jigsaw, bottomrule=.15mm, bottomtitle=1mm, title=\textcolor{quarto-callout-caution-color}{\faFire}\hspace{0.5em}{Exemple}, toptitle=1mm, opacityback=0, colbacktitle=quarto-callout-caution-color!10!white, coltitle=black, leftrule=.75mm, colback=white, toprule=.15mm, titlerule=0mm, arc=.35mm, breakable, rightrule=.15mm, left=2mm, opacitybacktitle=0.6]

Supposons implémenté le type abstrait \textbf{file}. Nous disposons
d'une interface composée des six primitives décrites ci-dessus. On
considère une file \texttt{F} composée des éléments suivants : 12, 14,
8, 7, 19 et 22 (la tête = premier élément entré dans la file est 22, le
dernier élément entré est 12). On exécute le code suivant ligne par
ligne :

\begin{Shaded}
\begin{Highlighting}[numbers=left,,]
\NormalTok{    enfiler(F,}\DecValTok{42}\NormalTok{)}
\NormalTok{    defiler(F)}
\NormalTok{    defiler(F)}
\NormalTok{    taille(F)}
\NormalTok{    estVide(F)}
\NormalTok{    tete(F)}
\end{Highlighting}
\end{Shaded}

\begin{itemize}
\tightlist
\item
  L'exécution de la ligne 1 ajoute l'élément 42 au bout de la file qui
  contient alors 42, 12, 14, 8, 7, 19, 22 ;
\item
  L'exécution de la ligne 2 affiche 22 et retire cet élément de la file
  qui contient maintenant 42, 12, 14, 8, 7, 19 ;
\item
  L'exécution de la ligne 3 affiche 19 et retire cet élément de la file
  qui contient maintenant 42, 12, 14, 8, 7 ;
\item
  La ligne 4 renvoie la taille de \texttt{F} : 5 ;
\item
  La file n'est pas vide, on obtient dont \texttt{False}.
\end{itemize}

\end{tcolorbox}

\hypertarget{du-point-de-vue-concepteur-impluxe9mentations}{%
\subsection{2. Du point de vue concepteur :
implémentation(s)}\label{du-point-de-vue-concepteur-impluxe9mentations}}

L'implémentation utilisant des listes Python est possible, mais
l'opération \texttt{defiler()} est inefficace dans ce cas (on a une
complexité en \(\mathcal{O}(n)\)).

Nous allons tout d'abord étudier une implémentation utilisant
\textbf{deux piles}.

\hypertarget{impluxe9mentation-utilisant-deux-piles}{%
\subsubsection{Implémentation utilisant deux
piles}\label{impluxe9mentation-utilisant-deux-piles}}

Comme le programme le suggère, il est possible d'implanter une file en
utilisant deux piles. Le procédé est le suivant :

\begin{itemize}
\item
  la file est, au départ, composée de deux piles vides ;
\item
  la première pile est une pile dite « d'entrée » et la seconde « de
  sortie » ;
\item
  quand on ajoute un élément dans la file, on le place dans la pile «
  d'entrée » ;
\item
  Quand on retire (ou qu'on accède) au premier élément de la file, on a
  deux cas :

  \begin{itemize}
  \tightlist
  \item
    soit la pile « de sortie » est vide et on dépile chaque élément de
    la pile « d'entrée » pour les empiler immédiatement dans la pile «
    de sortie » ;
  \item
    soit il y a au moins un élément dans la pile « de sortie », auquel
    cas on ne fait rien de plus.
  \item
    Enfin, on sélectionne le sommet de la pile « de sortie » ;
  \end{itemize}
\item
  comme il y a deux piles, la taille de la file (et le fait qu'elle soit
  vide ou non) doit se baser sur les éléments contenus dans les deux
  piles.
\end{itemize}

\includegraphics{file2piles.png}

Dans notre implémentation, on propose de matérialiser la file sous la
forme d'un tuple contenant deux piles, crées (et manipulées) avec les
méthodes du module modélisant le type abstrait de données pile en
utilisant les listes Python, définit plus tôt dans la section sur les
piles. Ce module, nommé \texttt{piles.py} sera importé dans le présent
fichier. On introduit également une nouvelle méthode « transferer » qui
sert à effectuer le transfert entre les piles (si nécessaire) avant de
retirer ou de récupérer le premier élément de la file.

\begin{Shaded}
\begin{Highlighting}[]
\ImportTok{import}\NormalTok{ piles}

\CommentTok{"""Implémentation du type abstrait "file" avec deux piles"""}


\KeywordTok{def}\NormalTok{ creer\_file():}
    \CommentTok{"""Retourne une file vide"""}
\NormalTok{    pile\_in }\OperatorTok{=}\NormalTok{ piles.creer()}
\NormalTok{    pile\_out }\OperatorTok{=}\NormalTok{ piles.creer()}
    \ControlFlowTok{return}\NormalTok{ (pile\_in, pile\_out)}


\KeywordTok{def}\NormalTok{ taille\_file(}\BuiltInTok{file}\NormalTok{):}
    \CommentTok{"""Retourne le nombre d\textquotesingle{}éléments dans la file"""}
    \ControlFlowTok{return}\NormalTok{ piles.taille(}\BuiltInTok{file}\NormalTok{[}\DecValTok{0}\NormalTok{]) }\OperatorTok{+}\NormalTok{ piles.taille(}\BuiltInTok{file}\NormalTok{[}\DecValTok{1}\NormalTok{])}


\KeywordTok{def}\NormalTok{ est\_vide\_file(}\BuiltInTok{file}\NormalTok{):}
    \CommentTok{"""Retourne True si la file est vide, False sinon"""}
    \ControlFlowTok{return}\NormalTok{ piles.est\_vide(}\BuiltInTok{file}\NormalTok{[}\DecValTok{0}\NormalTok{]) }\KeywordTok{and}\NormalTok{ piles.est\_vide(}\BuiltInTok{file}\NormalTok{[}\DecValTok{1}\NormalTok{])}


\KeywordTok{def}\NormalTok{ enfiler(}\BuiltInTok{file}\NormalTok{, element):}
    \CommentTok{"""Ajoute un nouvel élément à l\textquotesingle{}arrière de la file"""}
\NormalTok{    piles.empiler(}\BuiltInTok{file}\NormalTok{[}\DecValTok{0}\NormalTok{], element)}


\KeywordTok{def}\NormalTok{ transferer(}\BuiltInTok{file}\NormalTok{):}
    \CommentTok{"""Transfère les éléments de la pile d\textquotesingle{}entrée vers la pile de sortie"""}
    \ControlFlowTok{while}\NormalTok{ piles.taille(}\BuiltInTok{file}\NormalTok{[}\DecValTok{0}\NormalTok{]) }\OperatorTok{!=} \DecValTok{0}\NormalTok{:}
\NormalTok{        item }\OperatorTok{=}\NormalTok{ piles.depiler(}\BuiltInTok{file}\NormalTok{[}\DecValTok{0}\NormalTok{])}
\NormalTok{        piles.empiler(}\BuiltInTok{file}\NormalTok{[}\DecValTok{1}\NormalTok{], item)}


\KeywordTok{def}\NormalTok{ defiler(}\BuiltInTok{file}\NormalTok{):}
    \CommentTok{"""Retourne l\textquotesingle{}élément situé en tête de la file et le supprime de la file"""}
    \ControlFlowTok{if}\NormalTok{ taille\_file(}\BuiltInTok{file}\NormalTok{) }\OperatorTok{==} \DecValTok{0}\NormalTok{:}
        \ControlFlowTok{return} \VariableTok{None}
    \ControlFlowTok{else}\NormalTok{:}
        \ControlFlowTok{if}\NormalTok{ piles.sommet(}\BuiltInTok{file}\NormalTok{[}\DecValTok{1}\NormalTok{]) }\KeywordTok{is} \VariableTok{None}\NormalTok{:}
\NormalTok{            transferer(}\BuiltInTok{file}\NormalTok{)}
        \ControlFlowTok{return} \BuiltInTok{file}\NormalTok{[}\DecValTok{1}\NormalTok{].pop()}


\KeywordTok{def}\NormalTok{ tete\_file(}\BuiltInTok{file}\NormalTok{):}
    \CommentTok{"""Retourne l\textquotesingle{}élément situé en tête de la file"""}
    \ControlFlowTok{if}\NormalTok{ taille\_file(}\BuiltInTok{file}\NormalTok{) }\OperatorTok{==} \DecValTok{0}\NormalTok{:}
        \ControlFlowTok{return} \VariableTok{None}
    \ControlFlowTok{else}\NormalTok{:}
        \ControlFlowTok{if}\NormalTok{ piles.sommet(}\BuiltInTok{file}\NormalTok{[}\DecValTok{1}\NormalTok{]) }\KeywordTok{is} \VariableTok{None}\NormalTok{:}
\NormalTok{            transferer(}\BuiltInTok{file}\NormalTok{)}
        \ControlFlowTok{return}\NormalTok{ piles.sommet(}\BuiltInTok{file}\NormalTok{[}\DecValTok{1}\NormalTok{])}
\end{Highlighting}
\end{Shaded}

Cette implémentation sera testée en exercices.

\hypertarget{impluxe9mentation-utilisant-la-poo}{%
\subsubsection{Implémentation utilisant la
POO}\label{impluxe9mentation-utilisant-la-poo}}

On reprend l'idée du chaînon, mais cette fois-ci, un chaînon est lié à
son élément \textbf{précédent} dans la file, et non à son élément
suivant : en effet, quand un élément sort de la file, c'est le précédent
qui prend la tête.

\begin{Shaded}
\begin{Highlighting}[]
\CommentTok{""""Implémentation du type abstrait file en POO"""}


\KeywordTok{class}\NormalTok{ Chainon:}
    \KeywordTok{def} \FunctionTok{\_\_init\_\_}\NormalTok{(}\VariableTok{self}\NormalTok{, element}\OperatorTok{=}\VariableTok{None}\NormalTok{, precedent}\OperatorTok{=}\VariableTok{None}\NormalTok{):}
        \CommentTok{"""element est la valeur du chainon et precedent est le chainon qui suit"""}
        \VariableTok{self}\NormalTok{.element }\OperatorTok{=}\NormalTok{ element}
        \VariableTok{self}\NormalTok{.precedent }\OperatorTok{=}\NormalTok{ precedent}


\KeywordTok{class}\NormalTok{ File():}
    \KeywordTok{def} \FunctionTok{\_\_init\_\_}\NormalTok{(}\VariableTok{self}\NormalTok{):}
        \VariableTok{self}\NormalTok{.front }\OperatorTok{=} \VariableTok{None}
        \VariableTok{self}\NormalTok{.back }\OperatorTok{=} \VariableTok{None}

    \KeywordTok{def}\NormalTok{ taille(}\VariableTok{self}\NormalTok{) }\OperatorTok{{-}\textgreater{}} \BuiltInTok{int}\NormalTok{:}
        \CommentTok{"""Retourne le nombre d\textquotesingle{}éléments dans la file"""}
        \BuiltInTok{long} \OperatorTok{=} \DecValTok{0}
\NormalTok{        chainon }\OperatorTok{=} \VariableTok{self}\NormalTok{.front}
        \ControlFlowTok{while}\NormalTok{ chainon }\KeywordTok{is} \KeywordTok{not} \VariableTok{None}\NormalTok{:}
\NormalTok{            chainon }\OperatorTok{=}\NormalTok{ chainon.precedent}
            \BuiltInTok{long} \OperatorTok{=} \BuiltInTok{long} \OperatorTok{+} \DecValTok{1}
        \ControlFlowTok{return} \BuiltInTok{long}

    \KeywordTok{def}\NormalTok{ est\_vide(}\VariableTok{self}\NormalTok{) }\OperatorTok{{-}\textgreater{}} \BuiltInTok{bool}\NormalTok{:}
        \CommentTok{"""Retourne True si la file est vide et False sinon"""}
        \ControlFlowTok{return} \VariableTok{self}\NormalTok{.front }\KeywordTok{is} \VariableTok{None}

    \KeywordTok{def}\NormalTok{ enfiler(}\VariableTok{self}\NormalTok{, element):}
        \CommentTok{"""Ajoute un nouvel élément à l\textquotesingle{}arrière de la file"""}
\NormalTok{        new\_back }\OperatorTok{=}\NormalTok{ Chainon(element, }\VariableTok{None}\NormalTok{)   }\CommentTok{\# Création d\textquotesingle{}un nouveau chaînon}
        \ControlFlowTok{if} \VariableTok{self}\NormalTok{.taille() }\OperatorTok{==} \DecValTok{0}\NormalTok{:}
            \CommentTok{\# dans ce cas la file est vide et la tête est la queue}
            \VariableTok{self}\NormalTok{.front }\OperatorTok{=}\NormalTok{ new\_back}
        \ControlFlowTok{else}\NormalTok{:}
            \VariableTok{self}\NormalTok{.back.precedent }\OperatorTok{=}\NormalTok{ new\_back    }\CommentTok{\# On relie l\textquotesingle{}ancien dernier élément au nouveau}
        \VariableTok{self}\NormalTok{.back }\OperatorTok{=}\NormalTok{ new\_back    }\CommentTok{\# On définit le nouveau dernier élément}

    \KeywordTok{def}\NormalTok{ defiler(}\VariableTok{self}\NormalTok{):}
        \CommentTok{"""Retourne l\textquotesingle{}élément situé en tête de la file et le supprime de la file"""}
\NormalTok{        item }\OperatorTok{=} \VariableTok{self}\NormalTok{.front.element}
        \VariableTok{self}\NormalTok{.front }\OperatorTok{=} \VariableTok{self}\NormalTok{.front.precedent}
        \ControlFlowTok{return}\NormalTok{ item}

    \KeywordTok{def}\NormalTok{ tete(}\VariableTok{self}\NormalTok{):}
        \CommentTok{"""Retourne l\textquotesingle{}élément situé en tête de la file"""}
        \ControlFlowTok{return} \VariableTok{self}\NormalTok{.front.element}

    \KeywordTok{def} \FunctionTok{\_\_str\_\_}\NormalTok{(}\VariableTok{self}\NormalTok{):}
\NormalTok{        chainon }\OperatorTok{=} \VariableTok{self}\NormalTok{.front}
\NormalTok{        res }\OperatorTok{=} \BuiltInTok{str}\NormalTok{(chainon.element)}
        \ControlFlowTok{while}\NormalTok{ chainon.precedent }\KeywordTok{is} \KeywordTok{not} \VariableTok{None}\NormalTok{:}
\NormalTok{            res }\OperatorTok{=} \StringTok{" \textless{}{-}{-} "} \OperatorTok{+}\NormalTok{ res}
\NormalTok{            chainon }\OperatorTok{=}\NormalTok{ chainon.precedent}
\NormalTok{            res }\OperatorTok{=} \BuiltInTok{str}\NormalTok{(chainon.element) }\OperatorTok{+}\NormalTok{ res}
        \ControlFlowTok{return}\NormalTok{ res}
\end{Highlighting}
\end{Shaded}

La méthode \textbf{enfiler} demande un peu d'attention et doit être bien
comprise.

Exemple d'utilisation de cette interface en console :

\begin{Shaded}
\begin{Highlighting}[]
\OperatorTok{\textgreater{}\textgreater{}\textgreater{}}\NormalTok{ a }\OperatorTok{=}\NormalTok{ File()}
\OperatorTok{\textgreater{}\textgreater{}\textgreater{}}\NormalTok{ a.taille()}
\DecValTok{0}
\OperatorTok{\textgreater{}\textgreater{}\textgreater{}}\NormalTok{ a.est\_vide()}
\VariableTok{True}
\OperatorTok{\textgreater{}\textgreater{}\textgreater{}} \ControlFlowTok{for}\NormalTok{ k }\KeywordTok{in} \BuiltInTok{range}\NormalTok{(}\DecValTok{5}\NormalTok{):}
\NormalTok{        a.enfiler(}\DecValTok{10}\OperatorTok{*}\NormalTok{k)}
\OperatorTok{\textgreater{}\textgreater{}\textgreater{}}\NormalTok{ a.est\_vide()}
\VariableTok{False}
\OperatorTok{\textgreater{}\textgreater{}\textgreater{}}\NormalTok{ a.taille()}
\DecValTok{5}
\OperatorTok{\textgreater{}\textgreater{}\textgreater{}} \BuiltInTok{print}\NormalTok{(a)}
\DecValTok{40} \OperatorTok{\textless{}{-}{-}} \DecValTok{30} \OperatorTok{\textless{}{-}{-}} \DecValTok{20} \OperatorTok{\textless{}{-}{-}} \DecValTok{10} \OperatorTok{\textless{}{-}{-}} \DecValTok{0}
\OperatorTok{\textgreater{}\textgreater{}\textgreater{}}\NormalTok{ a.tete()}
\DecValTok{0}
\OperatorTok{\textgreater{}\textgreater{}\textgreater{}}\NormalTok{ a.defiler()}
\DecValTok{0}
\OperatorTok{\textgreater{}\textgreater{}\textgreater{}}\NormalTok{ a.defiler()}
\DecValTok{10}
\OperatorTok{\textgreater{}\textgreater{}\textgreater{}}\NormalTok{ a.taille()}
\DecValTok{3}
\end{Highlighting}
\end{Shaded}

\begin{tcolorbox}[enhanced jigsaw, bottomrule=.15mm, bottomtitle=1mm, title=\textcolor{quarto-callout-note-color}{\faInfo}\hspace{0.5em}{Remarque (extrait de la documentation Python)}, toptitle=1mm, opacityback=0, colbacktitle=quarto-callout-note-color!10!white, coltitle=black, leftrule=.75mm, colback=white, toprule=.15mm, titlerule=0mm, arc=.35mm, breakable, rightrule=.15mm, left=2mm, opacitybacktitle=0.6]

Il est également possible d'utiliser une liste comme une file, où le
premier élément ajouté est le premier récupéré (« premier entré, premier
sorti » ou FIFO pour first-in, first-out) ; toutefois, les listes ne
sont pas très efficaces pour réaliser ce type de traitement. Alors que
les ajouts et suppressions en fin de liste sont rapides, les insertions
ou les retraits en début de liste sont lents (car tous les autres
éléments doivent être décalés d'une position).

Pour implémenter une file, utilisez plutôt la classe
\texttt{collections.deque} qui a été conçue spécialement pour réaliser
rapidement les opérations d'ajout et de retrait aux deux extrémités. Par
exemple :

\begin{Shaded}
\begin{Highlighting}[]
\OperatorTok{\textgreater{}\textgreater{}\textgreater{}} \ImportTok{from}\NormalTok{ collections }\ImportTok{import}\NormalTok{ deque}
\OperatorTok{\textgreater{}\textgreater{}\textgreater{}}\NormalTok{ queue }\OperatorTok{=}\NormalTok{ deque([}\StringTok{"Eric"}\NormalTok{, }\StringTok{"John"}\NormalTok{, }\StringTok{"Michael"}\NormalTok{])}
\OperatorTok{\textgreater{}\textgreater{}\textgreater{}}\NormalTok{ queue.append(}\StringTok{"Terry"}\NormalTok{)           }\CommentTok{\# Terry arrives}
\OperatorTok{\textgreater{}\textgreater{}\textgreater{}}\NormalTok{ queue.append(}\StringTok{"Graham"}\NormalTok{)          }\CommentTok{\# Graham arrives}
\OperatorTok{\textgreater{}\textgreater{}\textgreater{}}\NormalTok{ queue.popleft()                 }\CommentTok{\# The first to arrive now leaves}
\CommentTok{\textquotesingle{}Eric\textquotesingle{}}
\OperatorTok{\textgreater{}\textgreater{}\textgreater{}}\NormalTok{ queue.popleft()                 }\CommentTok{\# The second to arrive now leaves}
\CommentTok{\textquotesingle{}John\textquotesingle{}}
\OperatorTok{\textgreater{}\textgreater{}\textgreater{}}\NormalTok{ queue                           }\CommentTok{\# Remaining queue in order of arrival}
\NormalTok{deque([}\StringTok{\textquotesingle{}Michael\textquotesingle{}}\NormalTok{, }\StringTok{\textquotesingle{}Terry\textquotesingle{}}\NormalTok{, }\StringTok{\textquotesingle{}Graham\textquotesingle{}}\NormalTok{])}
\end{Highlighting}
\end{Shaded}

\end{tcolorbox}



\end{document}
