% Options for packages loaded elsewhere
\PassOptionsToPackage{unicode}{hyperref}
\PassOptionsToPackage{hyphens}{url}
\PassOptionsToPackage{dvipsnames,svgnames,x11names}{xcolor}
%
\documentclass[
  a4paper,
  DIV=11,
  numbers=noendperiod]{scrartcl}

\usepackage{amsmath,amssymb}
\usepackage{lmodern}
\usepackage{iftex}
\ifPDFTeX
  \usepackage[T1]{fontenc}
  \usepackage[utf8]{inputenc}
  \usepackage{textcomp} % provide euro and other symbols
\else % if luatex or xetex
  \usepackage{unicode-math}
  \defaultfontfeatures{Scale=MatchLowercase}
  \defaultfontfeatures[\rmfamily]{Ligatures=TeX,Scale=1}
\fi
% Use upquote if available, for straight quotes in verbatim environments
\IfFileExists{upquote.sty}{\usepackage{upquote}}{}
\IfFileExists{microtype.sty}{% use microtype if available
  \usepackage[]{microtype}
  \UseMicrotypeSet[protrusion]{basicmath} % disable protrusion for tt fonts
}{}
\makeatletter
\@ifundefined{KOMAClassName}{% if non-KOMA class
  \IfFileExists{parskip.sty}{%
    \usepackage{parskip}
  }{% else
    \setlength{\parindent}{0pt}
    \setlength{\parskip}{6pt plus 2pt minus 1pt}}
}{% if KOMA class
  \KOMAoptions{parskip=half}}
\makeatother
\usepackage{xcolor}
\usepackage[top=20mm,bottom=20mm,left=20mm,right=20mm,heightrounded]{geometry}
\setlength{\emergencystretch}{3em} % prevent overfull lines
\setcounter{secnumdepth}{-\maxdimen} % remove section numbering
% Make \paragraph and \subparagraph free-standing
\ifx\paragraph\undefined\else
  \let\oldparagraph\paragraph
  \renewcommand{\paragraph}[1]{\oldparagraph{#1}\mbox{}}
\fi
\ifx\subparagraph\undefined\else
  \let\oldsubparagraph\subparagraph
  \renewcommand{\subparagraph}[1]{\oldsubparagraph{#1}\mbox{}}
\fi

\usepackage{color}
\usepackage{fancyvrb}
\newcommand{\VerbBar}{|}
\newcommand{\VERB}{\Verb[commandchars=\\\{\}]}
\DefineVerbatimEnvironment{Highlighting}{Verbatim}{commandchars=\\\{\}}
% Add ',fontsize=\small' for more characters per line
\usepackage{framed}
\definecolor{shadecolor}{RGB}{241,243,245}
\newenvironment{Shaded}{\begin{snugshade}}{\end{snugshade}}
\newcommand{\AlertTok}[1]{\textcolor[rgb]{0.68,0.00,0.00}{#1}}
\newcommand{\AnnotationTok}[1]{\textcolor[rgb]{0.37,0.37,0.37}{#1}}
\newcommand{\AttributeTok}[1]{\textcolor[rgb]{0.40,0.45,0.13}{#1}}
\newcommand{\BaseNTok}[1]{\textcolor[rgb]{0.68,0.00,0.00}{#1}}
\newcommand{\BuiltInTok}[1]{\textcolor[rgb]{0.00,0.23,0.31}{#1}}
\newcommand{\CharTok}[1]{\textcolor[rgb]{0.13,0.47,0.30}{#1}}
\newcommand{\CommentTok}[1]{\textcolor[rgb]{0.37,0.37,0.37}{#1}}
\newcommand{\CommentVarTok}[1]{\textcolor[rgb]{0.37,0.37,0.37}{\textit{#1}}}
\newcommand{\ConstantTok}[1]{\textcolor[rgb]{0.56,0.35,0.01}{#1}}
\newcommand{\ControlFlowTok}[1]{\textcolor[rgb]{0.00,0.23,0.31}{#1}}
\newcommand{\DataTypeTok}[1]{\textcolor[rgb]{0.68,0.00,0.00}{#1}}
\newcommand{\DecValTok}[1]{\textcolor[rgb]{0.68,0.00,0.00}{#1}}
\newcommand{\DocumentationTok}[1]{\textcolor[rgb]{0.37,0.37,0.37}{\textit{#1}}}
\newcommand{\ErrorTok}[1]{\textcolor[rgb]{0.68,0.00,0.00}{#1}}
\newcommand{\ExtensionTok}[1]{\textcolor[rgb]{0.00,0.23,0.31}{#1}}
\newcommand{\FloatTok}[1]{\textcolor[rgb]{0.68,0.00,0.00}{#1}}
\newcommand{\FunctionTok}[1]{\textcolor[rgb]{0.28,0.35,0.67}{#1}}
\newcommand{\ImportTok}[1]{\textcolor[rgb]{0.00,0.46,0.62}{#1}}
\newcommand{\InformationTok}[1]{\textcolor[rgb]{0.37,0.37,0.37}{#1}}
\newcommand{\KeywordTok}[1]{\textcolor[rgb]{0.00,0.23,0.31}{#1}}
\newcommand{\NormalTok}[1]{\textcolor[rgb]{0.00,0.23,0.31}{#1}}
\newcommand{\OperatorTok}[1]{\textcolor[rgb]{0.37,0.37,0.37}{#1}}
\newcommand{\OtherTok}[1]{\textcolor[rgb]{0.00,0.23,0.31}{#1}}
\newcommand{\PreprocessorTok}[1]{\textcolor[rgb]{0.68,0.00,0.00}{#1}}
\newcommand{\RegionMarkerTok}[1]{\textcolor[rgb]{0.00,0.23,0.31}{#1}}
\newcommand{\SpecialCharTok}[1]{\textcolor[rgb]{0.37,0.37,0.37}{#1}}
\newcommand{\SpecialStringTok}[1]{\textcolor[rgb]{0.13,0.47,0.30}{#1}}
\newcommand{\StringTok}[1]{\textcolor[rgb]{0.13,0.47,0.30}{#1}}
\newcommand{\VariableTok}[1]{\textcolor[rgb]{0.07,0.07,0.07}{#1}}
\newcommand{\VerbatimStringTok}[1]{\textcolor[rgb]{0.13,0.47,0.30}{#1}}
\newcommand{\WarningTok}[1]{\textcolor[rgb]{0.37,0.37,0.37}{\textit{#1}}}

\providecommand{\tightlist}{%
  \setlength{\itemsep}{0pt}\setlength{\parskip}{0pt}}\usepackage{longtable,booktabs,array}
\usepackage{calc} % for calculating minipage widths
% Correct order of tables after \paragraph or \subparagraph
\usepackage{etoolbox}
\makeatletter
\patchcmd\longtable{\par}{\if@noskipsec\mbox{}\fi\par}{}{}
\makeatother
% Allow footnotes in longtable head/foot
\IfFileExists{footnotehyper.sty}{\usepackage{footnotehyper}}{\usepackage{footnote}}
\makesavenoteenv{longtable}
\usepackage{graphicx}
\makeatletter
\def\maxwidth{\ifdim\Gin@nat@width>\linewidth\linewidth\else\Gin@nat@width\fi}
\def\maxheight{\ifdim\Gin@nat@height>\textheight\textheight\else\Gin@nat@height\fi}
\makeatother
% Scale images if necessary, so that they will not overflow the page
% margins by default, and it is still possible to overwrite the defaults
% using explicit options in \includegraphics[width, height, ...]{}
\setkeys{Gin}{width=\maxwidth,height=\maxheight,keepaspectratio}
% Set default figure placement to htbp
\makeatletter
\def\fps@figure{htbp}
\makeatother

\usepackage{fancyhdr} \pagestyle{fancy} \usepackage{lastpage}
\KOMAoption{captions}{tablesignature}
\makeatletter
\@ifpackageloaded{tcolorbox}{}{\usepackage[many]{tcolorbox}}
\@ifpackageloaded{fontawesome5}{}{\usepackage{fontawesome5}}
\definecolor{quarto-callout-color}{HTML}{909090}
\definecolor{quarto-callout-note-color}{HTML}{0758E5}
\definecolor{quarto-callout-important-color}{HTML}{CC1914}
\definecolor{quarto-callout-warning-color}{HTML}{EB9113}
\definecolor{quarto-callout-tip-color}{HTML}{00A047}
\definecolor{quarto-callout-caution-color}{HTML}{FC5300}
\definecolor{quarto-callout-color-frame}{HTML}{acacac}
\definecolor{quarto-callout-note-color-frame}{HTML}{4582ec}
\definecolor{quarto-callout-important-color-frame}{HTML}{d9534f}
\definecolor{quarto-callout-warning-color-frame}{HTML}{f0ad4e}
\definecolor{quarto-callout-tip-color-frame}{HTML}{02b875}
\definecolor{quarto-callout-caution-color-frame}{HTML}{fd7e14}
\makeatother
\makeatletter
\makeatother
\makeatletter
\makeatother
\makeatletter
\@ifpackageloaded{caption}{}{\usepackage{caption}}
\AtBeginDocument{%
\ifdefined\contentsname
  \renewcommand*\contentsname{Table des matières}
\else
  \newcommand\contentsname{Table des matières}
\fi
\ifdefined\listfigurename
  \renewcommand*\listfigurename{Liste des Figures}
\else
  \newcommand\listfigurename{Liste des Figures}
\fi
\ifdefined\listtablename
  \renewcommand*\listtablename{Liste des Tables}
\else
  \newcommand\listtablename{Liste des Tables}
\fi
\ifdefined\figurename
  \renewcommand*\figurename{Figure}
\else
  \newcommand\figurename{Figure}
\fi
\ifdefined\tablename
  \renewcommand*\tablename{Tableau}
\else
  \newcommand\tablename{Tableau}
\fi
}
\@ifpackageloaded{float}{}{\usepackage{float}}
\floatstyle{ruled}
\@ifundefined{c@chapter}{\newfloat{codelisting}{h}{lop}}{\newfloat{codelisting}{h}{lop}[chapter]}
\floatname{codelisting}{Listing}
\newcommand*\listoflistings{\listof{codelisting}{Liste des Listings}}
\makeatother
\makeatletter
\@ifpackageloaded{caption}{}{\usepackage{caption}}
\@ifpackageloaded{subcaption}{}{\usepackage{subcaption}}
\makeatother
\makeatletter
\@ifpackageloaded{tcolorbox}{}{\usepackage[many]{tcolorbox}}
\makeatother
\makeatletter
\@ifundefined{shadecolor}{\definecolor{shadecolor}{rgb}{.97, .97, .97}}
\makeatother
\makeatletter
\makeatother
\makeatletter
\@ifpackageloaded{fontawesome5}{}{\usepackage{fontawesome5}}
\makeatother
\ifLuaTeX
\usepackage[bidi=basic]{babel}
\else
\usepackage[bidi=default]{babel}
\fi
\babelprovide[main,import]{french}
% get rid of language-specific shorthands (see #6817):
\let\LanguageShortHands\languageshorthands
\def\languageshorthands#1{}
\ifLuaTeX
  \usepackage{selnolig}  % disable illegal ligatures
\fi
\IfFileExists{bookmark.sty}{\usepackage{bookmark}}{\usepackage{hyperref}}
\IfFileExists{xurl.sty}{\usepackage{xurl}}{} % add URL line breaks if available
\urlstyle{same} % disable monospaced font for URLs
\hypersetup{
  pdftitle={Listes (Cours)},
  pdflang={fr},
  colorlinks=true,
  linkcolor={blue},
  filecolor={Maroon},
  citecolor={Blue},
  urlcolor={Blue},
  pdfcreator={LaTeX via pandoc}}

\title{Listes (Cours)}
\usepackage{etoolbox}
\makeatletter
\providecommand{\subtitle}[1]{% add subtitle to \maketitle
  \apptocmd{\@title}{\par {\large #1 \par}}{}{}
}
\makeatother
\subtitle{S2 - Structures de données}
\author{}
\date{}

\begin{document}
\maketitle
\lhead{Spécialité NSI} \rhead{Terminale} \chead{} \cfoot{} \lfoot{Lycée \'Emile Duclaux} \rfoot{Page \thepage/\pageref{LastPage}} \renewcommand{\headrulewidth}{0pt} \renewcommand{\footrulewidth}{0pt} \thispagestyle{fancy} \vspace{-2cm}

\ifdefined\Shaded\renewenvironment{Shaded}{\begin{tcolorbox}[frame hidden, boxrule=0pt, breakable, borderline west={3pt}{0pt}{shadecolor}, enhanced, interior hidden, sharp corners]}{\end{tcolorbox}}\fi

\begin{tcolorbox}[enhanced jigsaw, arc=.35mm, title=\textcolor{quarto-callout-warning-color}{\faExclamationTriangle}\hspace{0.5em}{Attention !}, colframe=quarto-callout-warning-color-frame, breakable, toptitle=1mm, opacitybacktitle=0.6, colbacktitle=quarto-callout-warning-color!10!white, coltitle=black, leftrule=.75mm, left=2mm, bottomrule=.15mm, titlerule=0mm, rightrule=.15mm, bottomtitle=1mm, opacityback=0, colback=white, toprule=.15mm]

En Première, nous avons utilisé le type de données \texttt{list} de
Python pour représenter des \textbf{tableaux} de d'éléments de même
type. Le vocabulaire propre à Python peut induire en erreur et amener à
penser que le type ``liste'' est déjà connu. La structure \texttt{list}
de Python réalise en fait l'implémentation du type abstrait de données
``tableau dynamique'' et doit être laissée de côté, malgré l'utilisation
du même vocabulaire.

Cela ne nous empêchera pas d'implémenter le type abstrait de données
liste en utilisant des structures de type \texttt{list} en Python.

\end{tcolorbox}

\hypertarget{du-point-de-vue-utilisateur-interface}{%
\subsection{1. Du point de vue utilisateur :
interface}\label{du-point-de-vue-utilisateur-interface}}

\begin{tcolorbox}[enhanced jigsaw, arc=.35mm, title=\textcolor{quarto-callout-tip-color}{\faLightbulb}\hspace{0.5em}{Définition}, colframe=quarto-callout-tip-color-frame, breakable, toptitle=1mm, opacitybacktitle=0.6, colbacktitle=quarto-callout-tip-color!10!white, coltitle=black, leftrule=.75mm, left=2mm, bottomrule=.15mm, titlerule=0mm, rightrule=.15mm, bottomtitle=1mm, opacityback=0, colback=white, toprule=.15mm]

Une liste est une structure de données qui permet de stocker des données
et d'y accéder directement.

C'est un type abstrait de données :

\begin{itemize}
\tightlist
\item
  linéaire : les données sont stockées dans une structure
  unidimensionnelle ;
\item
  indexé : chaque donnée est associée à une valeur ;
\item
  ordonné : les données sont présentées les unes après les autres.
\end{itemize}

\end{tcolorbox}

Une liste est une collection finie de données. On appelle \textbf{tête}
le premier élément de la liste et \textbf{queue} la liste privée de son
premier élément. Il est seulement possible d'ajouter et de lire une
donnée en tête de la liste.

L'\textbf{interface} minimale permettant de définir le type abstrait de
données ``liste'' comporte cinq fonctions, qui sont appelées
\textbf{primitives} :

\begin{itemize}
\tightlist
\item
  \texttt{creer()}, qui crée une liste vide ;
\item
  \texttt{ajouter(element,\ liste)}, qui ajoute un élément en tête de
  liste ; ces deux première primitives peuvent parfois se regrouper en
  une seule ;
\item
  \texttt{tete(liste)}, qui renvoie la valeur de l'élément en tête de
  liste ;
\item
  \texttt{queue(liste)}, qui renvoie la liste privée de son premier
  élément ;
\item
  \texttt{est\_vide(liste)}, qui renvoie vrai si la liste est vide, faux
  sinon.
\end{itemize}

Ce type abstrait de données est non mutable (il n'y a pas de primitive
permettant de modifier la valeur d'un élément de la liste).

\textbf{Remarque} : on peut selon les besoins ajouter d'autres fonctions
permettant par exemple de renvoyer la longueur d'une liste, de
rechercher un élément ou d'accéder au ième élément \ldots{}

\begin{tcolorbox}[enhanced jigsaw, arc=.35mm, title=\textcolor{quarto-callout-caution-color}{\faFire}\hspace{0.5em}{Exemple}, colframe=quarto-callout-caution-color-frame, breakable, toptitle=1mm, opacitybacktitle=0.6, colbacktitle=quarto-callout-caution-color!10!white, coltitle=black, leftrule=.75mm, left=2mm, bottomrule=.15mm, titlerule=0mm, rightrule=.15mm, bottomtitle=1mm, opacityback=0, colback=white, toprule=.15mm]

Supposons implémenté le type abstrait \textbf{liste}. Nous disposons
d'une interface composée des cinq primitives décrites ci-dessus. On
exécute le code suivant ligne par ligne :

\begin{Shaded}
\begin{Highlighting}[numbers=left,,]
\NormalTok{L }\OperatorTok{=}\NormalTok{ creer()}
\NormalTok{est\_vide(L)}
\NormalTok{L1 }\OperatorTok{=}\NormalTok{ ajouter(}\DecValTok{12}\NormalTok{, L)}
\NormalTok{est\_vide(L1)}
\NormalTok{L1 }\OperatorTok{=}\NormalTok{ ajouter(}\DecValTok{15}\NormalTok{, L1)}
\NormalTok{L1 }\OperatorTok{=}\NormalTok{ ajouter(}\DecValTok{1}\NormalTok{, ajouter(}\DecValTok{11}\NormalTok{,L1))}
\NormalTok{tete(L1)}
\NormalTok{L2 }\OperatorTok{=}\NormalTok{ queue(L1)}
\end{Highlighting}
\end{Shaded}

\begin{itemize}
\tightlist
\item
  La ligne 1 créée une liste vide \texttt{L} ;
\item
  La ligne 2 affiche \texttt{True} car la liste \texttt{L} est vide ;
\item
  Après exécution de la ligne 3, la liste \texttt{L1} contient l'élément
  unique 12 ;
\item
  La ligne 4 affiche \texttt{False} car la liste \texttt{L1} n'est pas
  vide ;
\item
  Après exécution de la ligne 5, la liste \texttt{L1} contient les
  éléments 12 et 15 ;
\item
  La ligne 6 montre que l'on peut \textbf{composer} les ajouts pour
  ajouter en une seule fois plusieurs éléments. Après exécution de la
  ligne 6, la liste \texttt{L1} contient les éléments 12, 15, 11 et 1 ;
\item
  La ligne 7 affiche 1 : c'est la tête de la liste (il s'agit du dernier
  élément ajouté) ;
\item
  La ligne 8 définit une liste \texttt{L2} égale à la queue de la liste
  \texttt{L1}. \texttt{L2} contient donc les éléments 12, 15 et 11.
\end{itemize}

\end{tcolorbox}

\hypertarget{du-point-de-vue-concepteur-impluxe9mentations}{%
\subsection{2. Du point de vue concepteur :
implémentation(s)}\label{du-point-de-vue-concepteur-impluxe9mentations}}

Nous allons implémenter le type abstrait ``liste'' en Python de deux
façons différentes.

\hypertarget{impluxe9mentation-avec-des-tuples}{%
\subsubsection{Implémentation avec des
tuples}\label{impluxe9mentation-avec-des-tuples}}

Nous allons ici utiliser des tuples et la programmation fonctionnelle
(rappel : fonctions pures, pas d'affectations, pas de boucles).

\begin{Shaded}
\begin{Highlighting}[]
\CommentTok{""" Implémentation du type abstrait "liste" avec des tuples"""}


\KeywordTok{def}\NormalTok{ creer() }\OperatorTok{{-}\textgreater{}} \BuiltInTok{tuple}\NormalTok{:}
    \CommentTok{"""Retourne une liste vide"""}
    \ControlFlowTok{return}\NormalTok{ ()}


\KeywordTok{def}\NormalTok{ ajouter(element: }\BuiltInTok{all}\NormalTok{, liste: }\BuiltInTok{tuple}\NormalTok{) }\OperatorTok{{-}\textgreater{}} \BuiltInTok{tuple}\NormalTok{:}
    \CommentTok{"""Retourne la liste avec l\textquotesingle{}élément ajouté en tête de liste"""}
    \ControlFlowTok{return}\NormalTok{ (element, liste)}


\KeywordTok{def}\NormalTok{ est\_vide(liste: }\BuiltInTok{tuple}\NormalTok{) }\OperatorTok{{-}\textgreater{}} \BuiltInTok{bool}\NormalTok{:}
    \CommentTok{"""Retourne True si la liste est vide et False sinon"""}
    \ControlFlowTok{return}\NormalTok{ liste }\OperatorTok{==}\NormalTok{ ()}


\KeywordTok{def}\NormalTok{ tete(liste: }\BuiltInTok{tuple}\NormalTok{) }\OperatorTok{{-}\textgreater{}} \BuiltInTok{all}\NormalTok{:}
    \CommentTok{"""Retourne la tête de la liste"""}
    \ControlFlowTok{assert} \KeywordTok{not}\NormalTok{ est\_vide(liste), }\StringTok{"Erreur : liste vide"}
    \ControlFlowTok{return}\NormalTok{ liste[}\DecValTok{0}\NormalTok{]}


\KeywordTok{def}\NormalTok{ queue(liste: }\BuiltInTok{tuple}\NormalTok{) }\OperatorTok{{-}\textgreater{}} \BuiltInTok{tuple}\NormalTok{:}
    \CommentTok{"""Retourne la queue de la liste"""}
    \ControlFlowTok{assert} \KeywordTok{not}\NormalTok{ est\_vide(liste), }\StringTok{"Erreur : liste vide"}
    \ControlFlowTok{return}\NormalTok{ liste[}\DecValTok{1}\NormalTok{]}


\KeywordTok{def}\NormalTok{ longueur(liste: }\BuiltInTok{tuple}\NormalTok{) }\OperatorTok{{-}\textgreater{}} \BuiltInTok{int}\NormalTok{:}
    \CommentTok{"""Retourne le nombre d\textquotesingle{}éléments de la liste"""}
    \ControlFlowTok{if}\NormalTok{ est\_vide(liste):}
        \ControlFlowTok{return} \DecValTok{0}
    \ControlFlowTok{else}\NormalTok{:}
        \ControlFlowTok{return} \DecValTok{1} \OperatorTok{+}\NormalTok{ longueur(liste[}\DecValTok{1}\NormalTok{])}
\end{Highlighting}
\end{Shaded}

Avec cette représentation une liste est toujours un tuple à deux
éléments dont le premier est la tête de la liste (le dernier élément
ajouté) et le deuxième est la queue (c'est donc une liste).

\[L=(1, (11, (15, (12, ()))))\]

On remarquera que la fonction \texttt{longueur} est codée sans boucle,
mais de façon récursive, afin de correspondre au \textbf{paradigme
fonctionnel}.

\hypertarget{impluxe9mentation-en-poo}{%
\subsubsection{Implémentation en POO}\label{impluxe9mentation-en-poo}}

Conformément au programme, on se limite à une version \textbf{naïve} de
la POO. On pourra à titre d'exercice reprendre cette implémentation en
respectant les règles plus strictes édictées dans
\href{../langagesProgr/POO_complements.qmd}{les compléments de cours sur
la POO}.

On définit ci-dessous une \textbf{liste chaînée} : chaque chaînon est
constitué de l'élément qui fait partie de la liste et de la référence à
l'élément suivant. C'est la classe \texttt{Chainon} qui implémente cette
structure. L'objet \texttt{Liste} est défini à partir de son premier
élément (tête) et ses primitives sont définies sous forme de
\textbf{méthodes}.

\begin{Shaded}
\begin{Highlighting}[]
\CommentTok{"""Implémentation du type abstrait liste en POO"""}


\KeywordTok{class}\NormalTok{ Chainon:}
    \KeywordTok{def} \FunctionTok{\_\_init\_\_}\NormalTok{(}\VariableTok{self}\NormalTok{, element}\OperatorTok{=}\VariableTok{None}\NormalTok{, suivant}\OperatorTok{=}\VariableTok{None}\NormalTok{):}
        \CommentTok{"""element est la valeur du chainon et suivant est le chainon qui suit"""}
        \VariableTok{self}\NormalTok{.element }\OperatorTok{=}\NormalTok{ element}
        \VariableTok{self}\NormalTok{.suivant }\OperatorTok{=}\NormalTok{ suivant}


\KeywordTok{class}\NormalTok{ Liste:}
    \KeywordTok{def} \FunctionTok{\_\_init\_\_}\NormalTok{(}\VariableTok{self}\NormalTok{):}
        \CommentTok{"""Crée une liste vide"""}
        \VariableTok{self}\NormalTok{.head }\OperatorTok{=}\NormalTok{ Chainon()}

    \KeywordTok{def}\NormalTok{ est\_vide(}\VariableTok{self}\NormalTok{) }\OperatorTok{{-}\textgreater{}} \BuiltInTok{bool}\NormalTok{:}
        \CommentTok{"""Retourne True si la liste est vide et False sinon"""}
        \ControlFlowTok{return} \VariableTok{self}\NormalTok{.head.element }\KeywordTok{is} \VariableTok{None}

    \KeywordTok{def}\NormalTok{ ajouter(}\VariableTok{self}\NormalTok{, element):}
        \CommentTok{"""Ajoute element en tête de la liste"""}
        \VariableTok{self}\NormalTok{.head }\OperatorTok{=}\NormalTok{ Chainon(element, }\VariableTok{self}\NormalTok{.head)}

    \KeywordTok{def}\NormalTok{ tete(}\VariableTok{self}\NormalTok{):}
        \CommentTok{"""Retourne la valeur de la tête de la liste"""}
        \ControlFlowTok{return} \VariableTok{self}\NormalTok{.head.element}

    \KeywordTok{def}\NormalTok{ queue(}\VariableTok{self}\NormalTok{):}
        \CommentTok{"""Retourne la queue de la liste, c.{-}à{-}d. la liste privée de sa tête"""}
\NormalTok{        new\_liste }\OperatorTok{=}\NormalTok{ Liste()}
\NormalTok{        new\_liste.head }\OperatorTok{=} \VariableTok{self}\NormalTok{.head.suivant}
        \ControlFlowTok{return}\NormalTok{ new\_liste}

    \KeywordTok{def}\NormalTok{ longueur(}\VariableTok{self}\NormalTok{):}
        \CommentTok{"""Retourne la longueur de la liste"""}
        \BuiltInTok{long} \OperatorTok{=} \DecValTok{0}
\NormalTok{        chainon }\OperatorTok{=} \VariableTok{self}\NormalTok{.head}
        \ControlFlowTok{while}\NormalTok{ chainon.element }\KeywordTok{is} \KeywordTok{not} \VariableTok{None}\NormalTok{:}
\NormalTok{            chainon }\OperatorTok{=}\NormalTok{ chainon.suivant}
            \BuiltInTok{long} \OperatorTok{=} \BuiltInTok{long} \OperatorTok{+} \DecValTok{1}
        \ControlFlowTok{return} \BuiltInTok{long}
\end{Highlighting}
\end{Shaded}

On peut améliorer l'implémentation en redéfinissant les méthodes
spéciales \texttt{\_\_len\_\_} (qui permettra de taper \texttt{len(L)}
au lieu de \texttt{L.longueur()}) et \texttt{\_\_str\_\_} (qui permettra
d'utiliser l'instruction \texttt{print(L)}).

\begin{Shaded}
\begin{Highlighting}[]
    \KeywordTok{def} \FunctionTok{\_\_len\_\_}\NormalTok{(}\VariableTok{self}\NormalTok{):}
        \ControlFlowTok{return} \VariableTok{self}\NormalTok{.longueur()}

    \KeywordTok{def} \FunctionTok{\_\_str\_\_}\NormalTok{(}\VariableTok{self}\NormalTok{):}
\NormalTok{        rep }\OperatorTok{=} \StringTok{""}
\NormalTok{        chainon }\OperatorTok{=} \VariableTok{self}\NormalTok{.head}
        \ControlFlowTok{while}\NormalTok{ chainon.element }\KeywordTok{is} \KeywordTok{not} \VariableTok{None}\NormalTok{:}
\NormalTok{            rep }\OperatorTok{=}\NormalTok{ rep }\OperatorTok{+} \BuiltInTok{str}\NormalTok{(chainon.element) }\OperatorTok{+} \StringTok{" {-}{-}\textgreater{} "}
\NormalTok{            chainon }\OperatorTok{=}\NormalTok{ chainon.suivant}
        \ControlFlowTok{return}\NormalTok{ rep[:}\OperatorTok{{-}}\DecValTok{4}\NormalTok{]}
\end{Highlighting}
\end{Shaded}

\begin{Shaded}
\begin{Highlighting}[]
\OperatorTok{\textgreater{}\textgreater{}\textgreater{}}\NormalTok{ L}\OperatorTok{=}\NormalTok{Liste()}
\OperatorTok{\textgreater{}\textgreater{}\textgreater{}}\NormalTok{ L.ajouter(}\DecValTok{11}\NormalTok{)}
\OperatorTok{\textgreater{}\textgreater{}\textgreater{}}\NormalTok{ L.ajouter(}\DecValTok{12}\NormalTok{)}
\OperatorTok{\textgreater{}\textgreater{}\textgreater{}}\NormalTok{ L.ajouter(}\DecValTok{13}\NormalTok{)}
\OperatorTok{\textgreater{}\textgreater{}\textgreater{}} \BuiltInTok{print}\NormalTok{(L)}
\DecValTok{13} \OperatorTok{{-}{-}\textgreater{}} \DecValTok{12} \OperatorTok{{-}{-}\textgreater{}} \DecValTok{11} 
\end{Highlighting}
\end{Shaded}




\end{document}
