% Options for packages loaded elsewhere
\PassOptionsToPackage{unicode}{hyperref}
\PassOptionsToPackage{hyphens}{url}
\PassOptionsToPackage{dvipsnames,svgnames,x11names}{xcolor}
%
\documentclass[
  a4paper,
  DIV=11,
  numbers=noendperiod]{scrartcl}

\usepackage{amsmath,amssymb}
\usepackage{lmodern}
\usepackage{iftex}
\ifPDFTeX
  \usepackage[T1]{fontenc}
  \usepackage[utf8]{inputenc}
  \usepackage{textcomp} % provide euro and other symbols
\else % if luatex or xetex
  \usepackage{unicode-math}
  \defaultfontfeatures{Scale=MatchLowercase}
  \defaultfontfeatures[\rmfamily]{Ligatures=TeX,Scale=1}
\fi
% Use upquote if available, for straight quotes in verbatim environments
\IfFileExists{upquote.sty}{\usepackage{upquote}}{}
\IfFileExists{microtype.sty}{% use microtype if available
  \usepackage[]{microtype}
  \UseMicrotypeSet[protrusion]{basicmath} % disable protrusion for tt fonts
}{}
\makeatletter
\@ifundefined{KOMAClassName}{% if non-KOMA class
  \IfFileExists{parskip.sty}{%
    \usepackage{parskip}
  }{% else
    \setlength{\parindent}{0pt}
    \setlength{\parskip}{6pt plus 2pt minus 1pt}}
}{% if KOMA class
  \KOMAoptions{parskip=half}}
\makeatother
\usepackage{xcolor}
\usepackage[top=20mm,bottom=20mm,left=20mm,right=20mm,heightrounded]{geometry}
\setlength{\emergencystretch}{3em} % prevent overfull lines
\setcounter{secnumdepth}{-\maxdimen} % remove section numbering
% Make \paragraph and \subparagraph free-standing
\ifx\paragraph\undefined\else
  \let\oldparagraph\paragraph
  \renewcommand{\paragraph}[1]{\oldparagraph{#1}\mbox{}}
\fi
\ifx\subparagraph\undefined\else
  \let\oldsubparagraph\subparagraph
  \renewcommand{\subparagraph}[1]{\oldsubparagraph{#1}\mbox{}}
\fi


\providecommand{\tightlist}{%
  \setlength{\itemsep}{0pt}\setlength{\parskip}{0pt}}\usepackage{longtable,booktabs,array}
\usepackage{calc} % for calculating minipage widths
% Correct order of tables after \paragraph or \subparagraph
\usepackage{etoolbox}
\makeatletter
\patchcmd\longtable{\par}{\if@noskipsec\mbox{}\fi\par}{}{}
\makeatother
% Allow footnotes in longtable head/foot
\IfFileExists{footnotehyper.sty}{\usepackage{footnotehyper}}{\usepackage{footnote}}
\makesavenoteenv{longtable}
\usepackage{graphicx}
\makeatletter
\def\maxwidth{\ifdim\Gin@nat@width>\linewidth\linewidth\else\Gin@nat@width\fi}
\def\maxheight{\ifdim\Gin@nat@height>\textheight\textheight\else\Gin@nat@height\fi}
\makeatother
% Scale images if necessary, so that they will not overflow the page
% margins by default, and it is still possible to overwrite the defaults
% using explicit options in \includegraphics[width, height, ...]{}
\setkeys{Gin}{width=\maxwidth,height=\maxheight,keepaspectratio}
% Set default figure placement to htbp
\makeatletter
\def\fps@figure{htbp}
\makeatother

\usepackage{fancyhdr} \pagestyle{fancy} \usepackage{lastpage}
\KOMAoption{captions}{tablesignature}
\makeatletter
\@ifpackageloaded{tcolorbox}{}{\usepackage[many]{tcolorbox}}
\@ifpackageloaded{fontawesome5}{}{\usepackage{fontawesome5}}
\definecolor{quarto-callout-color}{HTML}{909090}
\definecolor{quarto-callout-note-color}{HTML}{0758E5}
\definecolor{quarto-callout-important-color}{HTML}{CC1914}
\definecolor{quarto-callout-warning-color}{HTML}{EB9113}
\definecolor{quarto-callout-tip-color}{HTML}{00A047}
\definecolor{quarto-callout-caution-color}{HTML}{FC5300}
\definecolor{quarto-callout-color-frame}{HTML}{acacac}
\definecolor{quarto-callout-note-color-frame}{HTML}{4582ec}
\definecolor{quarto-callout-important-color-frame}{HTML}{d9534f}
\definecolor{quarto-callout-warning-color-frame}{HTML}{f0ad4e}
\definecolor{quarto-callout-tip-color-frame}{HTML}{02b875}
\definecolor{quarto-callout-caution-color-frame}{HTML}{fd7e14}
\makeatother
\makeatletter
\makeatother
\makeatletter
\makeatother
\makeatletter
\@ifpackageloaded{caption}{}{\usepackage{caption}}
\AtBeginDocument{%
\ifdefined\contentsname
  \renewcommand*\contentsname{Table des matières}
\else
  \newcommand\contentsname{Table des matières}
\fi
\ifdefined\listfigurename
  \renewcommand*\listfigurename{Liste des Figures}
\else
  \newcommand\listfigurename{Liste des Figures}
\fi
\ifdefined\listtablename
  \renewcommand*\listtablename{Liste des Tables}
\else
  \newcommand\listtablename{Liste des Tables}
\fi
\ifdefined\figurename
  \renewcommand*\figurename{Figure}
\else
  \newcommand\figurename{Figure}
\fi
\ifdefined\tablename
  \renewcommand*\tablename{Tableau}
\else
  \newcommand\tablename{Tableau}
\fi
}
\@ifpackageloaded{float}{}{\usepackage{float}}
\floatstyle{ruled}
\@ifundefined{c@chapter}{\newfloat{codelisting}{h}{lop}}{\newfloat{codelisting}{h}{lop}[chapter]}
\floatname{codelisting}{Listing}
\newcommand*\listoflistings{\listof{codelisting}{Liste des Listings}}
\makeatother
\makeatletter
\@ifpackageloaded{caption}{}{\usepackage{caption}}
\@ifpackageloaded{subcaption}{}{\usepackage{subcaption}}
\makeatother
\makeatletter
\@ifpackageloaded{tcolorbox}{}{\usepackage[many]{tcolorbox}}
\makeatother
\makeatletter
\@ifundefined{shadecolor}{\definecolor{shadecolor}{rgb}{.97, .97, .97}}
\makeatother
\makeatletter
\makeatother
\ifLuaTeX
\usepackage[bidi=basic]{babel}
\else
\usepackage[bidi=default]{babel}
\fi
\babelprovide[main,import]{french}
% get rid of language-specific shorthands (see #6817):
\let\LanguageShortHands\languageshorthands
\def\languageshorthands#1{}
\ifLuaTeX
  \usepackage{selnolig}  % disable illegal ligatures
\fi
\IfFileExists{bookmark.sty}{\usepackage{bookmark}}{\usepackage{hyperref}}
\IfFileExists{xurl.sty}{\usepackage{xurl}}{} % add URL line breaks if available
\urlstyle{same} % disable monospaced font for URLs
\hypersetup{
  pdftitle={Interfaces et implémentations (Cours)},
  pdflang={fr},
  colorlinks=true,
  linkcolor={blue},
  filecolor={Maroon},
  citecolor={Blue},
  urlcolor={Blue},
  pdfcreator={LaTeX via pandoc}}

\title{Interfaces et implémentations (Cours)}
\usepackage{etoolbox}
\makeatletter
\providecommand{\subtitle}[1]{% add subtitle to \maketitle
  \apptocmd{\@title}{\par {\large #1 \par}}{}{}
}
\makeatother
\subtitle{S2 - Structures de données}
\author{}
\date{}

\begin{document}
\maketitle
\lhead{Spécialité NSI} \rhead{Terminale} \chead{} \cfoot{} \lfoot{Lycée \'Emile Duclaux} \rfoot{Page \thepage/\pageref{LastPage}} \renewcommand{\headrulewidth}{0pt} \renewcommand{\footrulewidth}{0pt} \thispagestyle{fancy} \vspace{-2cm}

\ifdefined\Shaded\renewenvironment{Shaded}{\begin{tcolorbox}[frame hidden, breakable, interior hidden, sharp corners, borderline west={3pt}{0pt}{shadecolor}, enhanced, boxrule=0pt]}{\end{tcolorbox}}\fi

Nous allons dans ce chapitre nous intéresser aux \textbf{structures de
données} comme les \textbf{listes}, les \textbf{piles}, les
\textbf{files} et les \textbf{dictionnaires}.

Les dictionnaires ont déjà été rencontrés en première. Plus précisément,
c'est l'\textbf{implémentation} du \textbf{type abstrait}
``dictionnaire'' en Python qui a été utilisée.

Toutes ces structures de données sont des \textbf{types abstraits} qui
doivent être définis dans un langage de programmation pour pouvoir être
utilisés.

\begin{tcolorbox}[enhanced jigsaw, bottomrule=.15mm, toptitle=1mm, colbacktitle=quarto-callout-tip-color!10!white, toprule=.15mm, bottomtitle=1mm, colback=white, rightrule=.15mm, leftrule=.75mm, left=2mm, titlerule=0mm, breakable, title=\textcolor{quarto-callout-tip-color}{\faLightbulb}\hspace{0.5em}{Définition}, arc=.35mm, opacitybacktitle=0.6, opacityback=0, coltitle=black]

\textbf{Implémenter} un algorithme, c'est le traduire dans un langage de
programmation.

\end{tcolorbox}

Pour un type abstrait donné, disons les dictionnaires, plusieurs
implémentations sont possibles. Elles peuvent se différencier par leur
rapidité d'exécution ou leur capacité à travailler avec des données de
grande taille par exemple.

Une fois implémenté, le type abstrait doit pouvoir être \textbf{utilisé}
par un programmeur (utilisateur). Cet utilisateur n'a pas besoin de
connaître comment le type ``dictionnaire'' a été concrètement implémenté
dans le langage de programmation qu'il utilise. Par contre, il faut
qu'il connaisse précisément les actions qu'i peut réaliser sur ce type
de données.

Par exemple, un dictionnaire associe à un ensemble de clés un ensemble
de valeurs (ce sont les données) et permet notamment les opérations :

\begin{itemize}
\tightlist
\item
  d'ajout d'un couple clé-valeur ;
\item
  de suppression d'une clé, et donc de la valeur correspondante ;
\item
  de modification de la valeur associée à une clé et ;
\item
  de recherche de la valeur correspondant à une clé.
\end{itemize}

L'ensemble des fonctions (méthodes) associées à un type abstrait
constitue son \textbf{interface}. Ces fonctions et leurs spécifications,
permettent à l'utilisateur d'utiliser le type abstrait dans son
programme.

Quand on utilise une bibliothèque contenant l'implémentation de
structures de données, l'ensemble de ces spécifications est nommée API
(Application Programming Interface, Interface de Programmation en
français).

\begin{tcolorbox}[enhanced jigsaw, bottomrule=.15mm, toptitle=1mm, colbacktitle=quarto-callout-tip-color!10!white, toprule=.15mm, bottomtitle=1mm, colback=white, rightrule=.15mm, leftrule=.75mm, left=2mm, titlerule=0mm, breakable, title=\textcolor{quarto-callout-tip-color}{\faLightbulb}\hspace{0.5em}{Définition (d'après Wikipedia)}, arc=.35mm, opacitybacktitle=0.6, opacityback=0, coltitle=black]

En informatique, une \textbf{interface de programmation} (souvent
désignée par le terme API pour Application Programming Interface) est un
ensemble normalisé de classes, de méthodes, de fonctions et de
constantes qui sert de \textbf{façade} par laquelle un logiciel offre
des services à d'autres logiciels. Elle est offerte par une bibliothèque
logicielle ou un service web, le plus souvent accompagnée d'une
description qui \textbf{spécifie} comment des \textbf{programmes
consommateurs} peuvent se servir des fonctionnalités du
\textbf{programme fournisseur}.

\end{tcolorbox}

L'usage des bibliothèques permet à chaque programmeur d'ajouter des
structures réalisant des types abstraits de données, cette
implémentation n'étant pas nécessairement connue de l'utilisateur de la
structure. Cette méthode de conception logicielle, utilisant
l'\textbf{encapsulation}, permet à la fois :

\begin{itemize}
\tightlist
\item
  le développement séparé de l'application et de l'implémentation de la
  structure ;
\item
  la modification de l'implémentation sans modification de ses
  utilisations (on préserve l'interface) ;
\item
  l'utilisation facile de l'implémentation de la structure dans des
  programmes à venir ;
\item
  la limitation des erreurs ;
\item
  l'ajout :

  \begin{itemize}
  \tightlist
  \item
    de vérifications sous forme d'assertions ;
  \item
    d'outils de correction des problèmes de programmation ;
  \end{itemize}
\item
  une meilleure lisibilité du code de l'application.
\end{itemize}

Dans la suite de ce chapitre, nous allons étudier successivement les
listes, les piles, les files et les dictionnaires du point de vue du
concepteur (implémentation) et de l'utilisateur (interface).



\end{document}
