% Options for packages loaded elsewhere
\PassOptionsToPackage{unicode}{hyperref}
\PassOptionsToPackage{hyphens}{url}
\PassOptionsToPackage{dvipsnames,svgnames,x11names}{xcolor}
%
\documentclass[
  letterpaper,
  DIV=11,
  numbers=noendperiod]{scrartcl}

\usepackage{amsmath,amssymb}
\usepackage{iftex}
\ifPDFTeX
  \usepackage[T1]{fontenc}
  \usepackage[utf8]{inputenc}
  \usepackage{textcomp} % provide euro and other symbols
\else % if luatex or xetex
  \usepackage{unicode-math}
  \defaultfontfeatures{Scale=MatchLowercase}
  \defaultfontfeatures[\rmfamily]{Ligatures=TeX,Scale=1}
\fi
\usepackage{lmodern}
\ifPDFTeX\else  
    % xetex/luatex font selection
\fi
% Use upquote if available, for straight quotes in verbatim environments
\IfFileExists{upquote.sty}{\usepackage{upquote}}{}
\IfFileExists{microtype.sty}{% use microtype if available
  \usepackage[]{microtype}
  \UseMicrotypeSet[protrusion]{basicmath} % disable protrusion for tt fonts
}{}
\makeatletter
\@ifundefined{KOMAClassName}{% if non-KOMA class
  \IfFileExists{parskip.sty}{%
    \usepackage{parskip}
  }{% else
    \setlength{\parindent}{0pt}
    \setlength{\parskip}{6pt plus 2pt minus 1pt}}
}{% if KOMA class
  \KOMAoptions{parskip=half}}
\makeatother
\usepackage{xcolor}
\usepackage[top=20mm,bottom=20mm,left=20mm,right=20mm,heightrounded]{geometry}
\setlength{\emergencystretch}{3em} % prevent overfull lines
\setcounter{secnumdepth}{-\maxdimen} % remove section numbering
% Make \paragraph and \subparagraph free-standing
\ifx\paragraph\undefined\else
  \let\oldparagraph\paragraph
  \renewcommand{\paragraph}[1]{\oldparagraph{#1}\mbox{}}
\fi
\ifx\subparagraph\undefined\else
  \let\oldsubparagraph\subparagraph
  \renewcommand{\subparagraph}[1]{\oldsubparagraph{#1}\mbox{}}
\fi

\usepackage{color}
\usepackage{fancyvrb}
\newcommand{\VerbBar}{|}
\newcommand{\VERB}{\Verb[commandchars=\\\{\}]}
\DefineVerbatimEnvironment{Highlighting}{Verbatim}{commandchars=\\\{\}}
% Add ',fontsize=\small' for more characters per line
\usepackage{framed}
\definecolor{shadecolor}{RGB}{241,243,245}
\newenvironment{Shaded}{\begin{snugshade}}{\end{snugshade}}
\newcommand{\AlertTok}[1]{\textcolor[rgb]{0.68,0.00,0.00}{#1}}
\newcommand{\AnnotationTok}[1]{\textcolor[rgb]{0.37,0.37,0.37}{#1}}
\newcommand{\AttributeTok}[1]{\textcolor[rgb]{0.40,0.45,0.13}{#1}}
\newcommand{\BaseNTok}[1]{\textcolor[rgb]{0.68,0.00,0.00}{#1}}
\newcommand{\BuiltInTok}[1]{\textcolor[rgb]{0.00,0.23,0.31}{#1}}
\newcommand{\CharTok}[1]{\textcolor[rgb]{0.13,0.47,0.30}{#1}}
\newcommand{\CommentTok}[1]{\textcolor[rgb]{0.37,0.37,0.37}{#1}}
\newcommand{\CommentVarTok}[1]{\textcolor[rgb]{0.37,0.37,0.37}{\textit{#1}}}
\newcommand{\ConstantTok}[1]{\textcolor[rgb]{0.56,0.35,0.01}{#1}}
\newcommand{\ControlFlowTok}[1]{\textcolor[rgb]{0.00,0.23,0.31}{#1}}
\newcommand{\DataTypeTok}[1]{\textcolor[rgb]{0.68,0.00,0.00}{#1}}
\newcommand{\DecValTok}[1]{\textcolor[rgb]{0.68,0.00,0.00}{#1}}
\newcommand{\DocumentationTok}[1]{\textcolor[rgb]{0.37,0.37,0.37}{\textit{#1}}}
\newcommand{\ErrorTok}[1]{\textcolor[rgb]{0.68,0.00,0.00}{#1}}
\newcommand{\ExtensionTok}[1]{\textcolor[rgb]{0.00,0.23,0.31}{#1}}
\newcommand{\FloatTok}[1]{\textcolor[rgb]{0.68,0.00,0.00}{#1}}
\newcommand{\FunctionTok}[1]{\textcolor[rgb]{0.28,0.35,0.67}{#1}}
\newcommand{\ImportTok}[1]{\textcolor[rgb]{0.00,0.46,0.62}{#1}}
\newcommand{\InformationTok}[1]{\textcolor[rgb]{0.37,0.37,0.37}{#1}}
\newcommand{\KeywordTok}[1]{\textcolor[rgb]{0.00,0.23,0.31}{#1}}
\newcommand{\NormalTok}[1]{\textcolor[rgb]{0.00,0.23,0.31}{#1}}
\newcommand{\OperatorTok}[1]{\textcolor[rgb]{0.37,0.37,0.37}{#1}}
\newcommand{\OtherTok}[1]{\textcolor[rgb]{0.00,0.23,0.31}{#1}}
\newcommand{\PreprocessorTok}[1]{\textcolor[rgb]{0.68,0.00,0.00}{#1}}
\newcommand{\RegionMarkerTok}[1]{\textcolor[rgb]{0.00,0.23,0.31}{#1}}
\newcommand{\SpecialCharTok}[1]{\textcolor[rgb]{0.37,0.37,0.37}{#1}}
\newcommand{\SpecialStringTok}[1]{\textcolor[rgb]{0.13,0.47,0.30}{#1}}
\newcommand{\StringTok}[1]{\textcolor[rgb]{0.13,0.47,0.30}{#1}}
\newcommand{\VariableTok}[1]{\textcolor[rgb]{0.07,0.07,0.07}{#1}}
\newcommand{\VerbatimStringTok}[1]{\textcolor[rgb]{0.13,0.47,0.30}{#1}}
\newcommand{\WarningTok}[1]{\textcolor[rgb]{0.37,0.37,0.37}{\textit{#1}}}

\providecommand{\tightlist}{%
  \setlength{\itemsep}{0pt}\setlength{\parskip}{0pt}}\usepackage{longtable,booktabs,array}
\usepackage{calc} % for calculating minipage widths
% Correct order of tables after \paragraph or \subparagraph
\usepackage{etoolbox}
\makeatletter
\patchcmd\longtable{\par}{\if@noskipsec\mbox{}\fi\par}{}{}
\makeatother
% Allow footnotes in longtable head/foot
\IfFileExists{footnotehyper.sty}{\usepackage{footnotehyper}}{\usepackage{footnote}}
\makesavenoteenv{longtable}
\usepackage{graphicx}
\makeatletter
\def\maxwidth{\ifdim\Gin@nat@width>\linewidth\linewidth\else\Gin@nat@width\fi}
\def\maxheight{\ifdim\Gin@nat@height>\textheight\textheight\else\Gin@nat@height\fi}
\makeatother
% Scale images if necessary, so that they will not overflow the page
% margins by default, and it is still possible to overwrite the defaults
% using explicit options in \includegraphics[width, height, ...]{}
\setkeys{Gin}{width=\maxwidth,height=\maxheight,keepaspectratio}
% Set default figure placement to htbp
\makeatletter
\def\fps@figure{htbp}
\makeatother

\usepackage{fancyhdr} \pagestyle{fancy} \usepackage{lastpage}
\KOMAoption{captions}{tablesignature}
\makeatletter
\@ifpackageloaded{tcolorbox}{}{\usepackage[skins,breakable]{tcolorbox}}
\@ifpackageloaded{fontawesome5}{}{\usepackage{fontawesome5}}
\definecolor{quarto-callout-color}{HTML}{909090}
\definecolor{quarto-callout-note-color}{HTML}{0758E5}
\definecolor{quarto-callout-important-color}{HTML}{CC1914}
\definecolor{quarto-callout-warning-color}{HTML}{EB9113}
\definecolor{quarto-callout-tip-color}{HTML}{00A047}
\definecolor{quarto-callout-caution-color}{HTML}{FC5300}
\definecolor{quarto-callout-color-frame}{HTML}{acacac}
\definecolor{quarto-callout-note-color-frame}{HTML}{4582ec}
\definecolor{quarto-callout-important-color-frame}{HTML}{d9534f}
\definecolor{quarto-callout-warning-color-frame}{HTML}{f0ad4e}
\definecolor{quarto-callout-tip-color-frame}{HTML}{02b875}
\definecolor{quarto-callout-caution-color-frame}{HTML}{fd7e14}
\makeatother
\makeatletter
\makeatother
\makeatletter
\makeatother
\makeatletter
\@ifpackageloaded{caption}{}{\usepackage{caption}}
\AtBeginDocument{%
\ifdefined\contentsname
  \renewcommand*\contentsname{Table des matières}
\else
  \newcommand\contentsname{Table des matières}
\fi
\ifdefined\listfigurename
  \renewcommand*\listfigurename{Liste des Figures}
\else
  \newcommand\listfigurename{Liste des Figures}
\fi
\ifdefined\listtablename
  \renewcommand*\listtablename{Liste des Tables}
\else
  \newcommand\listtablename{Liste des Tables}
\fi
\ifdefined\figurename
  \renewcommand*\figurename{Figure}
\else
  \newcommand\figurename{Figure}
\fi
\ifdefined\tablename
  \renewcommand*\tablename{Tableau}
\else
  \newcommand\tablename{Tableau}
\fi
}
\@ifpackageloaded{float}{}{\usepackage{float}}
\floatstyle{ruled}
\@ifundefined{c@chapter}{\newfloat{codelisting}{h}{lop}}{\newfloat{codelisting}{h}{lop}[chapter]}
\floatname{codelisting}{Listing}
\newcommand*\listoflistings{\listof{codelisting}{Liste des Listings}}
\makeatother
\makeatletter
\@ifpackageloaded{caption}{}{\usepackage{caption}}
\@ifpackageloaded{subcaption}{}{\usepackage{subcaption}}
\makeatother
\makeatletter
\@ifpackageloaded{tcolorbox}{}{\usepackage[skins,breakable]{tcolorbox}}
\makeatother
\makeatletter
\@ifundefined{shadecolor}{\definecolor{shadecolor}{rgb}{.97, .97, .97}}
\makeatother
\makeatletter
\makeatother
\makeatletter
\makeatother
\ifLuaTeX
\usepackage[bidi=basic]{babel}
\else
\usepackage[bidi=default]{babel}
\fi
\babelprovide[main,import]{french}
% get rid of language-specific shorthands (see #6817):
\let\LanguageShortHands\languageshorthands
\def\languageshorthands#1{}
\ifLuaTeX
  \usepackage{selnolig}  % disable illegal ligatures
\fi
\IfFileExists{bookmark.sty}{\usepackage{bookmark}}{\usepackage{hyperref}}
\IfFileExists{xurl.sty}{\usepackage{xurl}}{} % add URL line breaks if available
\urlstyle{same} % disable monospaced font for URLs
\hypersetup{
  pdftitle={Dictionnaires (Cours)},
  pdflang={fr},
  colorlinks=true,
  linkcolor={blue},
  filecolor={Maroon},
  citecolor={Blue},
  urlcolor={Blue},
  pdfcreator={LaTeX via pandoc}}

\title{Dictionnaires (Cours)}
\usepackage{etoolbox}
\makeatletter
\providecommand{\subtitle}[1]{% add subtitle to \maketitle
  \apptocmd{\@title}{\par {\large #1 \par}}{}{}
}
\makeatother
\subtitle{S2 - Structures de données}
\author{}
\date{}

\begin{document}
\maketitle
\lhead{Spécialité NSI} \rhead{Terminale} \chead{} \cfoot{} \lfoot{Lycée \'Emile Duclaux} \rfoot{Page \thepage/\pageref{LastPage}} \renewcommand{\headrulewidth}{0pt} \renewcommand{\footrulewidth}{0pt} \thispagestyle{fancy} \vspace{-3cm}

\ifdefined\Shaded\renewenvironment{Shaded}{\begin{tcolorbox}[boxrule=0pt, breakable, interior hidden, enhanced, frame hidden, borderline west={3pt}{0pt}{shadecolor}, sharp corners]}{\end{tcolorbox}}\fi

\hypertarget{du-point-de-vue-utilisateur-interface}{%
\subsection{1. Du point de vue utilisateur :
Interface}\label{du-point-de-vue-utilisateur-interface}}

Nous allons maintenant étudier un autre type abstrait de données : les
\textbf{dictionnaires} aussi appelés \textbf{tableaux associatifs}.

\begin{tcolorbox}[enhanced jigsaw, breakable, colframe=quarto-callout-tip-color-frame, coltitle=black, rightrule=.15mm, opacityback=0, toprule=.15mm, opacitybacktitle=0.6, bottomtitle=1mm, colback=white, bottomrule=.15mm, toptitle=1mm, arc=.35mm, title=\textcolor{quarto-callout-tip-color}{\faLightbulb}\hspace{0.5em}{Définition}, leftrule=.75mm, colbacktitle=quarto-callout-tip-color!10!white, titlerule=0mm, left=2mm]

Un dictionnaire est une structure de donnée permettant d'\textbf{indexer
des objets par leur nom} plutôt que par un nombre.

\end{tcolorbox}

On retrouve une structure qui ressemble, à première vue, beaucoup à un
tableau (à chaque élément on associe un indice de position). Mais au
lieu d'associer chaque élément à un indice de position, dans un
dictionnaire, on associe chaque élément (on parle de \textbf{valeur}
dans un dictionnaire) à une \textbf{clé}, on dit qu'un dictionnaire
contient des \textbf{couples clé:valeur} (chaque clé est associée à une
valeur).

Un dictionnaire associe une clé à une valeur. On peut voir les tableaux
comme un dictionnaire où les clés sont des entiers allant de 0 à la
longueur de la liste moins 1.

Mais cela peut être bien plus général :

\begin{itemize}
\tightlist
\item
  les clés peuvent être des mots et les valeurs un nombre. Cela permet
  par exemple de compter le nombre de fois où un chaque mot d'un texte
  apparaît.
\item
  associer un nom (valeur) à un numéro de téléphone (clé) sans avoir
  besoin d'une liste allant de 0 à numéro max de téléphone.
\end{itemize}

Les clés ne doivent pas changer une fois créées, sinon la serrure
fabriquée dans le dictionnaire ne fonctionne plus. On ne doit donc
utiliser que des objets non modifiables pour créer des clés d'un
dictionnaire Python. Comme :

\begin{itemize}
\tightlist
\item
  des entiers
\item
  des réels
\item
  des chaines de caractères
\item
  des tuples
\end{itemize}

\begin{tcolorbox}[enhanced jigsaw, breakable, colframe=quarto-callout-caution-color-frame, coltitle=black, rightrule=.15mm, opacityback=0, toprule=.15mm, opacitybacktitle=0.6, bottomtitle=1mm, colback=white, bottomrule=.15mm, toptitle=1mm, arc=.35mm, title=\textcolor{quarto-callout-caution-color}{\faFire}\hspace{0.5em}{Exemple}, leftrule=.75mm, colbacktitle=quarto-callout-caution-color!10!white, titlerule=0mm, left=2mm]

Exemples de couples clé:valeur :

\begin{itemize}
\tightlist
\item
  prenom:Kevin, nom:Durand, naissance:17-05-2005.
\item
  prenom, nom et naissance sont des \textbf{clés} ; Kevin, Durand et
  17-05-2005 sont des \textbf{valeurs}.
\end{itemize}

\end{tcolorbox}

Les méthodes primitives permettant de définir l'interface de la
structure de dictionnaire peuvent être les suivantes :

\begin{itemize}
\tightlist
\item
  \texttt{créer()} : création d'un nouveau dictionnaire vide ;
\item
  \texttt{ajouter(dict,\ clé,\ valeur)} : on associe une nouvelle valeur
  à une nouvelle clé ;
\item
  \texttt{modifier(dict,\ clé,\ valeur)} : on modifie un couple
  clé:valeur en remplaçant la valeur courante par une autre valeur (la
  clé restant identique) ;
\item
  \texttt{supprimer(dict,\ clé)} : on supprime une clé (et donc la
  valeur qui lui est associée) ;
\item
  \texttt{rechercher(dict,\ clé)} : on recherche une valeur à l'aide de
  la clé associée à cette valeur.
\end{itemize}

\begin{tcolorbox}[enhanced jigsaw, breakable, colframe=quarto-callout-caution-color-frame, coltitle=black, rightrule=.15mm, opacityback=0, toprule=.15mm, opacitybacktitle=0.6, bottomtitle=1mm, colback=white, bottomrule=.15mm, toptitle=1mm, arc=.35mm, title=\textcolor{quarto-callout-caution-color}{\faFire}\hspace{0.5em}{Exemple}, leftrule=.75mm, colbacktitle=quarto-callout-caution-color!10!white, titlerule=0mm, left=2mm]

Soit le dictionnaire D composé des couples clé:valeur suivants :
prenom:Kevin, nom:Durand, naissance:17-05-2005. Pour chaque exemple
ci-dessous on repart du dictionnaire d'origine :

\begin{Shaded}
\begin{Highlighting}[numbers=left,,]
\NormalTok{ajouter(D, tel, }\DecValTok{0}\ErrorTok{6060606}\NormalTok{)}
\NormalTok{modifier(D,nom,Dupont)}
\NormalTok{supprimer(D, naissance)}
\NormalTok{rechercher(D, prenom)}
\end{Highlighting}
\end{Shaded}

\begin{itemize}
\tightlist
\item
  Ligne 1 : le dictionnaire D est maintenant composé des couples
  suivants : prenom:Kevin, nom:Durand, date-naissance:17-05-2005,
  tel:06060606 ;
\item
  Ligne 2 : le dictionnaire D est maintenant composé des couples
  suivants : prenom:Kevin, nom:Dupont, date-naissance:17-05-2005 ;
\item
  Ligne 3 : le dictionnaire D est maintenant composé des couples
  suivants : prenom:Kevin, nom:Durand ;
\item
  Ligne 4 : la fonction renvoie Kevin.
\end{itemize}

\end{tcolorbox}

L'utilisation de la structure dictionnaire en Python a été étudiée en
première. Il faut donc revoir le cours correspondant.

\hypertarget{du-point-de-vue-concepteur-impluxe9mentation}{%
\subsection{2. Du point de vue concepteur :
Implémentation}\label{du-point-de-vue-concepteur-impluxe9mentation}}

L'implémentation du type abstrait dictionnaire est complexe et dépasse
le cadre du programme de NSI. Cette implémentation utilise des
\textbf{fonctions de hachage}.

\begin{figure}

{\centering \includegraphics[width=0.5\textwidth,height=\textheight]{hache.png}

}

\end{figure}

L'utilisation des tables et des fonctions de hachages est omniprésente
en informatique, il est donc utile pour votre ``culture générale
informatique'', de connaître le principe des fonctions de hachages.
Voici un texte qui vous permettra de comprendre le principe des
fonctions de hachages :
\href{https://culture-informatique.net/cest-quoi-hachage/}{c'est quoi le
hachage ?} . Pour avoir quelques idées sur le principe des tables de
hachages, je vous recommande le visionnage de cette vidéo :
\href{https://www.youtube.com/watch?v=CkLctGYWFPA}{les tables de
hachage}.

On peut retenir que

\begin{tcolorbox}[enhanced jigsaw, breakable, colframe=quarto-callout-important-color-frame, coltitle=black, rightrule=.15mm, opacityback=0, toprule=.15mm, opacitybacktitle=0.6, bottomtitle=1mm, colback=white, bottomrule=.15mm, toptitle=1mm, arc=.35mm, title=\textcolor{quarto-callout-important-color}{\faExclamation}\hspace{0.5em}{À retenir}, leftrule=.75mm, colbacktitle=quarto-callout-important-color!10!white, titlerule=0mm, left=2mm]

La complexité de recherche, d'ajout et de suppression d'un élément dans
un dictionnaire est en \(\mathcal{O}(1)\) : elle ne dépend pas du nombre
d'éléments présents dans le dictionnaire.

En comparaison, la complexité de l'algorithme de recherche dans un
tableau non trié est \(\mathcal{O}(n)\).

La structure de dictionnaire est donc une structure très efficace !
N'hésitez pas à l'utiliser car son temps moyen d'exécution est très
rapide.

\end{tcolorbox}



\end{document}
