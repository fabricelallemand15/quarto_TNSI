% Options for packages loaded elsewhere
\PassOptionsToPackage{unicode}{hyperref}
\PassOptionsToPackage{hyphens}{url}
\PassOptionsToPackage{dvipsnames,svgnames,x11names}{xcolor}
%
\documentclass[
  a4paper,
  DIV=11,
  numbers=noendperiod]{scrartcl}

\usepackage{amsmath,amssymb}
\usepackage{lmodern}
\usepackage{iftex}
\ifPDFTeX
  \usepackage[T1]{fontenc}
  \usepackage[utf8]{inputenc}
  \usepackage{textcomp} % provide euro and other symbols
\else % if luatex or xetex
  \usepackage{unicode-math}
  \defaultfontfeatures{Scale=MatchLowercase}
  \defaultfontfeatures[\rmfamily]{Ligatures=TeX,Scale=1}
\fi
% Use upquote if available, for straight quotes in verbatim environments
\IfFileExists{upquote.sty}{\usepackage{upquote}}{}
\IfFileExists{microtype.sty}{% use microtype if available
  \usepackage[]{microtype}
  \UseMicrotypeSet[protrusion]{basicmath} % disable protrusion for tt fonts
}{}
\makeatletter
\@ifundefined{KOMAClassName}{% if non-KOMA class
  \IfFileExists{parskip.sty}{%
    \usepackage{parskip}
  }{% else
    \setlength{\parindent}{0pt}
    \setlength{\parskip}{6pt plus 2pt minus 1pt}}
}{% if KOMA class
  \KOMAoptions{parskip=half}}
\makeatother
\usepackage{xcolor}
\usepackage[top=20mm,bottom=20mm,left=20mm,right=20mm,heightrounded]{geometry}
\setlength{\emergencystretch}{3em} % prevent overfull lines
\setcounter{secnumdepth}{-\maxdimen} % remove section numbering
% Make \paragraph and \subparagraph free-standing
\ifx\paragraph\undefined\else
  \let\oldparagraph\paragraph
  \renewcommand{\paragraph}[1]{\oldparagraph{#1}\mbox{}}
\fi
\ifx\subparagraph\undefined\else
  \let\oldsubparagraph\subparagraph
  \renewcommand{\subparagraph}[1]{\oldsubparagraph{#1}\mbox{}}
\fi

\usepackage{color}
\usepackage{fancyvrb}
\newcommand{\VerbBar}{|}
\newcommand{\VERB}{\Verb[commandchars=\\\{\}]}
\DefineVerbatimEnvironment{Highlighting}{Verbatim}{commandchars=\\\{\}}
% Add ',fontsize=\small' for more characters per line
\usepackage{framed}
\definecolor{shadecolor}{RGB}{241,243,245}
\newenvironment{Shaded}{\begin{snugshade}}{\end{snugshade}}
\newcommand{\AlertTok}[1]{\textcolor[rgb]{0.68,0.00,0.00}{#1}}
\newcommand{\AnnotationTok}[1]{\textcolor[rgb]{0.37,0.37,0.37}{#1}}
\newcommand{\AttributeTok}[1]{\textcolor[rgb]{0.40,0.45,0.13}{#1}}
\newcommand{\BaseNTok}[1]{\textcolor[rgb]{0.68,0.00,0.00}{#1}}
\newcommand{\BuiltInTok}[1]{\textcolor[rgb]{0.00,0.23,0.31}{#1}}
\newcommand{\CharTok}[1]{\textcolor[rgb]{0.13,0.47,0.30}{#1}}
\newcommand{\CommentTok}[1]{\textcolor[rgb]{0.37,0.37,0.37}{#1}}
\newcommand{\CommentVarTok}[1]{\textcolor[rgb]{0.37,0.37,0.37}{\textit{#1}}}
\newcommand{\ConstantTok}[1]{\textcolor[rgb]{0.56,0.35,0.01}{#1}}
\newcommand{\ControlFlowTok}[1]{\textcolor[rgb]{0.00,0.23,0.31}{#1}}
\newcommand{\DataTypeTok}[1]{\textcolor[rgb]{0.68,0.00,0.00}{#1}}
\newcommand{\DecValTok}[1]{\textcolor[rgb]{0.68,0.00,0.00}{#1}}
\newcommand{\DocumentationTok}[1]{\textcolor[rgb]{0.37,0.37,0.37}{\textit{#1}}}
\newcommand{\ErrorTok}[1]{\textcolor[rgb]{0.68,0.00,0.00}{#1}}
\newcommand{\ExtensionTok}[1]{\textcolor[rgb]{0.00,0.23,0.31}{#1}}
\newcommand{\FloatTok}[1]{\textcolor[rgb]{0.68,0.00,0.00}{#1}}
\newcommand{\FunctionTok}[1]{\textcolor[rgb]{0.28,0.35,0.67}{#1}}
\newcommand{\ImportTok}[1]{\textcolor[rgb]{0.00,0.46,0.62}{#1}}
\newcommand{\InformationTok}[1]{\textcolor[rgb]{0.37,0.37,0.37}{#1}}
\newcommand{\KeywordTok}[1]{\textcolor[rgb]{0.00,0.23,0.31}{#1}}
\newcommand{\NormalTok}[1]{\textcolor[rgb]{0.00,0.23,0.31}{#1}}
\newcommand{\OperatorTok}[1]{\textcolor[rgb]{0.37,0.37,0.37}{#1}}
\newcommand{\OtherTok}[1]{\textcolor[rgb]{0.00,0.23,0.31}{#1}}
\newcommand{\PreprocessorTok}[1]{\textcolor[rgb]{0.68,0.00,0.00}{#1}}
\newcommand{\RegionMarkerTok}[1]{\textcolor[rgb]{0.00,0.23,0.31}{#1}}
\newcommand{\SpecialCharTok}[1]{\textcolor[rgb]{0.37,0.37,0.37}{#1}}
\newcommand{\SpecialStringTok}[1]{\textcolor[rgb]{0.13,0.47,0.30}{#1}}
\newcommand{\StringTok}[1]{\textcolor[rgb]{0.13,0.47,0.30}{#1}}
\newcommand{\VariableTok}[1]{\textcolor[rgb]{0.07,0.07,0.07}{#1}}
\newcommand{\VerbatimStringTok}[1]{\textcolor[rgb]{0.13,0.47,0.30}{#1}}
\newcommand{\WarningTok}[1]{\textcolor[rgb]{0.37,0.37,0.37}{\textit{#1}}}

\providecommand{\tightlist}{%
  \setlength{\itemsep}{0pt}\setlength{\parskip}{0pt}}\usepackage{longtable,booktabs,array}
\usepackage{calc} % for calculating minipage widths
% Correct order of tables after \paragraph or \subparagraph
\usepackage{etoolbox}
\makeatletter
\patchcmd\longtable{\par}{\if@noskipsec\mbox{}\fi\par}{}{}
\makeatother
% Allow footnotes in longtable head/foot
\IfFileExists{footnotehyper.sty}{\usepackage{footnotehyper}}{\usepackage{footnote}}
\makesavenoteenv{longtable}
\usepackage{graphicx}
\makeatletter
\def\maxwidth{\ifdim\Gin@nat@width>\linewidth\linewidth\else\Gin@nat@width\fi}
\def\maxheight{\ifdim\Gin@nat@height>\textheight\textheight\else\Gin@nat@height\fi}
\makeatother
% Scale images if necessary, so that they will not overflow the page
% margins by default, and it is still possible to overwrite the defaults
% using explicit options in \includegraphics[width, height, ...]{}
\setkeys{Gin}{width=\maxwidth,height=\maxheight,keepaspectratio}
% Set default figure placement to htbp
\makeatletter
\def\fps@figure{htbp}
\makeatother

\usepackage{fancyhdr} \pagestyle{fancy} \usepackage{lastpage}
\KOMAoption{captions}{tablesignature}
\makeatletter
\@ifpackageloaded{tcolorbox}{}{\usepackage[many]{tcolorbox}}
\@ifpackageloaded{fontawesome5}{}{\usepackage{fontawesome5}}
\definecolor{quarto-callout-color}{HTML}{909090}
\definecolor{quarto-callout-note-color}{HTML}{0758E5}
\definecolor{quarto-callout-important-color}{HTML}{CC1914}
\definecolor{quarto-callout-warning-color}{HTML}{EB9113}
\definecolor{quarto-callout-tip-color}{HTML}{00A047}
\definecolor{quarto-callout-caution-color}{HTML}{FC5300}
\definecolor{quarto-callout-color-frame}{HTML}{acacac}
\definecolor{quarto-callout-note-color-frame}{HTML}{4582ec}
\definecolor{quarto-callout-important-color-frame}{HTML}{d9534f}
\definecolor{quarto-callout-warning-color-frame}{HTML}{f0ad4e}
\definecolor{quarto-callout-tip-color-frame}{HTML}{02b875}
\definecolor{quarto-callout-caution-color-frame}{HTML}{fd7e14}
\makeatother
\makeatletter
\makeatother
\makeatletter
\makeatother
\makeatletter
\@ifpackageloaded{caption}{}{\usepackage{caption}}
\AtBeginDocument{%
\ifdefined\contentsname
  \renewcommand*\contentsname{Table des matières}
\else
  \newcommand\contentsname{Table des matières}
\fi
\ifdefined\listfigurename
  \renewcommand*\listfigurename{Liste des Figures}
\else
  \newcommand\listfigurename{Liste des Figures}
\fi
\ifdefined\listtablename
  \renewcommand*\listtablename{Liste des Tables}
\else
  \newcommand\listtablename{Liste des Tables}
\fi
\ifdefined\figurename
  \renewcommand*\figurename{Figure}
\else
  \newcommand\figurename{Figure}
\fi
\ifdefined\tablename
  \renewcommand*\tablename{Tableau}
\else
  \newcommand\tablename{Tableau}
\fi
}
\@ifpackageloaded{float}{}{\usepackage{float}}
\floatstyle{ruled}
\@ifundefined{c@chapter}{\newfloat{codelisting}{h}{lop}}{\newfloat{codelisting}{h}{lop}[chapter]}
\floatname{codelisting}{Listing}
\newcommand*\listoflistings{\listof{codelisting}{Liste des Listings}}
\makeatother
\makeatletter
\@ifpackageloaded{caption}{}{\usepackage{caption}}
\@ifpackageloaded{subcaption}{}{\usepackage{subcaption}}
\makeatother
\makeatletter
\@ifpackageloaded{tcolorbox}{}{\usepackage[many]{tcolorbox}}
\makeatother
\makeatletter
\@ifundefined{shadecolor}{\definecolor{shadecolor}{rgb}{.97, .97, .97}}
\makeatother
\makeatletter
\makeatother
\ifLuaTeX
\usepackage[bidi=basic]{babel}
\else
\usepackage[bidi=default]{babel}
\fi
\babelprovide[main,import]{french}
% get rid of language-specific shorthands (see #6817):
\let\LanguageShortHands\languageshorthands
\def\languageshorthands#1{}
\ifLuaTeX
  \usepackage{selnolig}  % disable illegal ligatures
\fi
\IfFileExists{bookmark.sty}{\usepackage{bookmark}}{\usepackage{hyperref}}
\IfFileExists{xurl.sty}{\usepackage{xurl}}{} % add URL line breaks if available
\urlstyle{same} % disable monospaced font for URLs
\hypersetup{
  pdftitle={Piles (Cours)},
  pdflang={fr},
  colorlinks=true,
  linkcolor={blue},
  filecolor={Maroon},
  citecolor={Blue},
  urlcolor={Blue},
  pdfcreator={LaTeX via pandoc}}

\title{Piles (Cours)}
\usepackage{etoolbox}
\makeatletter
\providecommand{\subtitle}[1]{% add subtitle to \maketitle
  \apptocmd{\@title}{\par {\large #1 \par}}{}{}
}
\makeatother
\subtitle{S2 - Structures de données}
\author{}
\date{}

\begin{document}
\maketitle
\lhead{Spécialité NSI} \rhead{Terminale} \chead{} \cfoot{} \lfoot{Lycée \'Emile Duclaux} \rfoot{Page \thepage/\pageref{LastPage}} \renewcommand{\headrulewidth}{0pt} \renewcommand{\footrulewidth}{0pt} \thispagestyle{fancy} \vspace{-2cm}

\ifdefined\Shaded\renewenvironment{Shaded}{\begin{tcolorbox}[boxrule=0pt, borderline west={3pt}{0pt}{shadecolor}, enhanced, sharp corners, breakable, frame hidden, interior hidden]}{\end{tcolorbox}}\fi

\hypertarget{du-point-de-vue-utilisateur-interface}{%
\subsection{1. Du point de vue utilisateur :
interface}\label{du-point-de-vue-utilisateur-interface}}

\begin{tcolorbox}[enhanced jigsaw, left=2mm, bottomrule=.15mm, titlerule=0mm, opacitybacktitle=0.6, coltitle=black, bottomtitle=1mm, toptitle=1mm, colbacktitle=quarto-callout-tip-color!10!white, opacityback=0, colback=white, title=\textcolor{quarto-callout-tip-color}{\faLightbulb}\hspace{0.5em}{Définition}, arc=.35mm, leftrule=.75mm, toprule=.15mm, rightrule=.15mm, breakable]

La \textbf{pile} est une structure de données qui permet de stocker des
données et d'y accéder. Une pile se comporte comme une \textbf{pile
d'assiettes} :

\begin{itemize}
\tightlist
\item
  on ajoute des nouvelles assiettes au sommet de la pile ;
\item
  quand on veut en retirer une, on est obligé de prendre celle située au
  sommet.
\end{itemize}

\end{tcolorbox}

On parle de mode \textbf{LIFO} (Last In, First Out, en anglais, dernier
arrivé, premier sorti), c'est-à-dire que le dernier élément ajouté à la
structure sera le prochain élément auquel on accédera. Les premiers
éléments ayant été ajoutés devront « attendre » que tous les éléments
qui ont été ajoutés après eux soient sortis de la pile. Contrairement
aux listes, on ne peut donc pas accéder à n'importe quelle valeur de la
structure (pas d'index). Pour gérer cette contrainte, on définit le
\textbf{sommet} de la pile qui caractérise l'emplacement pour ajouter ou
retirer des éléments.

\begin{figure}

\begin{minipage}[t]{0.50\linewidth}

{\centering 

\raisebox{-\height}{

\includegraphics[width=3.125in,height=\textheight]{pile_assiettes.jpg}

}

}

\end{minipage}%
%
\begin{minipage}[t]{0.50\linewidth}

{\centering 

\raisebox{-\height}{

\includegraphics{Data_stack.png}

}

}

\end{minipage}%

\end{figure}

L'interface suivante permet de définir le type abstrait de données
\textbf{pile} :

\begin{itemize}
\tightlist
\item
  \texttt{creer()}, qui crée une pile vide ;
\item
  \texttt{taille(pile)}, qui permet de connaître le nombre d'éléments
  contenus dans la pile ;
\item
  \texttt{est\_vide(pile)}, qui renvoie vrai si la pile est vide, faux
  sinon ;
\item
  \texttt{empiler(pile,\ element)}, qui ajoute un élément au sommet de
  la pile (qui devient le nouveau sommet) ;
\item
  \texttt{depiler(pile)}, qui retire et renvoie l'élément situé au
  sommet de la pile (le nouveau sommet devient l'élément qui suivait
  l'ancien sommet) ;
\item
  \texttt{sommet(pile)}, qui renvoie l'élément situé au sommet de la
  pile (sans le retirer).
\end{itemize}

L'opération d'empilement se dit ``push'' en anglais, l'opération de
dépilement se dit ``pop''.

\begin{tcolorbox}[enhanced jigsaw, left=2mm, bottomrule=.15mm, titlerule=0mm, opacitybacktitle=0.6, coltitle=black, bottomtitle=1mm, toptitle=1mm, colbacktitle=quarto-callout-note-color!10!white, opacityback=0, colback=white, title=\textcolor{quarto-callout-note-color}{\faInfo}\hspace{0.5em}{La pile est utile dans différents types de problèmes}, arc=.35mm, leftrule=.75mm, toprule=.15mm, rightrule=.15mm, breakable]

\begin{itemize}
\tightlist
\item
  algorithme d'un navigateur pour pouvoir mémoriser les pages web et
  revenir en arrière (ou ré-avancer) sur certaines pages ;
\item
  stocker des actions et les annuler (ou les réappliquer), sur
  l'ordinateur (fonction CTRL+Z, et CTRL+Y) ;
\item
  coder une calculatrice en notation polonaise inversée (voir exercices)
  ;
\item
  algorithme du parcours en profondeur pour les arbres et les graphes,
  par exemple, pour résoudre un labyrinthe, trouver un trajet sur une
  carte\ldots{} (voir séquence 6) ;
\item
  écrire des versions itératives de certains algorithmes récursifs (voir
  séquence 1) ;
\item
  illustration du fonctionnement de la pile d'appels des fonctions lors
  de l'exécution d'un programme.
\end{itemize}

\end{tcolorbox}

\begin{tcolorbox}[enhanced jigsaw, left=2mm, bottomrule=.15mm, titlerule=0mm, opacitybacktitle=0.6, coltitle=black, bottomtitle=1mm, toptitle=1mm, colbacktitle=quarto-callout-caution-color!10!white, opacityback=0, colback=white, title=\textcolor{quarto-callout-caution-color}{\faFire}\hspace{0.5em}{Exemple}, arc=.35mm, leftrule=.75mm, toprule=.15mm, rightrule=.15mm, breakable]

Supposons implémenté le type abstrait \textbf{pile}. Nous disposons
d'une interface composée des six primitives décrites ci-dessus. On
considère une pile \texttt{P} composée des éléments suivants : 12, 14,
8, 7, 19 et 22 (le sommet de la pile est 22). On exécute le code suivant
ligne par ligne :

\begin{Shaded}
\begin{Highlighting}[numbers=left,,]
\NormalTok{    depiler(P)}
\NormalTok{    empiler(P,}\DecValTok{42}\NormalTok{)}
\NormalTok{    depiler(P)}
\NormalTok{    taille(P)}
\NormalTok{    estVide(P)}
\end{Highlighting}
\end{Shaded}

\begin{itemize}
\tightlist
\item
  L'exécution de la ligne 1 renvoie la valeur 22 et la pile est
  maintenant composée des éléments 12, 14, 8, 7 et 19 ;
\item
  L'exécution de la ligne 2 place l'élément 42 au sommet de la pile ;
\item
  L'exécution de la ligne 3 renvoie la valeur 42 et la pile est
  maintenant à nouveau composée des éléments 12, 14, 8, 7 et 19 ;
\item
  La ligne 4 renvoie la taille de \texttt{P} : 5 ;
\item
  La pile n'est pas vide, on obtient dont \texttt{False}.
\end{itemize}

\end{tcolorbox}

\hypertarget{du-point-de-vue-concepteur-impluxe9mentations}{%
\subsection{2. Du point de vue concepteur :
implémentation(s)}\label{du-point-de-vue-concepteur-impluxe9mentations}}

\hypertarget{impluxe9mentation-avec-listes-python}{%
\subsubsection{Implémentation avec listes
Python}\label{impluxe9mentation-avec-listes-python}}

Première solution, on peut implémenter une pile en utilisant des listes
Python. Cette solution est très facile, car les méthodes \texttt{append}
et \texttt{pop} des objets de type \texttt{list} correspondent
exactement aux primitives \textbf{empiler} et \textbf{dépiler} de la
structure pile.

\begin{Shaded}
\begin{Highlighting}[]
\CommentTok{"""Implémentation des piles avec des listes Python"""}


\KeywordTok{def}\NormalTok{ creer():}
    \CommentTok{"""Retourne une pile vide"""}
    \ControlFlowTok{return}\NormalTok{ []}


\KeywordTok{def}\NormalTok{ taille(pile):}
    \CommentTok{"""Retourne le nombre d\textquotesingle{}éléments de la pile"""}
    \ControlFlowTok{return} \BuiltInTok{len}\NormalTok{(pile)}


\KeywordTok{def}\NormalTok{ est\_vide(pile):}
    \CommentTok{"""Retourne True si la pile est vide, False sinon"""}
    \ControlFlowTok{return}\NormalTok{ pile }\OperatorTok{==}\NormalTok{ []}


\KeywordTok{def}\NormalTok{ empiler(pile, element):}
    \CommentTok{"""Empile un nouvel élément au sommet de la pile"""}
\NormalTok{    pile.append(element)}


\KeywordTok{def}\NormalTok{ depiler(pile):}
    \CommentTok{"""Retourne l\textquotesingle{}élément situé au sommet de la pile}
\CommentTok{    et le supprime de celle{-}ci"""}
    \ControlFlowTok{if} \KeywordTok{not}\NormalTok{ est\_vide(pile):}
        \ControlFlowTok{return}\NormalTok{ pile.pop()}
    \ControlFlowTok{else}\NormalTok{:}
        \ControlFlowTok{return} \VariableTok{None}


\KeywordTok{def}\NormalTok{ sommet(pile):}
    \CommentTok{"""Retourne l\textquotesingle{}élément situé au sommet de la pile"""}
    \ControlFlowTok{if} \KeywordTok{not}\NormalTok{ est\_vide(pile):}
        \ControlFlowTok{return}\NormalTok{ pile[}\OperatorTok{{-}}\DecValTok{1}\NormalTok{]}
    \ControlFlowTok{else}\NormalTok{:}
        \ControlFlowTok{return} \VariableTok{None}
\end{Highlighting}
\end{Shaded}

Cette implémentation sera testée en exercices.

\hypertarget{impluxe9mentation-en-poo}{%
\subsubsection{Implémentation en POO}\label{impluxe9mentation-en-poo}}

Conformément au programme, on se limite à une version \textbf{naïve} de
la POO. On pourra à titre d'exercice reprendre cette implémentation en
respectant les règles plus strictes édictées dans
\href{../langagesProgr/POO_complements.qmd}{les compléments de cours sur
la POO}.

On reprend l'idée de \textbf{chaînon} utilisée pour les listes.

\begin{Shaded}
\begin{Highlighting}[]
\CommentTok{"""Implémentation du type abstrait pile en POO"""}


\KeywordTok{class}\NormalTok{ Chainon:}
    \KeywordTok{def} \FunctionTok{\_\_init\_\_}\NormalTok{(}\VariableTok{self}\NormalTok{, element}\OperatorTok{=}\VariableTok{None}\NormalTok{, suivant}\OperatorTok{=}\VariableTok{None}\NormalTok{):}
        \CommentTok{"""element est la valeur du chainon et suivant est le chainon qui suit"""}
        \VariableTok{self}\NormalTok{.element }\OperatorTok{=}\NormalTok{ element}
        \VariableTok{self}\NormalTok{.suivant }\OperatorTok{=}\NormalTok{ suivant}


\KeywordTok{class}\NormalTok{ Pile:}
    \KeywordTok{def} \FunctionTok{\_\_init\_\_}\NormalTok{(}\VariableTok{self}\NormalTok{):}
        \CommentTok{"""Crée une pile vide"""}
        \VariableTok{self}\NormalTok{.summit }\OperatorTok{=}\NormalTok{ Chainon()}

    \KeywordTok{def}\NormalTok{ taille(}\VariableTok{self}\NormalTok{):}
        \CommentTok{"""Retourne le nombre d\textquotesingle{}éléments dans la pile"""}
        \BuiltInTok{long} \OperatorTok{=} \DecValTok{0}
\NormalTok{        chainon }\OperatorTok{=} \VariableTok{self}\NormalTok{.summit}
        \ControlFlowTok{while}\NormalTok{ chainon.element }\KeywordTok{is} \KeywordTok{not} \VariableTok{None}\NormalTok{:}
\NormalTok{            chainon }\OperatorTok{=}\NormalTok{ chainon.suivant}
            \BuiltInTok{long} \OperatorTok{=} \BuiltInTok{long} \OperatorTok{+} \DecValTok{1}
        \ControlFlowTok{return} \BuiltInTok{long}

    \KeywordTok{def}\NormalTok{ est\_vide(}\VariableTok{self}\NormalTok{) }\OperatorTok{{-}\textgreater{}} \BuiltInTok{bool}\NormalTok{:}
        \CommentTok{"""Retourne True si la pile est vide et False sinon"""}
        \ControlFlowTok{return} \VariableTok{self}\NormalTok{.summit.element }\KeywordTok{is} \VariableTok{None}

    \KeywordTok{def}\NormalTok{ empiler(}\VariableTok{self}\NormalTok{, element):}
        \CommentTok{"""Empile element qu sommet de la pile"""}
        \VariableTok{self}\NormalTok{.summit }\OperatorTok{=}\NormalTok{ Chainon(element, }\VariableTok{self}\NormalTok{.summit)}

    \KeywordTok{def}\NormalTok{ depiler(}\VariableTok{self}\NormalTok{):}
        \CommentTok{"""Retourne l\textquotesingle{}élément situé au sommet de la pile}
\CommentTok{        et le supprime de celle{-}ci"""}
\NormalTok{        item }\OperatorTok{=} \VariableTok{self}\NormalTok{.summit.element}
        \VariableTok{self}\NormalTok{.summit }\OperatorTok{=} \VariableTok{self}\NormalTok{.summit.suivant}
        \ControlFlowTok{return}\NormalTok{ item}

    \KeywordTok{def}\NormalTok{ sommet(}\VariableTok{self}\NormalTok{):}
        \CommentTok{"""Retourne la valeur du sommet de la pile"""}
        \ControlFlowTok{return} \VariableTok{self}\NormalTok{.summit.element}

\end{Highlighting}
\end{Shaded}

Cette implémentation sera testée en exercices.

\begin{tcolorbox}[enhanced jigsaw, left=2mm, bottomrule=.15mm, titlerule=0mm, opacitybacktitle=0.6, coltitle=black, bottomtitle=1mm, toptitle=1mm, colbacktitle=quarto-callout-important-color!10!white, opacityback=0, colback=white, title=\textcolor{quarto-callout-important-color}{\faExclamation}\hspace{0.5em}{Comparaison des deux implémentations}, arc=.35mm, leftrule=.75mm, toprule=.15mm, rightrule=.15mm, breakable]

On peut comparer en termes de temps d'exécution l'efficacité de ces deux
implémentations.

On utilise pour cela la bibliothèque \texttt{timeit} présentée dans
\href{https://www.flallemand.fr/wp/2022/06/05/mesurer-le-temps-dexecution-dun-fragment-de-code/}{cet
article}.

Pour cela, on ajoute les lignes suivantes au code de la version ``listes
Python'' :

\begin{Shaded}
\begin{Highlighting}[]
\ImportTok{import}\NormalTok{ timeit}

\NormalTok{ma\_pile }\OperatorTok{=}\NormalTok{ creer()}
\BuiltInTok{print}\NormalTok{(timeit.timeit(}\StringTok{\textquotesingle{}empiler(ma\_pile,1)\textquotesingle{}}\NormalTok{, number}\OperatorTok{=}\DecValTok{10000000}\NormalTok{, }\BuiltInTok{globals}\OperatorTok{=}\BuiltInTok{globals}\NormalTok{()))}
\end{Highlighting}
\end{Shaded}

on obtient :

\begin{Shaded}
\begin{Highlighting}[]
\FloatTok{2.2605971000011778}
\end{Highlighting}
\end{Shaded}

et les lignes suivantes au code de la version ``POO'' :

\begin{Shaded}
\begin{Highlighting}[]
\ImportTok{import}\NormalTok{ timeit}

\NormalTok{ma\_pile }\OperatorTok{=}\NormalTok{ Pile()}
\BuiltInTok{print}\NormalTok{(timeit.timeit(}\StringTok{\textquotesingle{}ma\_pile.empiler(1)\textquotesingle{}}\NormalTok{, number}\OperatorTok{=}\DecValTok{10000000}\NormalTok{, }\BuiltInTok{globals}\OperatorTok{=}\BuiltInTok{globals}\NormalTok{()))}
\end{Highlighting}
\end{Shaded}

on obtient :

\begin{Shaded}
\begin{Highlighting}[]
\FloatTok{7.292327400005888}
\end{Highlighting}
\end{Shaded}

On constate donc que la version utilisant les listes Python est beaucoup
plus efficace. Cela s'explique notamment par le fait que l'implantation
avec les listes Python repose sur une programmation avancée et
optimisée, contrairement à l'implantation objet qui a été construite
sans utiliser de structure externe.

Néanmoins, on peut montrer que, dans les deux implémentations, les
opérations d'empilement et de dépilement sont en \(\mathcal{O}(1)\), ce
qui signifie que le temps d'exécution ne dépend pas du nombre de
données.

L'utilisation des listes est la plus efficace. Néanmoins, l'implantation
objet a pour avantage de montrer et d'assimiler le fonctionnement
interne de cette structure. C'est donc un meilleur outil d'apprentissage
des concepts.

\end{tcolorbox}



\end{document}
